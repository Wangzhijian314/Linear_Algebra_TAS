\documentclass{ctexart}
\usepackage{amsmath}
\usepackage{amssymb}
\usepackage{enumitem}

\newenvironment{question}[5]{%
	\noindent\textbf{题型:}#1\\
	\textbf{主题:}#2\\
	\textbf{难度:}#3\\
	\textbf{题目:}#4\\
	\textbf{解析:}#5\\
	\vspace{1em}
}{}

\begin{document}
	
	\begin{question}
		{选择题}
		{行列式}
		{容易}
		{1.排列 41253 的逆序数为(  ).A. 4 B. 0 C. 5 D. 3}
		{41253 所含逆序为 $41,42,43,53$ ,所以 41253 的逆序数 $N(41253)=4$.}
	\end{question}
	
	\begin{question}
		{选择题}
		{行列式}
		{容易}
		{2.排列 3712456 的逆序数为(  ).A. 7 B. 6 C. 5 D. 10}
		{3712456 所含逆序为 $31,71,32,72,74,75,76$ ,所以 3712456 的逆序数 $N(3712456)=7$.}
	\end{question}
	
	\begin{question}
		{选择题}
		{行列式}
		{容易}
		{3.排列 36715284 的逆序数为(  ).A. 13 B. 9 C. 12 D. 10}
		{36715284 的逆序数为 $N(36715284)=2+4+4+0+2+0+1=13$.}
	\end{question}
	
	\begin{question}
		{选择题}
		{行列式}
		{容易}
		{4.排列 654321 的逆序数为(  ).A. 15 B. 9 C. 12 D. 11}
		{654321 的逆序数为 $N(654321)=15$.}
	\end{question}
	
	\begin{question}
		{选择题}
		{行列式}
		{容易}
		{5.排列 54321 的逆序数为(  ).A. 10 B. 9 C. 11 D. 12}
		{54321 的逆序数为 $N(54321)=10$.}
	\end{question}
	
	\begin{question}
		{选择题}
		{行列式}
		{容易}
		{6.排列 42153 的逆序数为(  ).A. 5 B. 0 C. 4 D. 3}
		{42153 的逆序数为 $N(42153)=5$.}
	\end{question}
	
	\begin{question}
		{选择题}
		{行列式}
		{容易}
		{7.排列 42153 的逆序数为(  ).A. 5 B. 0 C. 4 D. 3}
		{42153 的逆序数为 $N(42153)=5$.}
	\end{question}
	
	\begin{question}
		{选择题}
		{行列式}
		{容易}
		{8.排列 13725468 的逆序数为(  ).A. 6 B. 5 C. 7 D. 8}
		{13725468 的逆序数为 $N(13725468)=6$.}
	\end{question}
	
	\begin{question}
		{选择题}
		{行列式}
		{容易}
		{9.排列 361524 的逆序数为(  ).A. 8 B. 9 C. 11 D. 10}
		{361524 的逆序数为 $N(361524)=8$.}
	\end{question}
	
	\begin{question}
		{选择题}
		{行列式}
		{容易}
		{10.排列 634512 的逆序数为(  ).A. 11 B. 9 C. 12 D. 10}
		{634512 的逆序数为 $N(634512)=11$.}
	\end{question}
	
	
	\begin{question}
		{选择题}
		{行列式}
		{容易}
		{11.排列 $n(n-1)(n-2)\cdots321$ 的逆序数为( )。A. $\frac{n(n-1)}{2}$ B. $n$ C. $n-1$ D. 不确定}
		{$n(n-1)(n-2)\cdots321$ 的逆序数为 $(n-1)+(n-2)+\cdots+2+1=\frac{n(n-1)}{2}$.}
	\end{question}
	
	\begin{question}
		{选择题}
		{行列式}
		{容易}
		{12.下列排列是偶排列的是()。A. 12345 B. 53214 C. 654321 D. 32145}
		{$\tau(12345)=0,\ \tau(53214)=7,\ \tau(654321)=15,\ \tau(32145)=3$,所以只有 12345 是偶排列。}
	\end{question}
	
	\begin{question}
		{选择题}
		{行列式}
		{中等}
		{13.计算行列式 $\begin{matrix}0 & 1 & 0 & \cdots & 0 \\ 0 & 0 & 2 & \cdots & 0 \\ \vdots & \vdots & \vdots & & \vdots \\ 0 & 0 & 0 & \cdots & n-1 \\ n & 0 & 0 & \cdots & 0\end{matrix}$ 的值是( )。A. $(-1)^{n-1}n!$ B. $n$ C. $n(n+1)$ D. $n(n-1)$}
		{该行列式的非零项只有 $a_{12}a_{23}\cdots a_{n-1,n}a_{n1}$,其逆序数为 $n-1$,所以行列式值为 $(-1)^{n-1}\times1\times2\times\cdots\times n = (-1)^{n-1}n!$。}
	\end{question}
	
	\begin{question}
		{选择题}
		{行列式}
		{中等}
		{14.计算行列式 $\begin{matrix}0 & 0 & 1 & 0 \\ 0 & 1 & 0 & 0 \\ 0 & 0 & 0 & 1 \\ -1 & 0 & 0 & 0\end{matrix}$ 的值是( )。A. -1 B. 1 C. 2 D. 0}
		{仅有非零项为 $a_{13}a_{22}a_{34}a_{41}$,其对应排列为 $3241$,逆序数为4,故带正号;乘积为 $1\times1\times1\times(-1) = -1$。}
	\end{question}
	
	\begin{question}
		{选择题}
		{行列式}
		{容易}
		{15.判断 4 阶行列式中 $a_{11}a_{33}a_{44}a_{22}$ 和 $a_{24}a_{31}a_{13}a_{42}$ 的符号分别为( )。A. 正,正 B. 正,负 C. 负,正 D. 负,负}
		{$a_{11}a_{33}a_{44}a_{22}$ 实为 $a_{11}a_{22}a_{33}a_{44}$,为正;$a_{24}a_{31}a_{13}a_{42}$ 的列标为 3412,为偶排列,故也为正。}
	\end{question}
	
	\begin{question}
		{选择题}
		{行列式}
		{中等}
		{16.行列式 $\begin{matrix}x & x & 1 & 0 \\ 1 & x & 2 & 3 \\ 3 & 2 & x & 2 \\ 1 & 1 & 2 & x\end{matrix}$ 中 $x^4$ 的系数为( )。A. 1 B. 3 C. -1 D. 2}
		{含有 $x^4$ 的项只有 $x\cdot x\cdot x\cdot x$ 一项,带正号,所以系数为 1。}
	\end{question}
	
	\begin{question}
		{选择题}
		{行列式}
		{中等}
		{17.行列式 $\begin{matrix}x & 1 & 1 & 2 \\ 1 & x & 1 & -1 \\ 3 & 2 & x & 1 \\ 1 & 1 & 2x & 1\end{matrix}$ 中 $x^3$ 的系数为( )。A. -1 B. 3 C. 1 D. 2}
		{含 $x^3$ 的项有两项,一项为 $x\cdot x\cdot x\cdot1$,带正号;另一项为 $x\cdot x\cdot1\cdot2x$,带负号;系数和为 $1-2=-1$。}
	\end{question}
	
	\begin{question}
		{判断题}
		{行列式}
		{容易}
		{1.排列12543678是奇排列(  ).}
		{12543678 逆序数为 3 ,为奇排列.}
	\end{question}
	
	
	\begin{question}
		{判断题}
		{行列式}
		{容易}
		{2. $a_{21} a_{53} a_{16} a_{42} a_{65} a_{34}$ 在六阶行列式中是带负号的项()。}
		{$N(251463)+N(136254)=6+5=11$ 为奇数,所以 $a_{21} a_{53} a_{16} a_{42} a_{65} a_{34}$ 前面应冠以负号.}
	\end{question}
	
	\begin{question}
		{判断题}
		{行列式}
		{容易}
		{3. $a_{15} a_{23} a_{32} a_{44} a_{51} a_{66}$ 在六阶行列式中是带负号的项()。}
		{$N(532416)=8$ 为偶数,所以 $a_{15} a_{23} a_{32} a_{44} a_{51} a_{66}$ 前面应冠以正号,由此知结论错误.}
	\end{question}
	
	\begin{question}
		{判断题}
		{行列式}
		{容易}
		{4. $a_{11} a_{26} a_{32} a_{44} a_{53} a_{65}$ 在六阶行列式中是带负号的项()。}
		{$N(162435)=5$ 为奇数,所以 $a_{11} a_{26} a_{32} a_{44} a_{53} a_{65}$ 前面应冠以负号.}
	\end{question}
	
	\begin{question}
		{判断题}
		{行列式}
		{容易}
		{5. $a_{51} a_{32} a_{13} a_{44} a_{65} a_{26}$ 在六阶行列式中是带正号的项()。}
		{$N(531462)=8$ 为偶数,所以 $a_{51} a_{32} a_{13} a_{44} a_{65} a_{26}$ 前面应冠以正号.}
	\end{question}
	
	\begin{question}
		{判断题}
		{行列式}
		{容易}
		{6. $n$ 阶行列式中有一行元素为零,行列式为零()。}
		{因为 $n$ 阶行列式中有一行元素为零,则所有项均为零,因此行列式为零。}
	\end{question}
	
	\begin{question}
		{判断题}
		{行列式}
		{容易}
		{7. 排列 36715284 是偶排列()。}
		{36715284 的逆序数为 $N(36715284)=3+4+0+4+2+0+0+0=13$ ,为奇排列.}
	\end{question}
	
	\begin{question}
		{判断题}
		{行列式}
		{容易}
		{8. $a_{21} a_{53} a_{16} a_{42} a_{64} a_{35}$ 在六阶行列式中是带正号的项()。}
		{$N(251436)+N(136245)=5+4=9$ 为奇数,所以 $a_{21} a_{53} a_{16} a_{42} a_{64} a_{35}$ 前面应冠以负号.}
	\end{question}
	
	\begin{question}
		{判断题}
		{行列式}
		{容易}
		{9. 排列 31245678 是奇排列()。}
		{31245678 的逆序数为 2 ,为偶排列.}
	\end{question}
	
	\begin{question}
		{判断题}
		{行列式}
		{容易}
		{10. 排列 31765284 是奇排列()。}
		{$N(31765284)=12$ ,为偶排列.}
	\end{question}
	
	\begin{question}
		{判断题}
		{行列式}
		{容易}
		{11. 排列 3761524 是奇排列()。}
		{$N(3761524)=13$ ,为奇排列.}
	\end{question}
	
	\begin{question}
		{判断题}
		{行列式}
		{容易}
		{12. 排列 1234 是奇排列()。}
		{1234 没有逆序,故逆序数为 0 ,为偶排列.}
	\end{question}
	
	\begin{question}
		{判断题}
		{行列式}
		{容易}
		{13. $a_{61} a_{52} a_{43} a_{34} a_{15} a_{26}$ 在六阶行列式中是带正号的项()。}
		{$N(654312)=14$ 为偶数,所以 $a_{61} a_{52} a_{43} a_{34} a_{15} a_{26}$ 前面应冠以正号.}
	\end{question}
	
	\begin{question}
		{填空题}
		{行列式}
		{容易}
		{1.排列132487695的逆序数为(  ).}
		{此排列含 8 个逆序.}
	\end{question}
	
	\begin{question}
		{填空题}
		{行列式}
		{容易}
		{2.排列132487659的逆序数为(  ).}
		{此排列含 7 个逆序.}
	\end{question}
	
	\begin{question}
		{填空题}
		{行列式}
		{容易}
		{3.排列7613542的逆序数为(  ).}
		{此排列含 15 个逆序.}
	\end{question}
	
	\begin{question}
		{填空题}
		{行列式}
		{容易}
		{4.排列1324765的逆序数为(  ).}
		{此排列含 4 个逆序.}
	\end{question}
	
	\begin{question}
		{填空题}
		{行列式}
		{容易}
		{5.排列41325的奇偶性为(  ).}
		{此排列含 4 个逆序,所以是偶排列.}
	\end{question}
	
	\begin{question}
		{填空题}
		{行列式}
		{容易}
		{6.排列76813542的逆序数为(  ).}
		{此排列含 20 个逆序.}
	\end{question}
	
	\begin{question}
		{填空题}
		{行列式}
		{容易}
		{7.排列13248765的逆序数为(  ).}
		{此排列含 7 个逆序.}
	\end{question}
	
	\begin{question}
		{填空题}
		{行列式}
		{容易}
		{8.排列453126的奇偶性为(  ).}
		{此排列含 8 个逆序,所以是偶排列.}
	\end{question}
	
	\begin{question}
		{填空题}
		{行列式}
		{容易}
		{9.排列613542的逆序数为(  ).}
		{此排列含 9 个逆序.}
	\end{question}
	
	\begin{question}
		{填空题}
		{行列式}
		{容易}
		{10.排列132465的逆序数为(  ).}
		{此排列含 2 个逆序.}
	\end{question}
	
	
	\begin{question}
		{填空题}
		{行列式}
		{容易}
		{11.排列7613542的逆序数为(  ).}
		{此排列含 15 个逆序.}
	\end{question}
	
	\begin{question}
		{填空题}
		{行列式}
		{中等}
		{12.计算行列式 $D=\begin{vmatrix}a & a & a & 1 \\ a & a & 1 & 0 \\ a & 1 & 0 & 0 \\ 1 & 0 & 0 & 0\end{vmatrix}=$(  ).}
		{$D$ 只含一项非零项 $1 \cdot 1 \cdot 1 \cdot 1$,符号由 $N(4321)=6$ 确定,带正号,所以 $D=1$.}
	\end{question}
	
	\begin{question}
		{填空题}
		{行列式}
		{中等}
		{13.求行列式 $D=\begin{vmatrix}a & a & a & 2 \\ a & a & -2 & 0 \\ a & 2 & 0 & 0 \\ 2 & 0 & 0 & 0\end{vmatrix}=$(  ).}
		{$D$ 只含一项非零项 $2 \cdot (-2) \cdot 2 \cdot a$,符号由 $N(4321)=6$ 确定,带正号,所以 $D=-16$.}
	\end{question}
	
	\begin{question}
		{填空题}
		{行列式}
		{中等}
		{14.计算行列式 $D=\begin{vmatrix}0 & a_{12} & 0 & 0 \\ 0 & 0 & 0 & a_{24} \\ a_{31} & 0 & 0 & 0 \\ 0 & 0 & a_{43} & 0\end{vmatrix}=$(  ).}
		{只含一项非零项 $a_{12} a_{24} a_{31} a_{43}$,符号由 $N(2413)=3$,带负号,所以 $D=-a_{12} a_{24} a_{31} a_{43}$.}
	\end{question}
	
	\begin{question}
		{填空题}
		{行列式}
		{中等}
		{15.计算行列式 $D=\begin{vmatrix}0 & 1 & 0 & 0 \\ 0 & 0 & 0 & 2 \\ 3 & 0 & 0 & 0 \\ 0 & 0 & 4 & 0\end{vmatrix}=$(  ).}
		{只含一项非零项 $1 \cdot 2 \cdot 3 \cdot 4=24$,符号由 $N(2413)=3$,带负号,所以 $D=-24$.}
	\end{question}
	
	\begin{question}
		{填空题}
		{行列式}
		{中等}
		{16.计算行列式 $D=\begin{vmatrix}0 & a_{12} & 0 & 0 \\ 0 & 0 & 0 & -a_{24} \\ a_{31} & 0 & 0 & 0 \\ 0 & 0 & a_{43} & 0\end{vmatrix}=$(  ).}
		{只含一项非零项 $a_{12} \cdot (-a_{24}) \cdot a_{31} \cdot a_{43}$,符号由 $N(2413)=3$,带负号,所以 $D=a_{12} a_{24} a_{31} a_{43}$.}
	\end{question}
	
	\begin{question}
		{填空题}
		{行列式}
		{中等}
		{17.在六阶行列式中,$a_{15} a_{23} a_{32} a_{44} a_{51} a_{66}$ 应取的符号为(  ).}
		{由排列 532416 的逆序数为 8,故符号为 $(-1)^8 = 1$.}
	\end{question}
	
	\begin{question}
		{填空题}
		{行列式}
		{中等}
		{18.四阶行列式中含有因子 $a_{11} a_{23}$ 的项是(  )和(  ).}
		{含有 $a_{11} a_{23}$ 的项为 $(-1)^{r(1324)} a_{11} a_{23} a_{32} a_{44}$ 和 $(-1)^{r(1342)} a_{11} a_{23} a_{34} a_{42}$.}
	\end{question}
	
	
	\begin{question}
		{计算题}
		{行列式}
		{容易}
		{1.用定义计算行列式 $\begin{vmatrix}0 & \cdots & 0 & 1 \\ 0 & \cdots & 2 & 0 \\ \cdots & \cdots & \cdots & \cdots \\ n & \cdots & 0 & 0\end{vmatrix}$ 的值(  ).}
		{设 $\begin{vmatrix}0 & \cdots & 0 & 1 \\ 0 & \cdots & 2 & 0 \\ \cdots & \cdots & \cdots & \cdots \\ n & \cdots & 0 & 0\end{vmatrix}=a_{i j}$ ,根据行列式的定义,$a_{i j}$
			的展开式中,除 $a_{1 n} a_{2, n-1} \cdots a_{n 1}$ 连乘积这一项外,其他各项中至少含有一个零元素,故皆为零,因此:$a_{i j}=(-1)^{N(n \cdot 21)} a_{1 n} a_{2, n-1} \cdots a_{n 1}$ $=(-1)^{\frac{n(n-1)}{2}} 1 \cdot 2 \cdots n=(-1)^{\frac{n(n-1)}{2}} n!$ .}
	\end{question}
	
	\begin{question}
		{选择题}
		{行列式}
		{容易}
		{1.利用行列式的性质,计算行列式 $\begin{vmatrix}1 & 2 & 1 \\ 0 & 2 & 0 \\ 0 & 0 & 3\end{vmatrix}=$().A. 6 B.-6 C.0 D.8}
		{上三角形行列式的值等于主对角元素的乘积,因此 $\begin{vmatrix}1 & 2 & 1 \\ 0 & 2 & 0 \\ 0 & 0 & 3\end{vmatrix}=1 \times 2 \times 3=6$.}
	\end{question}	
	
	\begin{question}
		{选择题}
		{行列式}
		{容易}
		{2.利用行列式的性质,计算行列式 $\begin{vmatrix}-1 & -1 & -1 \\ 8 & -5 & -2 \\ 4 & 4 & 4\end{vmatrix}=$().A. 0 B. 1 C. -1 D. 32}
		{行列式有两行对应成比例,行列式的值为 0 ,因此 $\begin{vmatrix}-1 & -1 & -1 \\ 8 & -5 & -2 \\ 4 & 4 & 4\end{vmatrix}=0$.}
	\end{question}
	
	\begin{question}
		{选择题}
		{行列式}
		{容易}
		{3.利用行列式的性质,计算行列式 $\begin{vmatrix}1 & -2 & 2 \\ 1 & -10 & 2 \\ 1 & -8 & 2\end{vmatrix}=$().A. 0 B. 2 C. -32 D. -2}
		{行列式有两列对应成比例,行列式的值为 0 ,因此 $\begin{vmatrix}1 & -2 & 2 \\ 1 & -10 & 2 \\ 1 & -8 & 2\end{vmatrix}=0$.}
	\end{question}
	
	\begin{question}
		{选择题}
		{行列式}
		{容易}
		{4.利用行列式的性质,计算行列式 $\begin{vmatrix}a & 0 & 0 \\ 0 & b & 0 \\ 0 & 0 & c\end{vmatrix}=$().A. $abc$ B. $-abc$ C. $a^3$ D. $b^2$}
		{对角形行列式的值等于主对角元素的乘积,因此 $\begin{vmatrix}a & 0 & 0 \\ 0 & b & 0 \\ 0 & 0 & c\end{vmatrix}=a \times b \times c$.}
	\end{question}
	
	\begin{question}
		{选择题}
		{行列式}
		{容易}
		{5.利用行列式的性质,计算行列式 $\begin{vmatrix}7 & 8 & 9 \\ -2 & -2 & -2 \\ 1 & 1 & 1\end{vmatrix}=$().A. 0 B. 7 C. 8 D. 9}
		{行列式有两行对应成比例,行列式的值为 0 ,因此 $\begin{vmatrix}7 & 8 & 9 \\ -2 & -2 & -2 \\ 1 & 1 & 1\end{vmatrix}=0$.}
	\end{question}
	
	\begin{question}
		{选择题}
		{行列式}
		{容易}
		{6.利用行列式的性质,计算行列式 $\begin{vmatrix}8 & 5 & 0 \\ 6 & 9 & 0 \\ 0 & 7 & 0\end{vmatrix}=$().A. 0 B. 72 C. -1 D. 1}
		{行列式有一列元素全为零,行列式的值等于零,因此 $\begin{vmatrix}8 & 5 & 0 \\ 6 & 9 & 0 \\ 0 & 7 & 0\end{vmatrix}=0$.}
	\end{question}
	
	\begin{question}
		{选择题}
		{行列式}
		{容易}
		{7.利用行列式的性质,计算行列式 $\begin{vmatrix}1 & 2 & 1 \\ 0 & 3 & 0 \\ 0 & 0 & 3\end{vmatrix}=$().A. 9 B. -6 C. 0 D. 8}
		{上三角形行列式的值等于主对角元素的乘积,因此 $\begin{vmatrix}1 & 2 & 1 \\ 0 & 3 & 0 \\ 0 & 0 & 3\end{vmatrix}=1 \times 3 \times 3=9$.}
	\end{question}
	
	\begin{question}
		{选择题}
		{行列式}
		{容易}
		{8.利用行列式的性质,计算行列式 $\begin{vmatrix}9 & 4 & 7 \\ 4 & 4 & 4 \\ 3 & 3 & 3\end{vmatrix}=$().A. 0 B. 9 C. 4 D. 7}
		{行列式有两行对应成比例,行列式的值为 0 ,因此 $\begin{vmatrix}9 & 4 & 7 \\ 4 & 4 & 4 \\ 3 & 3 & 3\end{vmatrix}=0$.}
	\end{question}
	
	\begin{question}
		{选择题}
		{行列式}
		{容易}
		{9.利用行列式的性质,计算行列式 $\begin{vmatrix}3 & a & 7 \\ 3 & b & 7 \\ 3 & c & 7\end{vmatrix}=$().A. 0 B. $abc$ C. 21abc D. $-21abc$}
		{行列式有两列对应成比例,行列式的值为 0 ,因此 $\begin{vmatrix}3 & a & 7 \\ 3 & b & 7 \\ 3 & c & 7\end{vmatrix}=0$.}
	\end{question}
	
	\begin{question}
		{选择题}
		{行列式}
		{容易}
		{10.利用行列式的性质,计算行列式 $\begin{vmatrix}-4 & 2 & 1 \\ 0 & 4 & -2 \\ 0 & 0 & 1\end{vmatrix}=$().A. -16 B. 16 C. 64 D. 32}
		{上三角形行列式的值等于主对角元素的乘积,因此 $\begin{vmatrix}-4 & 2 & 1 \\ 0 & 4 & -2 \\ 0 & 0 & 1\end{vmatrix}=-4 \times 4 \times 1=-16$.}
	\end{question}
	
	\begin{question}
		{选择题}
		{行列式}
		{中等}
		{11.利用行列式的性质,计算行列式 $\begin{vmatrix}1 & 4 & 5 \\ 2 & 6 & 7 \\ 3 & 9 & 8\end{vmatrix}=$().A. 5 B. -5 C. 10 D. -10}
		{通过初等行变换将行列式化为上三角形式:$\begin{vmatrix}1 & 4 & 5 \\ 2 & 6 & 7 \\ 3 & 9 & 8\end{vmatrix}=\begin{vmatrix}1 & 4 & 5 \\ 0 & -2 & -3 \\ 0 & -3 & -7\end{vmatrix}=\begin{vmatrix}1 & 4 & 5 \\ 0 & -2 & -3 \\ 0 & 0 & -\frac{5}{2}\end{vmatrix}=5$.}
	\end{question}
	
	\begin{question}
		{选择题}
		{行列式}
		{中等}
		{12.如果行列式的所有元素变号,则()。A. 奇阶行列式变号 B. 行列式一定不变号 C. 偶阶行列式变号 D. 行列式一定变号}
		{每一行都提出一个负号,一共提$n$个,得$(-1)^n$,所以$n$为奇数时变号,为偶数时不变。}
	\end{question}
	
	\begin{question}
		{选择题}
		{行列式}
		{中等}
		{13.设 $\begin{vmatrix}a_1 & a_2 & a_3 \\ b_1 & b_2 & b_3 \\ c_1 & c_2 & c_3\end{vmatrix}=k$ ,则 $\begin{vmatrix}a_1 & a_2-2a_3 & -a_3 \\ b_1 & b_2-2b_3 & -b_3 \\ c_1 & c_2-2c_3 & -c_3\end{vmatrix}=$().A. $-k$ B. $-2k$ C. $k$ D. $2k$}
		{原行列式为$k$,第二列变为$a_2 - 2a_3$,第三列为$-a_3$,根据行列式的线性性质可得:该行列式为$\begin{vmatrix}a_1 & a_2 & -a_3 \\ b_1 & b_2 & -b_3 \\ c_1 & c_2 & -c_3\end{vmatrix} + \begin{vmatrix}a_1 & -2a_3 & -a_3 \\ b_1 & -2b_3 & -b_3 \\ c_1 & -2c_3 & -c_3\end{vmatrix} = -k$。}
	\end{question}
	
	\begin{question}
		{判断题}
		{行列式}
		{容易}
		{1.行列式有两行元素全相等,行列式的值一定为零(  ).}
		{行列式的性质:行列式有两行元素全相等行列式的值等于零.所以结论正确.}
	\end{question}
	
	
	\begin{question}
		{判断题}
		{行列式}
		{容易}
		{2.行列式有一行元素全为零,行列式的值一定为零(  ).}
		{行列式有一行元素全为零,行列式的值一定为零.所以结论正确.}
	\end{question}
	
	\begin{question}
		{判断题}
		{行列式}
		{容易}
		{3.行列式有两列元素对应成比例,行列式的值一定为零(  ).}
		{行列式的性质:行列式有两列元素对应成比例,行列式的值为 0.所以结论正确.}
	\end{question}
	
	\begin{question}
		{判断题}
		{行列式}
		{容易}
		{4.行列式有两行元素的总和成比例,行列式的值一定为零(  ).}
		{由行列式的性质,行列式有两行元素对应成比例,行列式的值为 0,但行列式两行元素的总和成比例,不一定行列式的值为零.所以结论错误.}
	\end{question}
	
	\begin{question}
		{判断题}
		{行列式}
		{容易}
		{5.行列式有一列元素全为零,行列式的值不一定为零(  ).}
		{行列式有一列元素全为零,行列式的值一定为零.所以结论错误.}
	\end{question}
	
	\begin{question}
		{判断题}
		{行列式}
		{容易}
		{6.$\begin{vmatrix}2 a_1 & 2 b_1 & 2 c_1 \\ 2 a_2 & 2 b_2 & 2 c_2 \\ 2 a_3 & 2 b_3 & 2 c_3\end{vmatrix}=2\begin{vmatrix}a_1 & b_1 & c_1 \\ a_2 & b_2 & c_2 \\ a_3 & b_3 & c_3\end{vmatrix}$ (  ).}
		{因为 $\begin{vmatrix}2 a_1 & 2 b_1 & 2 c_1 \\ 2 a_2 & 2 b_2 & 2 c_2 \\ 2 a_3 & 2 b_3 & 2 c_3\end{vmatrix}=2^3\begin{vmatrix}a_1 & b_1 & c_1 \\ a_2 & b_2 & c_2 \\ a_3 & b_3 & c_3\end{vmatrix}$,所以结论错误.}
	\end{question}
	
	\begin{question}
		{判断题}
		{行列式}
		{容易}
		{7.$n$ 阶行列式非零元素的个数少于 $n$ 个,行列式的值一定为零(  ).}
		{行列式非零元素的个数少于 $n$ 个,行列式一定有一行或一列全为零,行列式的值一定为零.所以结论正确.}
	\end{question}
	
	\begin{question}
		{判断题}
		{行列式}
		{容易}
		{8.$n$ 阶行列式主对角元素全为零,行列式的值一定为零(  ).}
		{行列式主对角元素全为零,行列式的值不一定为零.所以结论错误.}
	\end{question}
	
	\begin{question}
		{判断题}
		{行列式}
		{容易}
		{9.行列式各行元素之和为零,行列式的值一定为零(  ).}
		{行列式各行元素之和为零,利用性质,把各列加于第一列,则第一列全为零,因此行列式为零.所以结论正确.}
	\end{question}
	
	\begin{question}
		{判断题}
		{行列式}
		{容易}
		{10.行列式各列元素之和为零,行列式的值不一定为零(  ).}
		{行列式各列元素之和为零,利用性质,把各行加于第一行,则第一行全为零,因此行列式为零.所以结论错误.}
	\end{question}
	
	
	\begin{question}
		{判断题}
		{行列式}
		{容易}
		{11.设 $\begin{vmatrix}a_{11} & a_{12} & a_{13} \\ a_{21} & a_{22} & a_{23} \\ a_{31} & a_{32} & a_{33}\end{vmatrix}=1$ ,则 $\begin{vmatrix}a_{11} & -4 a_{12} & a_{13} \\ 3 a_{21} & -12 a_{22} & 3 a_{23} \\ a_{31} & -4 a_{32} & a_{33}\end{vmatrix}=12(  ).$}
		{$\begin{vmatrix}a_{11} & -4 a_{12} & a_{13} \\ 3 a_{21} & -12 a_{22} & 3 a_{23} \\ a_{31} & -4 a_{32} & a_{33}\end{vmatrix}=-4\begin{vmatrix}a_{11} & a_{12} & a_{13} \\ 3 a_{21} & 3 a_{22} & 3 a_{23} \\ a_{31} & a_{32} & a_{33}\end{vmatrix}=-12\begin{vmatrix}a_{11} & a_{12} & a_{13} \\ a_{21} & a_{22} & a_{23} \\ a_{31} & a_{32} & a_{33}\end{vmatrix}=-12$,所以结论错误.}
	\end{question}
	
	\begin{question}
		{判断题}
		{行列式}
		{容易}
		{12.设 $\begin{vmatrix}a_{11} & a_{12} & a_{13} \\ a_{21} & a_{22} & a_{23} \\ a_{31} & a_{32} & a_{33}\end{vmatrix}=1$ ,则
			$$
			\begin{vmatrix}
				a_{11} & -2 a_{12} & a_{13} \\
				3 a_{21} & -6 a_{22} & 3 a_{23} \\
				a_{31} & -2 a_{32} & a_{33}
			\end{vmatrix}=6(  ).
			$$
		}
		{$\begin{vmatrix}a_{11} & -2 a_{12} & a_{13} \\ 3 a_{21} & -6 a_{22} & 3 a_{23} \\ a_{31} & -2 a_{32} & a_{33}\end{vmatrix}=-2\begin{vmatrix}a_{11} & a_{12} & a_{13} \\ 3 a_{21} & 3 a_{22} & 3 a_{23} \\ a_{31} & a_{32} & a_{33}\end{vmatrix}=$ $-6\begin{vmatrix}a_{11} & a_{12} & a_{13} \\ a_{21} & a_{22} & a_{23} \\ a_{31} & a_{32} & a_{33}\end{vmatrix}=-6$,所以结论错误.}
	\end{question}
	
	\begin{question}
		{填空题}
		{行列式}
		{容易}
		{1.利用行列式的性质,计算行列式 $\begin{vmatrix}3 & 2 & 1 & 7 \\ 0 & 4 & 6 & -9 \\ 0 & 0 & 1 & 8 \\ 0 & 0 & 0 & -4\end{vmatrix}=$(  ).}
		{上三角形行列式的值等于主对角元素的乘积,因此$$
			\begin{vmatrix}
				3 & 2 & 1 & 7 \\
				0 & 4 & 6 & -9 \\
				0 & 0 & 1 & 8 \\
				0 & 0 & 0 & -4
			\end{vmatrix}=3 \times 4 \times 1 \times(-4)=-48
			$$}
	\end{question}	
	
	\begin{question}
		{填空题}
		{行列式}
		{容易}
		{2. 利用行列式的性质,计算行列式 $\begin{vmatrix}1 & 2 & 3 \\ 0 & 1 & 2 \\ 1 & 1 & 1\end{vmatrix}=$( )。}
		{第三行乘以1加到第二行,得到两行相等,因此
			$$
			\begin{vmatrix}
				1 & 2 & 3 \\
				0 & 1 & 2 \\
				1 & 1 & 1
			\end{vmatrix}
			=\begin{vmatrix}
				1 & 2 & 3 \\
				1 & 2 & 3 \\
				1 & 1 & 1
			\end{vmatrix}=0
			$$}
	\end{question}
	
	\begin{question}
		{填空题}
		{行列式}
		{容易}
		{3. 利用行列式的性质,计算行列式 $\begin{vmatrix}8 & 1 & 5 \\ 6 & 3 & -11 \\ 0 & 0 & 0\end{vmatrix}=$( )。}
		{行列式有一行元素全为零,行列式的值为零,因此
			$$
			\begin{vmatrix}
				8 & 1 & 5 \\
				6 & 3 & -11 \\
				0 & 0 & 0
			\end{vmatrix}=0
			$$}
	\end{question}
	
	\begin{question}
		{填空题}
		{行列式}
		{容易}
		{4. 利用行列式的性质,计算行列式 $\begin{vmatrix}1 & 2 & 3 \\ 3 & -1 & 7 \\ 3 & -1 & 7\end{vmatrix}=$( )。}
		{行列式有两行相等,值为零,因此
			$$
			\begin{vmatrix}
				1 & 2 & 3 \\
				3 & -1 & 7 \\
				3 & -1 & 7
			\end{vmatrix}=0
			$$}
	\end{question}
	
	\begin{question}
		{填空题}
		{行列式}
		{容易}
		{5. 利用行列式的性质,计算行列式 $\begin{vmatrix}2 & 0 & 5 \\ 6 & 0 & -1 \\ 3 & 0 & 4\end{vmatrix}=$( )。}
		{行列式有一列全为零,值为零,因此
			$$
			\begin{vmatrix}
				2 & 0 & 5 \\
				6 & 0 & -1 \\
				3 & 0 & 4
			\end{vmatrix}=0
			$$}
	\end{question}
	
	\begin{question}
		{填空题}
		{行列式}
		{容易}
		{6. 利用行列式的性质,计算行列式 $\begin{vmatrix}-3 & 2 & 1 & 7 \\ 0 & 2 & 6 & -9 \\ 0 & 0 & 1 & 8 \\ 0 & 0 & 0 & -2\end{vmatrix}=$( )。}
		{上三角行列式的值等于主对角线元素的乘积,因此
			$$
			\begin{vmatrix}
				-3 & 2 & 1 & 7 \\
				0 & 2 & 6 & -9 \\
				0 & 0 & 1 & 8 \\
				0 & 0 & 0 & -2
			\end{vmatrix}=(-3)\cdot2\cdot1\cdot(-2)=12
			$$}
	\end{question}
	
	\begin{question}
		{填空题}
		{行列式}
		{中等}
		{7. 计算行列式 $\begin{vmatrix}2 & 1 & 1 & 1 \\ 1 & 2 & 1 & 1 \\ 1 & 1 & 2 & 1 \\ 1 & 1 & 1 & 2\end{vmatrix}=$( )。}
		{将第2、3、4列分别乘以1加到第1列,得
			$$
			\begin{vmatrix}
				5 & 1 & 1 & 1 \\
				5 & 2 & 1 & 1 \\
				5 & 1 & 2 & 1 \\
				5 & 1 & 1 & 2
			\end{vmatrix}=5\begin{vmatrix}
				1 & 1 & 1 & 1 \\
				1 & 2 & 1 & 1 \\
				1 & 1 & 2 & 1 \\
				1 & 1 & 1 & 2
			\end{vmatrix}=5\cdot1\cdot1\cdot1\cdot1=5
			$$}
	\end{question}
	
	\begin{question}
		{填空题}
		{行列式}
		{中等}
		{8. 计算行列式 $\begin{vmatrix}3 & 1 & 1 & 1 \\ 1 & 3 & 1 & 1 \\ 1 & 1 & 3 & 1 \\ 1 & 1 & 1 & 3\end{vmatrix}=$( )。}
		{将第2、3、4列分别乘以1加到第1列,得
			$$
			\begin{vmatrix}
				6 & 1 & 1 & 1 \\
				6 & 3 & 1 & 1 \\
				6 & 1 & 3 & 1 \\
				6 & 1 & 1 & 3
			\end{vmatrix}=6\begin{vmatrix}
				1 & 1 & 1 & 1 \\
				1 & 3 & 1 & 1 \\
				1 & 1 & 3 & 1 \\
				1 & 1 & 1 & 3
			\end{vmatrix}=6\cdot1\cdot2\cdot2\cdot2=48
			$$}
	\end{question}
	
	
	\begin{question}
		{计算题}
		{行列式}
		{容易}
		{1.利用行列式的性质,计算行列式 $\begin{vmatrix}1 & 1 & 1 & 1 \\ -1 & 1 & 1 & 1 \\ -1 & -1 & 1 & 1 \\ -1 & -1 & -1 & 1\end{vmatrix}$ 的值(  ).}
		{用性质化为上三角形行列式 $\begin{vmatrix}1 & 1 & 1 & 1 \\ -1 & 1 & 1 & 1 \\ -1 & -1 & 1 & 1 \\ -1 & -1 & -1 & 1\end{vmatrix}=$ $\begin{vmatrix}1 & 1 & 1 & 1 \\ 0 & 2 & 2 & 2 \\ 0 & 0 & 2 & 2 \\ 0 & 0 & 0 & 2\end{vmatrix}=8$.}
	\end{question}	
	
	
	\begin{question}
		{计算题}
		{行列式}
		{容易}
		{2.利用行列式的性质,计算行列式 $\begin{vmatrix}2 & -5 & 3 & 1 \\ 1 & 3 & -1 & 3 \\ 0 & 1 & 1 & -5 \\ -1 & -4 & 2 & -3\end{vmatrix}$ 的值(  ).}
		{用性质化为上三角形行列式:
			\[
			\begin{aligned}
				&\begin{vmatrix}
					2 & -5 & 3 & 1 \\
					1 & 3 & -1 & 3 \\
					0 & 1 & 1 & -5 \\
					-1 & -4 & 2 & -3
				\end{vmatrix}
				= -\begin{vmatrix}
					1 & 3 & -1 & 3 \\
					2 & -5 & 3 & 1 \\
					0 & 1 & 1 & -5 \\
					-1 & -4 & 2 & -3
				\end{vmatrix} \\
				&= -\begin{vmatrix}
					1 & 3 & -1 & 3 \\
					0 & -11 & 5 & -5 \\
					0 & 1 & 1 & -5 \\
					0 & -1 & 1 & 0
				\end{vmatrix}
				= \begin{vmatrix}
					1 & 3 & -1 & 3 \\
					0 & 1 & 1 & -5 \\
					0 & 0 & 16 & -60 \\
					0 & 0 & 0 & \frac{5}{2}
				\end{vmatrix} = 40.
			\end{aligned}
			\]}
	\end{question}
	
	\begin{question}
		{计算题}
		{行列式}
		{容易}
		{3.计算行列式 $\begin{vmatrix}3 & 1 & -1 & 2 \\ -5 & 1 & 3 & -4 \\ 2 & 0 & 1 & -1 \\ 1 & -5 & 3 & -3\end{vmatrix}$ 的值(  ).}
		{
			\[
			D \stackrel{\substack{c_1-2c_3 \\ c_4+c_3}}{=}
			\begin{vmatrix}
				5 & 1 & -1 & 1 \\
				-11 & 1 & 3 & -1 \\
				0 & 0 & 1 & 0 \\
				-5 & -5 & 3 & 0
			\end{vmatrix}
			\]
			对第3行展开,得:
			\[
			(-1)^{3+3}\begin{vmatrix}
				5 & 1 & 1 \\
				-11 & 1 & -1 \\
				-5 & -5 & 0
			\end{vmatrix} \stackrel{r_2+r_1}{=}
			\begin{vmatrix}
				5 & 1 & 1 \\
				-6 & 2 & 0 \\
				-5 & -5 & 0
			\end{vmatrix}
			\]
			对第1行第3列展开,得:
			\[
			(-1)^{1+3} \begin{vmatrix}
				-6 & 2 \\
				-5 & -5
			\end{vmatrix} = 40.
			\]}
	\end{question}
	
	
	\begin{question}
		{计算题}
		{行列式}
		{容易}
		{4.计算行列式 $\begin{vmatrix}3 & 1 & 1 & 1 \\ 1 & 3 & 1 & 1 \\ 1 & 1 & 3 & 1 \\ 1 & 1 & 1 & 3\end{vmatrix}$ 的值(  ).}
		{
			\[
			D = \begin{vmatrix}
				6 & 1 & 1 & 1 \\
				6 & 3 & 1 & 1 \\
				6 & 1 & 3 & 1 \\
				6 & 1 & 1 & 3
			\end{vmatrix} = 6
			\begin{vmatrix}
				1 & 1 & 1 & 1 \\
				1 & 3 & 1 & 1 \\
				1 & 1 & 3 & 1 \\
				1 & 1 & 1 & 3
			\end{vmatrix}
			\]
			\[
			= 6
			\begin{vmatrix}
				1 & 1 & 1 & 1 \\
				0 & 2 & 0 & 0 \\
				0 & 0 & 2 & 0 \\
				0 & 0 & 0 & 2
			\end{vmatrix} = 6 \cdot (2 \cdot 2 \cdot 2) = 48.
			\]}
	\end{question}
	
	\begin{question}
		{计算题}
		{行列式}
		{容易}
		{5.计算行列式 $\begin{vmatrix}1 & 1 & 1 & 1 \\ 1 & 2 & 3 & 4 \\ 1 & 3 & 6 & 10 \\ 1 & 4 & 10 & 20\end{vmatrix}$ 的值(  ).}
		{
			\[
			D = \begin{vmatrix}
				1 & 1 & 1 & 1 \\
				1 & 2 & 3 & 4 \\
				1 & 3 & 6 & 10 \\
				1 & 4 & 10 & 20
			\end{vmatrix} =
			\begin{vmatrix}
				1 & 1 & 1 & 1 \\
				0 & 1 & 2 & 3 \\
				0 & 1 & 3 & 6 \\
				0 & 1 & 4 & 10
			\end{vmatrix}
			\]
			\[
			=
			\begin{vmatrix}
				1 & 1 & 1 & 1 \\
				0 & 1 & 2 & 3 \\
				0 & 0 & 1 & 3 \\
				0 & 0 & 1 & 4
			\end{vmatrix} =
			\begin{vmatrix}
				1 & 1 & 1 & 1 \\
				0 & 1 & 2 & 3 \\
				0 & 0 & 1 & 3 \\
				0 & 0 & 0 & 1
			\end{vmatrix} = 1.
			\]}
	\end{question}
	
	\begin{question}
		{选择题}
		{行列式}
		{容易}
		{1.已知 $n$ 阶行列式 $D=a_{i j}$ ,设 $A_{i j}$ 是 $n$ 阶行列式中 $a_{i j}$ 的代数余子式,若 $i \neq s$ ,则 $a_{i 1} A_{s 1}+a_{i 2} A_{s 2}+\ldots+a_{i n} A_{s n}=(  )$.A. 0 B.1 C.$D$ D.$-D$}
		{行列式的任意一行与另一行的代数余子式的乘积之和等于零.}
	\end{question}
	
	
	\begin{question}
		{选择题}
		{行列式}
		{容易}
		{2.已知 $n$ 阶行列式 $D=a_{i j}$ ,设 $A_{i j}$ 是 $n$ 阶行列式中 $a_{i j}$ 的代数余子式,若 $i=s$ ,则 $a_{i 1} A_{s 1}+a_{i 2} A_{s 2}+\ldots+a_{i n} A_{s n}=(  )$.A.$D$ B.1 C.0 D.$-D$}
		{行列式的任意一行与其代数余子式的乘积之和等于 $D$.}
	\end{question}
	
	\begin{question}
		{选择题}
		{行列式}
		{容易}
		{3.行列式 $\begin{vmatrix}-3 & 0 & 4 \\ 5 & 0 & 3 \\ x & y & 1\end{vmatrix}$ 中 $x$ 的代数余子式=(  ).A.0 B.12 C.3 D.-4}
		{该代数余子式为 $(-1)^{3+1}\begin{vmatrix}0 & 4 \\ 0 & 3\end{vmatrix}=0$.}
	\end{question}
	
	\begin{question}
		{选择题}
		{行列式}
		{容易}
		{4.行列式 $\begin{vmatrix}-3 & 0 & 4 \\ 5 & 0 & 3 \\ x & y & 1\end{vmatrix}$ 中 $y$ 的代数余子式=(  ).A.29 B.-29 C.11 D.-11}
		{$y$ 的代数余子式为 $(-1)^{3+2}\begin{vmatrix}-3 & 4 \\ 5 & 3\end{vmatrix}=29$.}
	\end{question}
	
	\begin{question}
		{选择题}
		{行列式}
		{容易}
		{5.已知四阶行列式 $D$ 中第3列元素依次为 $-1,2,1,1$ ,相应代数余子式为 $2,5,3,0$ ,则 $D=(  )$.A.11 B.-13 C.13 D.-11}
		{$D = (-1) \times 2 + 2 \times 5 + 1 \times 3 + 1 \times 0 = 11$.}
	\end{question}
	
	\begin{question}
		{选择题}
		{行列式}
		{容易}
		{6.已知三阶行列式 $D$ 中第1列元素依次为 $-1,2,1$ ,相应代数余子式为 $2,5,3$ ,则 $D=(  )$.A.11 B.-13 C.13 D.-11}
		{$D = (-1) \times 2 + 2 \times 5 + 1 \times 3 = 11$.}
	\end{question}
	
	\begin{question}
		{选择题}
		{行列式}
		{容易}
		{7.已知三阶行列式 $D$ 中第1行元素依次为 $1,-2,3$ ,相应代数余子式为 $4,7,2$ ,则 $D=(  )$.A.-4 B.4 C.6 D.-5}
		{$D = 1 \times 4 + (-2) \times 7 + 3 \times 2 = -4$.}
	\end{question}
	
	\begin{question}
		{选择题}
		{行列式}
		{容易}
		{8.已知三阶行列式 $D$ 中第3行元素依次为 $-1,-2,3$ ,相应代数余子式为 $4,7,2$ ,则 $D=(  )$.A.-12 B.12 C.6 D.-5}
		{$D = (-1) \times 4 + (-2) \times 7 + 3 \times 2 = -12$.}
	\end{question}
	
	\begin{question}
		{选择题}
		{行列式}
		{容易}
		{9.已知四阶行列式 $D$ 中第4列元素依次为 $-1,0,1,1$ ,相应代数余子式为 $2,7,3,0$ ,则 $D=(  )$.A.1 B.-1 C.4 D.-4}
		{$D = (-1) \times 2 + 0 \times 7 + 1 \times 3 + 1 \times 0 = 1$.}
	\end{question}
	
	\begin{question}
		{选择题}
		{行列式}
		{容易}
		{10.已知四阶行列式 $D$ 中第4行元素依次为 $-4,0,-1,1$ ,相应代数余子式为 $2,7,1,0$ ,则 $D=(  )$.A.-9 B.-10 C.7 D.-7}
		{$D = (-4) \times 2 + 0 \times 7 + (-1) \times 1 + 1 \times 0 = -9$.}
	\end{question}
	
	\begin{question}
		{选择题}
		{行列式}
		{容易}
		{11.已知三阶行列式 $D$ 中第1行元素依次为 $1,-2,-3$ ,相应代数余子式为 $4,7,2$ ,则 $D=(  )$.A.-16 B.14 C.16 D.-15}
		{$D = 1 \times 4 + (-2) \times 7 + (-3) \times 2 = -16$.}
	\end{question}
	
	
	\begin{question}
		{选择题}
		{行列式}
		{容易}
		{12.已知三阶行列式 $D$ 中第 1 行元素依次为 $1,2,-3$ 且相应的余子式依次为 $4,7,2$,则 $D=(\quad)$。A. $-16$ \quad B. $-26$ \quad C. $-4$ \quad D. $26$}
		{$D=1 \times(-1)^{1+1} \cdot 4 + 2 \times(-1)^{1+2} \cdot 7 + (-3) \times(-1)^{1+3} \cdot 2 = -16$。}
	\end{question}
	
	\begin{question}
		{选择题}
		{行列式}
		{容易}
		{13.已知三阶行列式 $D$ 中第 2 行元素依次为 $1,2,-3$ 且相应的余子式依次为 $4,7,2$,则 $D=(\quad)$。A. $16$ \quad B. $26$ \quad C. $-4$ \quad D. $-16$}
		{$D=1 \cdot (-1)^{2+1} \cdot 4 + 2 \cdot (-1)^{2+2} \cdot 7 + (-3) \cdot (-1)^{2+3} \cdot 2 = 16$。}
	\end{question}
	
	\begin{question}
		{选择题}
		{行列式}
		{容易}
		{14.行列式 $\begin{vmatrix}7 & -1 & 4 \\ 5 & 2 & 3 \\ x & y & 9\end{vmatrix}$ 中 $x$ 的代数余子式为(\quad)。A. $-11$ \quad B. $11$ \quad C. $7$ \quad D. $-5$}
		{$x$ 的代数余子式为 $(-1)^{3+1} \begin{vmatrix}-1 & 4 \\ 2 & 3\end{vmatrix} = -11$。}
	\end{question}
	
	\begin{question}
		{选择题}
		{行列式}
		{容易}
		{15.行列式 $\begin{vmatrix}7 & -1 & 4 \\ 5 & 2 & 3 \\ x & y & 9\end{vmatrix}$ 中 $y$ 的代数余子式为(\quad)。A. $-1$ \quad B. $1$ \quad C. $3$ \quad D. $-5$}
		{$y$ 的代数余子式为 $(-1)^{3+2} \begin{vmatrix}7 & 4 \\ 5 & 3\end{vmatrix} = -1$。}
	\end{question}
	
	\begin{question}
		{选择题}
		{行列式}
		{容易}
		{16.已知四阶行列式 $D$ 中第 3 列元素依次为 $-1,3,2,1$,且相应的代数余子式依次为 $1,5,4,0$,则 $D=(\quad)$。A. $22$ \quad B. $-13$ \quad C. $13$ \quad D. $-22$}
		{$D = (-1)\cdot 1 + 3\cdot 5 + 2\cdot 4 + 1\cdot 0 = 22$。}
	\end{question}
	
	\begin{question}
		{选择题}
		{行列式}
		{中等}
		{17.若 $\begin{vmatrix}a_{11} & a_{12} \\ a_{21} & a_{22}\end{vmatrix} = 1$,则 $\begin{vmatrix}-a_{11} & 3a_{12} & 0 \\ -a_{21} & 3a_{22} & 0 \\ 0 & 6 & 1\end{vmatrix}=(\quad)$。A. $-3$ \quad B. $3$ \quad C. $1$ \quad D. $6$}
		{可分块计算,该值为 $(-1) \cdot 3 \cdot \begin{vmatrix}a_{11} & a_{12} \\ a_{21} & a_{22}\end{vmatrix} = -3$。}
	\end{question}
	
	\begin{question}
		{选择题}
		{行列式}
		{容易}
		{18.行列式 $\begin{vmatrix}3 & 4 & 9 \\ 5 & 7 & -1 \\ 2 & 1 & 4\end{vmatrix}$ 中元素 $a_{23}$ 的代数余子式 $A_{23}=(\quad)$。A. $5$ \quad B. $-3$ \quad C. $3$ \quad D. $-5$}
		{$A_{23} = (-1)^{2+3} \begin{vmatrix}3 & 4 \\ 2 & 1\end{vmatrix} = 5$。}
	\end{question}
	
	\begin{question}
		{选择题}
		{行列式}
		{中等}
		{19.若 $\begin{vmatrix}a_{11} & a_{12} & a_{13} \\ a_{21} & a_{22} & a_{23} \\ a_{31} & a_{32} & a_{33}\end{vmatrix} = d$,则
			$$
			\begin{vmatrix}
				2 a_{11} & 2 a_{12} & 2 a_{13} \\
				3 a_{31} & 3 a_{32} & 3 a_{33} \\
				-a_{21} & -a_{22} & -a_{23}
			\end{vmatrix} = (\quad)
			$$
			A. $6d$ \quad B. $3d$ \quad C. $2d$ \quad D. $-6d$}
		{变换后为 $-6 \cdot \begin{vmatrix}a_{11} & a_{12} & a_{13} \\ a_{31} & a_{32} & a_{33} \\ a_{21} & a_{22} & a_{23}\end{vmatrix} = 6d$。}
	\end{question}
	
	\begin{question}
		{选择题}
		{行列式}
		{中等}
		{20.行列式 $\begin{vmatrix}1 & 0 & a & 1 \\ 0 & -1 & b & -1 \\ -1 & -1 & c & -1 \\ -1 & 1 & d & 0\end{vmatrix}$ 按第 3 列展开,则 $a$ 的符号为(\quad),行列式的值为(\quad)。A. $(-1)^{1+3},\ a + b + d$ \quad B. $(-1)^{1+3},\ a - b - c$ \quad C. $(-1)^{1+3},\ a + b - c$ \quad D. $(-1)^3,\ a + b + c$}
		{按第 3 列展开,整体值为 $a + b + d$,$a$ 的符号为 $(-1)^{1+3}$。}
	\end{question}
	
	\begin{question}
		{选择题}
		{行列式}
		{容易}
		{21. 将行列式 $\begin{vmatrix}1 & 0 & 2a & 1 \\ 0 & -1 & -b & -1 \\ -1 & -1 & c & -1 \\ -1 & 1 & d & 0\end{vmatrix}$ 按第三列展开,则 $2a$ 的代数余子式为(),行列式的值为()。\\
			A. $\begin{vmatrix}0 & -1 & -1 \\ -1 & -1 & -1 \\ -1 & 1 & 0\end{vmatrix}, 2a-b+d$ \quad
			B. $\begin{vmatrix}0 & -1 & -1 \\ -1 & -1 & -1 \\ -1 & 1 & 0\end{vmatrix}, 2a+b+d$ \quad
			C. $\begin{vmatrix}0 & -1 & -1 \\ -1 & -1 & -1 \\ -1 & 1 & 0\end{vmatrix}, 2a-b-d$ \quad
			D. $\begin{vmatrix}0 & -1 & -1 \\ -1 & -1 & -1 \\ -1 & 1 & 0\end{vmatrix}, 2a+b-d$}
		{按第三列展开行列式,$2a$ 的代数余子式为 $\begin{vmatrix}0 & -1 & -1 \\ -1 & -1 & -1 \\ -1 & 1 & 0\end{vmatrix}$,代入展开式得值为 $2a-b+d$.}
	\end{question}
	
	
	\begin{question}
		{选择题}
		{行列式}
		{容易}
		{22. 某四阶行列式 $D$ 的值为 1 ,它的第一行元素为 $1,7,2,-1$ ,而第一行元素对应的余子式分别为 $-1,0,k,4$ ,则 $k=(\quad)$。\\
			A. $-1$ \quad B. $-2$ \quad C. $1$ \quad D. $2$}
		{由 $D = a_{11}A_{11} + a_{12}A_{12} + a_{13}A_{13} + a_{14}A_{14} = 1$,代入已知值,得 $1 \times (-1) + 7 \times 0 + 2 \times k + (-1) \times (-4) = 1$,解得 $k = -1$.}
	\end{question}
	
	
	\begin{question}
		{选择题}
		{行列式}
		{容易}
		{23. 某四阶行列式 $D$ 的值为 1 ,它的第一行元素为 $1,3,k,-1$ ,而第二行元素对应的余子式分别为 $-1,0,1,2$ ,则 $k=(\quad)$。\\
			A. $-1$ \quad B. $-2$ \quad C. $1$ \quad D. $2$}
		{利用代数余子式的正交性有 $a_{11}A_{21} + a_{12}A_{22} + a_{13}A_{23} + a_{14}A_{24} = 0$,代入已知得 $1\times1 + 3\times0 + k\times(-1) + (-1)\times2 = 0$,解得 $k = -1$.}
	\end{question}
	
	\begin{question}
		{判断题}
		{行列式}
		{容易}
		{1.已知 7 阶行列式 $D=a_{i j}$ ,设 $A_{i j}$ 是 $n$ 阶行列式中 $a_{i j}$ 的代数余子式,则 $a_{11} A_{21}+a_{12} A_{22}+a_{13} A_{23}+a_{14} A_{24}+a_{15} A_{25}+a_{16} A_{26}+$ $a_{17} A_{27}=0$(  ).}
		{行列式的任意一行与另一行的代数余子式的乘积之和等于零.所以结论正确.}
	\end{question}	
	
	\begin{question}
		{判断题}
		{行列式}
		{容易}
		{2.设 $D=\begin{vmatrix}3 & -1 & 2 \\ -2 & -3 & 1 \\ 0 & 1 & -4\end{vmatrix}$ ,则 $2 A_{11}+A_{21}-4 A_{31}=0$(  ).}
		{$2 A_{11}+A_{21}-4 A_{31}=\begin{vmatrix}2 & -1 & 2 \\ 1 & -3 & 1 \\ -4 & 1 & -4\end{vmatrix}=0$ ,所以结论正确.}
	\end{question}
	
	\begin{question}
		{判断题}
		{行列式}
		{容易}
		{3.行列式 $D=\begin{vmatrix}-1 & 3 & 2 \\ -5 & 2 & 0 \\ 1 & 0 & -3\end{vmatrix}$ 中元素 $a_{32}$ 的代数余子式为 $-10$(  ).}
		{$A_{32}=(-1)^{3+2} M_{32}=(-1)^5 \begin{vmatrix}-1 & 2 \\ -5 & 0\end{vmatrix}=-10$ ,所以结论正确.}
	\end{question}
	
	\begin{question}
		{判断题}
		{行列式}
		{容易}
		{4.四阶行列式 $D$ 的某行元素依次为 $1,0,k,6$ ,它们的代数余子式分别为 $3,4,2,0$ ,且 $D=-9$ ,则 $k=1$(  ).}
		{$D=a_{i1} A_{i1}+a_{i2} A_{i2}+a_{i3} A_{i3}+a_{i4} A_{i4}=3+0+2k+0=3+2k$ ,由 $D=-9$ 得 $k=-6$ ,所以结论错误.}
	\end{question}
	
	\begin{question}
		{判断题}
		{行列式}
		{容易}
		{5.设 $D=\begin{vmatrix}3 & -1 & 1 \\ -2 & -3 & 1 \\ 0 & 1 & -1\end{vmatrix}$ ,则 $A_{11}+A_{21}-A_{31}=0$(  ).}
		{$A_{11}+A_{21}-A_{31}=\begin{vmatrix}1 & -1 & 1 \\ 1 & -3 & 1 \\ -1 & 1 & -1\end{vmatrix}=0$ ,所以结论正确.}
	\end{question}
	
	\begin{question}
		{判断题}
		{行列式}
		{容易}
		{6.行列式 $D=\begin{vmatrix}-1 & 3 & 2 \\ -5 & 2 & 0 \\ 1 & 0 & -3\end{vmatrix}$ 中元素 $a_{32}$ 的余子式为 $-10$(  ).}
		{$M_{32}=\begin{vmatrix}-1 & 2 \\ -5 & 0\end{vmatrix}=10$ ,所以结论错误.}
	\end{question}
	
	\begin{question}
		{判断题}
		{行列式}
		{容易}
		{7.已知 $D$ 为三阶行列式,其第三行元素分别为 $1,3,-2$ ,它们的代数余子式分别为 $3,-2,1$ ,则 $D=-5$(  ).}
		{$D=a_{31} A_{31}+a_{32} A_{32}+a_{33} A_{33}=1 \cdot 3 + 3 \cdot (-2) + (-2) \cdot 1 = -5$ ,所以结论正确.}
	\end{question}
	
	\begin{question}
		{判断题}
		{行列式}
		{容易}
		{8.设行列式 $D=\begin{vmatrix}a^2 & ab & b^2 \\ 2a & a+b & 2b \\ 1 & 1 & 1\end{vmatrix}$ ,则它的第一行元素的代数余子式之和 $A_{11}+A_{12}+A_{13}=0$(  ).}
		{$A_{11}+A_{12}+A_{13}=1 \cdot A_{11} + 1 \cdot A_{12} + 1 \cdot A_{13} = D = \begin{vmatrix}1 & 1 & 1 \\ 2a & a+b & 2b \\ 1 & 1 & 1\end{vmatrix}=0$ ,所以结论正确.}
	\end{question}
	
	\begin{question}
		{判断题}
		{行列式}
		{容易}
		{9.设行列式 $D=\begin{vmatrix}a & b & c \\ a^2 & b^2 & c^2 \\ b+c & c+a & a+b\end{vmatrix}$ ,则它的第三行元素的代数余子式之和 $A_{31}+A_{32}+A_{33}=0$(  ).}
		{$A_{31}+A_{32}+A_{33}=1 \cdot A_{31}+1 \cdot A_{32}+1 \cdot A_{33} = \begin{vmatrix}a & b & c \\ a^2 & b^2 & c^2 \\ 1 & 1 & 1\end{vmatrix}=\begin{vmatrix}1 & 1 & 1 \\ a & b & c \\ a^2 & b^2 & c^2\end{vmatrix}=(c-a)(c-b)(b-a)$ ,所以结论错误.}
	\end{question}
	
	\begin{question}
		{判断题}
		{行列式}
		{容易}
		{10.四阶行列式 $D$ 的某行元素依次为 $1,0,k,6$ ,它们的代数余子式分别为 $3,4,2,0$ ,且 $D=-7$ ,则 $k=2$(  ).}
		{$D=a_{i1} A_{i1}+a_{i2} A_{i2}+a_{i3} A_{i3}+a_{i4} A_{i4}=3+0+2k+0=3+2k$ ,由 $D=-7$ 得 $k=-5$ ,所以结论错误.}
	\end{question}
	
	\begin{question}
		{填空题}
		{行列式}
		{容易}
		{1.若
			$$
			\begin{vmatrix}
				a_{11} & a_{12} \\
				a_{21} & a_{22}
			\end{vmatrix}=1 \text {, 则 }\begin{vmatrix}
				a_{11} & 3 a_{12} & 0 \\
				a_{21} & 3 a_{22} & 0 \\
				0 & 6 & 1
			\end{vmatrix}= ( ).
			$$}
		{若 $\begin{vmatrix}a_{11} & a_{12} \\ a_{21} & a_{22}\end{vmatrix}=1$ ,则 $\begin{vmatrix}a_{11} & 3 a_{12} & 0 \\ a_{21} & 3 a_{22} & 0 \\ 0 & 6 & 1\end{vmatrix}$ $=(-1)^{3+3} \cdot 1\begin{vmatrix}a_{11} & 3 a_{12} \\ a_{21} & 3 a_{22}\end{vmatrix}=3\begin{vmatrix}a_{11} & a_{12} \\ a_{21} & a_{22}\end{vmatrix}=3$ .} 
	\end{question}
	
	
	\begin{question}
		{填空题}
		{行列式}
		{容易}
		{2.若 $\begin{vmatrix}1 & 0 & 2 \\ x & 3 & 1 \\ 4 & x & 5\end{vmatrix}$ 的代数余子式 $A_{12}=-1$ ,则代数余子式 $A_{21}=$( ).}
		{因为 $A_{12}=(-1)^{1+2}\begin{vmatrix}x & 1 \\ 4 & 5\end{vmatrix}=-(5x-4)=-1$,可得 $5x-4=1$,解得 $x=1$。代入后,$A_{21}=(-1)^{2+1}\begin{vmatrix}0 & 2 \\ 1 & 5\end{vmatrix}=(-1)\cdot(0\times5 - 2\times1) = (-1)\cdot(-2)=2$。}
	\end{question}
	
	\begin{question}
		{填空题}
		{行列式}
		{容易}
		{3.设 $D=\begin{vmatrix}1 & 0 & 0 & 0 \\ -2 & 7 & 6 & -3 \\ -4 & 8 & 3 & -5 \\ 9 & -7 & 2 & 5\end{vmatrix}$,则元素 8 的余子式为( ),其代数余子式为( )。}
		{元素 8 在第3行第2列,即 $a_{32}=8$。其余子式 $M_{32} = \begin{vmatrix}1 & 0 & 0 \\ -2 & 6 & -3 \\ 9 & 2 & 5\end{vmatrix} = 36$,代数余子式 $A_{32} = (-1)^{3+2} M_{32} = -36$。}
	\end{question}
	
	\begin{question}
		{填空题}
		{行列式}
		{容易}
		{4.已知行列式 $\begin{vmatrix}4 & 3 & 2 & 1 \\ 1 & 4 & 3 & 2 \\ 2 & 1 & 4 & 3 \\ 3 & 2 & 1 & 4\end{vmatrix}$,则 $3A_{14}+4A_{24}+A_{34}+2A_{44}=$( )。}
		{$3A_{14}+4A_{24}+A_{34}+2A_{44}$ 是第2列的元素与第4列对应代数余子式的乘积之和,按照行列式性质,该和等于行列式中第2列与第4列对应展开的混合式,即 $0$。}
	\end{question}
	
	\begin{question}
		{填空题}
		{行列式}
		{容易}
		{5.四阶行列式 $D$ 中第三列元素为 $1,2,3,4$,对应的余子式为 $1,-1,2,1$,则 $D$ 的值为( )。}
		{设 $a_{13}=1, a_{23}=2, a_{33}=3, a_{43}=4$,相应代数余子式 $A_{13}=1, A_{23}=1, A_{33}=2, A_{43}=-1$,所以 $D=1\times1 + 2\times1 + 3\times2 + 4\times(-1) = 1+2+6-4=5$。}
	\end{question}
	
	\begin{question}
		{计算题}
		{行列式}
		{容易}
		{1.已知 $D=\begin{vmatrix}1 & 0 & 0 \\ 2 & 3 & 0 \\ 4 & 5 & 6\end{vmatrix}$ ,求 $A_{11}+2 A_{21}+4 A_{31}$ ,其中 $A_{i j}$ 表示 $D$ 中 $a_{i j}$ 的代数余子式(  ).}
		{$A_{11}+2 A_{21}+4 A_{31}=D=18$ .}
	\end{question}
	
	
	\begin{question}
		{计算题}
		{行列式}
		{容易}
		{2. 已知 $D=\begin{vmatrix}1 & 0 & 0 \\ 2 & 3 & 0 \\ 4 & 5 & 6\end{vmatrix}$,求 $7 A_{31} + 8 A_{32} + 9 A_{33}$,其中 $A_{ij}$ 表示 $D$ 中 $a_{ij}$ 的代数余子式( )。}
		{$7 A_{31} + 8 A_{32} + 9 A_{33} = \begin{vmatrix}1 & 0 & 0 \\ 2 & 3 & 0 \\ 7 & 8 & 9\end{vmatrix} = 27$。}
	\end{question}
	
	
	\begin{question}
		{计算题}
		{行列式}
		{中等}
		{3. 设四阶行列式 $D_4=\begin{vmatrix}a & b & c & d \\ d & a & c & d \\ b & d & c & a \\ a & d & c & b\end{vmatrix}$,求 $A_{11} + A_{21} + A_{31} + A_{41}$ ( )。}
		{将第1列都变为1,可转化为:
			$
			A_{11} + A_{21} + A_{31} + A_{41} = \begin{vmatrix}1 & b & c & d \\ 1 & a & c & d \\ 1 & d & c & a \\ 1 & d & c & b\end{vmatrix} 
			$
			该行列式中第1列元素都为1,可将其展开:记第1列为常数列,可看作列向量线性相关,因而行列式为0,
			所以:$A_{11} + A_{21} + A_{31} + A_{41} = 0$。}
	\end{question}
	
	\begin{question}
		{计算题}
		{行列式}
		{容易}
		{4. 设 $D=\begin{vmatrix}1 & 2 & 3 \\ 1 & 0 & -1 \\ -2 & 0 & 3\end{vmatrix}$,其中 $A_{ij}$ 表示代数余子式,求 $A_{11}+A_{22}+A_{33}$ ( )。}
		{$A_{11} = (-1)^{1+1} \begin{vmatrix}0 & -1 \\ 0 & 3\end{vmatrix} = 1 \cdot (0 \cdot 3 - (-1)\cdot 0) = 0, $ $A_{22} = (-1)^{2+2} \begin{vmatrix}1 & 3 \\ -2 & 3\end{vmatrix} = (1)(1 \cdot 3 - 3 \cdot (-2)) = 9, $ $A_{33} = (-1)^{3+3} \begin{vmatrix}1 & 2 \\ 1 & 0\end{vmatrix} = 1 \cdot (1 \cdot 0 - 2 \cdot 1) = -2, A_{11} + A_{22} + A_{33} = 0 + 9 - 2 = 7.$}
	\end{question}
	
	\begin{question}
		{计算题}
		{行列式}
		{容易}
		{5.已知四阶行列式 \$D\$ 的第3行元素依次为 \$-1,2,0,1\$,它们的余子式分别为 \$5,3,-7,4\$,求行列式的值( )。}
		{由题设余子式 $M_{31}=5, M_{32}=3, M_{33}=-7, M_{34}=4$ ,故代数余子式为 $A_{31}=5, A_{32}=-3, A_{33}=-7, A_{34}=-4$ ,所以 $D=$ $(-1) \times 5+2 \times(-3)+0 \times(-7)+1 \times(-4)=-15$ .}
	\end{question}
	
	\begin{question}
		{计算题}
		{行列式}
		{中等}
		{6.设 $D=\begin{vmatrix}3 & -5 & 2 & 1 \\ 1 & 1 & 0 & -5 \\ -1 & 3 & 1 & 3 \\ 2 & -4 & -1 & -3\end{vmatrix}$ 的代数余子式为 $A_{ij}$,求 $A_{11}+2 A_{12}+A_{13}+A_{14}$( )。}
		{将行列式按照第 i 行展开:$|D|=a_{i 1} A_{i 1}+a_{i 2} A_{i 2}+$ $a_{i 3} A_{i 3}+a_{i 4} A_{i 4},(i=1,2,3,4)$ ,其中 $A_{i j}=(-1)^{i+j} M_{i j}$ ,令 $a_{11}=1, a_{12}=2, a_{13}=1, a_{14}=1$ ,则 $A_{11}+2 A_{12}+A_{13}+A_{14}$ $=\begin{vmatrix}1 & 2 & 1 & 1 \\ 1 & 1 & 0 & -5 \\ -1 & 3 & 1 & 3 \\ 2 & -4 & -1 & -3\end{vmatrix}=\begin{vmatrix}1 & 2 & 1 & 1 \\ 0 & -1 & -1 & -6 \\ 0 & 5 & 2 & 4 \\ 0 & -8 & -3 & -5\end{vmatrix}=$
			$\begin{vmatrix}1 & 2 & 1 & 1 \\ 0 & -1 & -1 & -6 \\ 0 & 0 & -3 & -26 \\ 0 & 0 & 5 & 43\end{vmatrix}=\begin{vmatrix}1 & 2 & 1 & 1 \\ 0 & -1 & -1 & -6 \\ 0 & 0 & -3 & -26 \\ 0 & 0 & 0 & -\frac{1}{3}\end{vmatrix}=-1$.}
	\end{question}
	
	\begin{question}
		{计算题}
		{行列式}
		{中等}
		{7.设 $D=\begin{vmatrix}3 & -5 & 2 & 1 \\ 1 & 1 & 0 & -5 \\ -1 & 3 & 1 & 3 \\ 2 & -4 & -1 & -3\end{vmatrix}$ 的余子式为 $M_{ij}$,求 $3 M_{21}+5 M_{22}+M_{23}+2 M_{24}$( )。}
		{$3 M_{21}+5 M_{22}+M_{23}+2 M_{24}=-3 A_{21}+5 A_{22}-A_{23}+$ $2 A_{24} =\begin{vmatrix}3 & -5 & 2 & 1 \\ -3 & 5 & -1 & 2 \\ -1 & 3 & 1 & 3 \\ 2 & -4 & -1 & -3\end{vmatrix}=-10$.}
	\end{question}
	
	\begin{question}
		{计算题}
		{行列式}
		{中等}
		{8.用降阶法计算行列式 $\begin{vmatrix}3 & 1 & -1 & 1 \\ -5 & 1 & 3 & -4 \\ 2 & 0 & 1 & 0 \\ 1 & -5 & 3 & -3\end{vmatrix}$( )。}
		{ $D \stackrel{c_1-2 c 3}{=}\begin{vmatrix}5 & 1 & -1 & 1 \\ -11 & 1 & 3 & -4 \\ 0 & 0 & 1 & 0 \\ -5 & -5 & 3 & -3\end{vmatrix}=$
			
			$$
			\begin{aligned}
				& (-1)^{3+3}\begin{vmatrix}
					5 & 1 & 1 \\
					-11 & 1 & -4 \\
					-5 & -5 & -3
				\end{vmatrix} \\
				=\begin{vmatrix}
					5 & 1 & 1 \\
					-16 & 0 & -5 \\
					20 & 0 & 2
				\end{vmatrix}= \\
				& (-1)^{1+2}\begin{vmatrix}
					-5 \\
					20 & 2
				\end{vmatrix}=-68 .
			\end{aligned}
			$$}
	\end{question}
	
	\begin{question}
		{计算题}
		{行列式}
		{中等}
		{9.用降阶法计算行列式 $\begin{vmatrix}1 & 1 & -1 & 2 \\ -5 & 1 & 3 & -1 \\ 2 & 0 & 1 & -1 \\ 1 & -5 & 3 & -3\end{vmatrix}$( )。}
		{$$
			\begin{aligned}
				&D^{\stackrel{c_1+(-2) c_3}{c_4+c_3}=}\begin{vmatrix}
					3 & 1 & -1 & 1 \\
					-11 & 1 & 3 & 2 \\
					0 & 0 & 1 & 0 \\
					-5 & -5 & 3 & 0
				\end{vmatrix}= \\
				& (-1)^{3+3}\begin{vmatrix}
					3 & 1 & 1 \\
					-11 & 1 & 2 \\
					-5 & -5 & 0
				\end{vmatrix} \stackrel{c_2+(-1) c_1}{=}\begin{vmatrix}
					3 & -2 & 1 \\
					-11 & 12 & 2 \\
					-5 & 0 & 0
				\end{vmatrix}=(-5) \times \\
				& (-1)^{3+1}\begin{vmatrix}
					1 \\
					12 & 2
				\end{vmatrix}=80
			\end{aligned}
			$$}
	\end{question}
	
	
	\begin{question}
		{计算题}
		{行列式}
		{中等}
		{10.设 $D=\begin{vmatrix}1 & 2 & 3 \\ 1 & 0 & -1 \\ -2 & 0 & 3\end{vmatrix}$ ,求 $D$ .( )}
		{$D=\begin{vmatrix}1 & 2 & 3 \\ 1 & 0 & -1 \\ -2 & 0 & 3\end{vmatrix}=2 \times(-1)^{1+2}\begin{vmatrix}1 & -1 \\ -2 & 3\end{vmatrix}=$ -2 .}
	\end{question}
	
	\begin{question}
		{计算题}
		{行列式}
		{中等}
		{11.设 $|A|=\begin{vmatrix}1 & -5 & 1 & 3 \\ 1 & 1 & 3 & 4 \\ 1 & 1 & 2 & 3 \\ 2 & 2 & 3 & 4\end{vmatrix}$ ,计算 $A_{41}+A_{42}+A_{43}+A_{44}$ 的值,其中 $A_{4 j}$ 是 $|A|$ 中元素 $a_{4 j}(j=1,2,3,4)$ 的代数余子式.}
		{$$
			\begin{aligned}
				& A_{41}+A_{42}+A_{43}+A_{44}=\begin{vmatrix}
					1 & -5 & 1 & 3 \\
					1 & 1 & 3 & 4 \\
					1 & 1 & 2 & 3 \\
					1 & 1 & 1 & 1
				\end{vmatrix} \stackrel{r_1 \leftrightarrows r_4}{=} \\
				& -\begin{vmatrix}
					1 & 1 & 1 & 1 \\
					1 & -1 & 3 & 4 \\
					1 & 1 & 2 & 3 \\
					1 & -5 & 1 & 3
				\end{vmatrix} =-\begin{vmatrix}
					1 & 1 & 1 & 1 \\
					= & r_4-r_1 \\
					0 & 0 & 2 & 3 \\
					0 & 0 & 1 & 2 \\
					0 & -6 & 0 & 2
				\end{vmatrix}= \\
				& (-1)\begin{vmatrix}
					0 & 2 & 3 \\
					0 & 1 & 2 \\
					-6 & 0 & 2
				\end{vmatrix}=(-1) \times(-6) \times(-1)^{3+1}\begin{vmatrix}
					2 & 3 \\
					1
				\end{vmatrix}=6 .
			\end{aligned}
			$$}
	\end{question}
	
	\begin{question}
		{计算题}
		{行列式}
		{中等}
		{12.已知行列式 $D=\begin{vmatrix}1 & 2 & 3 & 4 \\ 1 & 0 & 1 & 2 \\ 3 & -1 & -1 & 0 \\ 1 & 2 & 0 & -5\end{vmatrix}$ ,求余子式 $M_{13}$ 和代数余子式 $A_{43}$ .}
		{余子式 $M_{13}=\begin{vmatrix}1 & 0 & 2 \\ 3 & -1 & 0 \\ 1 & 2 & -5\end{vmatrix}=-19$ ,代数余子式 $A_{43}=$ $(-1)^{4+3}\begin{vmatrix}1 & 2 & 4 \\ 1 & 0 & 2 \\ 3 & -1 & 0\end{vmatrix}=-10$.}
	\end{question}
	
	
	
	\begin{question}
		{计算题}
		{行列式}
		{中等}
		{13.用行列式按行(列)展开定理计算行列式:$D=$
			
			$$
			\begin{vmatrix}
				1 & 2 & 3 & 4 \\
				1 & 0 & 1 & 2 \\
				3 & -1 & -1 & 0 \\
				1 & 2 & 0 & -5
			\end{vmatrix} .
			$$}
		{按第二行展开
			
			$$
			\begin{aligned}
				D & =1 \cdot A_{21}+0 \cdot A_{22}+1 \cdot A_{23}+2 A_{24} \\
				= & 1 \times(-1)^{2+1} 3+1 \times(-1)^{2+3} 63+2 \times(-1)^{2+4} 21=-3-63+42=-24
			\end{aligned}
			$$
			利用展开定理时,通常结合性质将展开行(列)的较多元素化为零.}
	\end{question}
	
	\begin{question}
		{计算题}
		{行列式}
		{中等}
		{14.设 $|A|=\begin{vmatrix}1 & -5 & 1 & 3 \\ 1 & 1 & 3 & 4 \\ 1 & 1 & 2 & 3 \\ 2 & 2 & 3 & 4\end{vmatrix}$ ,计算 $A_{41}+A_{42}+A_{43}+A_{44}$ 的值,其中 $A_{4 i}(i=1,2,3,4)$ 是对应元素的代数余子式.}
		{由行列式按行展开定理 $A_{41}+A_{42}+A_{43}+A_{44}=1 \cdot A_{41}+$
			
			$$
			1 \cdot A_{42}+1 \cdot A_{43}+1 \cdot A_{44}=\begin{vmatrix}
				1 & -5 & 1 & 3 \\
				1 & 1 & 3 & 4 \\
				1 & 1 & 2 & 3 \\
				1 & 1 & 1 & 1
			\end{vmatrix}=
			$$
			
			
			$$
			\begin{aligned}
				& \begin{vmatrix}
					1 & -5 & 1 & 3 \\
					0 & 6 & 2 & 1 \\
					0 & 6 & 1 & 0 \\
					0 & 6 & 0 & -2
				\end{vmatrix}=\begin{vmatrix}
					6 & 2 & 1 \\
					6 & 1 & 0 \\
					6 & 0 & -2
				\end{vmatrix}=\begin{vmatrix}
					6 & 2 & 1 \\
					0 & -1 & -1 \\
					0 & -2 & -3
				\end{vmatrix}= \\
				& 6\begin{vmatrix}
					-1 & -1 \\
					-2 & -3
				\end{vmatrix}=6 .
			\end{aligned}
			$$}
	\end{question}
	
	
\end{document}
