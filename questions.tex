\documentclass{ctexart}
\usepackage{amsmath}
\usepackage{amssymb}
\usepackage{enumitem}

\newenvironment{question}[5]{%
	\noindent\textbf{题型:}#1\\
	\textbf{主题:}#2\\
	\textbf{难度:}#3\\
	\textbf{题目:}#4\\
	\textbf{解析:}#5\\
	\vspace{1em}
}{}

\begin{document}
	\begin{question}
		{选择题}
		{行列式}
		{容易}
		{1.行列式 \(\begin{vmatrix} 1 & 3 \\ 1 & 4\end{vmatrix}=(\quad).\)
			A. 1 \quad B. 0 \quad C. -2 \quad D. 3}
		{\(\begin{vmatrix} 1 & 3 \\ 1 & 4\end{vmatrix}=1 \times 4-1 \times 3=1. \)}
	\end{question}
	
	\begin{question}
		{选择题}
		{行列式}
		{容易}
		{2.行列式 \(\begin{vmatrix}2 & 1 \\ -1 & 2\end{vmatrix}=(\quad).\)
			A. 5 \quad B. 3	\quad C. -1 \quad D. -3 }
		{\(\begin{vmatrix}2 & 1 \\ -1 & 2\end{vmatrix}=2 \times 2-1 \times(-1)=5.\) }
	\end{question}
	
	\begin{question}
		{选择题}
		{行列式}
		{容易}
		{3.行列式 \(\begin{vmatrix} 6 & 9 \\ 8 & 12\end{vmatrix}=( \quad ). \)
			A. 0 \quad B. 144 \quad C. 12 \quad D. -12}
		{\(\begin{vmatrix} 6 & 9 \\ 8 & 12\end{vmatrix}=6 \times 12-9 \times 8=0. \)}
	\end{question}
	
	\begin{question}
		{选择题}
		{行列式}
		{容易}
		{4.行列式 \(\begin{vmatrix} 1 & 3 \\ 1 & 5\end{vmatrix}=( \quad ).\)
			A. 2  \quad B. 8  \quad C. -8	 \quad D. 15}
		{\(\begin{vmatrix} 1 & 3 \\ 1 & 5\end{vmatrix}=1 \times 5-1 \times 3=2\) .}
	\end{question}
	
	\begin{question}
		{选择题}
		{行列式}
		{容易}
		{5.行列式 \(\begin{vmatrix} a & b \\ a^2 & b^2\end{vmatrix}=(\quad). \)A. \(a b(b-a)\) \quad B. \(a^3-b^2\) \quad C. \(a b(a-b)\) \quad D. \(a^3+b^2-1\)}
		{\(\begin{vmatrix} a & b \\ a^2 & b^2\end{vmatrix}=a \times b^2-b \times a^2=a b(b-a).\)}
	\end{question}
	
	\begin{question}
		{选择题}
		{行列式}
		{容易}
		{6.行列式 \(\begin{vmatrix} -6 & -2 \\ 4 & 1\end{vmatrix}=(\quad). \)A. 2 \quad B. 4 \quad C. 1 \quad D. -1}
		{\(\begin{vmatrix} -6 & -2 \\ 4 & 1\end{vmatrix}=-6 \times 1-(-2) \times 4=2.\)}
	\end{question}
	
	\begin{question}
		{选择题}
		{行列式}
		{容易}
		{7.行列式 \(\begin{vmatrix} 1 & 7 \\ 2 & 5\end{vmatrix}=(\quad). \)A. -9 \quad B. 7 \quad C. -8 \quad D. 11}
		{\(\begin{vmatrix} 1 & 7 \\ 2 & 5\end{vmatrix}=1 \times 5-2 \times 7=-9.\)}
	\end{question}
	
	\begin{question}
		{选择题}
		{行列式}
		{容易}
		{8.行列式 \(\begin{vmatrix} 1 & -7 \\ 3 & -5\end{vmatrix}=(\quad). \)A. 16 \quad B. -26 \quad C. -8 \quad D. 17}
		{\(\begin{vmatrix} 1 & -7 \\ 3 & -5\end{vmatrix}=1 \times(-5)-3 \times(-7)=16.\)}
	\end{question}
	
	
	\begin{question}
		{选择题}
		{行列式}
		{容易}
		{9.行列式 \(\begin{vmatrix} x+1 & 1 \\ 1 & x-1\end{vmatrix}=(\quad). \)A.\(x^2-2\) \quad B.\(x^2\) \quad C.\(x^2+1\) \quad D.\(x^2-1\)}
		{\(\begin{vmatrix} x+1 & 1 \\ 1 & x-1\end{vmatrix}=(x+1)(x-1)-1=x^2-2.\)}
	\end{question}
	
	\begin{question}
		{选择题}
		{行列式}
		{容易}
		{10.行列式 \(\begin{vmatrix} a & b \\ b & a\end{vmatrix}=(\quad). \)A. \(a^2-b^2\) \quad B. \(a^2\) \quad C. \(a^2+1\) \quad D. \(a^2+b^2\)}
		{\(\begin{vmatrix} a & b \\ b & a\end{vmatrix}=a^2-b^2\).}
	\end{question}
	
	\begin{question}
		{选择题}
		{行列式}
		{容易}
		{11.行列式 \(\begin{vmatrix} 2 & 4 \\ -1 & 3\end{vmatrix}=(\quad). \)A. 10 \quad B. 13 \quad C. -10 \quad D. 5}
		{\(\begin{vmatrix} 2 & 4 \\ -1 & 3\end{vmatrix}=2 \times 3-4 \times(-1)=10.\)}
	\end{question}
	
	\begin{question}
		{选择题}
		{行列式}
		{容易}
		{12.行列式 \(\begin{vmatrix} 4 & 3 \\ -1 & 4\end{vmatrix}=(\quad). \)A. 19 \quad B. 10 \quad C. -21 \quad D. 13}
		{\(\begin{vmatrix} 4 & 3 \\ -1 & 4\end{vmatrix}=4 \times 4-(-1) \times 3=19. \)}
	\end{question}
	
	\begin{question}
		{选择题}
		{行列式}
		{容易}
		{13. 行列式 \(\begin{vmatrix} 2 & 3 \\ -1 & 5\end{vmatrix}=(\quad). \)
			A. 13 \quad B. 3 \quad C. -1 \quad D. 5}
		{\(\begin{vmatrix} 2 & 3 \\ -1 & 5\end{vmatrix}=2 \times 5-3 \times(-1)=13. \)}
	\end{question}
	
	
	\begin{question}
		{选择题}
		{行列式}
		{容易}
		{14. 行列式 \(\begin{vmatrix} 5 & -2 \\ 8 & 1\end{vmatrix}=(\quad). \) A.  21 \quad B.  14 \quad C.  12 \quad D. -12}
		{\(\begin{vmatrix} 5 & -2 \\ 8 & 1\end{vmatrix}=5 \times 1-(-2) \times 8=21. \)}
	\end{question}
	
	\begin{question}
		{选择题}
		{行列式}
		{容易}
		{15. 行列式 \(\begin{vmatrix} 2 & 3 \\ 1 & 8\end{vmatrix}=(\quad). \) A.  13 \quad B.  18 \quad C. -18 \quad D.  15}
		{\(\begin{vmatrix} 2 & 3 \\ 1 & 8\end{vmatrix}=2 \times 8-1 \times 3=13\)}
	\end{question}
	
	\begin{question}
		{选择题}
		{行列式}
		{容易}
		{16. 行列式 \(\begin{vmatrix} 1 & -3 \\ -1 & 4\end{vmatrix}=(\quad). \) A.  1 \quad B.  0 \quad C. -2 \quad D.  3}
		{\(\begin{vmatrix} 1 & -3 \\ -1 & 4\end{vmatrix}=1 \times 4-(-1) \times(-3)=1.\)}
	\end{question}
	
	\begin{question}
		{选择题}
		{行列式}
		{容易}
		{17. 行列式 \(\begin{vmatrix} 2 & 1 \\ -1 & -2\end{vmatrix}=(\quad). \) A. -3 \quad B.  3 \quad C. -1 \quad D.  5}
		{\(\begin{vmatrix} 2 & 1 \\ -1 & -2\end{vmatrix}=2 \times(-2)-1 \times(-1)=-3.\)}
	\end{question}
	
	\begin{question}
		{选择题}
		{行列式}
		{容易}
		{18. 行列式 \(\begin{vmatrix} a & b \\ a^2 & b\end{vmatrix}=(\quad). \)A. \(a b(1-a)\) \quad B. \(a^3-b^2\) \quad C. \(a^3\) \quad D. \(a^3+b^2-1\)}
		{\(\begin{vmatrix} a & b \\ a^2 & b\end{vmatrix}=a b-b a^2=a b(1-a).\)}
	\end{question}
	
	
	\begin{question}
		{选择题}
		{行列式}
		{容易}
		{19. 行列式 \(\begin{vmatrix} 3 & 7 \\ 2 & 5\end{vmatrix}=(\quad). \)A.  1 \quad B.  7 \quad C. -8 \quad D.  11}
		{\(\begin{vmatrix} 3 & 7 \\ 2 & 5\end{vmatrix}=3 \times 5-2 \times 7=1\).}
	\end{question}
	
	\begin{question}
		{选择题}
		{行列式}
		{容易}
		{20.行列式 \(\begin{vmatrix} a & b \\ b & -a\end{vmatrix}=(\quad).\)
			A. \(-a^2-b^2\) \quad B. \(a^2\) \quad C. \(a^2+1\) \quad D. \(a^2+b^2\)}
		{\(\begin{vmatrix} a & b \\ b & -a\end{vmatrix}=-a^2-b^2. \)}
	\end{question}
	
	
	\begin{question}
		{选择题}
		{行列式}
		{容易}
		{21.行列式 \(\begin{vmatrix} 4 & 3 \\ 1 & 5\end{vmatrix}=(\quad). \) A.  17 \quad B.  18 \quad C.-18 \quad D.  15}
		{\(\begin{vmatrix} 4 & 3 \\ 1 & 5\end{vmatrix}=4 \times 5-1 \times 3=17. \)}
	\end{question}
	
	
	\begin{question}
		{选择题}
		{行列式}
		{容易}
		{22. 行列式 \(\begin{vmatrix} 1 & 2 \\ 3 & -5\end{vmatrix}=().\) A. -11 B.  16 C.-8 D.  17}
		{\(\begin{vmatrix} 1 & 2 \\ 3 & -5\end{vmatrix}=1 \times(-5)-3 \times 2=-11.\)}
	\end{question}
	
	\begin{question}
		{选择题}
		{行列式}
		{容易}
		{23. 行列式 \(\begin{vmatrix} -6 & -2 \\ 8 & 1\end{vmatrix}=\)A.  10 B.  14 C.  12 D. -12}
		{\(\begin{vmatrix} -6 & -2 \\ 8 & 1\end{vmatrix}=-6 \times 1-(-2) \times 8=10. \)}
	\end{question}
	
	\begin{question}
		{选择题}
		{行列式}
		{容易}
		{24.\(\begin{vmatrix} a & 1 & 1 \\ 0 & -1 & 0 \\ 4 & a & a\end{vmatrix}>0\) 的充分必要条件是(\quad).  A. \(|a|<2\) B. \(|a|=2\) C.\(|a|>2\) D.\(|a| \leq 2\)}
		{\(\begin{vmatrix} a & 1 & 1 \\ 0 & -1 & 0 \\ 4 & a & a\end{vmatrix}=-a^2+4. \) 若 \(\begin{vmatrix} a & 1 & 1 \\ 0 & -1 & 0 \\ 4 & a & a\end{vmatrix}>0\) . 则有 \(a^2<4\) . 即 \(|a|<2.\)}
	\end{question}
	
	\begin{question}
		{选择题}
		{行列式}
		{容易}
		{25. 当 \(a, b\) 满足条件()时. 行列式 \(\begin{vmatrix} a & b & 0 \\ -b & a & 0 \\ 1 & 0 & 1\end{vmatrix}=0.\) A. \(a=0\) 且 \(b=0\) B. \(a=0\) 或 \(b=0\) C. \(a=0\) D. \(b=0\)}
		{\(\begin{vmatrix} a & b & 0 \\ -b & a & 0 \\ 1 & 0 & 1\end{vmatrix}=a^2+b^2=0\).}
	\end{question}
	
	\begin{question}
		{选择题}
		{行列式}
		{容易}
		{26. \(\begin{vmatrix} k-1 & 2 \\ 2 & k-1\end{vmatrix} \neq 0\) 的充分必要条件是(). A. \(k \neq 3\) 且 \(k \neq-1\) B. \(k \neq 3\) C. \(k \neq-1\) D. \(k \neq 3\) 或 \(k \neq-1\)}
		{因为\(\begin{vmatrix} k-1 & 2 \\ 2 & k-1\end{vmatrix}=k^2-2 k-3=(k-3)(k+1)\) .所以若 \(\begin{vmatrix} k-1 & 2 \\ 2 & k-1\end{vmatrix} \neq 0\) .则 \(k \neq 3\) 且 \(k \neq-1\) .}
	\end{question}
	
	
	\begin{question}
		{选择题}
		{行列式}
		{容易}
		{27. \(\begin{vmatrix} k & 2 & 1 \\ 2 & k & 0 \\ 1 & -1 & 1\end{vmatrix}=0\) 的充分条件是(\qquad).A. \(k=-2\) 或 \(k=3\) B. \(k=3\) C. \(k=-1\) D. \(k=-2\) 且 \(k=3\)}
		{因为 \(\begin{vmatrix} k & 2 & 1 \\ 2 & k & 0 \\ 1 & -1 & 1\end{vmatrix}=(k+2)(k-3)\),所以 \((k+2)(k-3)=0\),即 \(k=-2\) 或 \(k=3\).}
	\end{question}
	
	\begin{question}
		{选择题}
		{行列式}
		{容易}
		{28. 行列式 \(\begin{vmatrix} 1 & 2 & 3 \\ 3 & 1 & 2 \\ 2 & 3 & 1\end{vmatrix}=\)(\qquad).A. 18 B. -6 C. -8 D. 15}
		{行列式计算为 \(1 \cdot 1 \cdot 1 + 2 \cdot 2 \cdot 2 + 3 \cdot 3 \cdot 3 - 1 \cdot 2 \cdot 3 - 2 \cdot 3 \cdot 1 - 3 \cdot 1 \cdot 2 = 1 + 8 + 27 - 6 - 6 - 6 = 18\).}
	\end{question}
	
	\begin{question}
		{选择题}
		{行列式}
		{容易}
		{29. 行列式 \(\begin{vmatrix} 1 & 1 & 1 \\ 3 & 1 & 4 \\ 8 & 9 & 5\end{vmatrix}=\)(\qquad).A. 5 B. -6 C. -5 D. 15}
		{计算得:\(1 \cdot 1 \cdot 5 + 1 \cdot 4 \cdot 8 + 1 \cdot 3 \cdot 9 - 1 \cdot 4 \cdot 9 - 1 \cdot 3 \cdot 5 - 1 \cdot 1 \cdot 8 = 5 + 32 + 27 - 36 - 15 - 8 = 5\).}
	\end{question}
	
	\begin{question}
		{选择题}
		{行列式}
		{容易}
		{30. 行列式 \(\begin{vmatrix} 0 & a & 0 \\ b & 0 & c \\ 0 & d & 0\end{vmatrix}=\)(\qquad).A. 0 B. \(ab\) C. \(a^2\) D. \(-ab\)}
		{此行列式中第一列和第三列全为 0,因此值为 0.}
	\end{question}
	
	\begin{question}
		{选择题}
		{行列式}
		{容易}
		{31. \(\begin{vmatrix} 2k - 2 & 2 \\ 4 & k - 1\end{vmatrix} > 0\) 的充分必要条件是(\qquad).A. \(k > 3\) 或 \(k < -1\) B. \(k \neq 3\) C. \(k \neq -1\) D. \(k \neq 3\) 且 \(k \neq -1\)}
		{行列式值为 \(2(k^2 - 2k - 3) = 2(k - 3)(k + 1)\),因此当 \((k - 3)(k + 1) > 0\),即 \(k < -1\) 或 \(k > 3\).}
	\end{question}
	
	\begin{question}
		{选择题}
		{行列式}
		{容易}
		{32. 行列式 \(\begin{vmatrix} 1 & 3 & 2 \\ 2 & 1 & 3 \\ 3 & 2 & 1\end{vmatrix}=\)(\qquad).A. 18 B. -6 C. -8 D. 15}
		{该行列式的值为 \(1 \cdot 1 \cdot 1 + 3 \cdot 3 \cdot 3 + 2 \cdot 2 \cdot 2 - 1 \cdot 3 \cdot 2 - 3 \cdot 2 \cdot 1 - 2 \cdot 1 \cdot 3 = 1 + 27 + 8 - 6 - 6 - 6 = 18\).}
	\end{question}
	
	\begin{question}
		{选择题}
		{行列式}
		{容易}
		{33. 行列式 \(\begin{vmatrix} 1 & 3 & 8 \\ 1 & 1 & 9 \\ 1 & 4 & 5\end{vmatrix}=\)(\qquad).A. 5 B. -6 C. -5 D. 15}
		{计算得 \(1 \cdot 1 \cdot 5 + 3 \cdot 9 \cdot 1 + 8 \cdot 1 \cdot 4 - 8 \cdot 1 \cdot 1 - 3 \cdot 1 \cdot 5 - 1 \cdot 9 \cdot 4 = 5 + 27 + 32 - 8 - 15 - 36 = 5\).}
	\end{question}
	
	\begin{question}
		{选择题}
		{行列式}
		{容易}
		{34. 行列式 \(\begin{vmatrix} 0 & -a & 0 \\ b & 0 & -c \\ 0 & d & 0\end{vmatrix}=\)(\qquad).A. 0 B. \(-ab\) C. \(a^2\) D. \(-ab\)}
		{由于第一列和第三列均为 0,因此该行列式为 0.}
	\end{question}
	
	\begin{question}
		{选择题}
		{行列式}
		{中等}
		{35. \(\begin{vmatrix} a & 0 & 1 \\ 0 & -1 & 0 \\ 4 & a & a\end{vmatrix} > 0\) 的充分必要条件是(\qquad).A. \(|a| < 2\) B. \(|a| = 2\) C. \(|a| > 2\) D. \(|a| \leq 2\)}
		{行列式计算为 \(-a^2 + 4\),令其大于 0,有 \(a^2 < 4\),即 \(|a| < 2\).}
	\end{question}
	
	\begin{question}
		{选择题}
		{行列式}
		{容易}
		{36. \(\begin{vmatrix} 1 - k & 2 \\ 2 & 1 - k\end{vmatrix} \neq 0\) 的充分必要条件是(\qquad).A. \(k \neq 3\) 且 \(k \neq -1\) B. \(k \neq 3\) C. \(k \neq -1\) D. \(k \neq 3\) 或 \(k \neq -1\)}
		{行列式为 \((1 - k)^2 - 4 = k^2 - 2k - 3 = (k - 3)(k + 1)\),要使其不为 0,必须 \(k \neq 3\) 且 \(k \neq -1\).}
	\end{question}
	
	\begin{question}
		{选择题}
		{行列式}
		{容易}
		{37. 当 \(a, b\) 满足条件()时,行列式 \(\begin{vmatrix} a & -b & 0 \\ b & a & 0 \\ 1 & 0 & 1\end{vmatrix}=0\)(\qquad).A. \(a=0\) 且 \(b=0\) B. \(a=0\) 或 \(b=0\) C. \(a=0\) D. \(b=0\)}
		{因为 \(\begin{vmatrix} a & -b & 0 \\ b & a & 0 \\ 1 & 0 & 1\end{vmatrix}=a^2+b^2\),所以 \(a^2+b^2=0\) 即 \(a=0\) 且 \(b=0\).}
	\end{question}
	
	\begin{question}
		{选择题}
		{行列式}
		{容易}
		{38. \(\begin{vmatrix} k & 2 & 1 \\ 2 & k & 0 \\ 1 & -1 & 1\end{vmatrix}>0\) 的必要条件是(\qquad).A. \(k>3\) 或 \(k<-2\) B. \(k>0\) C. \(k<0\) D. \(k=-2\) 且 \(k=3\)}
		{因为 \(\begin{vmatrix} k & 2 & 1 \\ 2 & k & 0 \\ 1 & -1 & 1\end{vmatrix}=(k+2)(k-3)\),若其大于 0,则有 \((k+2)(k-3)>0\),解得 \(k<-2\) 或 \(k>3\).}
	\end{question}
	
	\begin{question}
		{选择题}
		{行列式}
		{容易}
		{39. 行列式 \(\begin{vmatrix} 1 & 2 & -1 \\ 3 & 1 & -3 \\ 2 & 3 & -2\end{vmatrix}=\)(\qquad). 
			A. 0 \quad B. \(-6\) \quad C. \(-8\) \quad D. 15}
		{\(\begin{vmatrix} 1 & 2 & -1 \\ 3 & 1 & -3 \\ 2 & 3 & -2\end{vmatrix}=0\). }
	\end{question}
	
	\begin{question}
		{选择题}
		{行列式}
		{容易}
		{40. 行列式 \(\begin{vmatrix} 1 & 1 & 1 \\ 3 & 1 & 4 \\ 6 & 2 & 8\end{vmatrix}=\)(\qquad). 
			A. 0 \quad B. \(-6\) \quad C. \(-5\) \quad D. 15}
		{\(\begin{vmatrix} 1 & 1 & 1 \\ 3 & 1 & 4 \\ 6 & 2 & 8\end{vmatrix}=0\). }
	\end{question}
	
	\begin{question}
		{选择题}
		{行列式}
		{容易}
		{41. 行列式 \(\begin{vmatrix} 0 & a & 0 \\ b & 2 & c \\ 0 & d & 0\end{vmatrix}=\)(\qquad). 
			A. 0 \quad B. \(ab\) \quad C. \(a^2\) \quad D. \(-ab\)}
		{\(\begin{vmatrix} 0 & a & 0 \\ b & 2 & c \\ 0 & d & 0\end{vmatrix}=0\). }
	\end{question}
	
	\begin{question}
		{选择题}
		{行列式}
		{容易}
		{42. 行列式 \(\begin{vmatrix} 1 & 3 & 2 \\ 2 & 6 & 4 \\ 3 & 2 & 1\end{vmatrix}=\)(\qquad). 
			A. 0 \quad B. \(-6\) \quad C. \(-8\) \quad D. 15}
		{\(\begin{vmatrix} 1 & 3 & 2 \\ 2 & 6 & 4 \\ 3 & 2 & 1\end{vmatrix}=0\). }
	\end{question}
	
	\begin{question}
		{选择题}
		{行列式}
		{容易}
		{43. 行列式 \(\begin{vmatrix} 6 & 3 & 8 \\ 2 & 1 & 9 \\ 8 & 4 & 5\end{vmatrix}=\)(\qquad). 
			A. 0 \quad B. \(-6\) \quad C. \(-5\) \quad D. 15}
		{\(\begin{vmatrix} 6 & 3 & 8 \\ 2 & 1 & 9 \\ 8 & 4 & 5\end{vmatrix}=0\). }
	\end{question}
	
	\begin{question}
		{选择题}
		{行列式}
		{中等}
		{44. 当 \(x\)(\qquad)时,\(\begin{vmatrix} -3 & -1 & -x \\ -4 & -x & 0 \\ 1 & 0 & x\end{vmatrix}>0\). 
			A. \(x<0\) 或 \(x>2\) \quad B. \(0<x<2\) \quad C. \(x=2\) 时 \quad D. \(x\neq0\) 且 \(x\neq2\)}
		{\(\begin{vmatrix} -3 & -1 & -x \\ -4 & -x & 0 \\ 1 & 0 & x\end{vmatrix}=2x^2 - 4x = 2x(x - 2)\),当 \(x<0\) 或 \(x>2\) 时,行列式大于零. }
	\end{question}
	
	\begin{question}
		{选择题}
		{行列式}
		{中等}
		{45. 当 \(x\)(\qquad)时,\(\begin{vmatrix} 3 & -1 & -x \\ 4 & -x & 0 \\ 1 & 0 & -x\end{vmatrix}<0\). 
			A. \(0<x<2\) \quad B. \(x<0\) 或 \(x>2\) \quad C. \(x=2\) 时 \quad D. \(x\neq0\) 且 \(x\neq2\)}
		{\(\begin{vmatrix} 3 & -1 & -x \\ 4 & -x & 0 \\ 1 & 0 & -x\end{vmatrix}=2x^2 - 4x = 2x(x - 2)\),当 \(0<x<2\) 时,行列式小于零. }
	\end{question}
	
	\begin{question}
		{选择题}
		{行列式}
		{中等}
		{46. \(\begin{vmatrix} a & -1 & -1 \\ 0 & 1 & 0 \\ 4 & -a & -a\end{vmatrix}=0\) 的充分必要条件是(\qquad). 
			A. \(a=\pm2\) \quad B. \(a>2\) \quad C. \(|a|<2\) \quad D. \(|a|\leq2\)}
		{\(\begin{vmatrix} a & -1 & -1 \\ 0 & 1 & 0 \\ 4 & -a & -a\end{vmatrix} = -a^2 + 4\),使其为0,则 \(a=\pm2\). }
	\end{question}
	
	\begin{question}
		{选择题}
		{行列式}
		{中等}
		{47. \(\begin{vmatrix} k-1 & 6 \\ 2 & 3k-3\end{vmatrix}<0\) 的充分必要条件是(\qquad). 
			A. \(-1<k<3\) \quad B. \(k\neq3\) \quad C. \(k\neq-1\) \quad D. \(k\neq3\) 且 \(k\neq-1\)}
		{\(\begin{vmatrix} k-1 & 6 \\ 2 & 3k-3\end{vmatrix} = 3(k^2 - 2k - 3) = 3(k - 3)(k + 1)\),当 \(-1<k<3\) 时小于0. }
	\end{question}
	
	\begin{question}
		{选择题}
		{行列式}
		{容易}
		{48. 行列式 \(\begin{vmatrix} 1 & 3 & 2 \\ -1 & -3 & -2 \\ 3 & 2 & 1\end{vmatrix}=\)(\qquad). 
			A. 0 \quad B. \(-6\) \quad C. \(-8\) \quad D. 15}
		{\(\begin{vmatrix} 1 & 3 & 2 \\ -1 & -3 & -2 \\ 3 & 2 & 1\end{vmatrix}=0\). }
	\end{question}
	
	\begin{question}
		{选择题}
		{行列式}
		{容易}
		{49. 行列式 \(\begin{vmatrix} -3 & 3 & 8 \\ -1 & 1 & 9 \\ -4 & 4 & 5\end{vmatrix}=\)(\qquad). 
			A. 0 \quad B. \(-6\) \quad C. \(-5\) \quad D. 15}
		{\(\begin{vmatrix} -3 & 3 & 8 \\ -1 & 1 & 9 \\ -4 & 4 & 5\end{vmatrix}=0\). }
	\end{question}
	
	\begin{question}
		{选择题}
		{行列式}
		{中等}
		{50. 方程 \(\begin{vmatrix} x & 3 & 4 \\ -1 & x & 0 \\ 0 & x & 1\end{vmatrix}=0\) 的根是(\qquad). 
			A. \(x_1=1, x_2=3\) \quad B. \(x_1=2, x_2=3\) \quad C. \(x_1=0, x_2=1\) \quad D. \(x_1=-1, x_2=1\)}
		{\(\begin{vmatrix} x & 3 & 4 \\ -1 & x & 0 \\ 0 & x & 1\end{vmatrix} = x^2 - 4x + 3 = (x - 1)(x - 3)\),所以 \(x_1=1, x_2=3\). }
	\end{question}
	
	
	\begin{question}
		{选择题}
		{行列式}
		{容易}
		{39. 若 \(a+b+c=0\),则行列式 \(\begin{vmatrix} a & b & c \\ c & a & b \\ b & c & a\end{vmatrix}=\)(\qquad).A. \(3abc\) B. \(a^3+b^3+c^3\) C. 0 D. \(ab+bc+ca\)}
		{由对称性和行列式恒等变换可知,当 \(a+b+c=0\) 时,该行列式值恒为 0.}
	\end{question}
	
	\begin{question}
		{选择题}
		{行列式}
		{容易}
		{40. 若 \(a \neq 0\),则行列式 \(\begin{vmatrix} a & 0 & 0 \\ b & a & 0 \\ c & d & a\end{vmatrix}=\)(\qquad).A. \(a^3\) B. \(abc\) C. \(a^2\) D. 0}
		{该行列式是一个上三角行列式,对角线元素乘积即为 \(a^3\).}
	\end{question}
	
	\begin{question}
		{选择题}
		{行列式}
		{容易}
		{41. 行列式 \(\begin{vmatrix} 1 & 1 & 1 \\ x & y & z \\ x^2 & y^2 & z^2\end{vmatrix}\) 的值等于(\qquad).A. \((x-y)(y-z)(z-x)\) B. \((x+y+z)(x-y)(y-z)\) C. \((x-y)(y-z)(z-x)\) D. \((x-y)(y-z)(x-z)\)}
		{该行列式为范德蒙行列式,结果为 \((x - y)(y - z)(z - x)\).}
	\end{question}
	
	\begin{question}
		{选择题}
		{行列式}
		{容易}
		{42. 设 \(A=\begin{vmatrix} 1 & 1 & 1 \\ 1 & 2 & 3 \\ 1 & 3 & 6\end{vmatrix}\),则 \(A=\)(\qquad).A. 0 B. 1 C. 2 D. 3}
		{利用行列式展开或初等变换,得 \(A=1\).}
	\end{question}
	
	\begin{question}
		{选择题}
		{行列式}
		{容易}
		{43. 若行列式 \(\begin{vmatrix} 1 & x & x^2 \\ 1 & y & y^2 \\ 1 & z & z^2\end{vmatrix}=0\),则(\qquad).A. \(x+y+z=0\) B. \(x^2+y^2+z^2=0\) C. \(x=y=z\) D. \(x,y,z\) 中有相等者}
		{该行为范德蒙行列式,为 0 时说明 \(x,y,z\) 至少有两个相等.}
	\end{question}
	
	\begin{question}
		{选择题}
		{行列式}
		{容易}
		{44. 行列式 \(\begin{vmatrix} 1 & 2 & 3 \\ 0 & 1 & 4 \\ 0 & 0 & 1\end{vmatrix}\) 的值是(\qquad).A. 1 B. 2 C. 3 D. 4}
		{该为上三角行列式,对角线元素相乘得 \(1 \cdot 1 \cdot 1 = 1\).}
	\end{question}
	
	\begin{question}
		{选择题}
		{行列式}
		{容易}
		{45. 行列式 \(\begin{vmatrix} a & 1 & 0 \\ 0 & a & 1 \\ 0 & 0 & a\end{vmatrix}\) 的值为(\qquad).A. \(a^3\) B. \(3a\) C. \(a^2\) D. \(a\)}
		{该为上三角行列式,行列式值为对角线元素乘积 \(a^3\).}
	\end{question}
	
	\begin{question}
		{选择题}
		{行列式}
		{容易}
		{46. 若行列式 \(\begin{vmatrix} 1 & 2 & 3 \\ a & b & c \\ 4 & 5 & 6\end{vmatrix}=0\),则(\qquad).A. 向量 \((1,2,3)\) 和 \((a,b,c)\) 线性相关 B. 三行线性无关 C. 三列不相关 D. 行列式不为零}
		{由行列式为 0 可知三行线性相关,故向量 \((1,2,3)\) 与 \((a,b,c)\) 线性相关.}
	\end{question}
	
	\begin{question}
		{选择题}
		{行列式}
		{容易}
		{47. 行列式 \(\begin{vmatrix} 1 & 0 & 0 \\ 0 & \cos\theta & -\sin\theta \\ 0 & \sin\theta & \cos\theta\end{vmatrix}\) 的值为(\qquad).A. \(1\) B. \(0\) C. \(-1\) D. \(\cos\theta\)}
		{该行为旋转矩阵,对角线乘积为 \(\cos^2\theta + \sin^2\theta = 1\),因此行列式为 \(1\).}
	\end{question}
	
	\begin{question}
		{选择题}
		{行列式}
		{容易}
		{48. 行列式 \(\begin{vmatrix} 0 & 1 & 2 \\ 1 & 0 & 3 \\ 4 & -3 & 8\end{vmatrix}\) 的值是(\qquad).A. 0 B. 1 C. -1 D. 5}
		{使用展开计算,结果为 0.}
	\end{question}
	
	\begin{question}
		{选择题}
		{行列式}
		{容易}
		{49. 若 \(A\) 是一个三阶行列式,且 \(A\) 的第一行元素都乘以 \(3\),则新行列式的值是原来的(\qquad)倍.A. 3 B. 6 C. 9 D. 27}
		{行列式某一行乘以常数 \(k\),整体乘以 \(k\),所以是原来值的 3 倍.}
	\end{question}
	
	\begin{question}
		{选择题}
		{行列式}
		{容易}
		{50. 行列式 \(\begin{vmatrix} 1 & 2 & 3 \\ 2 & 4 & 6 \\ 1 & 1 & 1\end{vmatrix}=\)(\qquad).A. 0 B. 1 C. 2 D. 3}
		{第二行是第一行的两倍,因此行列式为 0.}
	\end{question}
	
	
	
	
	
	
	
	
	
	
	
	
	
	
	
	
	
	
	
	
	
	\begin{question}
		{判断题}
		{行列式}
		{容易}
		{1. \(\begin{vmatrix} x & y \\ -2 x & -2 y\end{vmatrix}=0\)().}
		{\(\begin{vmatrix} x & y \\ -2 x & -2 y\end{vmatrix}=0\) .由此知结论正确.}
	\end{question}
	
	\begin{question}
		{判断题}
		{行列式}
		{容易}
		{2. \(\begin{vmatrix} a & b \\ 2 a & 2 b\end{vmatrix}=0\)(). }
		{\(\begin{vmatrix} a & b \\ 2 a & 2 b\end{vmatrix}=0\) .由此知结论正确.}
	\end{question}
	
	\begin{question}
		{判断题}
		{行列式}
		{容易}
		{3. 设 \(\begin{vmatrix} 1 & 2 & 0 \\ 2 & 3 & 0 \\ 0 & 0 & -x\end{vmatrix}=0\),则 \(x=0\)(). }
		{\(\begin{vmatrix} 1 & 2 & 0 \\ 2 & 3 & 0 \\ 0 & 0 & -x\end{vmatrix}=1 \cdot 3 \cdot(-x) - 2 \cdot 2 \cdot(-x)=x=0\).}
	\end{question}
	
	\begin{question}
		{判断题}
		{行列式}
		{容易}
		{4. 设 \(\begin{vmatrix} 1 & 2 & 3 \\ x & x & 0 \\ 2 & 4 & 2\end{vmatrix}=0\),则 \(x=0\)(). }
		{\(\begin{vmatrix} 1 & 2 & 3 \\ x & x & 0 \\ 2 & 4 & 2\end{vmatrix}=1 \cdot x \cdot 2 + x \cdot 4 \cdot 3 - 3 \cdot x \cdot 2 - 2 \cdot x = 0\),可得 \(x=0\).}
	\end{question}
	
	\begin{question}
		{判断题}
		{行列式}
		{容易}
		{5. \(\begin{vmatrix} 2 & 0 & 1 \\ 1 & -4 & -1 \\ -1 & 8 & 3\end{vmatrix}=0\)(). }
		{\(\begin{vmatrix} 2 & 0 & 1 \\ 1 & -4 & -1 \\ -1 & 8 & 3\end{vmatrix} = -24 + 8 + 6 - 4 = -14 \ne 0\),所以结论错误.}
	\end{question}
	
	\begin{question}
		{判断题}
		{行列式}
		{容易}
		{6. 设 \(\begin{vmatrix} 1 & 0 & 2 \\ 2 & x & 2x \\ 3 & 0 & 3\end{vmatrix}=0\),则 \(x=0\)(). }
		{\(\begin{vmatrix} 1 & 0 & 2 \\ 2 & x & 2x \\ 3 & 0 & 3\end{vmatrix}=1 \cdot x \cdot 3 - 3 \cdot x \cdot 2 = -3x=0 \Rightarrow x=0\).}
	\end{question}
	
	\begin{question}
		{判断题}
		{行列式}
		{容易}
		{7. 设 \(\begin{vmatrix} 1 & x & 2 \\ 2 & x & 4 \\ 3 & 0 & 3\end{vmatrix}=0\),则 \(x=0\)(). }
		{\(\begin{vmatrix} 1 & x & 2 \\ 2 & x & 4 \\ 3 & 0 & 3\end{vmatrix} = 1 \cdot x \cdot 3 + x \cdot 4 \cdot 3 - 3 \cdot x \cdot 2 - 3 \cdot x = 0\),可得 \(x=0\).}
	\end{question}
	
	\begin{question}
		{判断题}
		{行列式}
		{容易}
		{8. \(\begin{vmatrix} 1 & 1 & 1 \\ 1 & 2 & 3 \\ 1 & 4 & 9\end{vmatrix}=2\)(). }
		{\(\begin{vmatrix} 1 & 1 & 1 \\ 1 & 2 & 3 \\ 1 & 4 & 9\end{vmatrix} = (2-1)(3-1)(4-2) = 2\).}
	\end{question}
	
	\begin{question}
		{判断题}
		{行列式}
		{容易}
		{9. \(\begin{vmatrix} 1 & 1 & 1 \\ 1 & 2 & 4 \\ 1 & 3 & 9\end{vmatrix}=2\)(). }
		{\(\begin{vmatrix} 1 & 1 & 1 \\ 1 & 2 & 4 \\ 1 & 3 & 9\end{vmatrix} = 2\).}
	\end{question}
	
	\begin{question}
		{判断题}
		{行列式}
		{容易}
		{10. \(\begin{vmatrix} 2 & 1 & 4 \\ 0 & -4 & 0 \\ 1 & -1 & 2\end{vmatrix}=0\)(). }
		{\(\begin{vmatrix} 2 & 1 & 4 \\ 0 & -4 & 0 \\ 1 & -1 & 2\end{vmatrix} = 2 \cdot (-4) \cdot 2 + 1 \cdot 0 \cdot 1 + 4 \cdot 0 \cdot 1 - 4 \cdot (-1) \cdot 2 - 2 \cdot 0 \cdot 2 - 1 \cdot (-4) \cdot 1 = -16 + 8 + 4 = -4\),不为 0,故结论错误.}
	\end{question}
	
	
	
	
	
	
	
	
	
	
	
	
	
	
	
	
	
	
	\begin{question}
		{填空题}
		{行列式}
		{容易}
		{1.行列式 \(\begin{vmatrix} 1 & 1 & 1 \\ 6 & 6 & 6 \\ a & b & c\end{vmatrix}=()\) .}
		{\(\begin{vmatrix} 1 & 1 & 1 \\ 6 & 6 & 6 \\ a & b & c\end{vmatrix}=0\).}
	\end{question}
	
	\begin{question}
		{填空题}
		{行列式}
		{容易}
		{2. 行列式 \(\begin{vmatrix} 1 & 1 & 1 \\ 0 & 2 & -2 \\ 0 & 0 & 5\end{vmatrix}=\)(\qquad). }
		{\(\begin{vmatrix} 1 & 1 & 1 \\ 0 & 2 & -2 \\ 0 & 0 & 5\end{vmatrix}=10\).}
	\end{question}
	
	\begin{question}
		{填空题}
		{行列式}
		{容易}
		{3. 行列式 \(\begin{vmatrix} 2 & 1 & 1 \\ 0 & 5 & -2 \\ 0 & 0 & 3\end{vmatrix}=\)(\qquad). }
		{\(\begin{vmatrix} 2 & 1 & 1 \\ 0 & 5 & -2 \\ 0 & 0 & 3\end{vmatrix}=30\).}
	\end{question}
	
	\begin{question}
		{填空题}
		{行列式}
		{容易}
		{4. 行列式 \(\begin{vmatrix} 3 & 1 & 1 \\ 0 & 4 & -2 \\ 0 & 0 & 3\end{vmatrix}=\)(\qquad). }
		{\(\begin{vmatrix} 3 & 1 & 1 \\ 0 & 4 & -2 \\ 0 & 0 & 3\end{vmatrix}=36\).}
	\end{question}
	
	\begin{question}
		{填空题}
		{行列式}
		{容易}
		{5. 行列式 \(\begin{vmatrix} 12 & -2 \\ 0 & 2\end{vmatrix}\) 等于(\qquad). }
		{\(\begin{vmatrix} 12 & -2 \\ 0 & 2\end{vmatrix}=24\).}
	\end{question}
	
	\begin{question}
		{填空题}
		{行列式}
		{容易}
		{6. 行列式 \(\begin{vmatrix} 3 & 12 \\ 1 & 0\end{vmatrix}\) 等于(\qquad). }
		{\(\begin{vmatrix} 3 & 12 \\ 1 & 0\end{vmatrix}=-12\).}
	\end{question}
	
	\begin{question}
		{填空题}
		{行列式}
		{容易}
		{7. 行列式 \(\begin{vmatrix} 1 & 1 & -2 \\ -1 & 0 & 1 \\ 2 & 1 & 2\end{vmatrix}=\)(\qquad). }
		{\(\begin{vmatrix} 1 & 1 & -2 \\ -1 & 0 & 1 \\ 2 & 1 & 2\end{vmatrix}=5\).}
	\end{question}
	
	\begin{question}
		{填空题}
		{行列式}
		{容易}
		{8. 行列式 \(\begin{vmatrix} 1 & -1 & 1 \\ 1 & 2 & -1 \\ 1 & 1 & 2\end{vmatrix}=\)(\qquad). }
		{\(\begin{vmatrix} 1 & -1 & 1 \\ 1 & 2 & -1 \\ 1 & 1 & 2\end{vmatrix}=7\).}
	\end{question}
	
	\begin{question}
		{填空题}
		{行列式}
		{容易}
		{9. 行列式 \(\begin{vmatrix} 1 & 1 & -2 \\ -1 & -1 & 1 \\ 2 & 1 & 0\end{vmatrix}=\)(\qquad). }
		{\(\begin{vmatrix} 1 & 1 & -2 \\ -1 & -1 & 1 \\ 2 & 1 & 0\end{vmatrix}=-1\).}
	\end{question}
	
	\begin{question}
		{填空题}
		{行列式}
		{容易}
		{10. 行列式 \(\begin{vmatrix} 0 & 1 & -2 \\ -1 & 0 & 1 \\ 2 & 1 & 0\end{vmatrix}=\)(\qquad). }
		{\(\begin{vmatrix} 0 & 1 & -2 \\ -1 & 0 & 1 \\ 2 & 1 & 0\end{vmatrix}=4\).}
	\end{question}
	
	\begin{question}
		{填空题}
		{行列式}
		{容易}
		{11. 设 \(\begin{vmatrix} 1 & 1 & 1 \\ 1 & 1-x & 1 \\ 1 & 1 & 2-x\end{vmatrix} \neq 0\) . 则 \(x\) 应满足条件(\qquad). }
		{\(x \neq 0\) 且 \(x \neq 1\).}
	\end{question}
	
	\begin{question}
		{填空题}
		{行列式}
		{容易}
		{12. 设 \(\begin{vmatrix} 1 & 1 & 1 \\ 1 & 1-x & 1 \\ 1 & 1 & 2-x\end{vmatrix}=0\) . 则 \(x=\)(\qquad). }
		{\(x=0\) 或 \(x=1\).}
	\end{question}
	
	\begin{question}
		{填空题}
		{行列式}
		{容易}
		{13. 设 \(\begin{vmatrix} 1 & 1 & 1 \\ 1 & 1-x & 1 \\ 1 & 1 & 2-x\end{vmatrix}>0\) . 则 \(x\) 满足(\qquad). }
		{\(x>1\) 或 \(x<0\).}
	\end{question}
	
	\begin{question}
		{填空题}
		{行列式}
		{容易}
		{14. 已知行列式 \(\begin{vmatrix} a & 2 & 1 \\ 2 & 3 & 0 \\ 1 & -1 & 1\end{vmatrix}=0\) . 则数 \(a=\)(\qquad). }
		{\(a=3\).}
	\end{question}
	
	\begin{question}
		{填空题}
		{行列式}
		{容易}
		{15. 行列式 \(\begin{vmatrix} 0 & 0 & 3 \\ 0 & 1 & 0 \\ 2 & 0 & 0\end{vmatrix}\) 等于(\qquad). }
		{\(\begin{vmatrix} 0 & 0 & 3 \\ 0 & 1 & 0 \\ 2 & 0 & 0\end{vmatrix}=-6\).}
	\end{question}	
	
	
	
	
	
	
	
	
	
	
	
	
	
	
	
	
	\begin{question} 
		{计算题} 
		{行列式}
		{容易}
		{1. 计算行列式 \(\begin{vmatrix} 1 & 2 & 3 \\ 3 & 1 & 2 \\ 2 & 3 & 1\end{vmatrix}\)().}
		{\(\begin{vmatrix} 1 & 2 & 3 \\ 3 & 1 & 2 \\ 2 & 3 & 1\end{vmatrix}=1 \times 1 \times 1+2 \times 2 \times 2+3 \times 3 \times 3-\) \(3 \times 1 \times 2-2 \times 3 \times 1-1 \times 2 \times 3=18\) .}
	\end{question}
	
	\begin{question} 
		{计算题} 
		{行列式}
		{容易}
		{2. 计算行列式 \(\begin{vmatrix} 1 & 2 & 5 \\ 3 & 4 & 5 \\ 0 & 0 & 4\end{vmatrix}\)().}
		{\(\begin{vmatrix} 1 & 2 & 5 \\ 3 & 4 & 5 \\ 0 & 0 & 4\end{vmatrix}=1 \times 4 \times 4+5 \times 3 \times 0+2 \times 5 \times 0-5 \times 4 \times 0-2 \times 3 \times 4-1 \times 5 \times 0=16-24=-8\).}
	\end{question}
	
	\begin{question} 
		{计算题} 
		{行列式}
		{容易}
		{3. 当 \(k\) 取何值时,\(\begin{vmatrix} k & 3 & 4 \\ -1 & k & 0 \\ 0 & k & 1\end{vmatrix}=0\)(\qquad).}
		{\(\begin{vmatrix} k & 3 & 4 \\ -1 & k & 0 \\ 0 & k & 1\end{vmatrix}=k^2-4k+3=(k-1)(k-3)\),当 \(k=1\) 或 \(k=3\) 时成立.}
	\end{question}
	
	\begin{question} 
		{计算题} 
		{行列式}
		{容易}
		{4. 若行列式 \(\begin{vmatrix} 1 & 2 & 7 \\ 5 & 3 & 4 \\ 0 & 0 & x\end{vmatrix}=0\),求 \(x\) 的值(\qquad).}
		{\(\begin{vmatrix} 1 & 2 & 7 \\ 5 & 3 & 4 \\ 0 & 0 & x\end{vmatrix}=3x - 10x = -7x=0\),所以 \(x=0\).}
	\end{question}
	
	\begin{question} 
		{计算题} 
		{行列式}
		{容易}
		{5. 解方程 \(\begin{vmatrix} 3 & 1 & 1 \\ x & 1 & 0 \\ x^2 & 3 & 1\end{vmatrix}=0\)(\qquad).}
		{\(\begin{vmatrix} 3 & 1 & 1 \\ x & 1 & 0 \\ x^2 & 3 & 1\end{vmatrix}=(x+1)(x-3)=0\),解得 \(x=-1\) 或 \(x=3\).}
	\end{question}
	
	\begin{question} 
		{计算题} 
		{行列式}
		{容易}
		{6. 行列式 \(\begin{vmatrix} a & 1 & 2 \\ 0 & -1 & 0 \\ 4 & a & 2a\end{vmatrix}>0\) 的充分必要条件是什么?(\qquad).}
		{\(\begin{vmatrix} a & 1 & 2 \\ 0 & -1 & 0 \\ 4 & a & 2a\end{vmatrix}=2(-a^2+4)\),要使其大于0,需 \(|a|<2\).}
	\end{question}
	
	\begin{question} 
		{计算题} 
		{行列式}
		{容易}
		{7. 若行列式 \(\begin{vmatrix} 1 & 2 & 9 \\ 5 & 6 & 4 \\ 0 & 0 & x\end{vmatrix}=0\),求 \(x\) 的值(\qquad).}
		{\(\begin{vmatrix} 1 & 2 & 9 \\ 5 & 6 & 4 \\ 0 & 0 & x\end{vmatrix}=6x - 10x = -4x=0\),所以 \(x=0\).}
	\end{question}
	
	\begin{question} 
		{计算题} 
		{行列式}
		{容易}
		{8. 计算行列式 \(D=\begin{vmatrix} a_0 & -1 & 0 \\ a_1 & x & -1 \\ a_2 & 0 & x\end{vmatrix}\)(\qquad).}
		{\(D = a_0 x^2 + a_1 x + a_2\).}
	\end{question}
	
	\begin{question} 
		{计算题} 
		{行列式}
		{容易}
		{9. 当 \(k\) 满足什么条件时,\(\begin{vmatrix} k & 3 & 4 \\ -2 & 2k & 0 \\ 0 & 3k & 3\end{vmatrix} \neq 0\)().}
		{\(\begin{vmatrix} k & 3 & 4 \\ -2 & 2k & 0 \\ 0 & 3k & 3\end{vmatrix}=6(k-1)(k-3)\),当 \(k \neq 1\) 且 \(k \neq 3\) 时,行列式不等于零.}
	\end{question}
	
	\begin{question} 
		{计算题} 
		{行列式}
		{容易}
		{10. 若行列式 \(\begin{vmatrix} 1 & -2 & 9 \\ 5 & 3 & -4 \\ 0 & 0 & x\end{vmatrix}=0\),求 \(x\) 的值(\qquad).}
		{\(\begin{vmatrix} 1 & -2 & 9 \\ 5 & 3 & -4 \\ 0 & 0 & x\end{vmatrix} = 3x + 10x = 13x = 0\),所以 \(x=0\).}
	\end{question}
	
	\begin{question} 
		{计算题} 
		{行列式}
		{中等}
		{11. 行列式 \(\begin{vmatrix} a & 1 & 1 \\ 0 & -1 & 0 \\ 4 & a & a\end{vmatrix} \neq 0\) 的充分必要条件是(). }
		{\(\begin{vmatrix} a & 1 & 1 \\ 0 & -1 & 0 \\ 4 & a & a\end{vmatrix} = -a^2 + 4\),所以充分必要条件是 \(a \neq \pm2\). }
	\end{question}
	
	\begin{question} 
		{计算题} 
		{行列式}
		{中等}
		{12. 解方程 \(\begin{vmatrix} 3 & 1 & 1 \\ -x & -1 & 0 \\ -x^2 & -3 & -1\end{vmatrix}=0\)(). }
		{\(\begin{vmatrix} 3 & 1 & 1 \\ -x & -1 & 0 \\ -x^2 & -3 & -1\end{vmatrix} = (x+1)(x-3) = 0\),所以 \(x = -1\) 或 \(x = 3\). }
	\end{question}
	
	\begin{question} 
		{计算题} 
		{行列式}
		{中等}
		{13. 行列式 \(\begin{vmatrix} a & 1 & 1 \\ 0 & -1 & 0 \\ 4 & a & a\end{vmatrix}<0\) 的充分必要条件是(). }
		{\(\begin{vmatrix} a & 1 & 1 \\ 0 & -1 & 0 \\ 4 & a & a\end{vmatrix} = -a^2 + 4 < 0\),所以 \(|a| > 2\). }
	\end{question}
	
	\begin{question} 
		{计算题} 
		{行列式}
		{容易}
		{14. 计算行列式 \(\begin{vmatrix} -a & 0 & 1 \\ 0 & -a & 0 \\ 1 & 0 & -a\end{vmatrix}\)(). }
		{\(\begin{vmatrix} -a & 0 & 1 \\ 0 & -a & 0 \\ 1 & 0 & -a\end{vmatrix} = -a^3 + a\). }
	\end{question}
	
	\begin{question} 
		{计算题} 
		{行列式}
		{容易}
		{15. 若行列式 \(\begin{vmatrix} 1 & -2 & 9 \\ 5 & 6 & -4 \\ 0 & 0 & x\end{vmatrix} = 0\),求 \(x\) 的值(). }
		{\(\begin{vmatrix} 1 & -2 & 9 \\ 5 & 6 & -4 \\ 0 & 0 & x\end{vmatrix} = 16x = 0\),所以 \(x = 0\). }
	\end{question}
	
	\begin{question} 
		{计算题} 
		{行列式}
		{中等}
		{16. 当 \(k\) 满足什么条件时,\(\begin{vmatrix} k & -3 & -4 \\ -1 & -k & 0 \\ 0 & -k & -1\end{vmatrix}>0\)(). }
		{\(\begin{vmatrix} k & -3 & -4 \\ -1 & -k & 0 \\ 0 & -k & -1\end{vmatrix} = (k - 1)(k - 3)\),所以当 \(k < 1\) 或 \(k > 3\) 时大于零. }
	\end{question}
	
	\begin{question} 
		{计算题} 
		{行列式}
		{中等}
		{17. 当 \(k\) 满足什么条件时,\(\begin{vmatrix} k & 3 & 4 \\ 1 & -k & 0 \\ 0 & -k & -1\end{vmatrix}<0\)(). }
		{\(\begin{vmatrix} k & 3 & 4 \\ 1 & -k & 0 \\ 0 & -k & -1\end{vmatrix} = (k - 1)(k - 3)\),所以当 \(1 < k < 3\) 时小于零. }
	\end{question}
	
	\begin{question} 
		{计算题} 
		{行列式}
		{中等}
		{18. 当 \(x\) 取何值时,行列式 \(\begin{vmatrix} 3 & -1 & -x \\ 4 & -x & 0 \\ 1 & 0 & -x\end{vmatrix}>0\)(). }
		{\(\begin{vmatrix} 3 & -1 & -x \\ 4 & -x & 0 \\ 1 & 0 & -x\end{vmatrix} = 2x(x - 2)\),所以当 \(x < 0\) 或 \(x > 2\) 时大于零. }
	\end{question}
	
	\begin{question} 
		{计算题} 
		{行列式}
		{中等}
		{19. 当 \(x\) 取何值时,行列式 \(\begin{vmatrix} 3 & 2 & 3x \\ 4 & 2x & 0 \\ 1 & 0 & 3x\end{vmatrix}<0\)(). }
		{\(\begin{vmatrix} 3 & 2 & 3x \\ 4 & 2x & 0 \\ 1 & 0 & 3x\end{vmatrix} = 12x(x - 2)\),所以当 \(0 < x < 2\) 时小于零. }
	\end{question}
	
	\begin{question} 
		{计算题} 
		{行列式}
		{中等}
		{20. 行列式 \(\begin{vmatrix} a & 1 & 3 \\ 0 & -1 & 0 \\ 4 & a & 3a\end{vmatrix}=0\) 的充分必要条件是(). }
		{\(\begin{vmatrix} a & 1 & 3 \\ 0 & -1 & 0 \\ 4 & a & 3a\end{vmatrix} = 3(-a^2 + 4)\),所以充分必要条件是 \(a = \pm2\). }
	\end{question}
	
	\begin{question} 
		{计算题} 
		{行列式}
		{中等}
		{21. 计算行列式 \(\begin{vmatrix} 1 & 1 & 1 \\ a & b & c \\ a^2 & b^2 & c^2\end{vmatrix}\)(). }
		{该行列式为范德蒙行列式,结果为 \((b - a)(c - a)(c - b)\). }
	\end{question}
	
	\begin{question} 
		{计算题} 
		{行列式}
		{中等}
		{22. 求解方程 \(\begin{vmatrix} x & 3 & 4 \\ -1 & x & 0 \\ 0 & x & 1\end{vmatrix} = 0\)(). }
		{\(x^2 - 4x + 3 = 0\),解得 \(x = 1\) 或 \(x = 3\). }
	\end{question}
	
	\begin{question} 
		{计算题} 
		{行列式}
		{中等}
		{23. 求解方程 \(\begin{vmatrix} 3 & 1 & x \\ 4 & x & 0 \\ 1 & 0 & x\end{vmatrix}=0\)(). }
		{展开计算得:\(3x^2 - x^2 - 4x = 0 \Rightarrow 2x^2 - 4x = 0\),所以 \(x = 0\) 或 \(x = 2\). }
	\end{question}
	
	\begin{question} 
		{计算题} 
		{行列式}
		{中等}
		{24. 计算行列式 \(\begin{vmatrix} 1+a & 1 & 1 \\ 1 & 1+b & 1 \\ 1 & 1 & 1+c\end{vmatrix}\)(). }
		{利用对角线展开法得:\(ab + bc + ca + abc\). }
	\end{question}
	
	\begin{question} 
		{计算题} 
		{行列式}
		{中等}
		{25. 设 \(f(x)=\begin{vmatrix} x & 0 & 1 \\ 1 & x & 1 \\ 3 & 2x & 1\end{vmatrix}\),求 \(f(x)\) 的最高次项的系数(). }
		{计算得 \(f(x) = -x^2 - x\),所以最高次项系数为 \(-1\). }
	\end{question}
	
	\begin{question} 
		{计算题} 
		{行列式}
		{容易}
		{26. 解不等式 \(\begin{vmatrix} 6 & 2 & 2 \\ -x & -1 & 0 \\ -x^2 & -3 & -1\end{vmatrix}>0\)(\qquad).}
		{\(\begin{vmatrix} 6 & 2 & 2 \\ -x & -1 & 0 \\ -x^2 & -3 & -1\end{vmatrix} = 2(x+1)(x-3)\),所以当 \(x<-1\) 或 \(x>3\) 时,行列式大于零.}
	\end{question}
	
	\begin{question} 
		{计算题} 
		{行列式}
		{容易}
		{27. 解不等式 \(\begin{vmatrix} 3 & 1 & 2 \\ -x & -1 & 0 \\ -x^2 & -3 & -2\end{vmatrix}<0\)(\qquad).}
		{\(\begin{vmatrix} 3 & 1 & 2 \\ -x & -1 & 0 \\ -x^2 & -3 & -2\end{vmatrix} = -2(x+1)(x-3)\),所以当 \(x<-1\) 或 \(x>3\) 时,行列式小于零.}
	\end{question}
	

	
	\begin{question}
		{证明题}
		{可逆矩阵}
		{困难}
		{证明:若 \( A \) 可逆,则 \( A^T \) 也可逆,并且 \( (A^T)^{-1} = (A^{-1})^T \). }
		{因为 \( A A^{-1} = I \),两边取转置得 \( (A^{-1})^T A^T = I \),说明 \( A^T \) 可逆,逆为 \( (A^{-1})^T \). }
	\end{question}








%%%%%%%%2025.5.12%%%%%%%%%%%%%%%%%%%%%%%%%%%%%%%%%%%


	
	\begin{question}
		{选择题}
		{行列式}
		{容易}
		{1.排列 41253 的逆序数为(  ).A. 4 B. 0 C. 5 D. 3}
		{41253 所含逆序为 \(41,42,43,53\) ,所以 41253 的逆序数 \(N(41253)=4\).}
	\end{question}
	
	\begin{question}
		{选择题}
		{行列式}
		{容易}
		{2.排列 3712456 的逆序数为(  ).A. 7 B. 6 C. 5 D. 10}
		{3712456 所含逆序为 \(31,71,32,72,74,75,76\) ,所以 3712456 的逆序数 \(N(3712456)=7\).}
	\end{question}
	
	\begin{question}
		{选择题}
		{行列式}
		{容易}
		{3.排列 36715284 的逆序数为(  ).A. 13 B. 9 C. 12 D. 10}
		{36715284 的逆序数为 \(N(36715284)=2+4+4+0+2+0+1=13\).}
	\end{question}
	
	\begin{question}
		{选择题}
		{行列式}
		{容易}
		{4.排列 654321 的逆序数为(  ).A. 15 B. 9 C. 12 D. 11}
		{654321 的逆序数为 \(N(654321)=15\).}
	\end{question}
	
	\begin{question}
		{选择题}
		{行列式}
		{容易}
		{5.排列 54321 的逆序数为(  ).A. 10 B. 9 C. 11 D. 12}
		{54321 的逆序数为 \(N(54321)=10\).}
	\end{question}
	
	\begin{question}
		{选择题}
		{行列式}
		{容易}
		{6.排列 42153 的逆序数为(  ).A. 5 B. 0 C. 4 D. 3}
		{42153 的逆序数为 \(N(42153)=5\).}
	\end{question}
	
	\begin{question}
		{选择题}
		{行列式}
		{容易}
		{7.排列 42153 的逆序数为(  ).A. 5 B. 0 C. 4 D. 3}
		{42153 的逆序数为 \(N(42153)=5\).}
	\end{question}
	
	\begin{question}
		{选择题}
		{行列式}
		{容易}
		{8.排列 13725468 的逆序数为(  ).A. 6 B. 5 C. 7 D. 8}
		{13725468 的逆序数为 \(N(13725468)=6\).}
	\end{question}
	
	\begin{question}
		{选择题}
		{行列式}
		{容易}
		{9.排列 361524 的逆序数为(  ).A. 8 B. 9 C. 11 D. 10}
		{361524 的逆序数为 \(N(361524)=8\).}
	\end{question}
	
	\begin{question}
		{选择题}
		{行列式}
		{容易}
		{10.排列 634512 的逆序数为(  ).A. 11 B. 9 C. 12 D. 10}
		{634512 的逆序数为 \(N(634512)=11\).}
	\end{question}
	
	
	\begin{question}
		{选择题}
		{行列式}
		{容易}
		{11.排列 \(n(n-1)(n-2)\cdots321\) 的逆序数为(\qquad). A. \(\frac{n(n-1)}{2}\) B. \(n\) C. \(n-1\) D. 不确定}
		{\(n(n-1)(n-2)\cdots321\) 的逆序数为 \((n-1)+(n-2)+\cdots+2+1=\frac{n(n-1)}{2}\).}
	\end{question}
	
	\begin{question}
		{选择题}
		{行列式}
		{容易}
		{12.下列排列是偶排列的是(). A. 12345 B. 53214 C. 654321 D. 32145}
		{\(\tau(12345)=0,\ \tau(53214)=7,\ \tau(654321)=15,\ \tau(32145)=3\),所以只有 12345 是偶排列. }
	\end{question}
	
	\begin{question}
		{选择题}
		{行列式}
		{中等}
		{13.计算行列式 \(\begin{matrix}0 & 1 & 0 & \cdots & 0 \\ 0 & 0 & 2 & \cdots & 0 \\ \vdots & \vdots & \vdots & & \vdots \\ 0 & 0 & 0 & \cdots & n-1 \\ n & 0 & 0 & \cdots & 0\end{matrix}\) 的值是(\qquad). A. \((-1)^{n-1}n!\) B. \(n\) C. \(n(n+1)\) D. \(n(n-1)\)}
		{该行列式的非零项只有 \(a_{12}a_{23}\cdots a_{n-1,n}a_{n1}\),其逆序数为 \(n-1\),所以行列式值为 \((-1)^{n-1}\times1\times2\times\cdots\times n = (-1)^{n-1}n!\). }
	\end{question}
	
	\begin{question}
		{选择题}
		{行列式}
		{中等}
		{14.计算行列式 \(\begin{matrix}0 & 0 & 1 & 0 \\ 0 & 1 & 0 & 0 \\ 0 & 0 & 0 & 1 \\ -1 & 0 & 0 & 0\end{matrix}\) 的值是(\qquad). A. -1 B. 1 C. 2 D. 0}
		{仅有非零项为 \(a_{13}a_{22}a_{34}a_{41}\),其对应排列为 \(3241\),逆序数为4,故带正号;乘积为 \(1\times1\times1\times(-1) = -1\). }
	\end{question}
	
	\begin{question}
		{选择题}
		{行列式}
		{容易}
		{15.判断 4 阶行列式中 \(a_{11}a_{33}a_{44}a_{22}\) 和 \(a_{24}a_{31}a_{13}a_{42}\) 的符号分别为(\qquad). A. 正,正 B. 正,负 C. 负,正 D. 负,负}
		{\(a_{11}a_{33}a_{44}a_{22}\) 实为 \(a_{11}a_{22}a_{33}a_{44}\),为正;\(a_{24}a_{31}a_{13}a_{42}\) 的列标为 3412,为偶排列,故也为正. }
	\end{question}
	
	\begin{question}
		{选择题}
		{行列式}
		{中等}
		{16.行列式 \(\begin{matrix}x & x & 1 & 0 \\ 1 & x & 2 & 3 \\ 3 & 2 & x & 2 \\ 1 & 1 & 2 & x\end{matrix}\) 中 \(x^4\) 的系数为(\qquad). A. 1 B. 3 C. -1 D. 2}
		{含有 \(x^4\) 的项只有 \(x\cdot x\cdot x\cdot x\) 一项,带正号,所以系数为 1. }
	\end{question}
	
	\begin{question}
		{选择题}
		{行列式}
		{中等}
		{17.行列式 \(\begin{matrix}x & 1 & 1 & 2 \\ 1 & x & 1 & -1 \\ 3 & 2 & x & 1 \\ 1 & 1 & 2x & 1\end{matrix}\) 中 \(x^3\) 的系数为(\qquad). A. -1 B. 3 C. 1 D. 2}
		{含 \(x^3\) 的项有两项,一项为 \(x\cdot x\cdot x\cdot1\),带正号;另一项为 \(x\cdot x\cdot1\cdot2x\),带负号;系数和为 \(1-2=-1\). }
	\end{question}
	
	\begin{question}
		{判断题}
		{行列式}
		{容易}
		{1.排列12543678是奇排列(  ).}
		{12543678 逆序数为 3 ,为奇排列.}
	\end{question}
	
	
	\begin{question}
		{判断题}
		{行列式}
		{容易}
		{2. \(a_{21} a_{53} a_{16} a_{42} a_{65} a_{34}\) 在六阶行列式中是带负号的项(). }
		{\(N(251463)+N(136254)=6+5=11\) 为奇数,所以 \(a_{21} a_{53} a_{16} a_{42} a_{65} a_{34}\) 前面应冠以负号.}
	\end{question}
	
	\begin{question}
		{判断题}
		{行列式}
		{容易}
		{3. \(a_{15} a_{23} a_{32} a_{44} a_{51} a_{66}\) 在六阶行列式中是带负号的项(). }
		{\(N(532416)=8\) 为偶数,所以 \(a_{15} a_{23} a_{32} a_{44} a_{51} a_{66}\) 前面应冠以正号,由此知结论错误.}
	\end{question}
	
	\begin{question}
		{判断题}
		{行列式}
		{容易}
		{4. \(a_{11} a_{26} a_{32} a_{44} a_{53} a_{65}\) 在六阶行列式中是带负号的项(). }
		{\(N(162435)=5\) 为奇数,所以 \(a_{11} a_{26} a_{32} a_{44} a_{53} a_{65}\) 前面应冠以负号.}
	\end{question}
	
	\begin{question}
		{判断题}
		{行列式}
		{容易}
		{5. \(a_{51} a_{32} a_{13} a_{44} a_{65} a_{26}\) 在六阶行列式中是带正号的项(). }
		{\(N(531462)=8\) 为偶数,所以 \(a_{51} a_{32} a_{13} a_{44} a_{65} a_{26}\) 前面应冠以正号.}
	\end{question}
	
	\begin{question}
		{判断题}
		{行列式}
		{容易}
		{6. \(n\) 阶行列式中有一行元素为零,行列式为零(). }
		{因为 \(n\) 阶行列式中有一行元素为零,则所有项均为零,因此行列式为零. }
	\end{question}
	
	\begin{question}
		{判断题}
		{行列式}
		{容易}
		{7. 排列 36715284 是偶排列(). }
		{36715284 的逆序数为 \(N(36715284)=3+4+0+4+2+0+0+0=13\) ,为奇排列.}
	\end{question}
	
	\begin{question}
		{判断题}
		{行列式}
		{容易}
		{8. \(a_{21} a_{53} a_{16} a_{42} a_{64} a_{35}\) 在六阶行列式中是带正号的项(). }
		{\(N(251436)+N(136245)=5+4=9\) 为奇数,所以 \(a_{21} a_{53} a_{16} a_{42} a_{64} a_{35}\) 前面应冠以负号.}
	\end{question}
	
	\begin{question}
		{判断题}
		{行列式}
		{容易}
		{9. 排列 31245678 是奇排列(). }
		{31245678 的逆序数为 2 ,为偶排列.}
	\end{question}
	
	\begin{question}
		{判断题}
		{行列式}
		{容易}
		{10. 排列 31765284 是奇排列(). }
		{\(N(31765284)=12\) ,为偶排列.}
	\end{question}
	
	\begin{question}
		{判断题}
		{行列式}
		{容易}
		{11. 排列 3761524 是奇排列(). }
		{\(N(3761524)=13\) ,为奇排列.}
	\end{question}
	
	\begin{question}
		{判断题}
		{行列式}
		{容易}
		{12. 排列 1234 是奇排列(). }
		{1234 没有逆序,故逆序数为 0 ,为偶排列.}
	\end{question}
	
	\begin{question}
		{判断题}
		{行列式}
		{容易}
		{13. \(a_{61} a_{52} a_{43} a_{34} a_{15} a_{26}\) 在六阶行列式中是带正号的项(). }
		{\(N(654312)=14\) 为偶数,所以 \(a_{61} a_{52} a_{43} a_{34} a_{15} a_{26}\) 前面应冠以正号.}
	\end{question}
	
	\begin{question}
		{填空题}
		{行列式}
		{容易}
		{1.排列132487695的逆序数为(  ).}
		{此排列含 8 个逆序.}
	\end{question}
	
	\begin{question}
		{填空题}
		{行列式}
		{容易}
		{2.排列132487659的逆序数为(  ).}
		{此排列含 7 个逆序.}
	\end{question}
	
	\begin{question}
		{填空题}
		{行列式}
		{容易}
		{3.排列7613542的逆序数为(  ).}
		{此排列含 15 个逆序.}
	\end{question}
	
	\begin{question}
		{填空题}
		{行列式}
		{容易}
		{4.排列1324765的逆序数为(  ).}
		{此排列含 4 个逆序.}
	\end{question}
	
	\begin{question}
		{填空题}
		{行列式}
		{容易}
		{5.排列41325的奇偶性为(  ).}
		{此排列含 4 个逆序,所以是偶排列.}
	\end{question}
	
	\begin{question}
		{填空题}
		{行列式}
		{容易}
		{6.排列76813542的逆序数为(  ).}
		{此排列含 20 个逆序.}
	\end{question}
	
	\begin{question}
		{填空题}
		{行列式}
		{容易}
		{7.排列13248765的逆序数为(  ).}
		{此排列含 7 个逆序.}
	\end{question}
	
	\begin{question}
		{填空题}
		{行列式}
		{容易}
		{8.排列453126的奇偶性为(  ).}
		{此排列含 8 个逆序,所以是偶排列.}
	\end{question}
	
	\begin{question}
		{填空题}
		{行列式}
		{容易}
		{9.排列613542的逆序数为(  ).}
		{此排列含 9 个逆序.}
	\end{question}
	
	\begin{question}
		{填空题}
		{行列式}
		{容易}
		{10.排列132465的逆序数为(  ).}
		{此排列含 2 个逆序.}
	\end{question}
	
	
	\begin{question}
		{填空题}
		{行列式}
		{容易}
		{11.排列7613542的逆序数为(  ).}
		{此排列含 15 个逆序.}
	\end{question}
	
	\begin{question}
		{填空题}
		{行列式}
		{中等}
		{12.计算行列式 \(D=\begin{vmatrix}a & a & a & 1 \\ a & a & 1 & 0 \\ a & 1 & 0 & 0 \\ 1 & 0 & 0 & 0\end{vmatrix}=\)(  ).}
		{\(D\) 只含一项非零项 \(1 \cdot 1 \cdot 1 \cdot 1\),符号由 \(N(4321)=6\) 确定,带正号,所以 \(D=1\).}
	\end{question}
	
	\begin{question}
		{填空题}
		{行列式}
		{中等}
		{13.求行列式 \(D=\begin{vmatrix}a & a & a & 2 \\ a & a & -2 & 0 \\ a & 2 & 0 & 0 \\ 2 & 0 & 0 & 0\end{vmatrix}=\)(  ).}
		{\(D\) 只含一项非零项 \(2 \cdot (-2) \cdot 2 \cdot a\),符号由 \(N(4321)=6\) 确定,带正号,所以 \(D=-16\).}
	\end{question}
	
	\begin{question}
		{填空题}
		{行列式}
		{中等}
		{14.计算行列式 \(D=\begin{vmatrix}0 & a_{12} & 0 & 0 \\ 0 & 0 & 0 & a_{24} \\ a_{31} & 0 & 0 & 0 \\ 0 & 0 & a_{43} & 0\end{vmatrix}=\)(  ).}
		{只含一项非零项 \(a_{12} a_{24} a_{31} a_{43}\),符号由 \(N(2413)=3\),带负号,所以 \(D=-a_{12} a_{24} a_{31} a_{43}\).}
	\end{question}
	
	\begin{question}
		{填空题}
		{行列式}
		{中等}
		{15.计算行列式 \(D=\begin{vmatrix}0 & 1 & 0 & 0 \\ 0 & 0 & 0 & 2 \\ 3 & 0 & 0 & 0 \\ 0 & 0 & 4 & 0\end{vmatrix}=\)(  ).}
		{只含一项非零项 \(1 \cdot 2 \cdot 3 \cdot 4=24\),符号由 \(N(2413)=3\),带负号,所以 \(D=-24\).}
	\end{question}
	
	\begin{question}
		{填空题}
		{行列式}
		{中等}
		{16.计算行列式 \(D=\begin{vmatrix}0 & a_{12} & 0 & 0 \\ 0 & 0 & 0 & -a_{24} \\ a_{31} & 0 & 0 & 0 \\ 0 & 0 & a_{43} & 0\end{vmatrix}=\)(  ).}
		{只含一项非零项 \(a_{12} \cdot (-a_{24}) \cdot a_{31} \cdot a_{43}\),符号由 \(N(2413)=3\),带负号,所以 \(D=a_{12} a_{24} a_{31} a_{43}\).}
	\end{question}
	
	\begin{question}
		{填空题}
		{行列式}
		{中等}
		{17.在六阶行列式中,\(a_{15} a_{23} a_{32} a_{44} a_{51} a_{66}\) 应取的符号为(  ).}
		{由排列 532416 的逆序数为 8,故符号为 \((-1)^8 = 1\).}
	\end{question}
	
	\begin{question}
		{填空题}
		{行列式}
		{中等}
		{18.四阶行列式中含有因子 \(a_{11} a_{23}\) 的项是(  )和(  ).}
		{含有 \(a_{11} a_{23}\) 的项为 \((-1)^{r(1324)} a_{11} a_{23} a_{32} a_{44}\) 和 \((-1)^{r(1342)} a_{11} a_{23} a_{34} a_{42}\).}
	\end{question}
	
	
	\begin{question}
		{计算题}
		{行列式}
		{容易}
		{1.用定义计算行列式 \(\begin{vmatrix}0 & \cdots & 0 & 1 \\ 0 & \cdots & 2 & 0 \\ \cdots & \cdots & \cdots & \cdots \\ n & \cdots & 0 & 0\end{vmatrix}\) 的值(  ).}
		{设 \(\begin{vmatrix}0 & \cdots & 0 & 1 \\ 0 & \cdots & 2 & 0 \\ \cdots & \cdots & \cdots & \cdots \\ n & \cdots & 0 & 0\end{vmatrix}=a_{i j}\) ,根据行列式的定义,\(a_{i j}\)
			的展开式中,除 \(a_{1 n} a_{2, n-1} \cdots a_{n 1}\) 连乘积这一项外,其他各项中至少含有一个零元素,故皆为零,因此:\(a_{i j}=(-1)^{N(n \cdot 21)} a_{1 n} a_{2, n-1} \cdots a_{n 1}\) \(=(-1)^{\frac{n(n-1)}{2}} 1 \cdot 2 \cdots n=(-1)^{\frac{n(n-1)}{2}} n!\) .}
	\end{question}
	
	\begin{question}
		{选择题}
		{行列式}
		{容易}
		{1.利用行列式的性质,计算行列式 \(\begin{vmatrix}1 & 2 & 1 \\ 0 & 2 & 0 \\ 0 & 0 & 3\end{vmatrix}=\)().A. 6 B.-6 C.0 D.8}
		{上三角形行列式的值等于主对角元素的乘积,因此 \(\begin{vmatrix}1 & 2 & 1 \\ 0 & 2 & 0 \\ 0 & 0 & 3\end{vmatrix}=1 \times 2 \times 3=6\).}
	\end{question}	
	
	\begin{question}
		{选择题}
		{行列式}
		{容易}
		{2.利用行列式的性质,计算行列式 \(\begin{vmatrix}-1 & -1 & -1 \\ 8 & -5 & -2 \\ 4 & 4 & 4\end{vmatrix}=\)().A. 0 B. 1 C. -1 D. 32}
		{行列式有两行对应成比例,行列式的值为 0 ,因此 \(\begin{vmatrix}-1 & -1 & -1 \\ 8 & -5 & -2 \\ 4 & 4 & 4\end{vmatrix}=0\).}
	\end{question}
	
	\begin{question}
		{选择题}
		{行列式}
		{容易}
		{3.利用行列式的性质,计算行列式 \(\begin{vmatrix}1 & -2 & 2 \\ 1 & -10 & 2 \\ 1 & -8 & 2\end{vmatrix}=\)().A. 0 B. 2 C. -32 D. -2}
		{行列式有两列对应成比例,行列式的值为 0 ,因此 \(\begin{vmatrix}1 & -2 & 2 \\ 1 & -10 & 2 \\ 1 & -8 & 2\end{vmatrix}=0\).}
	\end{question}
	
	\begin{question}
		{选择题}
		{行列式}
		{容易}
		{4.利用行列式的性质,计算行列式 \(\begin{vmatrix}a & 0 & 0 \\ 0 & b & 0 \\ 0 & 0 & c\end{vmatrix}=\)().A. \(abc\) B. \(-abc\) C. \(a^3\) D. \(b^2\)}
		{对角形行列式的值等于主对角元素的乘积,因此 \(\begin{vmatrix}a & 0 & 0 \\ 0 & b & 0 \\ 0 & 0 & c\end{vmatrix}=a \times b \times c\).}
	\end{question}
	
	\begin{question}
		{选择题}
		{行列式}
		{容易}
		{5.利用行列式的性质,计算行列式 \(\begin{vmatrix}7 & 8 & 9 \\ -2 & -2 & -2 \\ 1 & 1 & 1\end{vmatrix}=\)().A. 0 B. 7 C. 8 D. 9}
		{行列式有两行对应成比例,行列式的值为 0 ,因此 \(\begin{vmatrix}7 & 8 & 9 \\ -2 & -2 & -2 \\ 1 & 1 & 1\end{vmatrix}=0\).}
	\end{question}
	
	\begin{question}
		{选择题}
		{行列式}
		{容易}
		{6.利用行列式的性质,计算行列式 \(\begin{vmatrix}8 & 5 & 0 \\ 6 & 9 & 0 \\ 0 & 7 & 0\end{vmatrix}=\)().A. 0 B. 72 C. -1 D. 1}
		{行列式有一列元素全为零,行列式的值等于零,因此 \(\begin{vmatrix}8 & 5 & 0 \\ 6 & 9 & 0 \\ 0 & 7 & 0\end{vmatrix}=0\).}
	\end{question}
	
	\begin{question}
		{选择题}
		{行列式}
		{容易}
		{7.利用行列式的性质,计算行列式 \(\begin{vmatrix}1 & 2 & 1 \\ 0 & 3 & 0 \\ 0 & 0 & 3\end{vmatrix}=\)().A. 9 B. -6 C. 0 D. 8}
		{上三角形行列式的值等于主对角元素的乘积,因此 \(\begin{vmatrix}1 & 2 & 1 \\ 0 & 3 & 0 \\ 0 & 0 & 3\end{vmatrix}=1 \times 3 \times 3=9\).}
	\end{question}
	
	\begin{question}
		{选择题}
		{行列式}
		{容易}
		{8.利用行列式的性质,计算行列式 \(\begin{vmatrix}9 & 4 & 7 \\ 4 & 4 & 4 \\ 3 & 3 & 3\end{vmatrix}=\)().A. 0 B. 9 C. 4 D. 7}
		{行列式有两行对应成比例,行列式的值为 0 ,因此 \(\begin{vmatrix}9 & 4 & 7 \\ 4 & 4 & 4 \\ 3 & 3 & 3\end{vmatrix}=0\).}
	\end{question}
	
	\begin{question}
		{选择题}
		{行列式}
		{容易}
		{9.利用行列式的性质,计算行列式 \(\begin{vmatrix}3 & a & 7 \\ 3 & b & 7 \\ 3 & c & 7\end{vmatrix}=\)().A. 0 B. \(abc\) C. 21abc D. \(-21abc\)}
		{行列式有两列对应成比例,行列式的值为 0 ,因此 \(\begin{vmatrix}3 & a & 7 \\ 3 & b & 7 \\ 3 & c & 7\end{vmatrix}=0\).}
	\end{question}
	
	\begin{question}
		{选择题}
		{行列式}
		{容易}
		{10.利用行列式的性质,计算行列式 \(\begin{vmatrix}-4 & 2 & 1 \\ 0 & 4 & -2 \\ 0 & 0 & 1\end{vmatrix}=\)().A. -16 B. 16 C. 64 D. 32}
		{上三角形行列式的值等于主对角元素的乘积,因此 \(\begin{vmatrix}-4 & 2 & 1 \\ 0 & 4 & -2 \\ 0 & 0 & 1\end{vmatrix}=-4 \times 4 \times 1=-16\).}
	\end{question}
	
	\begin{question}
		{选择题}
		{行列式}
		{中等}
		{11.利用行列式的性质,计算行列式 \(\begin{vmatrix}1 & 4 & 5 \\ 2 & 6 & 7 \\ 3 & 9 & 8\end{vmatrix}=\)().A. 5 B. -5 C. 10 D. -10}
		{通过初等行变换将行列式化为上三角形式:\(\begin{vmatrix}1 & 4 & 5 \\ 2 & 6 & 7 \\ 3 & 9 & 8\end{vmatrix}=\begin{vmatrix}1 & 4 & 5 \\ 0 & -2 & -3 \\ 0 & -3 & -7\end{vmatrix}=\begin{vmatrix}1 & 4 & 5 \\ 0 & -2 & -3 \\ 0 & 0 & -\frac{5}{2}\end{vmatrix}=5\).}
	\end{question}
	
	\begin{question}
		{选择题}
		{行列式}
		{中等}
		{12.如果行列式的所有元素变号,则(). A. 奇阶行列式变号 B. 行列式一定不变号 C. 偶阶行列式变号 D. 行列式一定变号}
		{每一行都提出一个负号,一共提\(n\)个,得\((-1)^n\),所以\(n\)为奇数时变号,为偶数时不变. }
	\end{question}
	
	\begin{question}
		{选择题}
		{行列式}
		{中等}
		{13.设 \(\begin{vmatrix}a_1 & a_2 & a_3 \\ b_1 & b_2 & b_3 \\ c_1 & c_2 & c_3\end{vmatrix}=k\) ,则 \(\begin{vmatrix}a_1 & a_2-2a_3 & -a_3 \\ b_1 & b_2-2b_3 & -b_3 \\ c_1 & c_2-2c_3 & -c_3\end{vmatrix}=\)().A. \(-k\) B. \(-2k\) C. \(k\) D. \(2k\)}
		{原行列式为\(k\),第二列变为\(a_2 - 2a_3\),第三列为\(-a_3\),根据行列式的线性性质可得:该行列式为\(\begin{vmatrix}a_1 & a_2 & -a_3 \\ b_1 & b_2 & -b_3 \\ c_1 & c_2 & -c_3\end{vmatrix} + \begin{vmatrix}a_1 & -2a_3 & -a_3 \\ b_1 & -2b_3 & -b_3 \\ c_1 & -2c_3 & -c_3\end{vmatrix} = -k\). }
	\end{question}
	
	\begin{question}
		{判断题}
		{行列式}
		{容易}
		{1.行列式有两行元素全相等,行列式的值一定为零(  ).}
		{行列式的性质:行列式有两行元素全相等行列式的值等于零.所以结论正确.}
	\end{question}
	
	
	\begin{question}
		{判断题}
		{行列式}
		{容易}
		{2.行列式有一行元素全为零,行列式的值一定为零(  ).}
		{行列式有一行元素全为零,行列式的值一定为零.所以结论正确.}
	\end{question}
	
	\begin{question}
		{判断题}
		{行列式}
		{容易}
		{3.行列式有两列元素对应成比例,行列式的值一定为零(  ).}
		{行列式的性质:行列式有两列元素对应成比例,行列式的值为 0.所以结论正确.}
	\end{question}
	
	\begin{question}
		{判断题}
		{行列式}
		{容易}
		{4.行列式有两行元素的总和成比例,行列式的值一定为零(  ).}
		{由行列式的性质,行列式有两行元素对应成比例,行列式的值为 0,但行列式两行元素的总和成比例,不一定行列式的值为零.所以结论错误.}
	\end{question}
	
	\begin{question}
		{判断题}
		{行列式}
		{容易}
		{5.行列式有一列元素全为零,行列式的值不一定为零(  ).}
		{行列式有一列元素全为零,行列式的值一定为零.所以结论错误.}
	\end{question}
	
	\begin{question}
		{判断题}
		{行列式}
		{容易}
		{6.\(\begin{vmatrix}2 a_1 & 2 b_1 & 2 c_1 \\ 2 a_2 & 2 b_2 & 2 c_2 \\ 2 a_3 & 2 b_3 & 2 c_3\end{vmatrix}=2\begin{vmatrix}a_1 & b_1 & c_1 \\ a_2 & b_2 & c_2 \\ a_3 & b_3 & c_3\end{vmatrix}\) (  ).}
		{因为 \(\begin{vmatrix}2 a_1 & 2 b_1 & 2 c_1 \\ 2 a_2 & 2 b_2 & 2 c_2 \\ 2 a_3 & 2 b_3 & 2 c_3\end{vmatrix}=2^3\begin{vmatrix}a_1 & b_1 & c_1 \\ a_2 & b_2 & c_2 \\ a_3 & b_3 & c_3\end{vmatrix}\),所以结论错误.}
	\end{question}
	
	\begin{question}
		{判断题}
		{行列式}
		{容易}
		{7.\(n\) 阶行列式非零元素的个数少于 \(n\) 个,行列式的值一定为零(  ).}
		{行列式非零元素的个数少于 \(n\) 个,行列式一定有一行或一列全为零,行列式的值一定为零.所以结论正确.}
	\end{question}
	
	\begin{question}
		{判断题}
		{行列式}
		{容易}
		{8.\(n\) 阶行列式主对角元素全为零,行列式的值一定为零(  ).}
		{行列式主对角元素全为零,行列式的值不一定为零.所以结论错误.}
	\end{question}
	
	\begin{question}
		{判断题}
		{行列式}
		{容易}
		{9.行列式各行元素之和为零,行列式的值一定为零(  ).}
		{行列式各行元素之和为零,利用性质,把各列加于第一列,则第一列全为零,因此行列式为零.所以结论正确.}
	\end{question}
	
	\begin{question}
		{判断题}
		{行列式}
		{容易}
		{10.行列式各列元素之和为零,行列式的值不一定为零(  ).}
		{行列式各列元素之和为零,利用性质,把各行加于第一行,则第一行全为零,因此行列式为零.所以结论错误.}
	\end{question}
	
	
	\begin{question}
		{判断题}
		{行列式}
		{容易}
		{11.设 \(\begin{vmatrix}a_{11} & a_{12} & a_{13} \\ a_{21} & a_{22} & a_{23} \\ a_{31} & a_{32} & a_{33}\end{vmatrix}=1\) ,则 \(\begin{vmatrix}a_{11} & -4 a_{12} & a_{13} \\ 3 a_{21} & -12 a_{22} & 3 a_{23} \\ a_{31} & -4 a_{32} & a_{33}\end{vmatrix}=12(  ).\)}
		{\(\begin{vmatrix}a_{11} & -4 a_{12} & a_{13} \\ 3 a_{21} & -12 a_{22} & 3 a_{23} \\ a_{31} & -4 a_{32} & a_{33}\end{vmatrix}=-4\begin{vmatrix}a_{11} & a_{12} & a_{13} \\ 3 a_{21} & 3 a_{22} & 3 a_{23} \\ a_{31} & a_{32} & a_{33}\end{vmatrix}=-12\begin{vmatrix}a_{11} & a_{12} & a_{13} \\ a_{21} & a_{22} & a_{23} \\ a_{31} & a_{32} & a_{33}\end{vmatrix}=-12\),所以结论错误.}
	\end{question}
	
	\begin{question}
		{判断题}
		{行列式}
		{容易}
		{12.设 \(\begin{vmatrix}a_{11} & a_{12} & a_{13} \\ a_{21} & a_{22} & a_{23} \\ a_{31} & a_{32} & a_{33}\end{vmatrix}=1\) ,则
			\[
			\begin{vmatrix}
				a_{11} & -2 a_{12} & a_{13} \\
				3 a_{21} & -6 a_{22} & 3 a_{23} \\
				a_{31} & -2 a_{32} & a_{33}
			\end{vmatrix}=6(  ).
			\]
		}
		{\(\begin{vmatrix}a_{11} & -2 a_{12} & a_{13} \\ 3 a_{21} & -6 a_{22} & 3 a_{23} \\ a_{31} & -2 a_{32} & a_{33}\end{vmatrix}=-2\begin{vmatrix}a_{11} & a_{12} & a_{13} \\ 3 a_{21} & 3 a_{22} & 3 a_{23} \\ a_{31} & a_{32} & a_{33}\end{vmatrix}=\) \(-6\begin{vmatrix}a_{11} & a_{12} & a_{13} \\ a_{21} & a_{22} & a_{23} \\ a_{31} & a_{32} & a_{33}\end{vmatrix}=-6\),所以结论错误.}
	\end{question}
	
	\begin{question}
		{填空题}
		{行列式}
		{容易}
		{1.利用行列式的性质,计算行列式 \(\begin{vmatrix}3 & 2 & 1 & 7 \\ 0 & 4 & 6 & -9 \\ 0 & 0 & 1 & 8 \\ 0 & 0 & 0 & -4\end{vmatrix}=\)(  ).}
		{上三角形行列式的值等于主对角元素的乘积,因此\[
			\begin{vmatrix}
				3 & 2 & 1 & 7 \\
				0 & 4 & 6 & -9 \\
				0 & 0 & 1 & 8 \\
				0 & 0 & 0 & -4
			\end{vmatrix}=3 \times 4 \times 1 \times(-4)=-48
			\]}
	\end{question}	
	
	\begin{question}
		{填空题}
		{行列式}
		{容易}
		{2. 利用行列式的性质,计算行列式 \(\begin{vmatrix}1 & 2 & 3 \\ 0 & 1 & 2 \\ 1 & 1 & 1\end{vmatrix}=\)(\qquad). }
		{第三行乘以1加到第二行,得到两行相等,因此
			\[
			\begin{vmatrix}
				1 & 2 & 3 \\
				0 & 1 & 2 \\
				1 & 1 & 1
			\end{vmatrix}
			=\begin{vmatrix}
				1 & 2 & 3 \\
				1 & 2 & 3 \\
				1 & 1 & 1
			\end{vmatrix}=0
			\]}
	\end{question}
	
	\begin{question}
		{填空题}
		{行列式}
		{容易}
		{3. 利用行列式的性质,计算行列式 \(\begin{vmatrix}8 & 1 & 5 \\ 6 & 3 & -11 \\ 0 & 0 & 0\end{vmatrix}=\)(\qquad). }
		{行列式有一行元素全为零,行列式的值为零,因此
			\[
			\begin{vmatrix}
				8 & 1 & 5 \\
				6 & 3 & -11 \\
				0 & 0 & 0
			\end{vmatrix}=0
			\]}
	\end{question}
	
	\begin{question}
		{填空题}
		{行列式}
		{容易}
		{4. 利用行列式的性质,计算行列式 \(\begin{vmatrix}1 & 2 & 3 \\ 3 & -1 & 7 \\ 3 & -1 & 7\end{vmatrix}=\)(\qquad). }
		{行列式有两行相等,值为零,因此
			\[
			\begin{vmatrix}
				1 & 2 & 3 \\
				3 & -1 & 7 \\
				3 & -1 & 7
			\end{vmatrix}=0
			\]}
	\end{question}
	
	\begin{question}
		{填空题}
		{行列式}
		{容易}
		{5. 利用行列式的性质,计算行列式 \(\begin{vmatrix}2 & 0 & 5 \\ 6 & 0 & -1 \\ 3 & 0 & 4\end{vmatrix}=\)(\qquad). }
		{行列式有一列全为零,值为零,因此
			\[
			\begin{vmatrix}
				2 & 0 & 5 \\
				6 & 0 & -1 \\
				3 & 0 & 4
			\end{vmatrix}=0
			\]}
	\end{question}
	
	\begin{question}
		{填空题}
		{行列式}
		{容易}
		{6. 利用行列式的性质,计算行列式 \(\begin{vmatrix}-3 & 2 & 1 & 7 \\ 0 & 2 & 6 & -9 \\ 0 & 0 & 1 & 8 \\ 0 & 0 & 0 & -2\end{vmatrix}=\)(\qquad). }
		{上三角行列式的值等于主对角线元素的乘积,因此
			\[
			\begin{vmatrix}
				-3 & 2 & 1 & 7 \\
				0 & 2 & 6 & -9 \\
				0 & 0 & 1 & 8 \\
				0 & 0 & 0 & -2
			\end{vmatrix}=(-3)\cdot2\cdot1\cdot(-2)=12
			\]}
	\end{question}
	
	\begin{question}
		{填空题}
		{行列式}
		{中等}
		{7. 计算行列式 \(\begin{vmatrix}2 & 1 & 1 & 1 \\ 1 & 2 & 1 & 1 \\ 1 & 1 & 2 & 1 \\ 1 & 1 & 1 & 2\end{vmatrix}=\)(\qquad). }
		{将第2、3、4列分别乘以1加到第1列,得
			\[
			\begin{vmatrix}
				5 & 1 & 1 & 1 \\
				5 & 2 & 1 & 1 \\
				5 & 1 & 2 & 1 \\
				5 & 1 & 1 & 2
			\end{vmatrix}=5\begin{vmatrix}
				1 & 1 & 1 & 1 \\
				1 & 2 & 1 & 1 \\
				1 & 1 & 2 & 1 \\
				1 & 1 & 1 & 2
			\end{vmatrix}=5\cdot1\cdot1\cdot1\cdot1=5
			\]}
	\end{question}
	
	\begin{question}
		{填空题}
		{行列式}
		{中等}
		{8. 计算行列式 \(\begin{vmatrix}3 & 1 & 1 & 1 \\ 1 & 3 & 1 & 1 \\ 1 & 1 & 3 & 1 \\ 1 & 1 & 1 & 3\end{vmatrix}=\)(\qquad). }
		{将第2、3、4列分别乘以1加到第1列,得
			\[
			\begin{vmatrix}
				6 & 1 & 1 & 1 \\
				6 & 3 & 1 & 1 \\
				6 & 1 & 3 & 1 \\
				6 & 1 & 1 & 3
			\end{vmatrix}=6\begin{vmatrix}
				1 & 1 & 1 & 1 \\
				1 & 3 & 1 & 1 \\
				1 & 1 & 3 & 1 \\
				1 & 1 & 1 & 3
			\end{vmatrix}=6\cdot1\cdot2\cdot2\cdot2=48
			\]}
	\end{question}
	
	
	\begin{question}
		{计算题}
		{行列式}
		{容易}
		{1.利用行列式的性质,计算行列式 \(\begin{vmatrix}1 & 1 & 1 & 1 \\ -1 & 1 & 1 & 1 \\ -1 & -1 & 1 & 1 \\ -1 & -1 & -1 & 1\end{vmatrix}\) 的值(  ).}
		{用性质化为上三角形行列式 \(\begin{vmatrix}1 & 1 & 1 & 1 \\ -1 & 1 & 1 & 1 \\ -1 & -1 & 1 & 1 \\ -1 & -1 & -1 & 1\end{vmatrix}=\) \(\begin{vmatrix}1 & 1 & 1 & 1 \\ 0 & 2 & 2 & 2 \\ 0 & 0 & 2 & 2 \\ 0 & 0 & 0 & 2\end{vmatrix}=8\).}
	\end{question}	
	
	
	\begin{question}
		{计算题}
		{行列式}
		{容易}
		{2.利用行列式的性质,计算行列式 \(\begin{vmatrix}2 & -5 & 3 & 1 \\ 1 & 3 & -1 & 3 \\ 0 & 1 & 1 & -5 \\ -1 & -4 & 2 & -3\end{vmatrix}\) 的值(  ).}
		{用性质化为上三角形行列式:
			\[
			\begin{aligned}
				&\begin{vmatrix}
					2 & -5 & 3 & 1 \\
					1 & 3 & -1 & 3 \\
					0 & 1 & 1 & -5 \\
					-1 & -4 & 2 & -3
				\end{vmatrix}
				= -\begin{vmatrix}
					1 & 3 & -1 & 3 \\
					2 & -5 & 3 & 1 \\
					0 & 1 & 1 & -5 \\
					-1 & -4 & 2 & -3
				\end{vmatrix} \\
				&= -\begin{vmatrix}
					1 & 3 & -1 & 3 \\
					0 & -11 & 5 & -5 \\
					0 & 1 & 1 & -5 \\
					0 & -1 & 1 & 0
				\end{vmatrix}
				= \begin{vmatrix}
					1 & 3 & -1 & 3 \\
					0 & 1 & 1 & -5 \\
					0 & 0 & 16 & -60 \\
					0 & 0 & 0 & \frac{5}{2}
				\end{vmatrix} = 40.
			\end{aligned}
			\]}
	\end{question}
	
	\begin{question}
		{计算题}
		{行列式}
		{容易}
		{3.计算行列式 \(\begin{vmatrix}3 & 1 & -1 & 2 \\ -5 & 1 & 3 & -4 \\ 2 & 0 & 1 & -1 \\ 1 & -5 & 3 & -3\end{vmatrix}\) 的值(  ).}
		{
			\[
			D \stackrel{\substack{c_1-2c_3 \\ c_4+c_3}}{=}
			\begin{vmatrix}
				5 & 1 & -1 & 1 \\
				-11 & 1 & 3 & -1 \\
				0 & 0 & 1 & 0 \\
				-5 & -5 & 3 & 0
			\end{vmatrix}
			\]
			对第3行展开,得:
			\[
			(-1)^{3+3}\begin{vmatrix}
				5 & 1 & 1 \\
				-11 & 1 & -1 \\
				-5 & -5 & 0
			\end{vmatrix} \stackrel{r_2+r_1}{=}
			\begin{vmatrix}
				5 & 1 & 1 \\
				-6 & 2 & 0 \\
				-5 & -5 & 0
			\end{vmatrix}
			\]
			对第1行第3列展开,得:
			\[
			(-1)^{1+3} \begin{vmatrix}
				-6 & 2 \\
				-5 & -5
			\end{vmatrix} = 40.
			\]}
	\end{question}
	
	
	\begin{question}
		{计算题}
		{行列式}
		{容易}
		{4.计算行列式 \(\begin{vmatrix}3 & 1 & 1 & 1 \\ 1 & 3 & 1 & 1 \\ 1 & 1 & 3 & 1 \\ 1 & 1 & 1 & 3\end{vmatrix}\) 的值(  ).}
		{
			\[
			D = \begin{vmatrix}
				6 & 1 & 1 & 1 \\
				6 & 3 & 1 & 1 \\
				6 & 1 & 3 & 1 \\
				6 & 1 & 1 & 3
			\end{vmatrix} = 6
			\begin{vmatrix}
				1 & 1 & 1 & 1 \\
				1 & 3 & 1 & 1 \\
				1 & 1 & 3 & 1 \\
				1 & 1 & 1 & 3
			\end{vmatrix}
			\]
			\[
			= 6
			\begin{vmatrix}
				1 & 1 & 1 & 1 \\
				0 & 2 & 0 & 0 \\
				0 & 0 & 2 & 0 \\
				0 & 0 & 0 & 2
			\end{vmatrix} = 6 \cdot (2 \cdot 2 \cdot 2) = 48.
			\]}
	\end{question}
	
	\begin{question}
		{计算题}
		{行列式}
		{容易}
		{5.计算行列式 \(\begin{vmatrix}1 & 1 & 1 & 1 \\ 1 & 2 & 3 & 4 \\ 1 & 3 & 6 & 10 \\ 1 & 4 & 10 & 20\end{vmatrix}\) 的值(  ).}
		{
			\[
			D = \begin{vmatrix}
				1 & 1 & 1 & 1 \\
				1 & 2 & 3 & 4 \\
				1 & 3 & 6 & 10 \\
				1 & 4 & 10 & 20
			\end{vmatrix} =
			\begin{vmatrix}
				1 & 1 & 1 & 1 \\
				0 & 1 & 2 & 3 \\
				0 & 1 & 3 & 6 \\
				0 & 1 & 4 & 10
			\end{vmatrix}
			\]
			\[
			=
			\begin{vmatrix}
				1 & 1 & 1 & 1 \\
				0 & 1 & 2 & 3 \\
				0 & 0 & 1 & 3 \\
				0 & 0 & 1 & 4
			\end{vmatrix} =
			\begin{vmatrix}
				1 & 1 & 1 & 1 \\
				0 & 1 & 2 & 3 \\
				0 & 0 & 1 & 3 \\
				0 & 0 & 0 & 1
			\end{vmatrix} = 1.
			\]}
	\end{question}
	
	\begin{question}
		{选择题}
		{行列式}
		{容易}
		{1.已知 \(n\) 阶行列式 \(D=a_{i j}\) ,设 \(A_{i j}\) 是 \(n\) 阶行列式中 \(a_{i j}\) 的代数余子式,若 \(i \neq s\) ,则 \(a_{i 1} A_{s 1}+a_{i 2} A_{s 2}+\ldots+a_{i n} A_{s n}=(  )\).A. 0 B.1 C.\(D\) D.\(-D\)}
		{行列式的任意一行与另一行的代数余子式的乘积之和等于零.}
	\end{question}
	
	
	\begin{question}
		{选择题}
		{行列式}
		{容易}
		{2.已知 \(n\) 阶行列式 \(D=a_{i j}\) ,设 \(A_{i j}\) 是 \(n\) 阶行列式中 \(a_{i j}\) 的代数余子式,若 \(i=s\) ,则 \(a_{i 1} A_{s 1}+a_{i 2} A_{s 2}+\ldots+a_{i n} A_{s n}=(  )\).A.\(D\) B.1 C.0 D.\(-D\)}
		{行列式的任意一行与其代数余子式的乘积之和等于 \(D\).}
	\end{question}
	
	\begin{question}
		{选择题}
		{行列式}
		{容易}
		{3.行列式 \(\begin{vmatrix}-3 & 0 & 4 \\ 5 & 0 & 3 \\ x & y & 1\end{vmatrix}\) 中 \(x\) 的代数余子式=(  ).A.0 B.12 C.3 D.-4}
		{该代数余子式为 \((-1)^{3+1}\begin{vmatrix}0 & 4 \\ 0 & 3\end{vmatrix}=0\).}
	\end{question}
	
	\begin{question}
		{选择题}
		{行列式}
		{容易}
		{4.行列式 \(\begin{vmatrix}-3 & 0 & 4 \\ 5 & 0 & 3 \\ x & y & 1\end{vmatrix}\) 中 \(y\) 的代数余子式=(  ).A.29 B.-29 C.11 D.-11}
		{\(y\) 的代数余子式为 \((-1)^{3+2}\begin{vmatrix}-3 & 4 \\ 5 & 3\end{vmatrix}=29\).}
	\end{question}
	
	\begin{question}
		{选择题}
		{行列式}
		{容易}
		{5.已知四阶行列式 \(D\) 中第3列元素依次为 \(-1,2,1,1\) ,相应代数余子式为 \(2,5,3,0\) ,则 \(D=(  )\).A.11 B.-13 C.13 D.-11}
		{\(D = (-1) \times 2 + 2 \times 5 + 1 \times 3 + 1 \times 0 = 11\).}
	\end{question}
	
	\begin{question}
		{选择题}
		{行列式}
		{容易}
		{6.已知三阶行列式 \(D\) 中第1列元素依次为 \(-1,2,1\) ,相应代数余子式为 \(2,5,3\) ,则 \(D=(  )\).A.11 B.-13 C.13 D.-11}
		{\(D = (-1) \times 2 + 2 \times 5 + 1 \times 3 = 11\).}
	\end{question}
	
	\begin{question}
		{选择题}
		{行列式}
		{容易}
		{7.已知三阶行列式 \(D\) 中第1行元素依次为 \(1,-2,3\) ,相应代数余子式为 \(4,7,2\) ,则 \(D=(  )\).A.-4 B.4 C.6 D.-5}
		{\(D = 1 \times 4 + (-2) \times 7 + 3 \times 2 = -4\).}
	\end{question}
	
	\begin{question}
		{选择题}
		{行列式}
		{容易}
		{8.已知三阶行列式 \(D\) 中第3行元素依次为 \(-1,-2,3\) ,相应代数余子式为 \(4,7,2\) ,则 \(D=(  )\).A.-12 B.12 C.6 D.-5}
		{\(D = (-1) \times 4 + (-2) \times 7 + 3 \times 2 = -12\).}
	\end{question}
	
	\begin{question}
		{选择题}
		{行列式}
		{容易}
		{9.已知四阶行列式 \(D\) 中第4列元素依次为 \(-1,0,1,1\) ,相应代数余子式为 \(2,7,3,0\) ,则 \(D=(  )\).A.1 B.-1 C.4 D.-4}
		{\(D = (-1) \times 2 + 0 \times 7 + 1 \times 3 + 1 \times 0 = 1\).}
	\end{question}
	
	\begin{question}
		{选择题}
		{行列式}
		{容易}
		{10.已知四阶行列式 \(D\) 中第4行元素依次为 \(-4,0,-1,1\) ,相应代数余子式为 \(2,7,1,0\) ,则 \(D=(  )\).A.-9 B.-10 C.7 D.-7}
		{\(D = (-4) \times 2 + 0 \times 7 + (-1) \times 1 + 1 \times 0 = -9\).}
	\end{question}
	
	\begin{question}
		{选择题}
		{行列式}
		{容易}
		{11.已知三阶行列式 \(D\) 中第1行元素依次为 \(1,-2,-3\) ,相应代数余子式为 \(4,7,2\) ,则 \(D=(  )\).A.-16 B.14 C.16 D.-15}
		{\(D = 1 \times 4 + (-2) \times 7 + (-3) \times 2 = -16\).}
	\end{question}
	
	
	\begin{question}
		{选择题}
		{行列式}
		{容易}
		{12.已知三阶行列式 \(D\) 中第 1 行元素依次为 \(1,2,-3\) 且相应的余子式依次为 \(4,7,2\),则 \(D=(\quad)\). A. \(-16\) \quad B. \(-26\) \quad C. \(-4\) \quad D. \(26\)}
		{\(D=1 \times(-1)^{1+1} \cdot 4 + 2 \times(-1)^{1+2} \cdot 7 + (-3) \times(-1)^{1+3} \cdot 2 = -16\). }
	\end{question}
	
	\begin{question}
		{选择题}
		{行列式}
		{容易}
		{13.已知三阶行列式 \(D\) 中第 2 行元素依次为 \(1,2,-3\) 且相应的余子式依次为 \(4,7,2\),则 \(D=(\quad)\). A. \(16\) \quad B. \(26\) \quad C. \(-4\) \quad D. \(-16\)}
		{\(D=1 \cdot (-1)^{2+1} \cdot 4 + 2 \cdot (-1)^{2+2} \cdot 7 + (-3) \cdot (-1)^{2+3} \cdot 2 = 16\). }
	\end{question}
	
	\begin{question}
		{选择题}
		{行列式}
		{容易}
		{14.行列式 \(\begin{vmatrix}7 & -1 & 4 \\ 5 & 2 & 3 \\ x & y & 9\end{vmatrix}\) 中 \(x\) 的代数余子式为(\quad). A. \(-11\) \quad B. \(11\) \quad C. \(7\) \quad D. \(-5\)}
		{\(x\) 的代数余子式为 \((-1)^{3+1} \begin{vmatrix}-1 & 4 \\ 2 & 3\end{vmatrix} = -11\). }
	\end{question}
	
	\begin{question}
		{选择题}
		{行列式}
		{容易}
		{15.行列式 \(\begin{vmatrix}7 & -1 & 4 \\ 5 & 2 & 3 \\ x & y & 9\end{vmatrix}\) 中 \(y\) 的代数余子式为(\quad). A. \(-1\) \quad B. \(1\) \quad C. \(3\) \quad D. \(-5\)}
		{\(y\) 的代数余子式为 \((-1)^{3+2} \begin{vmatrix}7 & 4 \\ 5 & 3\end{vmatrix} = -1\). }
	\end{question}
	
	\begin{question}
		{选择题}
		{行列式}
		{容易}
		{16.已知四阶行列式 \(D\) 中第 3 列元素依次为 \(-1,3,2,1\),且相应的代数余子式依次为 \(1,5,4,0\),则 \(D=(\quad)\). A. \(22\) \quad B. \(-13\) \quad C. \(13\) \quad D. \(-22\)}
		{\(D = (-1)\cdot 1 + 3\cdot 5 + 2\cdot 4 + 1\cdot 0 = 22\). }
	\end{question}
	
	\begin{question}
		{选择题}
		{行列式}
		{中等}
		{17.若 \(\begin{vmatrix}a_{11} & a_{12} \\ a_{21} & a_{22}\end{vmatrix} = 1\),则 \(\begin{vmatrix}-a_{11} & 3a_{12} & 0 \\ -a_{21} & 3a_{22} & 0 \\ 0 & 6 & 1\end{vmatrix}=(\quad)\). A. \(-3\) \quad B. \(3\) \quad C. \(1\) \quad D. \(6\)}
		{可分块计算,该值为 \((-1) \cdot 3 \cdot \begin{vmatrix}a_{11} & a_{12} \\ a_{21} & a_{22}\end{vmatrix} = -3\). }
	\end{question}
	
	\begin{question}
		{选择题}
		{行列式}
		{容易}
		{18.行列式 \(\begin{vmatrix}3 & 4 & 9 \\ 5 & 7 & -1 \\ 2 & 1 & 4\end{vmatrix}\) 中元素 \(a_{23}\) 的代数余子式 \(A_{23}=(\quad)\). A. \(5\) \quad B. \(-3\) \quad C. \(3\) \quad D. \(-5\)}
		{\(A_{23} = (-1)^{2+3} \begin{vmatrix}3 & 4 \\ 2 & 1\end{vmatrix} = 5\). }
	\end{question}
	
	\begin{question}
		{选择题}
		{行列式}
		{中等}
		{19.若 \(\begin{vmatrix}a_{11} & a_{12} & a_{13} \\ a_{21} & a_{22} & a_{23} \\ a_{31} & a_{32} & a_{33}\end{vmatrix} = d\),则
			\[
			\begin{vmatrix}
				2 a_{11} & 2 a_{12} & 2 a_{13} \\
				3 a_{31} & 3 a_{32} & 3 a_{33} \\
				-a_{21} & -a_{22} & -a_{23}
			\end{vmatrix} = (\quad)
			\]
			A. \(6d\) \quad B. \(3d\) \quad C. \(2d\) \quad D. \(-6d\)}
		{变换后为 \(-6 \cdot \begin{vmatrix}a_{11} & a_{12} & a_{13} \\ a_{31} & a_{32} & a_{33} \\ a_{21} & a_{22} & a_{23}\end{vmatrix} = 6d\). }
	\end{question}
	
	\begin{question}
		{选择题}
		{行列式}
		{中等}
		{20.行列式 \(\begin{vmatrix}1 & 0 & a & 1 \\ 0 & -1 & b & -1 \\ -1 & -1 & c & -1 \\ -1 & 1 & d & 0\end{vmatrix}\) 按第 3 列展开,则 \(a\) 的符号为(\quad),行列式的值为(\quad). A. \((-1)^{1+3},\ a + b + d\) \quad B. \((-1)^{1+3},\ a - b - c\) \quad C. \((-1)^{1+3},\ a + b - c\) \quad D. \((-1)^3,\ a + b + c\)}
		{按第 3 列展开,整体值为 \(a + b + d\),\(a\) 的符号为 \((-1)^{1+3}\). }
	\end{question}
	
	\begin{question}
		{选择题}
		{行列式}
		{容易}
		{21. 将行列式 \(\begin{vmatrix}1 & 0 & 2a & 1 \\ 0 & -1 & -b & -1 \\ -1 & -1 & c & -1 \\ -1 & 1 & d & 0\end{vmatrix}\) 按第三列展开,则 \(2a\) 的代数余子式为(),行列式的值为(). \\
			A. \(\begin{vmatrix}0 & -1 & -1 \\ -1 & -1 & -1 \\ -1 & 1 & 0\end{vmatrix}, 2a-b+d\) \quad
			B. \(\begin{vmatrix}0 & -1 & -1 \\ -1 & -1 & -1 \\ -1 & 1 & 0\end{vmatrix}, 2a+b+d\) \quad
			C. \(\begin{vmatrix}0 & -1 & -1 \\ -1 & -1 & -1 \\ -1 & 1 & 0\end{vmatrix}, 2a-b-d\) \quad
			D. \(\begin{vmatrix}0 & -1 & -1 \\ -1 & -1 & -1 \\ -1 & 1 & 0\end{vmatrix}, 2a+b-d\)}
		{按第三列展开行列式,\(2a\) 的代数余子式为 \(\begin{vmatrix}0 & -1 & -1 \\ -1 & -1 & -1 \\ -1 & 1 & 0\end{vmatrix}\),代入展开式得值为 \(2a-b+d\).}
	\end{question}
	
	
	\begin{question}
		{选择题}
		{行列式}
		{容易}
		{22. 某四阶行列式 \(D\) 的值为 1 ,它的第一行元素为 \(1,7,2,-1\) ,而第一行元素对应的余子式分别为 \(-1,0,k,4\) ,则 \(k=(\quad)\). \\
			A. \(-1\) \quad B. \(-2\) \quad C. \(1\) \quad D. \(2\)}
		{由 \(D = a_{11}A_{11} + a_{12}A_{12} + a_{13}A_{13} + a_{14}A_{14} = 1\),代入已知值,得 \(1 \times (-1) + 7 \times 0 + 2 \times k + (-1) \times (-4) = 1\),解得 \(k = -1\).}
	\end{question}
	
	
	\begin{question}
		{选择题}
		{行列式}
		{容易}
		{23. 某四阶行列式 \(D\) 的值为 1 ,它的第一行元素为 \(1,3,k,-1\) ,而第二行元素对应的余子式分别为 \(-1,0,1,2\) ,则 \(k=(\quad)\). \\
			A. \(-1\) \quad B. \(-2\) \quad C. \(1\) \quad D. \(2\)}
		{利用代数余子式的正交性有 \(a_{11}A_{21} + a_{12}A_{22} + a_{13}A_{23} + a_{14}A_{24} = 0\),代入已知得 \(1\times1 + 3\times0 + k\times(-1) + (-1)\times2 = 0\),解得 \(k = -1\).}
	\end{question}
	
	\begin{question}
		{判断题}
		{行列式}
		{容易}
		{1.已知 7 阶行列式 \(D=a_{i j}\) ,设 \(A_{i j}\) 是 \(n\) 阶行列式中 \(a_{i j}\) 的代数余子式,则 \(a_{11} A_{21}+a_{12} A_{22}+a_{13} A_{23}+a_{14} A_{24}+a_{15} A_{25}+a_{16} A_{26}+\) \(a_{17} A_{27}=0\)(  ).}
		{行列式的任意一行与另一行的代数余子式的乘积之和等于零.所以结论正确.}
	\end{question}	
	
	\begin{question}
		{判断题}
		{行列式}
		{容易}
		{2.设 \(D=\begin{vmatrix}3 & -1 & 2 \\ -2 & -3 & 1 \\ 0 & 1 & -4\end{vmatrix}\) ,则 \(2 A_{11}+A_{21}-4 A_{31}=0\)(  ).}
		{\(2 A_{11}+A_{21}-4 A_{31}=\begin{vmatrix}2 & -1 & 2 \\ 1 & -3 & 1 \\ -4 & 1 & -4\end{vmatrix}=0\) ,所以结论正确.}
	\end{question}
	
	\begin{question}
		{判断题}
		{行列式}
		{容易}
		{3.行列式 \(D=\begin{vmatrix}-1 & 3 & 2 \\ -5 & 2 & 0 \\ 1 & 0 & -3\end{vmatrix}\) 中元素 \(a_{32}\) 的代数余子式为 \(-10\)(  ).}
		{\(A_{32}=(-1)^{3+2} M_{32}=(-1)^5 \begin{vmatrix}-1 & 2 \\ -5 & 0\end{vmatrix}=-10\) ,所以结论正确.}
	\end{question}
	
	\begin{question}
		{判断题}
		{行列式}
		{容易}
		{4.四阶行列式 \(D\) 的某行元素依次为 \(1,0,k,6\) ,它们的代数余子式分别为 \(3,4,2,0\) ,且 \(D=-9\) ,则 \(k=1\)(  ).}
		{\(D=a_{i1} A_{i1}+a_{i2} A_{i2}+a_{i3} A_{i3}+a_{i4} A_{i4}=3+0+2k+0=3+2k\) ,由 \(D=-9\) 得 \(k=-6\) ,所以结论错误.}
	\end{question}
	
	\begin{question}
		{判断题}
		{行列式}
		{容易}
		{5.设 \(D=\begin{vmatrix}3 & -1 & 1 \\ -2 & -3 & 1 \\ 0 & 1 & -1\end{vmatrix}\) ,则 \(A_{11}+A_{21}-A_{31}=0\)(  ).}
		{\(A_{11}+A_{21}-A_{31}=\begin{vmatrix}1 & -1 & 1 \\ 1 & -3 & 1 \\ -1 & 1 & -1\end{vmatrix}=0\) ,所以结论正确.}
	\end{question}
	
	\begin{question}
		{判断题}
		{行列式}
		{容易}
		{6.行列式 \(D=\begin{vmatrix}-1 & 3 & 2 \\ -5 & 2 & 0 \\ 1 & 0 & -3\end{vmatrix}\) 中元素 \(a_{32}\) 的余子式为 \(-10\)(  ).}
		{\(M_{32}=\begin{vmatrix}-1 & 2 \\ -5 & 0\end{vmatrix}=10\) ,所以结论错误.}
	\end{question}
	
	\begin{question}
		{判断题}
		{行列式}
		{容易}
		{7.已知 \(D\) 为三阶行列式,其第三行元素分别为 \(1,3,-2\) ,它们的代数余子式分别为 \(3,-2,1\) ,则 \(D=-5\)(  ).}
		{\(D=a_{31} A_{31}+a_{32} A_{32}+a_{33} A_{33}=1 \cdot 3 + 3 \cdot (-2) + (-2) \cdot 1 = -5\) ,所以结论正确.}
	\end{question}
	
	\begin{question}
		{判断题}
		{行列式}
		{容易}
		{8.设行列式 \(D=\begin{vmatrix}a^2 & ab & b^2 \\ 2a & a+b & 2b \\ 1 & 1 & 1\end{vmatrix}\) ,则它的第一行元素的代数余子式之和 \(A_{11}+A_{12}+A_{13}=0\)(  ).}
		{\(A_{11}+A_{12}+A_{13}=1 \cdot A_{11} + 1 \cdot A_{12} + 1 \cdot A_{13} = D = \begin{vmatrix}1 & 1 & 1 \\ 2a & a+b & 2b \\ 1 & 1 & 1\end{vmatrix}=0\) ,所以结论正确.}
	\end{question}
	
	\begin{question}
		{判断题}
		{行列式}
		{容易}
		{9.设行列式 \(D=\begin{vmatrix}a & b & c \\ a^2 & b^2 & c^2 \\ b+c & c+a & a+b\end{vmatrix}\) ,则它的第三行元素的代数余子式之和 \(A_{31}+A_{32}+A_{33}=0\)(  ).}
		{\(A_{31}+A_{32}+A_{33}=1 \cdot A_{31}+1 \cdot A_{32}+1 \cdot A_{33} = \begin{vmatrix}a & b & c \\ a^2 & b^2 & c^2 \\ 1 & 1 & 1\end{vmatrix}=\begin{vmatrix}1 & 1 & 1 \\ a & b & c \\ a^2 & b^2 & c^2\end{vmatrix}=(c-a)(c-b)(b-a)\) ,所以结论错误.}
	\end{question}
	
	\begin{question}
		{判断题}
		{行列式}
		{容易}
		{10.四阶行列式 \(D\) 的某行元素依次为 \(1,0,k,6\) ,它们的代数余子式分别为 \(3,4,2,0\) ,且 \(D=-7\) ,则 \(k=2\)(  ).}
		{\(D=a_{i1} A_{i1}+a_{i2} A_{i2}+a_{i3} A_{i3}+a_{i4} A_{i4}=3+0+2k+0=3+2k\) ,由 \(D=-7\) 得 \(k=-5\) ,所以结论错误.}
	\end{question}
	
	\begin{question}
		{填空题}
		{行列式}
		{容易}
		{1.若
			\[
			\begin{vmatrix}
				a_{11} & a_{12} \\
				a_{21} & a_{22}
			\end{vmatrix}=1 \text {, 则 }\begin{vmatrix}
				a_{11} & 3 a_{12} & 0 \\
				a_{21} & 3 a_{22} & 0 \\
				0 & 6 & 1
			\end{vmatrix}= ( ).
			\]}
		{若 \(\begin{vmatrix}a_{11} & a_{12} \\ a_{21} & a_{22}\end{vmatrix}=1\) ,则 \(\begin{vmatrix}a_{11} & 3 a_{12} & 0 \\ a_{21} & 3 a_{22} & 0 \\ 0 & 6 & 1\end{vmatrix}\) \(=(-1)^{3+3} \cdot 1\begin{vmatrix}a_{11} & 3 a_{12} \\ a_{21} & 3 a_{22}\end{vmatrix}=3\begin{vmatrix}a_{11} & a_{12} \\ a_{21} & a_{22}\end{vmatrix}=3\) .} 
	\end{question}
	
	
	\begin{question}
		{填空题}
		{行列式}
		{容易}
		{2.若 \(\begin{vmatrix}1 & 0 & 2 \\ x & 3 & 1 \\ 4 & x & 5\end{vmatrix}\) 的代数余子式 \(A_{12}=-1\) ,则代数余子式 \(A_{21}=\)(\qquad).}
		{因为 \(A_{12}=(-1)^{1+2}\begin{vmatrix}x & 1 \\ 4 & 5\end{vmatrix}=-(5x-4)=-1\),可得 \(5x-4=1\),解得 \(x=1\). 代入后,\(A_{21}=(-1)^{2+1}\begin{vmatrix}0 & 2 \\ 1 & 5\end{vmatrix}=(-1)\cdot(0\times5 - 2\times1) = (-1)\cdot(-2)=2\). }
	\end{question}
	
	\begin{question}
		{填空题}
		{行列式}
		{容易}
		{3.设 \(D=\begin{vmatrix}1 & 0 & 0 & 0 \\ -2 & 7 & 6 & -3 \\ -4 & 8 & 3 & -5 \\ 9 & -7 & 2 & 5\end{vmatrix}\),则元素 8 的余子式为(\qquad),其代数余子式为(\qquad). }
		{元素 8 在第3行第2列,即 \(a_{32}=8\). 其余子式 \(M_{32} = \begin{vmatrix}1 & 0 & 0 \\ -2 & 6 & -3 \\ 9 & 2 & 5\end{vmatrix} = 36\),代数余子式 \(A_{32} = (-1)^{3+2} M_{32} = -36\). }
	\end{question}
	
	\begin{question}
		{填空题}
		{行列式}
		{容易}
		{4.已知行列式 \(\begin{vmatrix}4 & 3 & 2 & 1 \\ 1 & 4 & 3 & 2 \\ 2 & 1 & 4 & 3 \\ 3 & 2 & 1 & 4\end{vmatrix}\),则 \(3A_{14}+4A_{24}+A_{34}+2A_{44}=\)(\qquad). }
		{\(3A_{14}+4A_{24}+A_{34}+2A_{44}\) 是第2列的元素与第4列对应代数余子式的乘积之和,按照行列式性质,该和等于行列式中第2列与第4列对应展开的混合式,即 \(0\). }
	\end{question}
	
	\begin{question}
		{填空题}
		{行列式}
		{容易}
		{5.四阶行列式 \(D\) 中第三列元素为 \(1,2,3,4\),对应的余子式为 \(1,-1,2,1\),则 \(D\) 的值为(\qquad). }
		{设 \(a_{13}=1, a_{23}=2, a_{33}=3, a_{43}=4\),相应代数余子式 \(A_{13}=1, A_{23}=1, A_{33}=2, A_{43}=-1\),所以 \(D=1\times1 + 2\times1 + 3\times2 + 4\times(-1) = 1+2+6-4=5\). }
	\end{question}
	
	\begin{question}
		{计算题}
		{行列式}
		{容易}
		{1.已知 \(D=\begin{vmatrix}1 & 0 & 0 \\ 2 & 3 & 0 \\ 4 & 5 & 6\end{vmatrix}\) ,求 \(A_{11}+2 A_{21}+4 A_{31}\) ,其中 \(A_{i j}\) 表示 \(D\) 中 \(a_{i j}\) 的代数余子式(  ).}
		{\(A_{11}+2 A_{21}+4 A_{31}=D=18\) .}
	\end{question}
	
	
	\begin{question}
		{计算题}
		{行列式}
		{容易}
		{2. 已知 \(D=\begin{vmatrix}1 & 0 & 0 \\ 2 & 3 & 0 \\ 4 & 5 & 6\end{vmatrix}\),求 \(7 A_{31} + 8 A_{32} + 9 A_{33}\),其中 \(A_{ij}\) 表示 \(D\) 中 \(a_{ij}\) 的代数余子式(\qquad). }
		{\(7 A_{31} + 8 A_{32} + 9 A_{33} = \begin{vmatrix}1 & 0 & 0 \\ 2 & 3 & 0 \\ 7 & 8 & 9\end{vmatrix} = 27\). }
	\end{question}
	
	
	\begin{question}
		{计算题}
		{行列式}
		{中等}
		{3. 设四阶行列式 \(D_4=\begin{vmatrix}a & b & c & d \\ d & a & c & d \\ b & d & c & a \\ a & d & c & b\end{vmatrix}\),求 \(A_{11} + A_{21} + A_{31} + A_{41}\) (\qquad). }
		{将第1列都变为1,可转化为:
			\(
			A_{11} + A_{21} + A_{31} + A_{41} = \begin{vmatrix}1 & b & c & d \\ 1 & a & c & d \\ 1 & d & c & a \\ 1 & d & c & b\end{vmatrix} 
			\)
			该行列式中第1列元素都为1,可将其展开:记第1列为常数列,可看作列向量线性相关,因而行列式为0,
			所以:\(A_{11} + A_{21} + A_{31} + A_{41} = 0\). }
	\end{question}
	
	\begin{question}
		{计算题}
		{行列式}
		{容易}
		{4. 设 \(D=\begin{vmatrix}1 & 2 & 3 \\ 1 & 0 & -1 \\ -2 & 0 & 3\end{vmatrix}\),其中 \(A_{ij}\) 表示代数余子式,求 \(A_{11}+A_{22}+A_{33}\) (\qquad). }
		{\(A_{11} = (-1)^{1+1} \begin{vmatrix}0 & -1 \\ 0 & 3\end{vmatrix} = 1 \cdot (0 \cdot 3 - (-1)\cdot 0) = 0, \) \(A_{22} = (-1)^{2+2} \begin{vmatrix}1 & 3 \\ -2 & 3\end{vmatrix} = (1)(1 \cdot 3 - 3 \cdot (-2)) = 9, \) \(A_{33} = (-1)^{3+3} \begin{vmatrix}1 & 2 \\ 1 & 0\end{vmatrix} = 1 \cdot (1 \cdot 0 - 2 \cdot 1) = -2, A_{11} + A_{22} + A_{33} = 0 + 9 - 2 = 7.\)}
	\end{question}
	
	\begin{question}
		{计算题}
		{行列式}
		{容易}
		{5.已知四阶行列式 \(D\) 的第3行元素依次为 \(-1,2,0,1\),它们的余子式分别为 \(5,3,-7,4\),求行列式的值(\qquad). }
		{由题设余子式 \(M_{31}=5, M_{32}=3, M_{33}=-7, M_{34}=4\) ,故代数余子式为 \(A_{31}=5, A_{32}=-3, A_{33}=-7, A_{34}=-4\) ,所以 \(D=\) \((-1) \times 5+2 \times(-3)+0 \times(-7)+1 \times(-4)=-15\) .}
	\end{question}
	
	\begin{question}
		{计算题}
		{行列式}
		{中等}
		{6.设 \(D=\begin{vmatrix}3 & -5 & 2 & 1 \\ 1 & 1 & 0 & -5 \\ -1 & 3 & 1 & 3 \\ 2 & -4 & -1 & -3\end{vmatrix}\) 的代数余子式为 \(A_{ij}\),求 \(A_{11}+2 A_{12}+A_{13}+A_{14}\)(\qquad). }
		{将行列式按照第 i 行展开:\(|D|=a_{i 1} A_{i 1}+a_{i 2} A_{i 2}+\) \(a_{i 3} A_{i 3}+a_{i 4} A_{i 4},(i=1,2,3,4)\) ,其中 \(A_{i j}=(-1)^{i+j} M_{i j}\) ,令 \(a_{11}=1, a_{12}=2, a_{13}=1, a_{14}=1\) ,则 \(A_{11}+2 A_{12}+A_{13}+A_{14}\) \(=\begin{vmatrix}1 & 2 & 1 & 1 \\ 1 & 1 & 0 & -5 \\ -1 & 3 & 1 & 3 \\ 2 & -4 & -1 & -3\end{vmatrix}=\begin{vmatrix}1 & 2 & 1 & 1 \\ 0 & -1 & -1 & -6 \\ 0 & 5 & 2 & 4 \\ 0 & -8 & -3 & -5\end{vmatrix}=\)
			\(\begin{vmatrix}1 & 2 & 1 & 1 \\ 0 & -1 & -1 & -6 \\ 0 & 0 & -3 & -26 \\ 0 & 0 & 5 & 43\end{vmatrix}=\begin{vmatrix}1 & 2 & 1 & 1 \\ 0 & -1 & -1 & -6 \\ 0 & 0 & -3 & -26 \\ 0 & 0 & 0 & -\frac{1}{3}\end{vmatrix}=-1\).}
	\end{question}
	
	\begin{question}
		{计算题}
		{行列式}
		{中等}
		{7.设 \(D=\begin{vmatrix}3 & -5 & 2 & 1 \\ 1 & 1 & 0 & -5 \\ -1 & 3 & 1 & 3 \\ 2 & -4 & -1 & -3\end{vmatrix}\) 的余子式为 \(M_{ij}\),求 \(3 M_{21}+5 M_{22}+M_{23}+2 M_{24}\)(\qquad). }
		{\(3 M_{21}+5 M_{22}+M_{23}+2 M_{24}=-3 A_{21}+5 A_{22}-A_{23}+\) \(2 A_{24} =\begin{vmatrix}3 & -5 & 2 & 1 \\ -3 & 5 & -1 & 2 \\ -1 & 3 & 1 & 3 \\ 2 & -4 & -1 & -3\end{vmatrix}=-10\).}
	\end{question}
	
	\begin{question}
		{计算题}
		{行列式}
		{中等}
		{8.用降阶法计算行列式 \(\begin{vmatrix}3 & 1 & -1 & 1 \\ -5 & 1 & 3 & -4 \\ 2 & 0 & 1 & 0 \\ 1 & -5 & 3 & -3\end{vmatrix}\)(\qquad). }
		{ \(D \stackrel{c_1-2 c 3}{=}\begin{vmatrix}5 & 1 & -1 & 1 \\ -11 & 1 & 3 & -4 \\ 0 & 0 & 1 & 0 \\ -5 & -5 & 3 & -3\end{vmatrix}=\)
			
			\[
			\begin{aligned}
				& (-1)^{3+3}\begin{vmatrix}
					5 & 1 & 1 \\
					-11 & 1 & -4 \\
					-5 & -5 & -3
				\end{vmatrix} \\
				=\begin{vmatrix}
					5 & 1 & 1 \\
					-16 & 0 & -5 \\
					20 & 0 & 2
				\end{vmatrix}= \\
				& (-1)^{1+2}\begin{vmatrix}
					-5 \\
					20 & 2
				\end{vmatrix}=-68 .
			\end{aligned}
			\]}
	\end{question}
	
	\begin{question}
		{计算题}
		{行列式}
		{中等}
		{9.用降阶法计算行列式 \(\begin{vmatrix}1 & 1 & -1 & 2 \\ -5 & 1 & 3 & -1 \\ 2 & 0 & 1 & -1 \\ 1 & -5 & 3 & -3\end{vmatrix}\)(\qquad). }
		{\[
			\begin{aligned}
				&D^{\stackrel{c_1+(-2) c_3}{c_4+c_3}=}\begin{vmatrix}
					3 & 1 & -1 & 1 \\
					-11 & 1 & 3 & 2 \\
					0 & 0 & 1 & 0 \\
					-5 & -5 & 3 & 0
				\end{vmatrix}= \\
				& (-1)^{3+3}\begin{vmatrix}
					3 & 1 & 1 \\
					-11 & 1 & 2 \\
					-5 & -5 & 0
				\end{vmatrix} \stackrel{c_2+(-1) c_1}{=}\begin{vmatrix}
					3 & -2 & 1 \\
					-11 & 12 & 2 \\
					-5 & 0 & 0
				\end{vmatrix}=(-5) \times \\
				& (-1)^{3+1}\begin{vmatrix}
					1 \\
					12 & 2
				\end{vmatrix}=80
			\end{aligned}
			\]}
	\end{question}
	
	
	\begin{question}
		{计算题}
		{行列式}
		{中等}
		{10.设 \(D=\begin{vmatrix}1 & 2 & 3 \\ 1 & 0 & -1 \\ -2 & 0 & 3\end{vmatrix}\) ,求 \(D\) .(\qquad)}
		{\(D=\begin{vmatrix}1 & 2 & 3 \\ 1 & 0 & -1 \\ -2 & 0 & 3\end{vmatrix}=2 \times(-1)^{1+2}\begin{vmatrix}1 & -1 \\ -2 & 3\end{vmatrix}=\) -2 .}
	\end{question}
	
	\begin{question}
		{计算题}
		{行列式}
		{中等}
		{11.设 \(|A|=\begin{vmatrix}1 & -5 & 1 & 3 \\ 1 & 1 & 3 & 4 \\ 1 & 1 & 2 & 3 \\ 2 & 2 & 3 & 4\end{vmatrix}\) ,计算 \(A_{41}+A_{42}+A_{43}+A_{44}\) 的值,其中 \(A_{4 j}\) 是 \(|A|\) 中元素 \(a_{4 j}(j=1,2,3,4)\) 的代数余子式.}
		{\[
			\begin{aligned}
				& A_{41}+A_{42}+A_{43}+A_{44}=\begin{vmatrix}
					1 & -5 & 1 & 3 \\
					1 & 1 & 3 & 4 \\
					1 & 1 & 2 & 3 \\
					1 & 1 & 1 & 1
				\end{vmatrix} \stackrel{r_1 \leftrightarrows r_4}{=} \\
				& -\begin{vmatrix}
					1 & 1 & 1 & 1 \\
					1 & -1 & 3 & 4 \\
					1 & 1 & 2 & 3 \\
					1 & -5 & 1 & 3
				\end{vmatrix} =-\begin{vmatrix}
					1 & 1 & 1 & 1 \\
					= & r_4-r_1 \\
					0 & 0 & 2 & 3 \\
					0 & 0 & 1 & 2 \\
					0 & -6 & 0 & 2
				\end{vmatrix}= \\
				& (-1)\begin{vmatrix}
					0 & 2 & 3 \\
					0 & 1 & 2 \\
					-6 & 0 & 2
				\end{vmatrix}=(-1) \times(-6) \times(-1)^{3+1}\begin{vmatrix}
					2 & 3 \\
					1
				\end{vmatrix}=6 .
			\end{aligned}
			\]}
	\end{question}
	
	\begin{question}
		{计算题}
		{行列式}
		{中等}
		{12.已知行列式 \(D=\begin{vmatrix}1 & 2 & 3 & 4 \\ 1 & 0 & 1 & 2 \\ 3 & -1 & -1 & 0 \\ 1 & 2 & 0 & -5\end{vmatrix}\) ,求余子式 \(M_{13}\) 和代数余子式 \(A_{43}\) .}
		{余子式 \(M_{13}=\begin{vmatrix}1 & 0 & 2 \\ 3 & -1 & 0 \\ 1 & 2 & -5\end{vmatrix}=-19\) ,代数余子式 \(A_{43}=\) \((-1)^{4+3}\begin{vmatrix}1 & 2 & 4 \\ 1 & 0 & 2 \\ 3 & -1 & 0\end{vmatrix}=-10\).}
	\end{question}
	
	
	
	\begin{question}
		{计算题}
		{行列式}
		{中等}
		{13.用行列式按行(列)展开定理计算行列式:\(D=\)
			
			\[
			\begin{vmatrix}
				1 & 2 & 3 & 4 \\
				1 & 0 & 1 & 2 \\
				3 & -1 & -1 & 0 \\
				1 & 2 & 0 & -5
			\end{vmatrix} .
			\]}
		{按第二行展开
			
			\[
			\begin{aligned}
				D & =1 \cdot A_{21}+0 \cdot A_{22}+1 \cdot A_{23}+2 A_{24} \\
				= & 1 \times(-1)^{2+1} 3+1 \times(-1)^{2+3} 63+2 \times(-1)^{2+4} 21=-3-63+42=-24
			\end{aligned}
			\]
			利用展开定理时,通常结合性质将展开行(列)的较多元素化为零.}
	\end{question}
	
	\begin{question}
		{计算题}
		{行列式}
		{中等}
		{14.设 \(|A|=\begin{vmatrix}1 & -5 & 1 & 3 \\ 1 & 1 & 3 & 4 \\ 1 & 1 & 2 & 3 \\ 2 & 2 & 3 & 4\end{vmatrix}\) ,计算 \(A_{41}+A_{42}+A_{43}+A_{44}\) 的值,其中 \(A_{4 i}(i=1,2,3,4)\) 是对应元素的代数余子式.}
		{由行列式按行展开定理 \(A_{41}+A_{42}+A_{43}+A_{44}=1 \cdot A_{41}+\)
			
			\[
			1 \cdot A_{42}+1 \cdot A_{43}+1 \cdot A_{44}=\begin{vmatrix}
				1 & -5 & 1 & 3 \\
				1 & 1 & 3 & 4 \\
				1 & 1 & 2 & 3 \\
				1 & 1 & 1 & 1
			\end{vmatrix}=
			\]
			
			
			\[
			\begin{aligned}
				& \begin{vmatrix}
					1 & -5 & 1 & 3 \\
					0 & 6 & 2 & 1 \\
					0 & 6 & 1 & 0 \\
					0 & 6 & 0 & -2
				\end{vmatrix}=\begin{vmatrix}
					6 & 2 & 1 \\
					6 & 1 & 0 \\
					6 & 0 & -2
				\end{vmatrix}=\begin{vmatrix}
					6 & 2 & 1 \\
					0 & -1 & -1 \\
					0 & -2 & -3
				\end{vmatrix}= \\
				& 6\begin{vmatrix}
					-1 & -1 \\
					-2 & -3
				\end{vmatrix}=6 .
			\end{aligned}
			\]}
	\end{question}
	
	
	\begin{question}
		{选择题}
		{矩阵}
		{容易}
		{1.\(\left(\begin{array}{ccc}1 & 6 & 4 \\ -4 & 2 & 8\end{array}\right)+\left(\begin{array}{ccc}-2 & 0 & 1 \\ 2 & -3 & 4\end{array}\right)=\)().
			A. \(\left(\begin{array}{ccc}-1 & 6 & 5 \\ -2 & -1 & 12\end{array}\right)\)
			B. \(\left(\begin{array}{ccc}-1 & 7 & 5 \\ -2 & -1 & 2\end{array}\right)\)
			C. \(\left(\begin{array}{ccc}1 & 6 & 5 \\ 2 & 1 & 12\end{array}\right)\)
			D. \(\left(\begin{array}{ccc}-1 & 6 & 5 \\ -3 & 0 & 10\end{array}\right)\)}
		{\(\left(\begin{array}{ccc}1 & 6 & 4 \\ -4 & 2 & 8\end{array}\right)+\left(\begin{array}{ccc}-2 & 0 & 1 \\ 2 & -3 & 4\end{array}\right)=\left(\begin{array}{ccc}-1 & 6 & 5 \\ -2 & -1 & 12\end{array}\right)\) .}
	\end{question}
	
	\begin{question}
		{选择题}
		{矩阵}
		{容易}
		{2. \(\left(\begin{array}{lll}1 & 0 & 4 \\ 1 & 2 & 0\end{array}\right)-\left(\begin{array}{lll}0 & 1 & 1 \\ 2 & 3 & 4\end{array}\right)=\)().
			A. \(\left(\begin{array}{ccc}1 & -1 & 3 \\ -1 & -1 & -4\end{array}\right)\)
			B. \(\left(\begin{array}{ccc}-1 & 7 & 5 \\ -2 & -1 & 2\end{array}\right)\)
			C. \(\left(\begin{array}{ccc}-1 & 6 & 5 \\ -3 & 0 & 10\end{array}\right)\)
			D. \(\left(\begin{array}{ccc}1 & 6 & 5 \\ 2 & 1 & 12\end{array}\right)\)}
		{\(\left(\begin{array}{lll}1 & 0 & 4 \\ 1 & 2 & 0\end{array}\right)-\left(\begin{array}{lll}0 & 1 & 1 \\ 2 & 3 & 4\end{array}\right)=\left(\begin{array}{ccc}1 & -1 & 3 \\ -1 & -1 & -4\end{array}\right)\).}
	\end{question}
	
	\begin{question}
		{选择题}
		{矩阵}
		{容易}
		{3. \(\left(\begin{array}{lll}1 & 0 & 4 \\ 1 & 2 & 0\end{array}\right)+\left(\begin{array}{lll}0 & 1 & 1 \\ 2 & 3 & 4\end{array}\right)=\)().
			A. \(\left(\begin{array}{lll}1 & 1 & 5 \\ 3 & 5 & 4\end{array}\right)\)
			B. \(\left(\begin{array}{lll}1 & 6 & 5 \\ 0 & 1 & 2\end{array}\right)\)
			C. \(\left(\begin{array}{ccc}-1 & 6 & 5 \\ -3 & 0 & 10\end{array}\right)\)
			D. \(\left(\begin{array}{ccc}-1 & 7 & 5 \\ -2 & -1 & 2\end{array}\right)\)}
		{\(\left(\begin{array}{lll}1 & 0 & 4 \\ 1 & 2 & 0\end{array}\right)+\left(\begin{array}{lll}0 & 1 & 1 \\ 2 & 3 & 4\end{array}\right)=\left(\begin{array}{lll}1 & 1 & 5 \\ 3 & 5 & 4\end{array}\right)\).}
	\end{question}
	
	\begin{question}
		{选择题}
		{矩阵}
		{容易}
		{4. \(\left(\begin{array}{ll}1 & 2 \\ 3 & 4\end{array}\right)-\left(\begin{array}{ll}1 & 1 \\ 1 & 1\end{array}\right)=\)().
			A. \(\left(\begin{array}{ll}0 & 1 \\ 2 & 3\end{array}\right)\)
			B. \(\left(\begin{array}{cc}0 & 1 \\ -2 & -3\end{array}\right)\)
			C. \(\left(\begin{array}{ll}1 & -1 \\ 2 & -3\end{array}\right)\)
			D. \(\left(\begin{array}{cc}0 & -1 \\ 2 & 3\end{array}\right)\)}
		{\(\left(\begin{array}{ll}1 & 2 \\ 3 & 4\end{array}\right)-\left(\begin{array}{ll}1 & 1 \\ 1 & 1\end{array}\right)=\left(\begin{array}{ll}0 & 1 \\ 2 & 3\end{array}\right)\).}
	\end{question}
	
	\begin{question}
		{选择题}
		{矩阵}
		{容易}
		{5. \(\left(\begin{array}{ll}2 & 0 \\ 3 & 4\end{array}\right)-\left(\begin{array}{ll}2 & 1 \\ 4 & 7\end{array}\right)=\)().
			A. \(\left(\begin{array}{cc}0 & -1 \\ -1 & -3\end{array}\right)\)
			B. \(\left(\begin{array}{cc}0 & 1 \\ -2 & -3\end{array}\right)\)
			C. \(\left(\begin{array}{cc}1 & -1 \\ 2 & 3\end{array}\right)\)
			D. \(\left(\begin{array}{cc}3 & -1 \\ 2 & 3\end{array}\right)\)}
		{\(\left(\begin{array}{ll}2 & 0 \\ 3 & 4\end{array}\right)-\left(\begin{array}{ll}2 & 1 \\ 4 & 7\end{array}\right)=\left(\begin{array}{cc}0 & -1 \\ -1 & -3\end{array}\right)\).}
	\end{question}
	
	\begin{question}
		{选择题}
		{矩阵}
		{容易}
		{6. \(\left(\begin{array}{ll}1 & 4 \\ 2 & 8\end{array}\right)+\left(\begin{array}{ll}1 & -1 \\ 2 & -3\end{array}\right)=\)().
			A. \(\left(\begin{array}{ll}2 & 3 \\ 4 & 5\end{array}\right)\)
			B. \(\left(\begin{array}{cc}2 & 1 \\ -4 & -3\end{array}\right)\)
			C. \(\left(\begin{array}{cc}1 & -1 \\ 2 & 8\end{array}\right)\)
			D. \(\left(\begin{array}{cc}2 & -1 \\ 2 & 3\end{array}\right)\)}
		{\(\left(\begin{array}{ll}1 & 4 \\ 2 & 8\end{array}\right)+\left(\begin{array}{ll}1 & -1 \\ 2 & -3\end{array}\right)=\left(\begin{array}{ll}2 & 3 \\ 4 & 5\end{array}\right)\).}
	\end{question}
	
	\begin{question}
		{选择题}
		{矩阵}
		{容易}
		{7. \(\left(\begin{array}{cc}1 & 4 \\ -3 & -1\end{array}\right)+\left(\begin{array}{cc}1 & -1 \\ 2 & -3\end{array}\right)=\)().
			A. \(\left(\begin{array}{cc}2 & 3 \\ -1 & -4\end{array}\right)\)
			B. \(\left(\begin{array}{cc}2 & 1 \\ -4 & -3\end{array}\right)\)
			C. \(\left(\begin{array}{ll}1 & -1 \\ 2 & -4\end{array}\right)\)
			D. \(\left(\begin{array}{cc}2 & -1 \\ -1 & 3\end{array}\right)\)}
		{\(\left(\begin{array}{cc}1 & 4 \\ -3 & -1\end{array}\right)+\left(\begin{array}{cc}1 & -1 \\ 2 & -3\end{array}\right)=\left(\begin{array}{cc}2 & 3 \\ -1 & -4\end{array}\right)\).}
	\end{question}
	
	\begin{question}
		{选择题}
		{矩阵}
		{容易}
		{8. \(\left(\begin{array}{cc}1 & 4 \\ -3 & -1\end{array}\right)+\left(\begin{array}{cc}1 & -1 \\ 2 & -3\end{array}\right)+\left(\begin{array}{cc}2 & 3 \\ -1 & -4\end{array}\right)=\)().
			A. \(\left(\begin{array}{cc}4 & 6 \\ -2 & -8\end{array}\right)\)
			B. \(\left(\begin{array}{cc}4 & 1 \\ -4 & -3\end{array}\right)\)
			C. \(\left(\begin{array}{cc}1 & 6 \\ 2 & -8\end{array}\right)\)
			D. \(\left(\begin{array}{cc}4 & -1 \\ -1 & 3\end{array}\right)\)}
		{\(\left(\begin{array}{cc}1 & 4 \\ -3 & -1\end{array}\right)+\left(\begin{array}{cc}1 & -1 \\ 2&-3\end{array}\right)+\left(\begin{array}{cc}2&3\\-1&-4\end{array}\right)=\left(\begin{array}{cc}4 & 6 \\ -2 & -8\end{array}\right)\).}
	\end{question}
	
	\begin{question}
		{选择题}
		{矩阵}
		{容易}
		{9. \(\left(\begin{array}{lll}1 & 0 & 4 \\ 5 & 2 & 0\end{array}\right)-\left(\begin{array}{lll}0 & 1 & 1 \\ 2 & 3 & 4\end{array}\right)=\)().
			A. \(\left(\begin{array}{ccc}1 & -1 & 3 \\ 3 & -1 & -4\end{array}\right)\)
			B. \(\left(\begin{array}{ccc}-1 & 7 & 5 \\ -2 & -1 & 2\end{array}\right)\)
			C. \(\left(\begin{array}{ccc}-1 & 6 & 5 \\ -3 & 0 & 10\end{array}\right)\)
			D. \(\left(\begin{array}{ccc}1 & 6 & 5 \\ 2 & 1 & 12\end{array}\right)\)}
		{\(\left(\begin{array}{lll}1 & 0 & 4 \\ 5 & 2 & 0\end{array}\right)-\left(\begin{array}{lll}0 & 1 & 1 \\ 2 & 3 & 4\end{array}\right)=\left(\begin{array}{ccc}1 & -1 & 3 \\ 3 & -1 & -4\end{array}\right)\).}
	\end{question}
	
	\begin{question}
		{选择题}
		{矩阵}
		{容易}
		{10. \(\left(\begin{array}{cc}1 & 4 \\ 3 & -1\end{array}\right)+\left(\begin{array}{cc}1 & -1 \\ -1 & -3\end{array}\right)+\left(\begin{array}{cc}2 & 5 \\ 5 & -4\end{array}\right)=\)().
			A. \(\left(\begin{array}{cc}4 & 8 \\ 7 & -8\end{array}\right)\)
			B. \(\left(\begin{array}{cc}4 & 1 \\ -4 & 0\end{array}\right)\)
			C. \(\left(\begin{array}{ll}1 & 6 \\ 2 & 0\end{array}\right)\)
			D. \(\left(\begin{array}{cc}4 & -1 \\ -1 & 0\end{array}\right)\)}
		{\(\left(\begin{array}{cc}1 & 4 \\ 3 & -1\end{array}\right)+\left(\begin{array}{cc}1 & -1 \\ -1&-3\end{array}\right)+\left(\begin{array}{cc}2&5\\5&-4\end{array}\right)=\left(\begin{array}{cc}4 & 8 \\ 7 & -8\end{array}\right)\).}
	\end{question}
	
	
	\begin{question}
		{选择题}
		{矩阵}
		{中等}
		{11. 设 \(\mathbf{A}\) 是 \(m \times n\) 阶矩阵,\(\mathbf{B}\) 是 \(n \times m\) 阶矩阵 \((m \neq n)\),则下列运算结果是 \(n\) 阶的是().
			A. \(\mathbf{B A}\)
			B. \(\mathbf{A B}\)
			C. \(\mathbf{A}-\mathbf{B}\)
			D. \(\mathbf{A}+\mathbf{B}\)}
		{\(\mathbf{B}_{n \times m} \mathbf{A}_{m \times n}=\mathbf{C}_{n \times n}\),为 \(n\) 阶矩阵.}
	\end{question}
	
	\begin{question}
		{选择题}
		{矩阵}
		{中等}
		{12. 若(),则矩阵 \(\mathbf{A B}\) 与 \(\mathbf{B A}\) 都有意义.
			A. \(\mathbf{A}\) 与 \(\mathbf{B}\) 为同阶方阵
			B. \(\mathbf{A}\) 与 \(\mathbf{B}\) 都是 3 行 4 列的矩阵
			C. \(\mathbf{A}\) 与 \(\mathbf{B}\) 都是 4 行 3 列的矩阵
			D. \(\mathbf{A}\) 与 \(\mathbf{B}\) 都是 5 行 6 列的矩阵}
		{矩阵的乘法运算是左矩阵的列与右矩阵的行相等,才有意义,所以要使 \(\mathbf{A B}\) 与 \(\mathbf{B A}\) 都有意义,只有选项 \(\mathbf{A}\) 与 \(\mathbf{B}\) 为同阶方阵正确.}
	\end{question}
	
	\begin{question}
		{选择题}
		{矩阵}
		{中等}
		{13. 设 \(\mathbf{A}\) 是 \(m \times n\) 阶矩阵,\(\mathbf{B}\) 是 \(n \times m\) 阶矩阵 \((m \neq n)\),则下列运算结果是 \(n \times n\) 阶矩阵的是().
			A. \(\mathbf{B A}\)
			B. \(\mathbf{A B}\)
			C. \(\mathbf{A}-\mathbf{B}\)
			D. \(\mathbf{B}-\mathbf{A}\)}
		{\(\mathbf{B}_{n \times m} \mathbf{A}_{m \times n}=\mathbf{C}_{n \times n}\),所以选项 \(\mathbf{B A}\) 正确.}
	\end{question}
	
	\begin{question}
		{选择题}
		{矩阵}
		{中等}
		{14. 设 \(\mathbf{A}\) 是 \(m \times n\) 阶矩阵,\(\mathbf{B}\) 是 \(n \times m\) 阶矩阵 \((m \neq n)\),则下列运算结果是 \(m\) 阶的是().
			A. \(\mathbf{A B}\)
			B. \(\mathbf{B A}\)
			C. \(\mathbf{A}-\mathbf{B}\)
			D. \(\mathbf{B}-\mathbf{A}\)}
		{\(\mathbf{A}_{m \times n} \mathbf{B}_{n \times m}=\mathbf{C}_{m \times m}\) 为 \(m\) 阶矩阵.}
	\end{question}
	
	\begin{question}
		{选择题}
		{矩阵}
		{中等}
		{15. 设 \(\mathbf{A}\) 是 \(l \times m\) 阶矩阵,\(\mathbf{B}\) 是 \(m \times n\) 阶矩阵 \((l, m, n\) 不相等\()\),则下列运算结果有意义的是().
			A. \(\mathbf{A B}\)
			B. \(\mathbf{B A}\)
			C. \(\mathbf{A}-\mathbf{B}\)
			D. \(\mathbf{B}-\mathbf{A}\)}
		{\(\mathbf{A}_{l \times m} \mathbf{B}_{m \times n}=\mathbf{C}_{l \times n}\).}
	\end{question}
	
	\begin{question}
		{选择题}
		{矩阵}
		{中等}
		{16. 设 \(\mathbf{A}\) 是 \(m \times l\) 矩阵,\(\mathbf{B}\) 是 \(n \times m\) 矩阵 \((m \neq l)\),如果 \(\mathbf{A C B}\) 矩阵有意义,则 \(\mathbf{C}\) 是()阶矩阵.
			A. \(l \times n\)
			B. \(m \times n\)
			C. \(n \times m\)
			D. \(m \times l\)}
		{\(\mathbf{A}_{m \times l} \mathbf{C}_{l \times n} \mathbf{B}_{n \times m}=\mathbf{D}_{m \times n}\).}
	\end{question}
	
	\begin{question}
		{选择题}
		{矩阵}
		{中等}
		{17. 设 \(\mathbf{A}\) 是 \(m \times l\) 矩阵,\(\mathbf{B}\) 是 \(n \times m\) 矩阵 \((m \neq l)\),如果 \(\mathbf{A C B}\) 矩阵有意义,则 \(\mathbf{A C B}\) 是()阶矩阵.
			A. \(m \times m\)
			B. \(n \times n\)
			C. \(n \times m\)
			D. \(m \times n\)}
		{\(\mathbf{A}_{m \times l} \mathbf{C}_{l \times n} \mathbf{B}_{n \times m}=\mathbf{D}_{m \times m}\).}
	\end{question}
	
	\begin{question}
		{选择题}
		{矩阵}
		{中等}
		{18. 设 \(\mathbf{A}=\left(\begin{array}{ll}1 & 2 \\ 3 & 4 \\ 5 & 6\end{array}\right), \mathbf{B}=\left(\begin{array}{lll}1 & 2 & 3 \\ 4 & 5 & 6\end{array}\right)\),则 \(\mathbf{A}\) 与 \(\mathbf{B}\) 两个矩阵可以做()运算.
			A. 乘法
			B. 加法
			C. 减法
			D. 不能做任何运算}
		{乘法运算,因为 \(\mathbf{A B}=\left(\begin{array}{cc}1 & 2 \\ 3 & 4 \\ 5 & 6\end{array}\right)\left(\begin{array}{ccc}1 & 2 & 3 \\ 4 & 5 & 6\end{array}\right)=\left(\begin{array}{ccc}9 & 12 & 15 \\ 19 & 26 & 33 \\ 29 & 40 & 51\end{array}\right)\).}
	\end{question}
	
	\begin{question}
		{选择题}
		{矩阵}
		{中等}
		{19. 设 \(\mathbf{A}\) 是 \(m \times n\) 阶矩阵 \((m \neq n)\),则 \(\mathbf{A}_{m \times n}\) 可以().
			A. 左乘 \(m\) 阶单位矩阵,右乘 \(n\) 阶单位矩阵
			B. 左乘 \(n\) 阶单位矩阵
			C. 右乘 \(m\) 阶单位矩阵
			D. 乘任意阶单位矩阵}
		{对于矩阵 \(\mathbf{A}_{m \times n}(m \neq n)\),则 \(\mathbf{A}\) 只能左乘 \(m\) 阶单位矩阵,右乘 \(n\) 阶单位矩阵.}
	\end{question}
	
	\begin{question}
		{选择题}
		{矩阵}
		{容易}
		{20. \(\left(\begin{array}{ll}1 & 2 \\ 3 & 4\end{array}\right)\left(\begin{array}{ll}1 & 1 \\ 1 & 1\end{array}\right)=\)().
			A. \(\left(\begin{array}{ll}3 & 3 \\ 7 & 7\end{array}\right)\)
			B. \(\left(\begin{array}{ll}3 & 7 \\ 3 & 7\end{array}\right)\)
			C. \(\left(\begin{array}{ll}3 & 7 \\ 7 & 3\end{array}\right)\)
			D. \(\left(\begin{array}{ll}7 & 3 \\ 7 & 3\end{array}\right)\)}
		{\(\left(\begin{array}{ll}1 & 2 \\ 3 & 4\end{array}\right)\left(\begin{array}{ll}1 & 1 \\ 1 & 1\end{array}\right)=\left(\begin{array}{ll}3 & 3 \\ 7 & 7\end{array}\right)\).}
	\end{question}
	
	
	\begin{question}
		{选择题}
		{矩阵}
		{中等}
		{21. \(\left(\begin{array}{ll}2 & 0 \\ 3 & 4\end{array}\right)\left(\begin{array}{ll}2 & 1 \\ 4 & 7\end{array}\right)=\)().
			A. \(\left(\begin{array}{cc}4 & 2 \\ 22 & 31\end{array}\right)\)
			B. \(\left(\begin{array}{cc}4 & 0 \\ 22 & 31\end{array}\right)\)
			C. \(\left(\begin{array}{cc}4 & 2 \\ 20 & 31\end{array}\right)\)
			D. \(\left(\begin{array}{cc}4 & 2 \\ 22 & 28\end{array}\right)\)}
		{\(\left(\begin{array}{ll}2 & 0 \\ 3 & 4\end{array}\right)\left(\begin{array}{ll}2 & 1 \\ 4 & 7\end{array}\right)=\left(\begin{array}{cc}4 & 2 \\ 22 & 31\end{array}\right)\).}
	\end{question}
	
	\begin{question}
		{选择题}
		{矩阵}
		{中等}
		{22. \(\left(\begin{array}{cc}1 & 4 \\ -3 & 1\end{array}\right)\left(\begin{array}{cc}1 & 1 \\ 2 & -3\end{array}\right)=\)().
			A. \(\left(\begin{array}{cc}9 & -11 \\ -1 & -6\end{array}\right)\)
			B. \(\left(\begin{array}{cc}9 & 11 \\ -5 & 0\end{array}\right)\)
			C. \(\left(\begin{array}{cc}9 & -11 \\ 5 & 6\end{array}\right)\)
			D. \(\left(\begin{array}{cc}9 & -11 \\ -5 & -6\end{array}\right)\)}
		{\(\left(\begin{array}{cc}1 & 4 \\ -3 & 1\end{array}\right)\left(\begin{array}{cc}1 & 1 \\ 2 & -3\end{array}\right)=\left(\begin{array}{cc}9 & -11 \\ -1 & -6\end{array}\right)\).}
	\end{question}
	
	\begin{question}
		{选择题}
		{矩阵}
		{中等}
		{23. \(\left(\begin{array}{ll}1 & -1 \\ 2 & -3\end{array}\right)\left(\begin{array}{cc}2 & 3 \\ -1 & -4\end{array}\right)+\left(\begin{array}{cc}1 & 4 \\ -3 & -1\end{array}\right)=\)().
			A. \(\left(\begin{array}{ll}4 & 11 \\ 4 & 17\end{array}\right)\)
			B. \(\left(\begin{array}{cc}5 & -10 \\ 6 & 14\end{array}\right)\)
			C. \(\left(\begin{array}{ll}5 & 10 \\ 6 & 18\end{array}\right)\)
			D. \(\left(\begin{array}{ll}5 & 10 \\ 8 & 14\end{array}\right)\)}
		{\(\left(\begin{array}{ll}1 & -1 \\ 2 & -3\end{array}\right)\left(\begin{array}{cc}2 & 3 \\ -1 & -4\end{array}\right)+\left(\begin{array}{cc}1 & 4 \\ -3 &-1\end{array}\right)=\left(\begin{array}{cc}3 & 7 \\ 7 &18\end{array}\right)+\left(\begin{array}{cc}1 & 4 \\ -3 &-1\end{array}\right)=\left(\begin{array}{ll}4 & 11 \\ 4 & 17\end{array}\right)\).}
	\end{question}
	
	\begin{question}
		{选择题}
		{矩阵}
		{中等}
		{24. \(\left(\begin{array}{cc}1 & 4 \\ -3 & -1\end{array}\right)\left(\begin{array}{cc}1 & -1 \\ 2 & -3\end{array}\right)+\left(\begin{array}{cc}-3 & 3 \\ -1 & 4\end{array}\right)=\)().
			A. \(\left(\begin{array}{cc}6 & -10 \\ -6 & 10\end{array}\right)\)
			B. \(\left(\begin{array}{cc}11 & 10 \\ -6 & 2\end{array}\right)\)
			C. \(\left(\begin{array}{cc}11 & -10 \\ 6 & 2\end{array}\right)\)
			D. \(\left(\begin{array}{cc}11 & 10 \\ 6 & 2\end{array}\right)\)}
		{\(\left(\begin{array}{cc}1 & 4 \\ -3 & -1\end{array}\right)\left(\begin{array}{cc}1 & -1 \\ 2 & -3\end{array}\right)+\left(\begin{array}{cc}-3 & 3 \\ -1 & 4\end{array}\right)=\left(\begin{array}{cc}9 & -13 \\ -5 & 6\end{array}\right)+\left(\begin{array}{cc}-3 & 3 \\ -1 & 4\end{array}\right)=\left(\begin{array}{cc}6 & -10 \\ -6 & 10\end{array}\right)\).}
	\end{question}
	
	\begin{question}
		{选择题}
		{矩阵}
		{容易}
		{25. \(\left(\begin{array}{ll}1 & \lambda \\ 0 & 1\end{array}\right)^2=\)().
			A. \(\left(\begin{array}{cc}1 & 2 \lambda \\ 0 & 1\end{array}\right)\)
			B. \(\left(\begin{array}{ll}1 & \lambda \\ 0 & 1\end{array}\right)\)
			C. \(\left(\begin{array}{cc}1 & 3 \lambda \\ 0 & 1\end{array}\right)\)
			D. \(\left(\begin{array}{cc}1 & 4 \lambda \\ 0 & 1\end{array}\right)\)}
		{\(\left(\begin{array}{ll}1 & \lambda \\ 0 & 1\end{array}\right)\left(\begin{array}{ll}1 & \lambda \\ 0 & 1\end{array}\right)=\left(\begin{array}{cc}1 & 2 \lambda \\ 0 & 1\end{array}\right)\).}
	\end{question}
	
	\begin{question}
		{选择题}
		{矩阵}
		{中等}
		{26. \(\left(\begin{array}{ll}1 & -1 \\ 2 & -3\end{array}\right)^2=\)().
			A. \(\left(\begin{array}{ll}-1 & 2 \\ -4 & 7\end{array}\right)\)
			B. \(\left(\begin{array}{cc}-1 & 2 \\ 4 & 7\end{array}\right)\)
			C. \(\left(\begin{array}{cc}1 & 2 \\ -4 & 7\end{array}\right)\)
			D. \(\left(\begin{array}{cc}-1 & 2 \\ 4 & -7\end{array}\right)\)}
		{\(\left(\begin{array}{ll}1 & -1 \\ 2 & -3\end{array}\right)\left(\begin{array}{ll}1 & -1 \\ 2 & -3\end{array}\right)=\left(\begin{array}{ll}-1 & 2 \\ -4 & 7\end{array}\right)\).}
	\end{question}
	
	\begin{question}
		{选择题}
		{矩阵}
		{中等}
		{27. \(\left(\begin{array}{cc}1 & -1 \\ -2 & 3\end{array}\right)^2=\)().
			A. \(\left(\begin{array}{cc}3 & -4 \\ -8 & 11\end{array}\right)\)
			B. \(\left(\begin{array}{cc}3 & 4 \\ -8 & 11\end{array}\right)\)
			C. \(\left(\begin{array}{ll}3 & -4 \\ 8 & 11\end{array}\right)\)
			D. \(\left(\begin{array}{cc}3 & -4 \\ -8 & -11\end{array}\right)\)}
		{\(\left(\begin{array}{cc}1 & -1 \\ -2 & 3\end{array}\right)\left(\begin{array}{cc}1 & -1 \\ -2 & 3\end{array}\right)=\left(\begin{array}{cc}3 & -4 \\ -8 & 11\end{array}\right)\).}
	\end{question}
	
	\begin{question}
		{选择题}
		{矩阵}
		{困难}
		{28. 设 \(f(x)=a x^2+b x+c\),\(\mathbf{A}\) 为 2 阶矩阵,\(\mathbf{E}\) 为 2 阶单位矩阵,定义 \(f(\mathbf{A})=a \mathbf{A}^2+b \mathbf{A}+c \mathbf{E}\). 已知 \(f(x)=x^2-5 x+3\),\(\mathbf{A}=\left(\begin{array}{cc}2 & -1 \\ -3 & 3\end{array}\right)\),则 \(f(\mathbf{A})=\)().
			A. \(\left(\begin{array}{ll}0 & 0 \\ 0 & 0\end{array}\right)\)
			B. \(\left(\begin{array}{ll}1 & 0 \\ 0 & 1\end{array}\right)\)
			C. \(\left(\begin{array}{ll}1 & 0 \\ 0 & 0\end{array}\right)\)
			D. \(\left(\begin{array}{ll}0 & 1 \\ 0 & 0\end{array}\right)\)}
		{\(f(\mathbf{A})=\mathbf{A}^2-5 \mathbf{A}+3 \mathbf{E}=\left(\begin{array}{cc}2 & -1 \\ -3 & 3\end{array}\right)^2-5\left(\begin{array}{cc}2 & -1 \\ -3 & 3\end{array}\right)+3\left(\begin{array}{ll}1 & 0 \\ 0 & 1\end{array}\right)=\left(\begin{array}{cc}7 & -5 \\ -15 & 12\end{array}\right)-\left(\begin{array}{cc}10 & -5 \\ -15 & 15\end{array}\right)+\left(\begin{array}{ll}3 & 0 \\ 0 & 3\end{array}\right)=\left(\begin{array}{ll}0 & 0 \\ 0 & 0\end{array}\right)\).}
	\end{question}
	
	\begin{question}
		{选择题}
		{矩阵}
		{困难}
		{29. 设 \(f(x)=a x^2+b x+c\),\(\mathbf{A}\) 为 2 阶矩阵,\(\mathbf{E}\) 为 2 阶单位矩阵. 定义 \(f(\mathbf{A})=a \mathbf{A}^2+b \mathbf{A}+c \mathbf{E}\). 已知 \(f(x)=x^2-2 x+5\),\(\mathbf{A}=\left(\begin{array}{cc}2 & -1 \\ -3 & 3\end{array}\right)\),则 \(f(\mathbf{A})=\)().
			A. \(\left(\begin{array}{cc}8 & -3 \\ -9 & 11\end{array}\right)\)
			B. \(\left(\begin{array}{cc}8 & 3 \\ -9 & 11\end{array}\right)\)
			C. \(\begin{array}{rr}8 & -3 \\ 0 & 11\end{array}\)
			D. \(\left(\begin{array}{cc}8 & 3 \\ 9 & 11\end{array}\right)\)}
		{\(f(\mathbf{A})=\mathbf{A}^2-2 \mathbf{A}+5 \mathbf{E}=\left(\begin{array}{cc}2 & -1 \\ -3 & 3\end{array}\right)^2-2\left(\begin{array}{cc}2 & -1 \\ -3 & 3\end{array}\right)+5\left(\begin{array}{ll}1 & 0 \\ 0 & 1\end{array}\right)=\left(\begin{array}{cc}7 & -5 \\ -15 & 12\end{array}\right)-\left(\begin{array}{cc}4 & -2 \\ -6 & 6\end{array}\right)+\left(\begin{array}{cc}5 & 0 \\ 0 & 5\end{array}\right)=\left(\begin{array}{cc}8 & -3 \\ -9 & 11\end{array}\right)\).}
	\end{question}
	
	\begin{question}
		{选择题}
		{矩阵}
		{困难}
		{30. 设 \(f(x)=a x^2+b x+c\),\(\mathbf{A}\) 为 2 阶矩阵,\(\mathbf{E}\) 为 2 阶单位矩阵. 定义 \(f(\mathbf{A})=a \mathbf{A}^2+b \mathbf{A}+c \mathbf{E}\). 已知 \(f(x)=x^2+x-5\),\(\mathbf{A}=\left(\begin{array}{cc}2 & -1 \\ -3 & 3\end{array}\right)\),则 \(f(\mathbf{A})=\)().
			A. \(\left(\begin{array}{cc}4 & -6 \\ -18 & 10\end{array}\right)\)
			B. \(\left(\begin{array}{cc}4 & 6 \\ -18 & 10\end{array}\right)\)
			C. \(\left(\begin{array}{cc}4 & -6 \\ 18 & 10\end{array}\right)\)
			D. \(\left(\begin{array}{cc}4 & 6 \\ 18 & 10\end{array}\right)\)}
		{\(f(\mathbf{A})=\mathbf{A}^2+\mathbf{A}-5 \mathbf{E}=\left(\begin{array}{cc}2 & -1 \\ -3 & 3\end{array}\right)^2+\left(\begin{array}{cc}2 & -1 \\ -3 & 3\end{array}\right)-5\left(\begin{array}{ll}1 & 0 \\ 0 & 1\end{array}\right)=\left(\begin{array}{cc}7 & -5 \\ -15 & 12\end{array}\right)+\left(\begin{array}{cc}2 & -1 \\ -3 & 3\end{array}\right)-\left(\begin{array}{cc}5 & 0 \\ 0 & 5\end{array}\right)=\left(\begin{array}{cc}4 & -6 \\ -18 & 10\end{array}\right)\).}
	\end{question}
	
	
	
	
	\begin{question}
		{选择题}
		{矩阵}
		{容易}
		{31.\(\left(\begin{array}{ll}1 & 0 \\ 0 & 2\end{array}\right)^3=()\).
			A. \(\left(\begin{array}{ll}1 & 0 \\ 0 & 8\end{array}\right)\)
			B. \(\left(\begin{array}{ll}1 & 0 \\ 0 & 1\end{array}\right)\)
			C. \(\left(\begin{array}{ll}1 & 2 \\ 0 & 1\end{array}\right)\)
			D. \(\left(\begin{array}{ll}1 & 4 \\ 0 & 1\end{array}\right)\)}
		{\(\left(\begin{array}{ll}1 & 0 \\ 0 & 2\end{array}\right)^3=\left(\begin{array}{ll}1 & 0 \\ 0 & 8\end{array}\right)\).}
	\end{question}
	
	\begin{question}
		{选择题}
		{矩阵}
		{容易}
		{32.\(\left(\begin{array}{cc}-1 & 0 \\ 0 & 2\end{array}\right)^3=()\).
			A. \(\left(\begin{array}{cc}-1 & 0 \\ 0 & 8\end{array}\right)\)
			B. \(\left(\begin{array}{ll}1 & 0 \\ 0 & 1\end{array}\right)\)
			C. \(\left(\begin{array}{ll}1 & 2 \\ 0 & 1\end{array}\right)\)
			D. \(\left(\begin{array}{ll}1 & 4 \\ 0 & 1\end{array}\right)\)}
		{\(\left(\begin{array}{cc}-1 & 0 \\ 0 & 2\end{array}\right)^3=\left(\begin{array}{cc}-1 & 0 \\ 0 & 8\end{array}\right)\).}
	\end{question}
	
	\begin{question}
		{选择题}
		{矩阵}
		{容易}
		{33.\(\left(\begin{array}{cc}-2 & 0 \\ 0 & -2\end{array}\right)^3=()\).
			A. \(\left(\begin{array}{cc}-8 & 0 \\ 0 & -8\end{array}\right)\)
			B. \(\left(\begin{array}{ll}1 & 0 \\ 0 & 1\end{array}\right)\)
			C. \(\left(\begin{array}{ll}1 & 2 \\ 0 & 1\end{array}\right)\)
			D. \(\left(\begin{array}{ll}1 & 4 \\ 0 & 1\end{array}\right)\)}
		{\(\left(\begin{array}{cc}-2 & 0 \\ 0 & -2\end{array}\right)^3=\left(\begin{array}{cc}-8 & 0 \\ 0 & -8\end{array}\right)\).}
	\end{question}
	
	\begin{question}
		{选择题}
		{矩阵}
		{中等}
		{34.矩阵\(\left(\begin{array}{ll}3 & 2 \\ 1 & 4\end{array}\right)\)的行列式值为().
			A. 10
			B. -10
			C. 4
			D. 3}
		{\(\left|\begin{array}{ll}3 & 2 \\ 1 & 4\end{array}\right|=10\).}
	\end{question}
	
	\begin{question}
		{选择题}
		{矩阵}
		{容易}
		{35.\((k \mathbf{A})^T=()\).
			A. \(k \mathbf{A}^T\)
			B. \(\frac{1}{k} \mathbf{A}^T\)
			C. \(-k \mathbf{A}^T\)
			D. \(k \mathbf{A}\)}
		{\((k \mathrm{A})^T=k \mathrm{A}^T\).}
	\end{question}
	
	\begin{question}
		{选择题}
		{矩阵}
		{中等}
		{36.矩阵\(\left(\begin{array}{cc}3 & 10 \\ -1 & -4\end{array}\right)\)的行列式值为().
			A. -2
			B. -10
			C. 4
			D. 3}
		{\(\left|\begin{array}{cc}3 & 10 \\ -1 & -4\end{array}\right|=-2\).}
	\end{question}
	
	\begin{question}
		{选择题}
		{矩阵}
		{中等}
		{37.\((\mathbf{A B})^T=()\).
			A. \(\mathbf{B}^T \mathbf{A}^T\)
			B. \(\mathbf{A}^T \mathbf{B}^T\)
			C. \(\mathbf{A B}\)
			D. \(\mathbf{B A}\)}
		{\((\mathbf{A B})^T=\mathbf{B}^T \mathbf{A}^T\).}
	\end{question}
	
	\begin{question}
		{选择题}
		{矩阵}
		{中等}
		{38.设\(\mathbf{A}\)为3阶方阵,\(|\mathbf{A}|=-2\),则\(|2 \mathbf{A}|=\)().
			A. -16
			B. -4
			C. -2
			D. 16}
		{\(|2 \mathbf{A}|=2^3|\mathbf{A}|=-16\).}
	\end{question}
	
	\begin{question}
		{选择题}
		{矩阵}
		{困难}
		{39.设\(\mathbf{A}\)为\(n\)阶方阵,\(|\mathbf{A}|=2\),则\(\left|3 \mathbf{A}^2\right|=\)().
			A. \(3^n \cdot 4\)
			B. 12
			C. \(3^n 2\)
			D. 6}
		{\(\left|3 \mathbf{A}^2\right|=3^n|\mathbf{A}|^2=3^n \cdot 4\).}
	\end{question}
	
	\begin{question}
		{选择题}
		{矩阵}
		{困难}
		{40.设\(\mathbf{A}\)为3阶方阵,\(|\mathbf{A}|=2\),则\(\left|3 \mathbf{A}^2\right|=()\).
			A. 108
			B. 12
			C. 6
			D. 54}
		{\(\left|3 \mathbf{A}^2\right|=3^3|\mathbf{A}|^2=108\).}
	\end{question}
	
	
	\begin{question}
		{选择题}
		{矩阵}
		{中等}
		{41.设\(A\)与\(B\)都是同阶方阵,则下列运算错误的是().
			A. \((\mathbf{A B})^T=\mathbf{A}^T \mathbf{B}^T\)
			B. \((\mathbf{A}+\mathbf{B})^T=\mathbf{A}^T+\mathbf{B}^T\)
			C. \((k \mathbf{A})^T=k \mathbf{A}^T\)
			D. \(\left(\mathbf{A}^T\right)^T=\mathbf{A}\)}
		{因为\((\mathbf{A B})^T=\mathbf{B}^T \mathbf{A}^T\),所以选项\((\mathbf{A B})^T=\mathbf{A}^T \mathbf{B}^T\)是错误的.}
	\end{question}
	
	\begin{question}
		{选择题}
		{矩阵}
		{中等}
		{42.设\(\mathbf{A}\)为3阶方阵,\(|\mathbf{A}|=2\),则\(\left|2 \mathbf{A}^2\right|=\)().
			A. 32
			B. 12
			C. 8
			D. 6}
		{\(\left|2 \mathbf{A}^2\right|=2^3|\mathbf{A}|^2=32\).}
	\end{question}
	
	\begin{question}
		{选择题}
		{矩阵}
		{中等}
		{43.设\(\mathbf{A}\)为2阶方阵,\(|\mathbf{A}|=2\),则\(\left|3 \mathbf{A}^2\right|=\)().
			A. 36
			B. 12
			C. 6
			D. 54}
		{\(\left|3 \mathbf{A}^2\right|=3^2|\mathbf{A}|^2=36\).}
	\end{question}
	
	\begin{question}
		{选择题}
		{矩阵}
		{中等}
		{44.设\(\mathbf{A}\)为3阶方阵,\(|\mathbf{A}|=-1\),则\(\left|2 \mathbf{A}^2\right|=\)().
			A. 8
			B. 10
			C. 12
			D. 6}
		{\(\left|2 \mathbf{A}^2\right|=2^3|\mathbf{A}|^2=8\).}
	\end{question}
	
	\begin{question}
		{选择题}
		{矩阵}
		{容易}
		{45.设\(\mathbf{A}\)为3阶方阵,\(|\mathbf{A}|=-1\),则\(\left|\mathbf{A}^2\right|=\)().
			A. 1
			B. 10
			C. 12
			D. 6}
		{\(\left|\mathbf{A}^2\right|=|\mathbf{A}|^2=1\).}
	\end{question}
	
	\begin{question}
		{选择题}
		{矩阵}
		{容易}
		{46.设\(\mathbf{A}\)为3阶方阵,\(|\mathbf{A}|=1\),则\(|-2 \mathbf{A}|=\)().
			A. -8
			B. 10
			C. 12
			D. 6}
		{\(|-2 \mathbf{A}|=-2^3|\mathbf{A}|=-8\).}
	\end{question}
	
	\begin{question}
		{选择题}
		{矩阵}
		{中等}
		{47.设\(\mathbf{A}\)为4阶方阵,\(|\mathbf{A}|=-1\),则\(\left|2 \mathbf{A}^2\right|=\)().
			A. 16
			B. 10
			C. 12
			D. 6}
		{\(\left|2 \mathbf{A}^2\right|=2^4|\mathbf{A}|^2=16\).}
	\end{question}
	
	\begin{question}
		{选择题}
		{矩阵}
		{容易}
		{48.设\(\mathbf{A}\)为\(n\)阶方阵,\(|\mathbf{A}|=2\),则\(\left|\mathbf{A}^2\right|=\)().
			A. 4
			B. 10
			C. 12
			D. 6}
		{\(\left|\mathbf{A}^2\right|=|\mathbf{A}|^2=4\).}
	\end{question}
	
	\begin{question}
		{选择题}
		{矩阵}
		{困难}
		{49.设\(\mathbf{A}, \mathbf{B}\)均为\(n\)阶方阵,\(\mathbf{A B}\)不可逆,则下列正确的是().
			A. \(\mathbf{A}, \mathbf{B}\)中至少有一个不可逆
			B. \(\mathbf{A}, \mathbf{B}\)都不可逆
			C. \(\mathbf{A}, \mathbf{B}\)都可逆
			D. \(\mathbf{A}, \mathbf{B}\)中至少有一个可逆}
		{由\(\mathbf{A B}\)不可逆知\(|\mathbf{A B}|=0 \Rightarrow|\mathbf{A}||\mathbf{B}|=0 \Rightarrow|\mathbf{A}|=0\)或\(|\mathbf{B}|=0\),故\(\mathbf{A}, \mathbf{B}\)中至少有一个不可逆.}
	\end{question}
	
	\begin{question}
		{选择题}
		{矩阵}
		{困难}
		{50.下列矩阵中,可逆的是().
			A. \(\left(\begin{array}{lll}1 & 0 & 0 \\ 1 & 1 & 1 \\ 1 & 0 & 1\end{array}\right)\)
			B. \(\left(\begin{array}{lll}1 & 1 & 0 \\ 2 & 2 & 0 \\ 0 & 0 & 1\end{array}\right)\)
			C. \(\left(\begin{array}{lll}1 & 1 & 0 \\ 0 & 1 & 1 \\ 1 & 2 & 1\end{array}\right)\)
			D. \(\left(\begin{array}{lll}0 & 0 & 0 \\ 0 & 1 & 0 \\ 0 & 0 & 1\end{array}\right)\)}
		{因为\(\left|\begin{array}{lll}1 & 0 & 0 \\ 1 & 1 & 1 \\ 1 & 0 & 1\end{array}\right|=1 \neq 0\),故\(\left(\begin{array}{lll}1 & 0 & 0 \\ 1 & 1 & 1 \\ 1 & 0 & 1\end{array}\right)\)可逆.}
	\end{question}
	
	\begin{question}
		{选择题}
		{矩阵}
		{中等}
		{51.设\(\mathbf{A}, \mathbf{B}\)均为\(n\)阶可逆矩阵,则下列各式中不正确的是().
			A. \((\mathbf{A}+\mathbf{B})^{-1}=\mathbf{A}^{-1}+\mathbf{B}^{-1}\)
			B. \((\mathbf{A}+\mathbf{B})^T=\mathbf{A}^T+\mathbf{B}^T\)
			C. \((\mathbf{A B})^{-1}=\mathbf{B}^{-1} \mathbf{A}^{-1}\)
			D. \((\mathbf{A B})^T=\mathbf{B}^T \mathbf{A}^T\)}
		{\(\mathbf{A}, \mathbf{B}\)都可逆,\(\mathbf{A}+\mathbf{B}\)不一定可逆,故\((\mathbf{A}+\mathbf{B})^{-1}=\mathbf{A}^{-1}+\mathbf{B}^{-1}\)不正确.}
	\end{question}
	
	\begin{question}
		{选择题}
		{矩阵}
		{困难}
		{52.若三阶方阵\(\mathbf{A}\)满足\(|\mathbf{A}|=3\),则\(\left|\mathbf{A}^*\right|=\)().
			A. 9
			B. 27
			C. 6
			D. -9}
		{由\(\mathbf{A}^*=|\mathbf{A}| \mathbf{A}^{-1}\)知\(\left|\mathbf{A}^*\right|=\left||\mathbf{A}| \mathbf{A}^{-1}\right|=|\mathbf{A}|^3 \cdot\left|\mathbf{A}^{-1}\right|=\frac{|\mathbf{A}|^3}{|\mathbf{A}|}=|\mathbf{A}|^2=9\).}
	\end{question}
	
	\begin{question}
		{选择题}
		{矩阵}
		{中等}
		{53.若\(\mathbf{A}, \mathbf{B}\)都是方阵且\(|\mathbf{A}|=2,|\mathbf{B}|=-1\),则\(\left|\mathbf{A}^{-1} \mathbf{B}\right|=\)().
			A. \(-\frac{1}{2}\)
			B. 2
			C. \(\frac{1}{2}\)
			D. -2}
		{\(\left|\mathbf{A}^{-1} \mathbf{B}\right|=\left|\mathbf{A}^{-1}\right| \cdot|\mathbf{B}|=\frac{1}{|\mathbf{A}|} \cdot|\mathbf{B}|=-\frac{1}{2}\).}
	\end{question}
	
	\begin{question}
		{选择题}
		{矩阵}
		{中等}
		{54.矩阵\(\mathbf{A}=\left(\begin{array}{ll}1 & 2 \\ 3 & 4\end{array}\right)\)的伴随矩阵\(\mathbf{A}^*=\)().
			A. \(\left(\begin{array}{cc}4 & -2 \\ -3 & 1\end{array}\right)\)
			B. \(\left(\begin{array}{ll}4 & 2 \\ 3 & 1\end{array}\right)\)
			C. \(\left(\begin{array}{cc}4 & -3 \\ -2 & 1\end{array}\right)\)
			D. \(\left(\begin{array}{cc}-4 & 2 \\ 3 & -1\end{array}\right)\)}
		{因为\(A_{11}=4, A_{21}=-2, A_{12}=-3, A_{22}=1\),故\(\mathbf{A}^*=\left(\begin{array}{cc}A_{11} & A_{21} \\ A_{12} & A_{22}\end{array}\right)=\left(\begin{array}{cc}4 & -2 \\ -3 & 1\end{array}\right)\).}
	\end{question}
	
	\begin{question}
		{选择题}
		{矩阵}
		{困难}
		{55.若矩阵\(\mathbf{A}=\left(\begin{array}{lll}0 & 2 & 0 \\ 0 & 0 & 3 \\ 4 & 0 & 0\end{array}\right)\),则\(\mathbf{A}\)的伴随矩阵\(\mathbf{A}^*=\)().
			A. \(\left(\begin{array}{ccc}0 & 0 & 6 \\ 12 & 0 & 0 \\ 0 & 8 & 0\end{array}\right)\)
			B. \(\left(\begin{array}{ccc}0 & 12 & 0 \\ 0 & 0 & 8 \\ 6 & 0 & 0\end{array}\right)\)
			C. \(\left(\begin{array}{ccc}0 & -12 & 0 \\ 0 & 0 & -8 \\ -6 & 0 & 0\end{array}\right)\)
			D. \(\left(\begin{array}{ccc}0 & 0 & -6 \\ -12 & 0 & 0 \\ 0 & -8 & 0\end{array}\right)\)}
		{因为\(A_{11}=0, A_{21}=0, A_{31}=6, A_{12}=12, A_{22}=0, A_{32}=0, A_{13}=0, A_{23}=8, A_{33}=0\),故\(\mathbf{A}^*=\left(\begin{array}{ccc}A_{11} & A_{21} & A_{31} \\ A_{12} & A_{22} & A_{32} \\ A_{13} & A_{23} & A_{33}\end{array}\right)=\left(\begin{array}{ccc}0 & 0 & 6 \\ 12 & 0 & 0 \\ 0 & 8 & 0\end{array}\right)\).}
	\end{question}
	
	\begin{question}
		{选择题}
		{矩阵}
		{容易}
		{56.若矩阵\(\mathbf{A}=\left(\begin{array}{lll}1 & 0 & 0 \\ 0 & 2 & 0 \\ 0 & 0 & 3\end{array}\right)\),则\(\mathbf{A}\)的逆矩阵\(\mathbf{A}^{-1}=\)().
			A. \(\left(\begin{array}{ccc}1 & 0 & 0 \\ 0 & \frac{1}{2} & 0 \\ 0 & 0 & \frac{1}{3}\end{array}\right)\)
			B. \(\left(\begin{array}{ccc}\frac{1}{3} & 0 & 0 \\ 0 & \frac{1}{2} & 0 \\ 0 & 0 & 1\end{array}\right)\)
			C. \(\left(\begin{array}{ccc}\frac{1}{3} & 0 & 0 \\ 0 & 1 & 0 \\ 0 & 0 & \frac{1}{2}\end{array}\right)\)
			D. \(\left(\begin{array}{ccc}\frac{1}{2} & 0 & 0 \\ 0 & \frac{1}{3} & 0 \\ 0 & 0 & 1\end{array}\right)\)}
		{因为\(|\mathbf{A}|=6 \neq 0\),故\(\mathbf{A}\)可逆.又\(\mathbf{A}\)是对角阵,故\(\mathbf{A}^{-1}=\left(\begin{array}{ccc}1 & 0 & 0 \\ 0 & \frac{1}{2} & 0 \\ 0 & 0 & \frac{1}{3}\end{array}\right)\).}
	\end{question}
	
	\begin{question}
		{选择题}
		{矩阵}
		{中等}
		{57.设\(\mathbf{A}, \mathbf{B}\)均为\(n\)阶方阵,\(\mathbf{A}\)可逆,则下列命题正确的是().
			A. 若\(\mathbf{A B}=\mathbf{O}\),则\(\mathbf{B}=\mathbf{O}\)
			B. 若\(\mathbf{A B} \neq \mathbf{O}\),则\(\mathbf{B}\)可逆
			C. 若\(\mathbf{A B} \neq \mathbf{O}\),则\(\mathbf{B}\)不可逆
			D. 若\(\mathbf{A B}=\mathbf{B A}\),则\(\mathbf{B}=\mathbf{E}\)}
		{因为\(\mathbf{A}\)可逆且\(\mathbf{A B}=\mathbf{O}\),故\(\mathbf{B}=\mathbf{A}^{-1} \mathbf{O}=\mathbf{O}\).}
	\end{question}
	
	\begin{question}
		{选择题}
		{矩阵}
		{困难}
		{58.设\(\mathbf{A}\)为\(n\)阶矩阵,\(\mathbf{A}^*\)是\(\mathbf{A}\)的伴随矩阵,则下列正确的是().
			A. \(\left|\mathbf{A}^*\right|=|\mathbf{A}|^{n-1}\)
			B. \(\left|\mathbf{A}^*\right|=|\mathbf{A}|\)
			C. \(\left|\mathbf{A}^*\right|=|\mathbf{A}|^n\)
			D. \(\left|\mathbf{A}^*\right|=\left|\mathbf{A}^{-1}\right|\)}
		{1. \(\mathbf{A}\)为可逆矩阵时,\(\left|\mathbf{A}^*\right|=|\mathbf{A}|^{n-1}\);2. \(\mathbf{A}\)为不可逆矩阵时,\(\left|\mathbf{A}^*\right|=0=|\mathbf{A}|^{n-1}\).}
	\end{question}
	
	\begin{question}
		{选择题}
		{矩阵}
		{困难}
		{59.设\(\mathbf{A}\)为\(n\)阶可逆矩阵\((n \geq 2)\),\(\mathbf{A}^*\)是\(\mathbf{A}\)的伴随矩阵,则下列正确的是().
			A. \(\left(\mathbf{A}^*\right)^*=|\mathbf{A}|^{n-2} \mathbf{A}\)
			B. \(\left(\mathbf{A}^*\right)^*=|\mathbf{A}|^{n-1} \mathbf{A}\)
			C. \(\left(\mathbf{A}^*\right)^*=|\mathbf{A}|^{n+1} \mathbf{A}\)
			D. \(\left(\mathbf{A}^*\right)^*=|\mathbf{A}|^{n+2} \mathbf{A}\)}
		{由\(\mathbf{A}^*=|\mathbf{A}| \mathbf{A}^{-1}\)知\(\left(\mathbf{A}^*\right)^*=|\mathbf{A}|^{n-2} \mathbf{A}\).}
	\end{question}
	
	\begin{question}
		{选择题}
		{矩阵}
		{中等}
		{60.设\(\mathbf{A}\)为任意\(n\)阶矩阵\((n \geq 3)\),\(k\)为常数且\(k \neq 0, \pm 1\),则必有\((k \mathbf{A})^*=\)().
			A. \(k^{n-1} \mathbf{A}^*\)
			B. \(k \mathbf{A}^*\)
			C. \(k^n \mathbf{A}^*\)
			D. \(\frac{1}{k} \mathbf{A}^*\)}
		{由\(\mathbf{A}^*=|\mathbf{A}| \mathbf{A}^{-1}\)知\((k \mathbf{A})^*=k^{n-1} \mathbf{A}^*\).}
	\end{question}
	\begin{question}
		{选择题}
		{矩阵}
		{中等}
		{61. 若三阶方阵 \(\mathbf{A}\) 满足 \(\mathbf{A}=-2\),则 \(\left|2 \mathbf{A}^{-1}\right|=\)().
			A. -4
			B. -1
			C. 4
			D. 1}
		{因为 \(\left|\mathbf{A}^{-1}\right|=\frac{1}{|\mathbf{A}|}=-\frac{1}{2}\),所以 \(\left|2 \mathbf{A}^{-1}\right|=2^3\left|\mathbf{A}^{-1}\right|=-4\).}
	\end{question}
	
	\begin{question}
		{选择题}
		{矩阵}
		{中等}
		{62. 若矩阵 \(x\) 满足 \(\left(\begin{array}{cc}2 & 0 \\ -1 & 1\end{array}\right) \mathbf{X}=\left(\begin{array}{cc}3 & 1 \\ 0 & -1\end{array}\right)\),则 \(X=\)().
			A. \(\left(\begin{array}{cc}\frac{3}{2} & \frac{1}{2} \\ \frac{3}{2} & -\frac{1}{2}\end{array}\right)\)
			B. \(\left(\begin{array}{rr}3 & -\frac{1}{2} \\ 0 & \frac{1}{2}\end{array}\right)\)
			C. \(\left(\begin{array}{cc}-\frac{3}{2} & \frac{1}{2} \\ -\frac{3}{2} & -\frac{1}{2}\end{array}\right)\)
			D. \(\left(\begin{array}{ll}\frac{3}{2} & -\frac{1}{2} \\ \frac{1}{2} & -\frac{1}{2}\end{array}\right)\)}
		{由 \(\left|\begin{array}{cc}2 & 0 \\ -1 & 1\end{array}\right|=2 \neq 0\) 知 \(\left(\begin{array}{cc}2 & 0 \\ -1 & 1\end{array}\right)\) 可逆,所以 \(\mathbf{X}=\left(\begin{array}{cc}2 & 0 \\ -1 & 1\end{array}\right)^{-1}\left(\begin{array}{cc}3 & 1 \\ 0 & -1\end{array}\right)=\left(\begin{array}{ll}\frac{1}{2} & 0 \\ \frac{1}{2} & 1\end{array}\right)\left(\begin{array}{cc}3 & 1 \\ 0 & -1\end{array}\right)=\left(\begin{array}{cc}\frac{3}{2} & \frac{1}{2} \\ \frac{3}{2} & -\frac{1}{2}\end{array}\right)\).}
	\end{question}
	
	\begin{question}
		{选择题}
		{矩阵}
		{困难}
		{63. 若矩阵 \(\mathbf{A}=\left(\begin{array}{ccc}2 & 0 & 0 \\ 0 & -1 & -1 \\ 0 & 1 & 2\end{array}\right)\),则 \(\mathbf{A}\) 的逆矩阵 \(\mathbf{A}^{-1}=\)().
			A. \(\left(\begin{array}{ccc}\frac{1}{2} & 0 & 0 \\ 0 & -2 & -1 \\ 0 & 1 & 1\end{array}\right)\)
			B. \(\left(\begin{array}{ccc}\frac{1}{2} & 0 & 0 \\ 0 & 2 & 1 \\ 0 & -1 & -1\end{array}\right)\)
			C. \(\left(\begin{array}{ccc}2 & 1 & 0 \\ -1 & -1 & 0 \\ 0 & 0 & \frac{1}{2}\end{array}\right)\)
			D. \(\left(\begin{array}{ccc}-2 & -1 & 0 \\ 1 & 1 & 0 \\ 0 & 0 & 2\end{array}\right)\)}
		{因为 \(|\mathbf{A}|=\left|\begin{array}{ccc}2 & 0 & 0 \\ 0 & -1 & -1 \\ 0 & 1 & 2\end{array}\right|=-2 \neq 0\),所以 \(\mathbf{A}\) 可逆,又 \(A_{11}=-1, A_{21}=0, A_{31}=0, A_{12}=0, A_{22}=4, A_{32}=2, A_{13}=0 A_{23}=-2, A_{33}=-2\) 故 \(\mathbf{A}^*=\left(\begin{array}{ccc}-1 & 0 & 0 \\ 0 & 4 & 2 \\ 0 & -2 & -2\end{array}\right)\),所以 \(\mathbf{A}^{-1}=\frac{\mathbf{A}^*}{|\mathbf{A}|}=\left(\begin{array}{ccc}\frac{1}{2} & 0 & 0 \\ 0 & -2 & -1 \\ 0 & 1 & 1\end{array}\right)\).}
	\end{question}
	
	\begin{question}
		{选择题}
		{矩阵}
		{中等}
		{64. 若 \(\mathbf{A}\) 是二阶可逆方阵且 \(\mathbf{A}^{-1}=\left(\begin{array}{cc}-3 & 7 \\ 1 & -2\end{array}\right)\),则 \(\mathbf{A}=\)().
			A. \(\left(\begin{array}{ll}2 & 7 \\ 1 & 3\end{array}\right)\)
			B. \(\left(\begin{array}{cc}-2 & 7 \\ 1 & -3\end{array}\right)\)
			C. \(\left(\begin{array}{cc}2 & -7 \\ -1 & 3\end{array}\right)\)
			D. \(\left(\begin{array}{ll}3 & 7 \\ 1 & 2\end{array}\right)\)}
		{因为 \(\left|\mathbf{A}^{-1}\right|=\left|\begin{array}{cc}-3 & 7 \\ 1 & -2\end{array}\right|=-1 \neq 0\),又 \(\mathbf{A}^{-1}\) 的伴随矩阵 \(\left(\mathbf{A}^{-1}\right)^*=\left(\begin{array}{cc}-2 & -7 \\ -1 & -3\end{array}\right)\),所以 \(\mathbf{A}=\frac{\left(\mathbf{A}^{-1}\right)^*}{\left|\mathbf{A}^{-1}\right|}=\left(\begin{array}{ll}2 & 7 \\ 1 & 3\end{array}\right)\).}
	\end{question}
	
	\begin{question}
		{选择题}
		{矩阵}
		{中等}
		{65. 设 \(\mathbf{A}\) 为 \(n\) 阶 \((n \geq 2)\) 方阵,\(|\mathbf{A}|=a \neq 0\),则 \(\mathbf{A}^* \mid=\)().
			A. \(a^{n-1}\)
			B. \(a^{-1}\)
			C. \(a\)
			D. \(a^n\)}
		{由 \(\mathbf{A}^*=|\mathbf{A}| \mathbf{A}^{-1}\) 知 \(\left|\mathbf{A}^*\right|=\left||\mathbf{A}| \mathbf{A}^{-1}\right|=|\mathbf{A}|^n \cdot\left|\mathbf{A}^{-1}\right|=\frac{|\mathbf{A}|^n}{|\mathbf{A}|}=|\mathbf{A}|^{n-1}=a^{n-1}\).}
	\end{question}
	
	\begin{question}
		{选择题}
		{矩阵}
		{困难}
		{66. 若三阶方阵 \(\mathbf{A}\) 满足 \(|\mathbf{A}|=\frac{1}{2}\),则 \(\left|(2 \mathbf{A})^{-1}-5 \mathbf{A}^*\right|=\)().
			A. -16
			B. \(-\frac{1}{16}\)
			C. 16
			D. \(\frac{1}{16}\)}
		{因为 \(\mathbf{A}^*=|\mathbf{A}| \mathbf{A}^{-1}\),所以 \(\left|(2 \mathbf{A})^{-1}-5 \mathbf{A}^*\right|=\left|\frac{1}{2} \mathbf{A}^{-1}-\frac{5}{2} \mathbf{A}^{-1}\right|=\left|-2 \mathbf{A}^{-1}\right|=(-2)^3 \frac{1}{|\mathbf{A}|}=-16\).}
	\end{question}
	
	\begin{question}
		{选择题}
		{矩阵}
		{困难}
		{67. 若三阶方阵 \(\mathbf{A}\) 满足 \(|\mathbf{A}|=\frac{1}{4}\),则 \(\left|(3 \mathbf{A})^{-1}-4 \mathbf{A}^*\right|=\)().
			A. \(-\frac{32}{27}\)
			B. \(\frac{16}{27}\)
			C. \(\frac{8}{9}\)
			D. \(-\frac{8}{9}\)}
		{因为 \(\mathbf{A}^*=|\mathbf{A}| \mathbf{A}^{-1}\),所以 \(\left|(3 \mathbf{A})^{-1}-4 \mathbf{A}^*\right|=\left|-\frac{2}{3} \mathbf{A}^{-1}\right|=\left(-\frac{2}{3}\right)^3 \frac{1}{|\mathbf{A}|}=-\frac{32}{27}\).}
	\end{question}
	
	\begin{question}
		{选择题}
		{矩阵}
		{容易}
		{68. 克莱姆法则适用于下面哪种类型的方程组().
			A. 方程的个数等于末知数的个数
			B. 方程的个数小于未知数的个数
			C. 方程的个数大于未知数的个数
			D. 任意}
		{克莱姆法则适用于方程的个数等于未知数的个数的方程组.}
	\end{question}
	
	\begin{question}
		{选择题}
		{矩阵}
		{容易}
		{69. 以下不能用克莱姆法则求解的方程组是().
			A. \(\left\{\begin{array}{l}x_1-x_2+2 x_4=-5 \\ 3 x_1+2 x_2-x_3-2 x_4=6 \\ 4 x_1+3 x_2-x_3-x_4=0\end{array}\right.\)
			B. \(\left\{\begin{array}{l}6 x_1-4 x_2=10 \\ 5 x_1+7 x_2=29\end{array}\right.\)
			C. \(\left\{\begin{array}{l}x_1-x_2+2 x_4=-5 \\ 3 x_1+2 x_2-x_3-2 x_4=6 \\ 4 x_1+3 x_2-x_3-x_4=0 \\ 2 x_1-x_3=0\end{array}\right.\)
			D. \(\left\{\begin{array}{l}x_1+x_2-2 x_3=-3 \\ 5 x_1-2 x_2+7 x_3=22 \\ 2 x_1-5 x_2+4 x_3=4\end{array}\right.\)}
		{因为 \(\left\{\begin{array}{l}x_1-x_2+2 x_4=-5 \\ 3 x_1+2 x_2-x_3-2 x_4=6 \\ 4 x_1+3 x_2-x_3-x_4=0\end{array}\right.\) 中方程个数与未知数的个数不相等,所以不能用克莱姆法则.}
	\end{question}
	
	\begin{question}
		{选择题}
		{矩阵}
		{中等}
		{70. 当()时,齐次线性方程组 \(\left\{\begin{array}{c}x+2 y-z=0 \\ x+y-2 z=0 \\ x+k z=0\end{array}\right.\) 仅有零解.
			A. \(k \neq-3\)
			B. \(k \neq 3\)
			C. \(k=-3\)
			D. \(k=3\)}
		{方程组的系数行列式 \(\left|\begin{array}{lll}1 & 2 & -1 \\ 1 & 1 & -2 \\ 1 & 0 & k\end{array}\right|=-k-3 \neq 0 \Rightarrow k \neq-3\),方程组仅有零解.}
	\end{question}
	
	
	\begin{question}
		{选择题}
		{矩阵}
		{中等}
		{71. 当()时,齐次线性方程组 \(\left\{\begin{array}{l}3 x-z=0 \\ k x+y-2 z=0 \\ -y+3 z=0\end{array}\right.\) 有非零解.
			A. \(k=-3\)
			B. \(k \neq 6\)
			C. \(k \neq-6\)
			D. \(k=-6\)}
		{方程组的系数行列式 \(\operatorname{det} A=\left|\begin{array}{ccc}3 & 0 & -1 \\ k & 1 & -2 \\ 0 & -1 & 3\end{array}\right|=3+k\),当 \(k=-3\) 时,\(\operatorname{det} A=0\),方程组有非零解.}
	\end{question}
	
	\begin{question}
		{选择题}
		{矩阵}
		{中等}
		{72. 当()时,齐次线性方程组 \(\left\{\begin{array}{l}x+z=0 \\ 2 x+k y-2 z=0 \\ -y+2 z=0\end{array}\right.\) 仅有零解.
			A. \(k \neq 2\)
			B. \(k \neq-2\)
			C. \(k=2\)
			D. \(k=-2\)}
		{方程组的系数行列式 \(\operatorname{det} A=\left|\begin{array}{ccc}1 & 0 & 1 \\ 2 & k & -2 \\ 0 & -1 & 2\end{array}\right|=2 k-4\),当 \(k \neq 2\) 时,\(\operatorname{det} A \neq 0\),方程组仅有零解.}
	\end{question}
	
	\begin{question}
		{选择题}
		{矩阵}
		{困难}
		{73. 当()时,齐次线性方程组 \(\left\{\begin{array}{c}3 x+2 y-z=0 \\ k x+7 y-2 z=0 \\ 2 x-y+3 z=0\end{array}\right.\) 仅有零解.
			A. \(k \neq \frac{63}{5}\)
			B. \(k=\frac{63}{5}\)
			C. \(k \neq-\frac{63}{5}\)
			D. \(k=-\frac{63}{5}\)}
		{方程组的系数行列式 \(\operatorname{det} A=\left|\begin{array}{ccc}3 & 2 & -1 \\ k & 7 & -2 \\ 2 & -1 & 3\end{array}\right|=63-5 k\),当 \(k \neq \frac{63}{5}\) 时,\(\operatorname{det} A \neq 0\),方程组仅有零解.}
	\end{question}
	
	\begin{question}
		{选择题}
		{矩阵}
		{中等}
		{74. 当()时,齐次线性方程组 \(\left\{\begin{array}{l}k x+2 y=0 \\ x+2 y-2 z=0 \\ 2 x+z=0\end{array}\right.\) 仅有零解.
			A. \(k \neq 5\)
			B. \(k=5\)
			C. \(k \neq-5\)
			D. \(k=-5\)}
		{方程组的系数行列式 \(\operatorname{det} A=\left|\begin{array}{ccc}k & 2 & 0 \\ 1 & 2 & -2 \\ 2 & 0 & 1\end{array}\right|=2 k-10\),当 \(k \neq 5\) 时,\(\operatorname{det} A \neq 0\),方程组仅有零解.}
	\end{question}
	
	
	
	%%%%%%%%%%%%%%%%%%%%%%%%%%%%%%%%%%%%%%%%%%%%%%%%%%%%%%%%%%%%%%%%%%%%%%
	%%%%%%%%%%%%%%%%%%%%%%%%%%%%%%%%%%%%%%%%%%%%%%%%%%%%%%%%%%%%%%%%%%%%%%
	%%%%%%%%%%%%%%%%%%%%%%%%%%%%%%%%%%%%%%%%%%%%%%%%%%%%%%%%%%%%%%%%%%%%%%
	
	\begin{question}
		{判断题}
		{矩阵}
		{容易}
		{1.任意两个矩阵都能相减().}
		{两个同型矩阵才能相减,由此知结论错误.}
	\end{question}
	
	\begin{question}
		{判断题}
		{矩阵}
		{容易}
		{2. 两个同型矩阵一定相等().}
		{两个同型矩阵所有对应元素都对应相等,这两矩阵一定相等,所以结论错误.}
	\end{question}
	
	\begin{question}
		{判断题}
		{矩阵}
		{容易}
		{3. 两个同型矩阵相加,是所有对应位置元素对应相加().}
		{两个同型矩阵相加,是所有对应位置元素对应相加,所以结论正确.}
	\end{question}
	
	\begin{question}
		{判断题}
		{矩阵}
		{容易}
		{4. 同型矩阵才能相加().}
		{两个同型矩阵才能相加,所以结论正确.}
	\end{question}
	
	\begin{question}
		{判断题}
		{矩阵}
		{容易}
		{5. 同型矩阵相加不满足交换律().}
		{同型矩阵相加满足交换律,所以结论错误.}
	\end{question}
	
	\begin{question}
		{判断题}
		{矩阵}
		{容易}
		{6. 同型矩阵相加不满足结合律().}
		{同型矩阵相加满足结合律,所以结论错误.}
	\end{question}
	
	\begin{question}
		{判断题}
		{矩阵}
		{容易}
		{7. 两个同型矩阵不一定相等().}
		{两个同型矩阵所有对应元素都对应相等,这两矩阵一定相等,所以结论错误.}
	\end{question}
	
	\begin{question}
		{判断题}
		{矩阵}
		{中等}
		{8. \(\mathbf{A}=\left(\begin{array}{cc}x & 0 \\ 7 & y\end{array}\right), \mathbf{B}=\left(\begin{array}{cc}4 & -2 \\ y & 2\end{array}\right), \mathbf{C}=\left(\begin{array}{cc}3 & -4 \\ x & -2\end{array}\right)\) 且 \(\mathbf{A}+2\mathbf{B}-\mathbf{C}=\mathbf{O}\),则\(x, y\)的值为\(x=5, y=6\)().}
		{由\(\mathbf{A}+2\mathbf{B}-\mathbf{C}=\mathbf{O}\)得\(\left(\begin{array}{cc}x+5 & 0 \\ 7+2y-x & y+6\end{array}\right)=\mathbf{0}\),有\(x=-5, y=-6\),所以结论错误.}
	\end{question}
	
	\begin{question}
		{判断题}
		{矩阵}
		{中等}
		{9. \(\mathbf{A}=\left(\begin{array}{cc}x & 0 \\ 7 & -6\end{array}\right), \mathbf{B}=\left(\begin{array}{cc}u & -2 \\ -6 & 2\end{array}\right), \mathbf{C}=\left(\begin{array}{cc}3 & -4 \\ x & -2\end{array}\right)\) 且 \(\mathbf{A}+2\mathbf{B}-\mathbf{C}=\mathbf{O}\),则\(x, u\)的值为\(x=5, u=-4\)().}
		{由\(\mathbf{A}+2\mathbf{B}-\mathbf{C}=\mathbf{O}\)得\(\left(\begin{array}{cc}x+2u-3 & 0 \\ -5-x & 0\end{array}\right)=\mathbf{0}\),有\(x=-5, u=4\),所以结论错误.}
	\end{question}
	
	\begin{question}
		{判断题}
		{矩阵}
		{困难}
		{10. \(\mathbf{A}=\left(\begin{array}{cc}-5 & 0 \\ 7 & y\end{array}\right), \mathbf{B}=\left(\begin{array}{ll}u & v \\ y & 2\end{array}\right), \mathbf{C}=\left(\begin{array}{cc}3 & -4 \\ -5 & v\end{array}\right)\) 且 \(\mathbf{A}+2\mathbf{B}-\mathbf{C}=\mathbf{O}\),则\(v, u, y\)的值为\(v=-2, u=4, y=-6\)().}
		{由\(\mathbf{A}+2\mathbf{B}-\mathbf{C}=\mathbf{O}\)得\(\left(\begin{array}{cc}-8+2u & 2v+4 \\ 12+2y & y+4-v\end{array}\right)=\mathbf{0}\),所以\(v=-2, u=4, y=-6\),结论正确.}
	\end{question}
	
	
	\begin{question}
		{判断题}
		{矩阵}
		{中等}
		{11. \(\mathbf{A}=\left(\begin{array}{cc}x & 0 \\ 7 & y\end{array}\right), \mathbf{B}=\left(\begin{array}{cc}u & -2 \\ y & 2\end{array}\right), \mathbf{C}=\left(\begin{array}{cc}3 & -4 \\ x & -2\end{array}\right)\) 且 \(\mathbf{A}+2\mathbf{B}-\mathbf{C}=\mathbf{O}\),则\(x, y\)的值为\(x=-5, y=-6, u=4\)().}
		{由\(\mathbf{A}+2\mathbf{B}-\mathbf{C}=\mathbf{O}\)得\(\left(\begin{array}{cc}x+2u-3 & 0 \\ 7+2y-x & y+6\end{array}\right)=\mathbf{0}\),有\(x=-5, y=-6, u=4\),所以结论正确.}
	\end{question}
	
	\begin{question}
		{判断题}
		{矩阵}
		{中等}
		{12. \(\mathbf{A}=\left(\begin{array}{ll}x & 0 \\ 7 & y\end{array}\right), \mathbf{B}=\left(\begin{array}{cc}4 & -2 \\ y & 2\end{array}\right), \mathbf{C}=\left(\begin{array}{cc}3 & -4 \\ x & -2\end{array}\right)\) 且 \(\mathbf{A}+2\mathbf{B}-\mathbf{C}=\mathbf{O}\),则\(x, y\)的值为\(x=5, y=6\)().}
		{由\(\mathbf{A}+2\mathbf{B}-\mathbf{C}=\mathbf{O}\)得\(\left(\begin{array}{cc}x+5 & 0 \\ 7+2y-x & y+6\end{array}\right)=\mathbf{O}\),有\(x=-5, y=-6\),所以结论错误.}
	\end{question}
	
	\begin{question}
		{判断题}
		{矩阵}
		{容易}
		{13. 矩阵乘法满足交换律().}
		{矩阵乘法不满足交换律,例如\(\mathbf{AB}\neq\mathbf{BA}\),所以结论错误.}
	\end{question}
	
	\begin{question}
		{判断题}
		{矩阵}
		{容易}
		{14. 矩阵乘法不满足结合律().}
		{矩阵乘法满足结合律,所以结论错误.}
	\end{question}
	
	\begin{question}
		{判断题}
		{矩阵}
		{容易}
		{15. 矩阵乘法不满足分配律().}
		{矩阵乘法满足分配律,所以结论错误.}
	\end{question}
	
	\begin{question}
		{判断题}
		{矩阵}
		{容易}
		{16. 两同阶矩阵不能相乘().}
		{两同阶矩阵不一定不能相乘,例如同阶方阵可以相乘,所以结论错误.}
	\end{question}
	
	\begin{question}
		{判断题}
		{矩阵}
		{中等}
		{17. 两同阶下三角矩阵乘积仍为下三角矩阵().}
		{两同阶下三角矩阵乘积仍为下三角矩阵,所以结论正确.}
	\end{question}
	
	\begin{question}
		{判断题}
		{矩阵}
		{中等}
		{18. 两同阶上三角形矩阵乘积不一定是上三角形矩阵().}
		{两同阶上三角形矩阵乘积仍为上三角形矩阵,所以结论错误.}
	\end{question}
	
	\begin{question}
		{判断题}
		{矩阵}
		{容易}
		{19. 因为\(m \neq n\),所以\(\mathbf{A}_{m \times s}, \mathbf{B}_{s \times n}\)不同阶,因此不能相乘().}
		{可以相乘,\(\mathbf{A}_{m \times s}\mathbf{B}_{s \times n}=\mathbf{C}_{m \times n}\),所以结论错误.}
	\end{question}
	
	\begin{question}
		{判断题}
		{矩阵}
		{中等}
		{20. 若\(\mathbf{AB}\)与\(\mathbf{BA}\)都有意义,则\(\mathbf{A}\)、\(\mathbf{B}\)一定为同阶方阵().}
		{不一定为同阶方阵,例如\(\mathbf{A}_{n \times 1}\)与\(\mathbf{B}_{n \times 1}\)的情况,所以结论错误.}
	\end{question}
	
	\begin{question}
		{判断题}
		{矩阵}
		{中等}
		{21. 两个矩阵如果可以相乘的话,其结果不可能是一个常数().}
		{例如\(\mathbf{A}_{1 \times n}\mathbf{B}_{n \times 1}\)的结果是\(1\times1\)矩阵(常数),所以结论错误.}
	\end{question}
	
	\begin{question}
		{判断题}
		{矩阵}
		{容易}
		{22. 单位矩阵乘任何矩阵仍为任何矩阵().}
		{单位矩阵不一定能乘任何矩阵(需满足乘法条件),所以结论错误.}
	\end{question}
	
	\begin{question}
		{判断题}
		{矩阵}
		{中等}
		{23. \(\mathbf{AX}=\mathbf{AY}\),则\(\mathbf{X}=\mathbf{Y}\)().}
		{矩阵乘法不满足消去律,所以结论错误.}
	\end{question}
	
	\begin{question}
		{判断题}
		{矩阵}
		{中等}
		{24. 两同阶对角矩阵乘积仍为对角矩阵().}
		{两同阶对角矩阵乘积仍为对角矩阵,所以结论正确.}
	\end{question}
	
	\begin{question}
		{判断题}
		{矩阵}
		{困难}
		{25. 对于矩阵\(\mathbf{A}, \mathbf{B}\)有\((\mathbf{A}+\mathbf{B})^2=\mathbf{A}^2+2\mathbf{AB}+\mathbf{B}^2\)().}
		{仅当\(\mathbf{AB}=\mathbf{BA}\)时成立,所以结论错误.}
	\end{question}
	
	
	\begin{question}
		{判断题}
		{矩阵}
		{中等}
		{26. 设\(\mathbf{A},\mathbf{B}\)都是\(n\)阶矩阵,若\(\mathbf{AB}=\mathbf{B}\),则\(\mathbf{A}\)是常数1().}
		{由\(n\)阶矩阵\(\mathbf{AB}=\mathbf{B}\),则\(\mathbf{A}\)是\(n\)阶单位矩阵,由此知结论错误.}
	\end{question}
	
	\begin{question}
		{判断题}
		{矩阵}
		{中等}
		{27. \(\mathbf{A}^2=\mathbf{O}\),则\(\mathbf{A}\)一定为零矩阵().}
		{由\(\mathbf{A}^2=\mathbf{O}\),\(\mathbf{A}\)不一定为零矩阵,所以结论错误.}
	\end{question}
	
	\begin{question}
		{判断题}
		{矩阵}
		{中等}
		{28. \((\mathbf{AB})^2=\mathbf{A}^2\mathbf{B}^2\)().}
		{\((\mathbf{AB})^2=\mathbf{ABAB}\),\(\mathbf{AB}\)不一定可交换,所以结论错误.}
	\end{question}
	
	\begin{question}
		{判断题}
		{矩阵}
		{中等}
		{29. \((\mathbf{AB})^k=\mathbf{A}^k\mathbf{B}^k\)().}
		{\((\mathbf{AB})^k=\overbrace{\mathbf{AB}\cdots\mathbf{AB}}^k\),\(\mathbf{AB}\)不一定可交换,所以结论错误.}
	\end{question}
	
	\begin{question}
		{判断题}
		{矩阵}
		{容易}
		{30. \(\mathbf{E}\)为单位阵,\(k\)为正整数,\(\mathbf{E}^k=\mathbf{E}\)().}
		{\(\mathbf{E}^k=\mathbf{E}\),\(k\)为正整数,所以结论正确.}
	\end{question}
	
	\begin{question}
		{判断题}
		{矩阵}
		{中等}
		{31. \(\mathbf{A}^2=\mathbf{A}\),则\(\mathbf{A}\)一定为零矩阵或单位阵().}
		{由\(\mathbf{A}^2=\mathbf{A}\),\(\mathbf{A}\)不一定为零矩阵或单位阵,所以结论错误.}
	\end{question}
	
	\begin{question}
		{判断题}
		{矩阵}
		{中等}
		{32. \(\mathbf{A}^2-\mathbf{E}^2=(\mathbf{A}+\mathbf{E})(\mathbf{A}-\mathbf{E})\)().}
		{矩阵乘法满足此展开式,所以结论正确.}
	\end{question}
	
	\begin{question}
		{判断题}
		{矩阵}
		{中等}
		{33. \(\mathbf{A}^2=\mathbf{O}\),则\(\mathbf{A}\)不一定为零矩阵().}
		{由\(\mathbf{A}^2=\mathbf{O}\),\(\mathbf{A}\)不一定为零矩阵,所以结论正确.}
	\end{question}
	
	\begin{question}
		{判断题}
		{矩阵}
		{中等}
		{34. \(\mathbf{A}^2=\mathbf{A}\),则\(\mathbf{A}\)一定为单位阵().}
		{由\(\mathbf{A}^2=\mathbf{A}\),\(\mathbf{A}\)不一定为单位阵,所以结论错误.}
	\end{question}
	
	\begin{question}
		{判断题}
		{矩阵}
		{困难}
		{35. \(\mathbf{A}^2=\mathbf{E}\),\(\mathbf{E}\)为单位矩阵,则\(\mathbf{A}=\mathbf{E}\)().}
		{反例:\(\mathbf{A}=\left(\begin{array}{ll}0 & 1 \\ 1 & 0\end{array}\right)\)满足\(\mathbf{A}^2=\mathbf{E}\)但\(\mathbf{A}\neq\mathbf{E}\),所以结论错误.}
	\end{question}
	
	\begin{question}
		{判断题}
		{矩阵}
		{困难}
		{36. \((\mathbf{A}-\mathbf{B})^2=\mathbf{A}^2-2\mathbf{AB}+\mathbf{B}^2\)成立的充要条件为\(\mathbf{A}=\mathbf{E}\)().}
		{充要条件是\(\mathbf{AB}=\mathbf{BA}\),所以结论错误.}
	\end{question}
	
	\begin{question}
		{判断题}
		{矩阵}
		{困难}
		{37. \((\mathbf{A}+\mathbf{B})^2=\mathbf{A}^2+2\mathbf{AB}+\mathbf{B}^2\)成立的充要条件为\(\mathbf{AB}=\mathbf{BA}\)().}
		{当且仅当\(\mathbf{AB}=\mathbf{BA}\)时成立,所以结论正确.}
	\end{question}
	
	\begin{question}
		{判断题}
		{行列式}
		{中等}
		{38. 设\(\mathbf{A}\)为\(n\)阶矩阵,则\(|-2\mathbf{A}|=-2|\mathbf{A}|\)().}
		{\(|-2\mathbf{A}|=(-2)^n|\mathbf{A}|\),所以结论错误.}
	\end{question}
	
	\begin{question}
		{判断题}
		{行列式}
		{中等}
		{39. 设\(\mathbf{A}\)为2阶矩阵,\(k\)为常数,则\(|k\mathbf{A}|=2k|\mathbf{A}|\)().}
		{\(|k\mathbf{A}|=k^2|\mathbf{A}|\),所以结论错误.}
	\end{question}
	
	\begin{question}
		{判断题}
		{行列式}
		{中等}
		{40. 设\(\mathbf{A}\)为\(n\)阶矩阵,\(\mathbf{A}^T\)为\(\mathbf{A}\)的转置矩阵,则\(|\mathbf{AA}^T|=|\mathbf{A}|^2\)().}
		{\(|\mathbf{AA}^T|=|\mathbf{A}||\mathbf{A}^T|=|\mathbf{A}|^2\),所以结论正确.}
	\end{question}
	
	\begin{question}
		{判断题}
		{行列式}
		{中等}
		{41. 设\(\mathbf{A}\)为\(n\)阶矩阵,\(k\)为常数,则\(|k\mathbf{A}|=k|\mathbf{A}|\)().}
		{\(|k\mathbf{A}|=k^n|\mathbf{A}|\),所以结论错误.}
	\end{question}
	
	\begin{question}
		{判断题}
		{矩阵}
		{容易}
		{42. 设\(\mathbf{A}\)与\(\mathbf{B}\)为同阶矩阵,则\((\mathbf{A}+\mathbf{B})^T=\mathbf{A}^T+\mathbf{B}^T\)().}
		{转置运算的线性性质,所以结论正确.}
	\end{question}
	
	\begin{question}
		{判断题}
		{矩阵}
		{容易}
		{43. 设\(\mathbf{A}\)与\(\mathbf{B}\)为同阶矩阵,则\((\mathbf{A}-\mathbf{B})^T=\mathbf{A}^T-\mathbf{B}^T\)().}
		{转置运算的线性性质,所以结论正确.}
	\end{question}
	
	\begin{question}
		{判断题}
		{行列式}
		{困难}
		{44. 设\(\mathbf{A}\)为\(n\)阶矩阵,\(k\)为常数,则\(|k\mathbf{A}^{-1}|=k^n|\mathbf{A}|^{-1}\)().}
		{逆矩阵行列式性质,所以结论正确.}
	\end{question}
	
	\begin{question}
		{判断题}
		{行列式}
		{中等}
		{45. 设\(\mathbf{A}\)为\(n\)阶矩阵,则\(|2\mathbf{A}|=2|\mathbf{A}|\)().}
		{\(|2\mathbf{A}|=2^n|\mathbf{A}|\),所以结论错误.}
	\end{question}
	
	\begin{question}
		{判断题}
		{行列式}
		{中等}
		{46. 设\(\mathbf{A}\)为3阶矩阵,\(k\)为常数,则\(|k\mathbf{A}|=k|\mathbf{A}|\)().}
		{\(|k\mathbf{A}|=k^3|\mathbf{A}|\),所以结论错误.}
	\end{question}
	
	\begin{question}
		{判断题}
		{行列式}
		{容易}
		{47. 设\(\mathbf{A}\)为\(n\)阶方阵,则矩阵\(\mathbf{A}\)与\(\mathbf{A}\)的行列式相等,即\(\mathbf{A}=|\mathbf{A}|\)().}
		{矩阵与行列式是不同概念,所以结论错误.}
	\end{question}
	
	\begin{question}
		{判断题}
		{行列式}
		{困难}
		{48. 设\(\mathbf{A}\)为\(n\)阶矩阵,\(\mathbf{E}\)为\(n\)阶单位阵,则\(|\mathbf{A}+\mathbf{E}|=|\mathbf{A}|+1\)().}
		{反例存在不成立情况,所以结论错误.}
	\end{question}
	
	\begin{question}
		{判断题}
		{行列式}
		{困难}
		{49. 设\(\mathbf{A},\mathbf{B}\)为\(n\)阶矩阵,\(|\mathbf{A}-\mathbf{B}| \geq |\mathbf{A}|-|\mathbf{B}|\)().}
		{存在反例使不等式不成立,所以结论错误.}
	\end{question}
	
	\begin{question}
		{判断题}
		{矩阵}
		{中等}
		{50. 若\(\mathbf{A}\)为\(n\)阶可逆矩阵,则\(\mathbf{A}^*\)也可逆().}
		{由\(|\mathbf{A}^*|=|\mathbf{A}|^{n-1}\neq 0\)可知结论正确.}
	\end{question}
	
	\begin{question}
		{判断题}
		{矩阵}
		{困难}
		{51. 若\(\mathbf{AA}^T=\mathbf{E}\),则\((\mathbf{A}^*)^T=(\mathbf{A}^*)^{-1}\)().}
		{通过伴随矩阵与逆矩阵关系可证,所以结论正确.}
	\end{question}
	
	\begin{question}
		{判断题}
		{矩阵}
		{中等}
		{52. 若\(\mathbf{A}\)为\(n\)阶可逆矩阵,则\((\mathbf{A}^*)^T=(\mathbf{A}^T)^*\)().}
		{通过伴随矩阵定义可证,所以结论正确.}
	\end{question}
	
	\begin{question}
		{判断题}
		{矩阵}
		{中等}
		{53. 设\(\mathbf{A},\mathbf{B}\)均为\(n\)阶方阵,若\(\mathbf{A}+\mathbf{B}\)与\(\mathbf{A}-\mathbf{B}\)均可逆,则\(\mathbf{A},\mathbf{B}\)一定可逆().}
		{不能推出\(\mathbf{A},\mathbf{B}\)可逆的结论,所以结论错误.}
	\end{question}
	
	\begin{question}
		{判断题}
		{行列式}
		{中等}
		{54. 若三阶方阵\(\mathbf{A}=\left(\begin{array}{lll}2 & 1 & 0 \\ 1 & 1 & 0 \\ 0 & 0 & 2\end{array}\right)\),则\(|\mathbf{A}^*|=4\)().}
		{计算得\(|\mathbf{A}|=2\),\(|\mathbf{A}^*|=|\mathbf{A}|^{n-1}=4\),所以结论正确.}
	\end{question}
	
	\begin{question}
		{判断题}
		{矩阵}
		{中等}
		{55. 若矩阵\(\mathbf{A}\)可逆,则\((\mathbf{A}^T)^{-1}=(\mathbf{A}^{-1})^T\)().}
		{通过逆矩阵性质可证,所以结论正确.}
	\end{question}
	
	\begin{question}
		{判断题}
		{矩阵}
		{容易}
		{56. 设\(\mathbf{A},\mathbf{B}\)均可逆,则\(\mathbf{AB}\)也可逆,且\((\mathbf{AB})^{-1}=\mathbf{A}^{-1}\mathbf{B}^{-1}\)().}
		{正确关系应为\((\mathbf{AB})^{-1}=\mathbf{B}^{-1}\mathbf{A}^{-1}\),所以结论错误.}
	\end{question}
	
	\begin{question}
		{判断题}
		{矩阵}
		{中等}
		{57. 若方阵\(\mathbf{A}\)满足\(\mathbf{A}^2-2\mathbf{A}-4\mathbf{E}=\mathbf{O}\),则\(\mathbf{A}^{-1}=\frac{\mathbf{A}-2\mathbf{E}}{2}\)().}
		{正确逆矩阵应为\(\frac{\mathbf{A}-2\mathbf{E}}{4}\),所以结论错误.}
	\end{question}
	
	\begin{question}
		{判断题}
		{矩阵}
		{中等}
		{58. 若方阵\(\mathbf{A}\)满足\(\mathbf{A}^2-\mathbf{A}-2\mathbf{E}=\mathbf{O}\),则\(\mathbf{A}\)不可逆().}
		{\(\mathbf{A}\)实际可逆且\(\mathbf{A}^{-1}=\frac{\mathbf{A}-\mathbf{E}}{2}\),所以结论错误.}
	\end{question}
	
	\begin{question}
		{判断题}
		{矩阵}
		{中等}
		{59. 设\(\mathbf{A},\mathbf{B},\mathbf{C}\)均为\(n\)阶方阵,若\(\mathbf{ABC}=\mathbf{E}\),则\(\mathbf{C}^{-1}=\mathbf{B}^{-1}\mathbf{A}^{-1}\)().}
		{正确关系应为\(\mathbf{C}^{-1}=\mathbf{AB}\),所以结论错误.}
	\end{question}
	
	
	
	%%%%%%%%%%%%%%%%%%%%%%%%%%%%%%%%%%%%%%%%%%%%%%%%%%%%%%%%%%%%%%%%%%%%%%%%%%%%%%%%%%%%%%%%%%%%%%%%
	%%%%%%%%%%%%%%%%%%%%%%%%%%%%%%%%%%%%%%%%%%%%%%%%%%%%%%%%%%%%%%%%%%%%%%%%%%%%%%%%%%%%%%%%%%%%%%%%
	%%%%%%%%%%%%%%%%%%%%%%%%%%%%%%%%%%%%%%%%%%%%%%%%%%%%%%%%%%%%%%%%%%%%%%%%%%%%%%%%%%%%%%%%%%%%%%%%
	
	\begin{question}
		{填空题}
		{矩阵}
		{容易}
		{1.\(3\left(\begin{array}{lll}1 & 0 & 4 \\ 1 & 2 & 0\end{array}\right)-2\left(\begin{array}{lll}0 & 1 & 1 \\ 2 & 3 & 1\end{array}\right)=()\).}
		{\(3\left(\begin{array}{lll}1 & 0 & 4 \\ 1 & 2 & 0\end{array}\right)-2\left(\begin{array}{lll}0 & 1 & 1 \\ 2 & 3 & 1\end{array}\right)=\left(\begin{array}{ccc}3 & -2 & 10 \\ -1 & 0 & -2\end{array}\right)\).}
	\end{question}
	
	\begin{question}
		{填空题}
		{矩阵}
		{容易}
		{2.\(\left(\begin{array}{lll}1 & 0 & 4 \\ 1 & 2 & 0\end{array}\right)-4\left(\begin{array}{lll}0 & 1 & 1 \\ 2 & 1 & 4\end{array}\right)=()\).}
		{\(\left(\begin{array}{lll}1 & 0 & 4 \\ 1 & 2 & 0\end{array}\right)-4\left(\begin{array}{lll}0 & 1 & 1 \\ 2 & 1 & 4\end{array}\right)=\left(\begin{array}{ccc}1 & -4 & 0 \\ -7 & -2 & -16\end{array}\right)\).}
	\end{question}
	
	\begin{question}
		{填空题}
		{矩阵}
		{容易}
		{3.\(3\left(\begin{array}{lll}1 & 0 & 4 \\ 1 & 2 & 0\end{array}\right)-2\left(\begin{array}{lll}0 & 1 & 1 \\ 2 & 3 & 4\end{array}\right)=()\).}
		{\(3\left(\begin{array}{lll}1 & 0 & 4 \\ 1 & 2 & 0\end{array}\right)-2\left(\begin{array}{lll}0 & 1 & 1 \\ 2 & 3 & 4\end{array}\right)=\left(\begin{array}{ccc}3 & -2 & 10 \\ -1 & 0 & -8\end{array}\right)\).}
	\end{question}
	
	\begin{question}
		{填空题}
		{矩阵}
		{容易}
		{4.\(\left(\begin{array}{lll}1 & 0 & 4 \\ 1 & 2 & 0\end{array}\right)-4\left(\begin{array}{lll}0 & 1 & 1 \\ 2 & 3 & 4\end{array}\right)=()\).}
		{\(\left(\begin{array}{lll}1 & 0 & 4 \\ 1 & 2 & 0\end{array}\right)-4\left(\begin{array}{lll}0 & 1 & 1 \\ 2 & 3 & 4\end{array}\right)=\left(\begin{array}{ccc}1 & -4 & 0 \\ -7 & -10 & -16\end{array}\right)\).}
	\end{question}
	
	\begin{question}
		{填空题}
		{矩阵}
		{中等}
		{5.\(\left(\begin{array}{lll}1 & 2 & 1\end{array}\right)\left(\begin{array}{c}-1 \\ 2 \\ 3\end{array}\right)=()\).}
		{\(\left(\begin{array}{lll}1 & 2 & 1\end{array}\right)\left(\begin{array}{c}-1 \\ 2 \\ 3\end{array}\right)=6\).}
	\end{question}
	
	\begin{question}
		{填空题}
		{矩阵}
		{中等}
		{6.\(\left(\begin{array}{lll}1 & 2 & 0\end{array}\right)\left(\begin{array}{l}1 \\ 2 \\ 3\end{array}\right)=()\).}
		{\(\left(\begin{array}{lll}1 & 2 & 0\end{array}\right)\left(\begin{array}{l}1 \\ 2 \\ 3\end{array}\right)=5\).}
	\end{question}
	
	\begin{question}
		{填空题}
		{矩阵}
		{中等}
		{7.\(\left(\begin{array}{l}1 \\ 2 \\ 3\end{array}\right)\left(\begin{array}{ll}-1 & 2\end{array}\right)=()\).}
		{\(\left(\begin{array}{l}1 \\ 2 \\ 3\end{array}\right)\left(\begin{array}{ll}-1 & 2\end{array}\right)=\left(\begin{array}{ll}-1 & 2 \\ -2 & 4 \\ -3 & 6\end{array}\right)\).}
	\end{question}
	
	\begin{question}
		{填空题}
		{矩阵}
		{中等}
		{8.\(\left(\begin{array}{l}1 \\ 2 \\ 3\end{array}\right)\left(\begin{array}{lll}3 & 2 & 1\end{array}\right)=()\).}
		{\(\left(\begin{array}{l}1 \\ 2 \\ 3\end{array}\right)\left(\begin{array}{lll}3 & 2 & 1\end{array}\right)=\left(\begin{array}{lll}3 & 2 & 1 \\ 6 & 4 & 2 \\ 9 & 6 & 3\end{array}\right)\).}
	\end{question}
	
	\begin{question}
		{填空题}
		{矩阵}
		{困难}
		{9.\(\left(\begin{array}{lll}5 & 0 & 0 \\ 0 & 3 & 1 \\ 0 & 2 & 1\end{array}\right)\left(\begin{array}{c}1 \\ -2 \\ 3\end{array}\right)=()\).}
		{\(\left(\begin{array}{lll}5 & 0 & 0 \\ 0 & 3 & 1 \\ 0 & 2 & 1\end{array}\right)\left(\begin{array}{c}1 \\ -2 \\ 3\end{array}\right)=\left(\begin{array}{c}5 \\ -3 \\ -1\end{array}\right)\).}
	\end{question}
	
	\begin{question}
		{填空题}
		{矩阵}
		{中等}
		{10.\(\left(\begin{array}{lll}3 & 2 & 1\end{array}\right)\left(\begin{array}{l}1 \\ 2 \\ 0\end{array}\right)=()\).}
		{\(\left(\begin{array}{lll}3 & 2 & 1\end{array}\right)\left(\begin{array}{l}1 \\ 2 \\ 0\end{array}\right)=7\).}
	\end{question}
	
	\begin{question}
		{填空题}
		{矩阵}
		{容易}
		{11.\(\left(\begin{array}{ll}1 & 0 \\ 0 & 3\end{array}\right)^2=()\). }
		{\(\left(\begin{array}{ll}1 & 0 \\ 0 & 3\end{array}\right)\left(\begin{array}{ll}1 & 0 \\ 0 & 3\end{array}\right)=\left(\begin{array}{ll}1 & 0 \\ 0 & 9\end{array}\right)\). }
	\end{question}
	
	\begin{question}
		{填空题}
		{矩阵}
		{中等}
		{12.\(\left(\begin{array}{ll}1 & 1 \\ 0 & 1\end{array}\right)^3=()\). }
		{\(\left(\begin{array}{ll}1 & 1 \\ 0 & 1\end{array}\right)^3=\left(\begin{array}{ll}1 & 3 \\ 0 & 1\end{array}\right)\). }
	\end{question}
	
	\begin{question}
		{填空题}
		{矩阵}
		{中等}
		{13.\(\left(\begin{array}{ll}a & 0 \\ 0 & b\end{array}\right)^n=()\). }
		{\(\left(\begin{array}{ll}a & 0 \\ 0 & b\end{array}\right)^n=\left(\begin{array}{cc}a^n & 0 \\ 0 & b^n\end{array}\right)\). }
	\end{question}
	
	\begin{question}
		{填空题}
		{矩阵}
		{中等}
		{14.设矩阵\(\mathbf{A}\)为偶数阶矩阵,则\(|-\mathbf{A}|=()\). }
		{\(|-\mathbf{A}|=(-1)^{2n}|\mathbf{A}|=|\mathbf{A}|\). }
	\end{question}
	
	\begin{question}
		{填空题}
		{矩阵}
		{困难}
		{15.设\(\mathbf{A}\)为3阶方阵,\(|\mathbf{A}|=2\),则\(\left|\frac{1}{2}\mathbf{A}^2\right|=()\). }
		{\(\left|\frac{1}{2}\mathbf{A}^2\right|=\left(\frac{1}{2}\right)^3|\mathbf{A}|^2=\frac{1}{2}\). }
	\end{question}
	
	\begin{question}
		{填空题}
		{矩阵}
		{中等}
		{16.设\(\mathbf{A}\)为2阶方阵,\(|\mathbf{A}|=3\),则\(\left|\frac{1}{3}\mathbf{A}^2\right|=()\). }
		{\(\left|\frac{1}{3}\mathbf{A}^2\right|=\left(\frac{1}{3}\right)^2|\mathbf{A}|^2=1\). }
	\end{question}
	
	\begin{question}
		{填空题}
		{矩阵}
		{困难}
		{17.设\(\mathbf{A}\)为\(n\)阶方阵,\(|\mathbf{A}|=2\),则\(\left|\frac{1}{2}\mathbf{A}^2\right|=()\). }
		{\(\left|\frac{1}{2}\mathbf{A}^2\right|=\left(\frac{1}{2}\right)^n|\mathbf{A}|^2=\left(\frac{1}{2}\right)^n\cdot4\). }
	\end{question}
	
	\begin{question}
		{填空题}
		{矩阵}
		{中等}
		{18.设\(\mathbf{A}\)为\(n\)阶方阵,\(|\mathbf{A}|=2\),则\(\left|2\mathbf{A}^T\right|=()\). }
		{\(\left|2\mathbf{A}^T\right|=2^n|\mathbf{A}|=2^{n+1}\). }
	\end{question}
	
	\begin{question}
		{填空题}
		{矩阵}
		{中等}
		{19.设\(\mathbf{A}=\left(\begin{array}{ll}a & b \\ c & d\end{array}\right)\)且\(ad-bc\neq0\),则\(\mathbf{A}^{-1}=()\). }
		{\(\mathbf{A}^{-1}=\frac{1}{ad-bc}\left(\begin{array}{rr}d & -b \\ -c & a\end{array}\right)\). }
	\end{question}
	
	\begin{question}
		{填空题}
		{矩阵}
		{中等}
		{20.设矩阵\(\mathbf{A}=\left(\begin{array}{cc}1 & -4 \\ 0 & 3\end{array}\right)\),则\(\mathbf{A}^{-1}=()\). }
		{\(\mathbf{A}^{-1}=\left(\begin{array}{cc}1 & \frac{4}{3} \\ 0 & \frac{1}{3}\end{array}\right)\). }
	\end{question}
	
	\begin{question}
		{填空题}
		{矩阵}
		{困难}
		{21.设二阶方阵\(\mathbf{A}\)满足\(|\mathbf{A}|=\frac{1}{2}\),则\(\left|2\mathbf{A}^*\right|=()\). }
		{\(\left|2\mathbf{A}^*\right|=2\). }
	\end{question}
	
	\begin{question}
		{填空题}
		{矩阵}
		{困难}
		{22.设四阶方阵\(\mathbf{A}\)满足\(|\mathbf{A}|=-2\),则\(\left|-\mathbf{A}^*\right|=()\). }
		{\(\left|-\mathbf{A}^*\right|=-8\). }
	\end{question}
	
	\begin{question}
		{填空题}
		{矩阵}
		{困难}
		{23.若方阵\(\mathbf{A}\)满足\(\mathbf{A}^2+\mathbf{A}+\mathbf{E}=\mathbf{0}\),则\(\mathbf{A}^{-1}=()\). }
		{\(\mathbf{A}^{-1}=-\mathbf{A}-\mathbf{E}\). }
	\end{question}
	
	\begin{question}
		{填空题}
		{矩阵}
		{困难}
		{24.若矩阵\(\mathbf{X}\)满足\(\left(\begin{array}{ll}2 & 1 \\ 1 & 2\end{array}\right)\mathbf{X}=\left(\begin{array}{ll}1 & 2 \\ 1 & 4\end{array}\right)\),则\(\mathbf{X}=()\). }
		{\(\mathbf{X}=\left(\begin{array}{cc}\frac{1}{3} & 0 \\ \frac{1}{3} & 2\end{array}\right)\). }
	\end{question}
	
	\begin{question}
		{填空题}
		{矩阵}
		{困难}
		{25.已知矩阵\(\mathbf{A}=\left(\begin{array}{cc}2 & 0 \\ 0 & 1\end{array}\right),\mathbf{B}=\left(\begin{array}{cc}-1 & 1 \\ 2 & 5\end{array}\right)\),则\(\mathbf{B}^2-\mathbf{A}^2\mathbf{A}^{-1}\mathbf{B}=()\). }
		{\(\mathbf{B}^2-\mathbf{A}^2\mathbf{A}^{-1}\mathbf{B}=\left(\begin{array}{cc}5 & 2 \\ 6 & 22\end{array}\right)\). }
	\end{question}
	
	
	
	%%%%%%%%%%%%%%%%%%%%%%%%%%%%%%%%%%%%%%%%%%%%%%%%%%%%%%%%%%%%%%%%%%%%%%%%%%%%%%%%%%%%%%%%%%%%%%%%
	%%%%%%%%%%%%%%%%%%%%%%%%%%%%%%%%%%%%%%%%%%%%%%%%%%%%%%%%%%%%%%%%%%%%%%%%%%%%%%%%%%%%%%%%%%%%%%%%
	%%%%%%%%%%%%%%%%%%%%%%%%%%%%%%%%%%%%%%%%%%%%%%%%%%%%%%%%%%%%%%%%%%%%%%%%%%%%%%%%%%%%%%%%%%%%%%%%
	
	\begin{question}
		{计算题}
		{矩阵}
		{容易}
		{1.设 \(\mathbf{A}=\left(\begin{array}{lll}1 & 2 & 1 \\ 2 & 1 & 2 \\ 1 & 2 & 3\end{array}\right), \mathbf{B}=\left(\begin{array}{ccc}4 & 3 & 2 \\ -2 & 1 & -2 \\ 0 & -1 & 0\end{array}\right)\) ,若 \((2 \mathbf{A}-\mathbf{X})+2(\mathbf{B}-\mathbf{X})=\mathbf{O}\) ,则 \(\mathbf{X}=()\).}
		{\(\mathbf{X}=\frac{2}{3}(\mathbf{A}+\mathbf{B})=\frac{2}{3}\left[\left(\begin{array}{lll}1 & 2 & 1 \\ 2 & 1 & 2 \\ 1 & 2 & 3\end{array}\right)+\left(\begin{array}{ccc}4 & 3 & 2 \\ -2 & 1 & -2 \\ 0 & -1 & 0\end{array}\right)\right]=\frac{2}{3}\left(\begin{array}{lll}5 & 5 & 3 \\ 0 & 2 & 0 \\ 1 & 1 & 3\end{array}\right)=\) \(\left(\begin{array}{ccc}\frac{10}{3} & \frac{10}{3} & 2 \\ 0 & \frac{4}{3} & 0 \\ \frac{2}{3} & \frac{2}{3} & 2\end{array}\right)\).}
	\end{question}
	
	\begin{question}
		{计算题}
		{矩阵}
		{容易}
		{2.设 \(\mathbf{A}=\left(\begin{array}{lll}1 & 2 & 1 \\ 2 & 1 & 2 \\ 1 & 2 & 3\end{array}\right), \mathbf{B}=\left(\begin{array}{ccc}4 & 3 & 2 \\ -2 & 1 & -2 \\ 0 & -1 & 0\end{array}\right)\) ,若 \(\frac{1}{2} \mathbf{X}-\mathbf{B}=\mathbf{A}\) ,则 \(\mathbf{X}=()\).}
		{\(\mathbf{X}=2(\mathbf{A}+\mathbf{B})=2\left[\left(\begin{array}{ccc}1 & 2 & 1 \\ 2 & 1 & 2 \\ 1 & 2 & 3\end{array}\right)+\left(\begin{array}{ccc}4 & 3 & 2 \\ -2 & 1 & -2 \\ 0 & -1 & 0\end{array}\right)\right]=\left(\begin{array}{ccc}10 & 10 & 6 \\ 0 & 4 & 0 \\ 2 & 2 & 6\end{array}\right)\).}
	\end{question}
	
	\begin{question}
		{计算题}
		{矩阵}
		{容易}
		{3.设 \(\mathbf{A}=\left(\begin{array}{ccc}1 & -2 & 3 \\ 4 & 3 & -1\end{array}\right), \mathbf{B}=\left(\begin{array}{ccc}2 & 0 & 2 \\ 3 & 1 & 4\end{array}\right)\) ,若 \(3(\mathbf{X}-\mathbf{A})=3 \mathbf{B}+2 \mathbf{X}\) ,则 \(\mathbf{X}=()\).}
		{\(\mathbf{X}=3(\mathbf{A}+\mathbf{B})=3\left[\left(\begin{array}{ccc}1 & -2 & 3 \\ 4 & 3 & -1\end{array}\right)+\left(\begin{array}{ccc}2 & 0 & 2 \\ 3 & 1 & 4\end{array}\right)\right]=3\left(\begin{array}{ccc}3 & -2 & 5 \\ 7 & 4 & 3\end{array}\right)\).}
	\end{question}
	
	\begin{question}
		{计算题}
		{矩阵}
		{中等}
		{4.设 \(\mathbf{A}=\left(\begin{array}{ccc}1 & -2 & 3 \\ 4 & 3 & -1\end{array}\right), \quad \mathbf{B}=\left(\begin{array}{ccc}2 & 0 & 2 \\ 3 & -1 & 4\end{array}\right)\) ,若 \(2(\mathbf{A}-\mathbf{X})=3 \mathbf{B}-\mathbf{X}\) ,则 \(\mathbf{X}=()\).}
		{\(\mathbf{X}=2 \mathbf{A}-3 \mathbf{B}=2\left(\begin{array}{ccc}1 & -2 & 3 \\ 4 & 3 & -1\end{array}\right)-3\left(\begin{array}{ccc}2 & 0 & 2 \\ 3 & -1 & 4\end{array}\right)=\left(\begin{array}{ccc}-4 & -4 & 0 \\ -1 & 9 & -14\end{array}\right)\).}
	\end{question}
	
	\begin{question}
		{计算题}
		{矩阵}
		{容易}
		{5.计算 \(\left(\begin{array}{cccc}1 & -3 & 3 & 1 \\ 2 & 1 & 2 & -1 \\ 4 & 7 & -2 & 3 \\ -2 & 0 & -4 & -3\end{array}\right)\left(\begin{array}{l}1 \\ 1 \\ 1 \\ 1\end{array}\right)=\)(). }
		{\(\left(\begin{array}{cccc}1 & -3 & 3 & 1 \\ 2 & 1 & 2 & -1 \\ 4 & 7 & -2 & 3 \\ -2 & 0 & -4 & -3\end{array}\right)\left(\begin{array}{l}1 \\ 1 \\ 1 \\ 1\end{array}\right)=\left(\begin{array}{c}2 \\ 4 \\ 12 \\ -9\end{array}\right)\).}
	\end{question}
	
	\begin{question}
		{计算题}
		{矩阵}
		{中等}
		{6.已知 \(\alpha=\left(\begin{array}{lll}1 & 2 & 3\end{array}\right), \boldsymbol{\beta}=\left(\begin{array}{lll}1 & \frac{1}{2} & \frac{1}{3}\end{array}\right)\) ,令 \(\mathbf{A}=\boldsymbol{\alpha}^{\mathrm{T}} \boldsymbol{\beta}\) ,求 \(\mathbf{A}^{\mathbf{n}}=\)(). }
		{因为 \(\mathbf{A}^n=\left(\boldsymbol{\alpha}^{\mathrm{T}} \boldsymbol{\beta}\right)\left(\boldsymbol{\alpha}^{\mathrm{T}} \boldsymbol{\beta}\right) \cdots\left(\boldsymbol{\alpha}^{\mathrm{T}} \boldsymbol{\beta}\right)=\boldsymbol{\alpha}^{\mathrm{T}} \underbrace{\left(\boldsymbol{\beta} \boldsymbol{\alpha}^{\mathrm{T}}\right) \cdots\left(\boldsymbol{\beta} \boldsymbol{\alpha}^{\mathrm{T}}\right) \boldsymbol{\beta}}_{n-1 }\) ,其中 \(\boldsymbol{\beta} \boldsymbol{\alpha}^{\mathrm{T}}=\) \(\left(\begin{array}{lll}1 & \frac{1}{2} & \frac{1}{3}\end{array}\right)\left(\begin{array}{l}1 \\ 2 \\ 3\end{array}\right)=3\) ,所以 \(\mathbf{A}^n=3^{n-1} \boldsymbol{\alpha}^{\mathrm{T}} \boldsymbol{\beta}=3^{n-1}\left(\begin{array}{ccc}1 & \frac{1}{2} & \frac{1}{3} \\ 2 & 1 & \frac{2}{3} \\ 3 & \frac{3}{2} & 1\end{array}\right)\).}
	\end{question}
	
	\begin{question}
		{计算题}
		{矩阵}
		{困难}
		{7.设 \(\mathbf{A}\) 为 3 阶矩阵, \(\mathbf{A}^*\) 为 \(\mathbf{A}\) 的伴随矩阵,且已知 \(|\mathbf{A}|=\frac{1}{2}\) ,求行列式 \(\left|(3 \mathbf{A})^{-1}-2 \mathbf{A}^*\right|=\)(). }
		{因 \((3 \mathbf{A})^{-1}=\frac{1}{3} \mathbf{A}^{-1},\left|\mathbf{A}^{-1}\right|=|\mathbf{A}|^{-1}, \mathbf{A}^*=|\mathbf{A}| \mathbf{A}^{-1}\) ,故 \(\left|(3 \mathbf{A})^{-1}-2 \mathbf{A}^*\right|=\) \(\left|\frac{1}{3} \mathbf{A}^{-1}-\mathbf{A}^{-1}\right|=\left|-\frac{2}{3} \mathbf{A}^{-1}\right|=\left(-\frac{2}{3}\right)^3|\mathbf{A}|^{-1}=-\frac{16}{27}\).}
	\end{question}
	
	\begin{question}
		{计算题}
		{矩阵}
		{中等}
		{8.判断矩阵 \(\mathbf{A}=\left(\begin{array}{lll}1 & 2 & 3 \\ 2 & 1 & 2 \\ 1 & 3 & 3\end{array}\right)\) 是否可逆,若可逆,利用伴随矩阵求出其逆矩阵. }
		{因为 \(|\mathbf{A}|=\left|\begin{array}{lll}1 & 2 & 3 \\ 2 & 1 & 2 \\ 1 & 3 & 3\end{array}\right|=4 \neq 0\) ,所以 \(\mathbf{A}\) 可逆.  \(\mathbf{A}\) 中各元素的代数余子式为:
			\[
			A_{11}=-3,\quad A_{12}=-4,\quad A_{13}=5,\quad A_{21}=3,\quad A_{22}=0,\quad A_{23}=-1,\quad A_{31}=1,\quad A_{32}=4,\quad A_{33}=-3
			\]
			所以 \(\mathbf{A}^*=\left(\begin{array}{ccc}-3 & 3 & 1 \\ -4 & 0 & 4 \\ 5 & -1 & -3\end{array}\right)^{\mathrm{T}}=\left(\begin{array}{ccc}-3 & -4 & 5 \\ 3 & 0 & -1 \\ 1 & 4 & -3\end{array}\right)\),则
			\[
			\mathbf{A}^{-1}=\frac{1}{|\mathbf{A}|}\mathbf{A}^*=\frac{1}{4}\left(\begin{array}{ccc}-3 & -4 & 5 \\ 3 & 0 & -1 \\ 1 & 4 & -3\end{array}\right)
			\]}
	\end{question}
	
	
	\begin{question}
		{计算题}
		{矩阵}
		{中等}
		{9.若方阵 \(\mathbf{A}\) 满足 \(\mathbf{A}^2 - 3\mathbf{A} - 10\mathbf{E} = \mathbf{O}\),求 \(\mathbf{A}^{-1}\) 和 \((\mathbf{A} - 4\mathbf{E})^{-1} = ()\). }
		{由 \(\mathbf{A}^2 - 3\mathbf{A} - 10\mathbf{E} = \mathbf{O} \Rightarrow \mathbf{A}(\mathbf{A} - 3\mathbf{E}) = 10\mathbf{E} \Rightarrow \mathbf{A} \left( \frac{1}{10}(\mathbf{A} - 3\mathbf{E}) \right) = \mathbf{E}\),所以 \(\mathbf{A}\) 可逆且 \(\mathbf{A}^{-1} = \frac{1}{10}(\mathbf{A} - 3\mathbf{E})\). \\
			再由 \(\mathbf{A}^2 - 3\mathbf{A} - 10\mathbf{E} = \mathbf{O} \Rightarrow (\mathbf{A} + \mathbf{E})(\mathbf{A} - 4\mathbf{E}) = 6\mathbf{E} \Rightarrow \frac{1}{6}(\mathbf{A} + \mathbf{E})(\mathbf{A} - 4\mathbf{E}) = \mathbf{E}\),所以 \(\mathbf{A} - 4\mathbf{E}\) 可逆,\((\mathbf{A} - 4\mathbf{E})^{-1} = \frac{1}{6}(\mathbf{A} + \mathbf{E})\). }
	\end{question}
	
	\begin{question}
		{计算题}
		{矩阵}
		{中等}
		{10.若矩阵 \(\mathbf{X}\) 满足 \(\mathbf{A} \mathbf{X} \mathbf{B} = \mathbf{C}\),其中 \(\mathbf{A} = \left(\begin{array}{ccc}-1 & 0 & 0 \\ 0 & 5 & 3 \\ 0 & 2 & 1\end{array}\right)\),\(\mathbf{B} = \left(\begin{array}{cc}-2 & -3 \\ 3 & 5\end{array}\right)\),\(\mathbf{C} = \left(\begin{array}{cc}2 & 3 \\ 1 & 2 \\ -1 & -2\end{array}\right)\),求矩阵 \(\mathbf{X} = ()\). }
		{由于 \(\mathbf{A}, \mathbf{B}\) 可逆,故 \(\mathbf{X} = \mathbf{A}^{-1} \mathbf{C} \mathbf{B}^{-1}\). \\
			计算可得:\(\mathbf{A}^{-1} = \left(\begin{array}{ccc}-1 & 0 & 0 \\ 0 & -1 & 3 \\ 0 & 2 & -5\end{array}\right)\),\(\mathbf{B}^{-1} = \left(\begin{array}{cc}-5 & -3 \\ 3 & 2\end{array}\right)\),\\
			因此 \(\mathbf{X} = \left(\begin{array}{cc}1 & 0 \\ -4 & -4 \\ 7 & 7\end{array}\right)\). }
	\end{question}
	
	\begin{question}
		{计算题}
		{矩阵}
		{中等}
		{11.若矩阵 \(\mathbf{A} = \left(\begin{array}{ccc}1 & 2 & -2 \\ 0 & 1 & 4 \\ 0 & 0 & 1\end{array}\right)\) 且 \(\mathbf{A}^2 - \mathbf{A}\mathbf{B} = \mathbf{E}\),求矩阵 \(\mathbf{B} = ()\). }
		{由 \(\mathbf{A}^2 - \mathbf{A}\mathbf{B} = \mathbf{E} \Rightarrow \mathbf{A}(\mathbf{A} - \mathbf{B}) = \mathbf{E} \Rightarrow \mathbf{A} - \mathbf{B} = \mathbf{A}^{-1} \Rightarrow \mathbf{B} = \mathbf{A} - \mathbf{A}^{-1}\). \\
			又因为 \(|\mathbf{A}| = 1 \Rightarrow \mathbf{A}^{-1} = \frac{\mathbf{A}^*}{|\mathbf{A}|} = \left(\begin{array}{ccc}1 & -2 & 10 \\ 0 & 1 & -4 \\ 0 & 0 & 1\end{array}\right)\),\\
			故 \(\mathbf{B} = \left(\begin{array}{ccc}0 & 4 & -12 \\ 0 & 0 & 8 \\ 0 & 0 & 0\end{array}\right)\). }
	\end{question}
	
	\begin{question}
		{计算题}
		{矩阵}
		{中等}
		{12.已知三阶方阵 \(\mathbf{A}\) 的逆矩阵为 \(\mathbf{A}^{-1} = \left(\begin{array}{ccc}1 & 1 & 1 \\ 1 & 2 & 1 \\ 1 & 1 & 3\end{array}\right)\),求 \(\left(\mathbf{A}^*\right)^{-1} = ()\). }
		{由 \(\mathbf{A}^* = |\mathbf{A}| \mathbf{A}^{-1}\) 得:\(\left(\mathbf{A}^*\right)^{-1} = \left(|\mathbf{A}| \mathbf{A}^{-1} \right)^{-1} = \frac{1}{|\mathbf{A}|} (\mathbf{A}^{-1})^{-1}\). \\
			由 \(\left|\mathbf{A}^{-1}\right| = \frac{1}{|\mathbf{A}|}\),可得 \(|\mathbf{A}| = \frac{1}{\left|\mathbf{A}^{-1}\right|}\). \\
			计算 \(|\mathbf{A}^{-1}| = \left|\begin{array}{ccc}1 & 1 & 1 \\ 1 & 2 & 1 \\ 1 & 1 & 3\end{array}\right| = 2\),故 \(|\mathbf{A}| = \frac{1}{2}\),\\
			于是 \(\left(\mathbf{A}^*\right)^{-1} = \frac{1}{\frac{1}{2}} \cdot \mathbf{A} = 2\mathbf{A} = 2 \left(\mathbf{A}^{-1}\right)^{-1} = 2 \left(\begin{array}{ccc}\frac{5}{2} & -1 & -\frac{1}{2} \\ -1 & 1 & 0 \\ -\frac{1}{2} & 0 & \frac{1}{2}\end{array}\right) = \left(\begin{array}{ccc}5 & -1 & -1 \\ -2 & 2 & 0 \\ -1 & 0 & 1\end{array}\right)\). }
	\end{question}
	
	\begin{question}
		{计算题}
		{矩阵}
		{中等}
		{13.若矩阵 \(\mathbf{X}\) 满足 \(\mathbf{X} \left(\begin{array}{ccc}2 & 1 & -1 \\ 1 & 1 & 1 \\ 3 & 2 & 1\end{array}\right) = \left(\begin{array}{ccc}1 & -1 & 3 \\ 4 & 3 & 2 \\ 2 & -2 & 5\end{array}\right)\),求 \(\mathbf{X} = ()\). }
		{由于右乘矩阵可逆,故 \(\mathbf{X} = \mathbf{M} \mathbf{N}^{-1}\),\\
			其中 \(\mathbf{M} = \left(\begin{array}{ccc}1 & -1 & 3 \\ 4 & 3 & 2 \\ 2 & -2 & 5\end{array}\right)\),\(\mathbf{N} = \left(\begin{array}{ccc}2 & 1 & -1 \\ 1 & 1 & 1 \\ 3 & 2 & 1\end{array}\right)\). \\
			计算得 \(\mathbf{N}^{-1} = \left(\begin{array}{ccc}-1 & -3 & 2 \\ 2 & 5 & -3 \\ -1 & -1 & 1\end{array}\right)\),\\
			所以 \(\mathbf{X} = \mathbf{M}\mathbf{N}^{-1} = \left(\begin{array}{ccc}-6 & -11 & 8 \\ 0 & 1 & 1 \\ -11 & -21 & 15\end{array}\right)\). }
	\end{question}
	
	\begin{question}
		{计算题}
		{矩阵}
		{中等}
		{14.设矩阵 \(\mathbf{A}, \mathbf{B}\) 满足 \(\mathbf{A} \mathbf{B} = 2\mathbf{B} + \mathbf{A}\),且 \(\mathbf{A} = \left(\begin{array}{ccc}3 & 0 & 1 \\ 1 & 1 & 0 \\ 0 & 1 & 4\end{array}\right)\),求矩阵 \(\mathbf{B} = ()\). }
		{由题意 \(\mathbf{A} \mathbf{B} = 2\mathbf{B} + \mathbf{A} \Rightarrow (\mathbf{A} - 2\mathbf{E}) \mathbf{B} = \mathbf{A}\),\\
			即 \(\mathbf{B} = (\mathbf{A} - 2\mathbf{E})^{-1} \mathbf{A}\). \\
			计算 \(\mathbf{A} - 2\mathbf{E} = \left(\begin{array}{ccc}1 & 0 & 1 \\ -1 & -1 & 0 \\ 0 & 1 & 2\end{array}\right)\),\\
			其逆为 \((\mathbf{A} - 2\mathbf{E})^{-1} = \left(\begin{array}{ccc}1 & 0 & -\frac{1}{2} \\ -1 & -1 & \frac{1}{2} \\ 0 & 0 & \frac{1}{2}\end{array}\right)\),\\
			则 \(\mathbf{B} = \left(\begin{array}{ccc}1 & 0 & -\frac{1}{2} \\ -1 & -1 & \frac{1}{2} \\ 0 & 0 & \frac{1}{2}\end{array}\right) \left(\begin{array}{ccc}3 & 0 & 1 \\ 1 & 1 & 0 \\ 0 & 1 & 4\end{array}\right) = \left(\begin{array}{ccc}3 & -\frac{1}{2} & -1 \\ 2 & -\frac{1}{2} & 1 \\ 0 & \frac{1}{2} & 2\end{array}\right)\). }
	\end{question}
	
	
	\begin{question}
		{计算题}
		{矩阵}
		{中等}
		{15.设矩阵 \(\mathbf{A}=\left(\begin{array}{lll}3 & 0 & 1 \\ 1 & 1 & 0 \\ 0 & 1 & 4\end{array}\right)\) ,求矩阵 \(\mathbf{B}\) 使其满足 \(\mathbf{A B}=\mathbf{A}+2 \mathbf{B}\) . }
		{\(\mathbf{A B}=\mathbf{A}+2 \mathbf{B} \Rightarrow (\mathbf{A}-2\mathbf{E})\mathbf{B}=\mathbf{A}\) ,即 \(\mathbf{B}=(\mathbf{A}-2 \mathbf{E})^{-1} \mathbf{A}\) . 由 \(\mathbf{A}-2\mathbf{E}=\left(\begin{array}{ccc}1 & 0 & 1 \\ 1 & -1 & 0 \\ 0 & 1 & 2\end{array}\right)\),计算其逆矩阵 \((\mathbf{A}-2 \mathbf{E})^{-1}=\left(\begin{array}{ccc}2 & -1 & -1 \\ 2 & -2 & -1 \\ -1 & 1 & 1\end{array}\right)\),从而得 \(\mathbf{B}=\left(\begin{array}{ccc}5 & -2 & -2 \\ 4 & -3 & -2 \\ -2 & 2 & 3\end{array}\right)\). }
	\end{question}
	
	\begin{question}
		{计算题}
		{矩阵}
		{中等}
		{16.若方阵 \(\mathbf{A}\) 满足 \(\mathbf{A}^2-\mathbf{A}-2 \mathbf{E}=\mathbf{O}\),求 \(\mathbf{A}^{-1}\) 和 \((\mathbf{A}+2 \mathbf{E})^{-1}\). }
		{由 \(\mathbf{A}^2-\mathbf{A}-2 \mathbf{E}=\mathbf{O} \Rightarrow \mathbf{A}(\mathbf{A}-\mathbf{E})=2 \mathbf{E}\),所以 \(\mathbf{A}^{-1}=\frac{1}{2}(\mathbf{A}-\mathbf{E})\). 又有 \((\mathbf{A}+2 \mathbf{E})(\mathbf{A}-3 \mathbf{E})=-4 \mathbf{E} \Rightarrow (\mathbf{A}+2 \mathbf{E})^{-1}=-\frac{1}{4}(\mathbf{A}-3 \mathbf{E})\). }
	\end{question}
	
	\begin{question}
		{计算题}
		{矩阵}
		{容易}
		{17.用克莱姆法则求解方程组 \(\left\{\begin{array}{l}2x+5y=1 \\ 3x+7y=2\end{array}\right.\). }
		{\(D=\left|\begin{array}{cc}2 & 5 \\ 3 & 7\end{array}\right|=-1\), \(D_1=\left|\begin{array}{cc}1 & 5 \\ 2 & 7\end{array}\right|=-3\), \(D_2=\left|\begin{array}{cc}2 & 1 \\ 3 & 2\end{array}\right|=1\),解得 \(x=\frac{D_1}{D}=3\), \(y=\frac{D_2}{D}=-1\). }
	\end{question}
	
	\begin{question}
		{计算题}
		{矩阵}
		{容易}
		{18.用克莱姆法则求解方程组 \(\left\{\begin{array}{l}4x_1+5x_2=0 \\ 3x_1-7x_2=0\end{array}\right.\). }
		{\(D=\left|\begin{array}{cc}4 & 5 \\ 3 & -7\end{array}\right|=-43\), \(D_1=\left|\begin{array}{cc}0 & 5 \\ 0 & -7\end{array}\right|=0\), \(D_2=\left|\begin{array}{cc}4 & 0 \\ 3 & 0\end{array}\right|=0\),故 \(x_1=\frac{D_1}{D}=0\), \(x_2=\frac{D_2}{D}=0\),解为零解. }
	\end{question}
	
	\begin{question}
		{计算题}
		{矩阵}
		{中等}
		{19.用克莱姆法则求解方程组 \(\left\{\begin{array}{l}x_1+x_2-2x_3=-3 \\ 5x_1-2x_2+7x_3=22 \\ 2x_1-5x_2+4x_3=4\end{array}\right.\). }
		{\(D=\left|\begin{array}{ccc}1 & 1 & -2 \\ 5 & -2 & 7 \\ 2 & -5 & 4\end{array}\right|=63\), \(D_1=\left|\begin{array}{ccc}-3 & 1 & -2 \\ 22 & -2 & 7 \\ 4 & -5 & 4\end{array}\right|=63\), \(D_2=\left|\begin{array}{ccc}1 & -3 & -2 \\ 5 & 22 & 7 \\ 2 & 4 & 4\end{array}\right|=126\), \(D_3=\left|\begin{array}{ccc}1 & 1 & -3 \\ 5 & -2 & 22 \\ 2 & -5 & 4\end{array}\right|=189\),解为 \(x_1=\frac{D_1}{D}=1\), \(x_2=\frac{D_2}{D}=2\), \(x_3=\frac{D_3}{D}=3\). }
	\end{question}
	
	\begin{question}
		{计算题}
		{矩阵}
		{困难}
		{20.用克莱姆法则求解方程组 \(\left\{\begin{array}{l}bx - ay = -2ab \\ -2cy + 3bz = bc \\ cx + az = 0 \quad (abc \neq 0)\end{array}\right.\). }
		{\(D=\left|\begin{array}{ccc}b & -a & 0 \\ 0 & -2c & 3b \\ c & 0 & a\end{array}\right|=-5abc\), \(D_1=\left|\begin{array}{ccc}-2ab & -a & 0 \\ bc & -2c & 3b \\ 0 & 0 & a\end{array}\right|=5a^2bc\), \(D_2=\left|\begin{array}{ccc}b & -2ab & 0 \\ 0 & bc & 3b \\ c & 0 & a\end{array}\right|=-5ab^2c\), \(D_3=\left|\begin{array}{ccc}b & -a & -2ab \\ 0 & -2c & bc \\ c & 0 & 0\end{array}\right|=-5abc^2\),因此 \(x=\frac{D_1}{D}=-a\), \(y=\frac{D_2}{D}=b\), \(z=\frac{D_3}{D}=c\). }
	\end{question}
	
	
		
	\begin{question}
		{填空题}
		{线性方程组}
		{容易}
		{1.\(3\left(\begin{array}{lll}1 & 0 & 4 \\ 1 & 2 & 0\end{array}\right)-2\left(\begin{array}{lll}0 & 1 & 1 \\ 2 & 3 & 1\end{array}\right)=()\).}
		{\(3\left(\begin{array}{lll}1 & 0 & 4 \\ 1 & 2 & 0\end{array}\right)-2\left(\begin{array}{lll}0 & 1 & 1 \\ 2 & 3 & 1\end{array}\right)=\left(\begin{array}{ccc}3 & -2 & 10 \\ -1 & 0 & -2\end{array}\right)\).}
	\end{question}
	
	
	\begin{question}
		{选择题}
		{线性方程组}
		{中等}
		{2. 设矩阵 \(\mathbf{A}=\left(\begin{array}{ccc}0 & 2 & -1 \\ 1 & 1 & 2 \\ -1 & -1 & -1\end{array}\right)\) 则 \(\mathbf{A}\) 的逆矩阵 \(\mathbf{A}^{-1}=\)(). 
			A. \(\left(\begin{array}{ccc}-\frac{1}{2} & -\frac{3}{2} & -\frac{5}{2} \\ \frac{1}{2} & \frac{1}{2} & \frac{1}{2} \\ 0 & 1 & 1\end{array}\right)\)
			B. \(\left(\begin{array}{ccc}\frac{1}{2} & \frac{3}{2} & \frac{5}{2} \\ \frac{1}{2} & \frac{1}{2} & \frac{1}{2} \\ 0 & 1 & 1\end{array}\right)\)
			C. \(\left(\begin{array}{ccc}-\frac{1}{2} & \frac{3}{2} & -\frac{5}{2} \\ \frac{1}{2} & -\frac{1}{2} & \frac{1}{2} \\ 0 & 0 & 1\end{array}\right)\)
			D. \(\left(\begin{array}{ccc}-\frac{1}{2} & -\frac{3}{2} & -\frac{5}{2} \\ -\frac{1}{2} & -\frac{1}{2} & \frac{1}{2} \\ 1 & 0 & 1\end{array}\right)\)}
		{由 \((\mathbf{A} \mid \mathbf{E})=\left(\begin{array}{ccc|ccc}0 & 2 & -1 & 1 & 0 & 0 \\ 1 & 1 & 2 & 0 & 1 & 0 \\ -1 & -1 & -1 & 0 & 0 & 1\end{array}\right) \xrightarrow[r_3 \times(-1)]{r_2+r_3}\left(\begin{array}{ccc|ccc}0 & 2 & -1 & 1 & 0 & 0 \\ 0 & 0 & 1 & 0 & 1 & 1 \\ 1 & 1 & 1 & 0 & 0 & -1\end{array}\right)\)
			\(\xrightarrow[r_3-r_2]{\left(r_1+r_2\right) \times \frac{1}{2}}\left(\begin{array}{ccc|ccc}0 & 1 & 0 & \frac{1}{2} & \frac{1}{2} & \frac{1}{2} \\ 0 & 0 & 1 & 0 & 1 & 1 \\ 1 & 1 & 0 & 0 & -1 & -2\end{array}\right) \xrightarrow{r_3-r_1}\left(\begin{array}{ccc|ccc}1 & 0 & 0 & \frac{1}{2} & -\frac{3}{2} & -\frac{5}{2} \\ 0 & 1 & 0 & \frac{1}{2} & \frac{1}{2} & \frac{1}{2} \\ 0 & 0 & 1 & 0 & 1 & 1\end{array}\right)\) 知 \(\mathbf{A}^{-1}=\) \(\left(\begin{array}{ccc}-\frac{1}{2} & -\frac{3}{2} & -\frac{5}{2} \\ \frac{1}{2} & \frac{1}{2} & \frac{1}{2} \\ 0 & 1 & 1\end{array}\right)\). }
	\end{question}
	
	\begin{question}
		{选择题}
		{线性方程组}
		{中等}
		{3. 设矩阵 \(\mathbf{A}=\left(\begin{array}{lll}1 & 2 & 3 \\ 2 & 2 & 1 \\ 3 & 4 & 3\end{array}\right)\) 则 \(\mathbf{A}\) 的逆矩阵 \(\mathbf{A}^{-1}=\)(). 
			A. \(\left(\begin{array}{ccc}1 & 3 & -2 \\ -\frac{3}{2} & -3 & \frac{5}{2} \\ 1 & 1 & -1\end{array}\right)\)
			B. \(\left(\begin{array}{ccc}1 & -3 & -2 \\ 0 & 3 & 2 \\ 1 & 1 & -1\end{array}\right)\)
			C. \(\left(\begin{array}{ccc}1 & -3 & 2 \\ \frac{3}{2} & -3 & \frac{5}{2} \\ 1 & 1 & -1\end{array}\right)\)
			D. \(\left(\begin{array}{ccc}1 & 3 & -2 \\ \frac{3}{2} & -3 & \frac{5}{2} \\ -1 & 1 & 0\end{array}\right)\)}
		{
			\((\mathbf{A} \mid \mathbf{E})=\left(\begin{array}{ccc|ccc}
				1 & 2 & 3 & 1 & 0 & 0 \\
				2 & 2 & 1 & 0 & 1 & 0 \\
				3 & 4 & 3 & 0 & 0 & 1
			\end{array}\right) \xrightarrow[r_2-r_1]{r_3-r_2-r_1}\left(\begin{array}{ccc|ccc}
				1 & 2 & 3 & 1 & 0 & 0 \\
				1 & 0 & -2 & -1 & 1 & 0 \\
				0 & 0 & -1 & -1 & -1 & 1
			\end{array}\right)\) \\
			\(\xrightarrow[(-1) \times r_3]{\substack{r_1+3 r_3 \\
					r_2-2 r_3}}\left(\begin{array}{ccc|ccc}
				1 & 2 & 0 & -2 & -3 & 3 \\
				1 & 0 & 0 & 1 & 3 & -2 \\
				0 & 0 & 1 & 1 & 1 & -1
			\end{array}\right) \xrightarrow{\frac{1}{2}\left(r_1-r_2\right)}\left(\begin{array}{ccc|ccc}
				0 & 1 & 0 & -\frac{3}{2} & -3 & \frac{5}{2} \\
				1 & 0 & 0 & 1 & 3 & -2 \\
				0 & 0 & 1 & 1 & 1 & -1
			\end{array}\right) \rightarrow\) \\
			\(\left(\begin{array}{ccc}
				1 & 3 & -2 \\
				-\frac{3}{2} & -3 & \frac{5}{2} \\
				1 & 1 & -1
			\end{array}\right)\) 知 \(A^{-1}=\left(\begin{array}{ccc}
				1 & 3 & -2 \\
				-\frac{3}{2} & -3 & \frac{5}{2} \\
				1 & 1 & -1
			\end{array}\right)\). }
	\end{question}
	
	
	\begin{question}
		{选择题}
		{线性方程组}
		{中等}
		{4. 设矩阵 \(\mathbf{A}, \mathbf{B}, \mathbf{X}\) 满足 \(\mathbf{A} \mathbf{X}=\mathbf{B}\) ,其中 \(\mathbf{A}=\left(\begin{array}{lll}1 & 2 & 3 \\ 2 & 2 & 1 \\ 3 & 4 & 3\end{array}\right), \mathbf{B}=\left(\begin{array}{ll}2 & 5 \\ 3 & 1 \\ 4 & 3\end{array}\right)\) ,则矩阵 \(\mathbf{X}=\)(). 
			A. \(\left(\begin{array}{cc}3 & 2 \\ -2 & -3 \\ 1 & 3\end{array}\right)\)
			B. \(\left(\begin{array}{cc}1 & 2 \\ -2 & 3 \\ 1 & 3\end{array}\right)\)
			C. \(\left(\begin{array}{cc}3 & 2 \\ -2 & 0 \\ 1 & -3\end{array}\right)\)
			D. \(\left(\begin{array}{cc}3 & 1 \\ -2 & -3 \\ -1 & 3\end{array}\right)\)}
		{可知 \(A\) 可逆,则 \(\mathbf{X}=\mathbf{A}^{-1} \mathbf{B}\) 故由 \((\mathbf{A} \mid \mathbf{B})=\left(\begin{array}{lll|ll}1 & 2 & 3 & 2 & 5 \\ 2 & 2 & 1 & 3 & 1 \\ 3 & 4 & 3 & 4 & 3\end{array}\right) \xrightarrow[r_2-r_1]{r_3-r_2-r_1}\)
			
			\(\left(\begin{array}{ccc|cc}
				1 & 2 & 3 & 2 & 5 \\
				1 & 0 & -2 & 1 & -4 \\
				0 & 0 & -1 & -1 & -3
			\end{array}\right) \xrightarrow[\substack{r_2+2 r_3 \\
				r_1-3 r_3}]{r_3 \times(-1)}\left(\begin{array}{ccc|cc}
				1 & 2 & 0 & -1 & -4 \\
				1 & 0 & 0 & 3 & 2 \\
				0 & 0 & 1 & 1 & 3
			\end{array}\right) \xrightarrow{\substack{\left(r_1-r_2\right) \times \frac{1}{2} \\
					0 \\
					1 \\
					0}}\left(\begin{array}{ll}
				1 & 0 \\
				0 & 0
			\end{array} \left\lvert\, \begin{array}{cc}
				-2 & -3 \\
				3 & 2 \\
				1 & 3
			\end{array}\right.\right) \rightarrow\left(\begin{array}{cc}
				3 & 2 \\
				-2 & -3 \\
				1 & 3
			\end{array}\right)\) 知 \(X=\left(\begin{array}{cc}
				3 & 2 \\
				-2 & -3 \\
				1 & 3
			\end{array}\right)\). }
	\end{question}
	
	\begin{question}
		{选择题}
		{线性方程组}
		{困难}
		{5. 设 \(\mathbf{A}, \mathbf{B}\) 均为 \(\boldsymbol{n}\) 阶矩阵,\(\mathbf{A}\) 与 \(\mathbf{B}\) 等价,则下列命题中错误的是(). 
			A. 若 \(|\mathbf{A}|>0\) ,则 \(|\mathbf{B}|>0\)
			B. 若 \(|\mathbf{A}| \neq 0\) ,则 \(\mathbf{B}\) 也可逆
			C. 若 \(\mathbf{A}\) 与 \(\mathbf{E}\) 等价,则 \(\mathbf{B}\) 与 \(\mathbf{E}\) 等价
			D. 存在可逆矩阵 \(\mathbf{P}, \mathbf{Q}\) ,使得 \(\mathbf{P} \mathbf{A} \mathbf{Q}=\mathbf{B}\)}
		{因为 \(\mathbf{A}\) 与 \(\mathbf{B}\) 等价,所以存在可逆矩阵 \(\mathbf{P}, \mathbf{Q}\) ,使得 \(\mathbf{P} \mathbf{A Q}=\mathbf{B}\) ,从而 \(|\mathbf{P} \mathbf{A Q}|=|\mathbf{B}| \Rightarrow|\mathbf{P}|\).  \(|\mathbf{A}| \cdot|\mathbf{Q}|=|\mathbf{B}|\) ,可知 \(|\mathbf{A}|\) 与 \(|\mathbf{B}|\) 相差一个非零倍数,故若 \(|\mathbf{A}| \neq 0\) ,则 \(|\mathbf{B}| \neq 0\) ,即 \(B\) 也可逆. 由等价的传递性知选项C正确. 故选A. }
	\end{question}
	
	\begin{question}
		{选择题}
		{线性方程组}
		{中等}
		{6. 设 \(\mathbf{A}, \mathbf{B}\) 均为 \(\boldsymbol{n}\) 阶非零矩阵,且 \(\mathbf{A B}=\mathbf{O}\) ,则 \(\mathbf{A}\) 和 \(\mathbf{B}\) 的秩(). 
			A. 均小于 \(n\)
			B. 均等于零
			C. 一个小于 \(n\) ,一个等于 \(n\)
			D. 均等于 \(n\)}
		{因为 \(\mathbf{A}, \mathbf{B}\) 均为非零矩阵,如 \(r(\mathbf{A})=n\) ,则 \(A\) 可逆,由 \(\mathbf{A B}=\mathbf{O}\) ,则 \(\mathbf{B}=\mathbf{O}\). 矛盾,所以只能是 \(\mathbf{A}\) 和 \(\mathbf{B}\) 的秩均小于 \(n\). }
	\end{question}
	
	\begin{question}
		{选择题}
		{线性方程组}
		{容易}
		{7. 设 \(\mathbf{A}\) 为三阶方阵,\(r(\mathbf{A})=1\) ,则下列正确的是(). 
			A. \(r\left(\mathbf{A}^*\right)=0\)
			B. \(r\left(\mathbf{A}^*\right)=1\)
			C. \(r\left(\mathbf{A}^*\right)=2\)
			D. \(r\left(\mathbf{A}^*\right)=3\)}
		{因为 \(r(\mathbf{A})=1\) ,所以 \(\mathbf{A}\) 的任意二阶子式都为零,从而 \(\mathbf{A}\) 的任意元素的代数余子式都为零,故 \(r\left(\mathbf{A}^*\right)=0\). }
	\end{question}
	
	\begin{question}
		{选择题}
		{线性方程组}
		{容易}
		{8. 设矩阵 \(\mathbf{A}\) 的秩 \(r(\mathbf{A})=r\) ,则下列正确的是(). 
			A. \(\mathbf{A}\) 中所有 \(r+1\) 阶子式均等于零
			B. \(\mathbf{A}\) 中所有 \(r\) 阶子式均不等于零
			C. \(\mathbf{A}\) 中所有 \(r-1\) 阶子式均不等于零
			D. \(\mathbf{A}\) 中只有一个 \(r-1\) 阶子式不等于零}
		{由矩阵的秩的定义可知,\(\mathbf{A}\) 中所有 \(r+1\) 阶子式均等于零. }
	\end{question}
	
	\begin{question}
		{选择题}
		{线性方程组}
		{容易}
		{9. 若矩阵 \(\mathbf{A}=\left(\begin{array}{cccc}1 & 0 & -1 & 0 \\ 0 & -2 & 3 & 4 \\ 0 & 0 & 0 & 5\end{array}\right)\) ,则 \(\mathbf{A}\) 中(). 
			A. 存在一个三阶子式不等于零
			B. 所有二阶子式都不等于零
			C. 所有二阶子式都等于零
			D. 所有三阶子式都不等于零}
		{由 \(\mathbf{A}=\left(\begin{array}{cccc}1 & 0 & -1 & 0 \\ 0 & -2 & 3 & 4 \\ 0 & 0 & 0 & 5\end{array}\right)\) 知 \(r(\mathbf{A})=3\) ,所以 \(\mathbf{A}\) 中存在一个三阶子式不等于零. }
	\end{question}
	
	\begin{question}
		{选择题}
		{线性方程组}
		{容易}
		{10. 设 \(\mathbf{A}=\left(\begin{array}{cc}2 & 3 \\ 1 & -1 \\ -1 & 2\end{array}\right)\) ,则 \(\mathbf{A}\) 的秩 \(r(\mathbf{A})=\)(). 
			A. 2
			B. 1
			C. 3
			D. 0}
		{因为 \(\left|\begin{array}{cc}2 & 3 \\ 1 & -1\end{array}\right|=-5 \neq 0\) ,所以 \(r(\mathbf{A})=2\). }
	\end{question}
	
	\begin{question}
		{选择题}
		{线性方程组}
		{中等}
		{11. 设 \(n\) 阶矩阵 \(\mathbf{A}\) 与 \(\mathbf{B}\) 等价,则必有(). 
			A. 当 \(|\mathbf{A}|=0\) 时,\(|\mathbf{B}|=0\)
			B. 当 \(|\mathbf{A}|=a(a \neq 0)\) 时,\(|\mathbf{B}|=a\)
			C. 当 \(|\mathbf{A}|=a(a \neq 0)\) 时,\(|\mathbf{B}|=-a\)
			D. 当 \(|\mathbf{A}| \neq 0\) 时,\(|\mathbf{B}|=0\)}
		{因为当 \(|\mathbf{A}|=0\) 时,\(r(\mathbf{A})<n\) 又 \(A\) 与 \(B\) 等价,故 \(r(\mathbf{B})<n\),即 \(|\mathbf{B}|=0\). }
	\end{question}
	
	\begin{question}
		{选择题}
		{线性方程组}
		{中等}
		{12. 设 \(\mathbf{A}\) 是 \(m \times n\) 矩阵,\(\mathbf{B}\) 是 \(n \times m\) 矩阵,则下列正确的是(). 
			A. 当 \(m>n\) 时,必有 \(|\mathbf{A B}|=0\)
			B. 当 \(m>n\) 时,必有 \(|\mathbf{A B}| \neq 0\)
			C. 当 \(n>m\) 时,必有 \(|\mathbf{A B}|=0\)
			D. 当 \(n>m\) 时,必有 \(|\mathbf{A B}| \neq 0\)}
		{由 \(\mathbf{A}_{m \times n} \mathbf{B}_{n \times m}\) 是 \(m \times m\) 方阵,而 \(r(\mathbf{A B}) \leq \min \{r(\mathbf{A}), r(\mathbf{B})\} \leq n<m \Rightarrow r(\mathbf{A B})<m\),所以 \(|\mathbf{A B}|=0\). }
	\end{question}
	
	\begin{question}
		{选择题}
		{线性方程组}
		{中等}
		{13. 设矩阵 \(\mathbf{A}\) 与 \(\mathbf{B}\) 等价,\(A\) 有一个 \(k\) 阶子式不等于零,则 \(r(\mathbf{B})\) 与 \(k\) 的大小关系是(). 
			A. \(r(\mathbf{B}) \geq k\)
			B. \(r(\mathbf{B})<k\)
			C. \(r(\mathbf{B})=k\)
			D. \(r(\mathbf{B}) \leq k\)}
		{因为 \(A\) 有一个 \(k\) 阶子式不等于零,所以 \(r(\mathbf{A}) \geq k\),又由 \(\mathbf{A}\) 与 \(\mathbf{B}\) 等价知 \(r(\mathbf{A})=r(\mathbf{B})\),故 \(r(\mathbf{B}) \geq k\). }
	\end{question}
	
	\begin{question}
		{选择题}
		{线性方程组}
		{中等}
		{14. 设 \(\mathbf{A}\) 是 \(m \times n\) 矩阵,\(\mathbf{C}\) 与 \(n\) 阶单位矩阵等价,\(\mathbf{B}=\mathbf{A C}\),则 \(r(\mathbf{A})\) 与 \(r(\mathbf{B})\) 的大小关系为(). 
			A. \(r(\mathbf{A})=r(\mathbf{B})\)
			B. \(r(\mathbf{A})>r(\mathbf{B})\)
			C. \(r(\mathbf{A})<r(\mathbf{B})\)
			D. \(r(\mathbf{A})\) 与 \(r(\mathbf{B})\) 的大小关系依 \(\mathbf{C}\) 而定}
		{因为 \(\mathbf{C}\) 与 \(n\) 阶单位矩阵等价,所以 \(r(\mathbf{C})=n \Rightarrow \mathbf{C}\) 可逆,从而 \(r(\mathbf{B})=r(\mathbf{A C}) \leq r(\mathbf{A})\),又 \(r(\mathbf{A})=r\left(\mathbf{A} \mathbf{C C}^{-1}\right)=r\left(\mathbf{B C}^{-1}\right) \leq r(\mathbf{B})\),故 \(r(\mathbf{A})=r(\mathbf{B})\). }
	\end{question}
	
	\begin{question}
		{选择题}
		{线性方程组}
		{容易}
		{15. 设 \(\mathbf{A}=\left(\begin{array}{lll}1 & 2 & 3 \\ 2 & 3 & 1 \\ 3 & 1 & 2\end{array}\right)\),则 \(\mathbf{A}\) 的秩 \(r(\mathbf{A})=\)(). 
			A. 3
			B. 1
			C. 2
			D. 0}
		{因为 \(|\mathbf{A}|=\left|\begin{array}{lll}1 & 2 & 3 \\ 2 & 3 & 1 \\ 3 & 1 & 2\end{array}\right|=-18 \neq 0\),所以 \(r(\mathbf{A})=3\). }
	\end{question}
	
	\begin{question}
		{选择题}
		{线性方程组}
		{容易}
		{16. 设 \(\mathbf{A}=\left(\begin{array}{lll}2 & -1 & 1 \\ 4 & -2 & 2 \\ 6 & -3 & 3\end{array}\right)\),则 \(\mathbf{A}\) 的秩 \(r(\mathbf{A})=\)(). 
			A. 1
			B. 2
			C. 3
			D. 0}
		{由 \(\mathbf{A}=\left(\begin{array}{ccc}2 & -1 & 1 \\ 4 & -2 & 2 \\ 6 & -3 & 3\end{array}\right) \rightarrow\left(\begin{array}{ccc}2 & -1 & 1 \\ 0 & 0 & 0 \\ 0 & 0 & 0\end{array}\right)\) 知 \(r(\mathbf{A})=1\). }
	\end{question}
	
	\begin{question}
		{选择题}
		{线性方程组}
		{中等}
		{17. 设 \(\mathbf{A}=\left(\begin{array}{ccc}1 & 1 & 1 \\ 1 & 2 & 1 \\ 2 & 3 & \lambda+1\end{array}\right)\) 的秩 \(r(\mathbf{A})=2\),则 \(\lambda=\)(). 
			A. 1
			B. 2
			C. 0
			D. -1}
		{由 \(\mathbf{A}=\left(\begin{array}{ccc}1 & 1 & 1 \\ 1 & 2 & 1 \\ 2 & 3 & \lambda+1\end{array}\right) \rightarrow\left(\begin{array}{ccc}1 & 1 & 1 \\ 0 & 1 & 0 \\ 0 & 0 & \lambda-1\end{array}\right)\) 和 \(r(\mathbf{A})=2\) 知 \(\lambda=1\). }
	\end{question}
	
	\begin{question}
		{选择题}
		{线性方程组}
		{中等}
		{18. 设 \(\mathbf{A}=\left(\begin{array}{ccc}1 & -2 & 3 \\ 1 & 3 & -2 \\ 2 & 1 & 1\end{array}\right)\),则 \(\mathbf{A}\) 的秩 \(r(\mathbf{A})=\)(). 
			A. 2
			B. 1
			C. 3
			D. 0}
		{由 \(\mathbf{A}=\left(\begin{array}{ccc}1 & -2 & 3 \\ 1 & 3 & -2 \\ 2 & 1 & 1\end{array}\right) \frac{r_3-2 r_1}{r_2-r_1}\left(\begin{array}{ccc}1 & -2 & 3 \\ 0 & 5 & -5 \\ 0 & 5 & -5\end{array}\right) \rightarrow\left(\begin{array}{ccc}1 & -2 & 3 \\ 0 & 5 & -5 \\ 0 & 0 & 0\end{array}\right)\) 知 \(r(\mathbf{A})=2\). }
	\end{question}
	
	\begin{question}
		{选择题}
		{线性方程组}
		{困难}
		{19. 设 \(\mathbf{A}=\left(\begin{array}{cccc}a & 1 & 1 & 1 \\ 1 & a & 1 & 1 \\ 1 & 1 & a & 1 \\ 1 & 1 & 1 & a\end{array}\right)\) 的秩 \(r(\mathbf{A})=3\),则 \(a=\)(). 
			A. -3
			B. 2
			C. 0
			D. 3}
		{因为 \(r(\mathbf{A})=3\),所以 \(\mathbf{A}\) 中非零子式的最高阶数是 3,从而有 \(|\mathbf{A}|=(a+3)(a-1)^3=0 \Rightarrow a=1\) 或 \(a=-3\). 当 \(a=1\) 时,显然有 \(r(\mathbf{A})=1\);而当 \(a=-3\) 时,\(A\) 中存在不为零的三阶子式且不存在更高阶的非零子式,故当且仅当 \(a=-3\) 时,\(r(\mathbf{A})=3\). }
	\end{question}
	
	
	
	\begin{question}
		{选择题}
		{线性方程组}
		{中等}
		{20. 非齐次线性方程组 \(\mathbf{A x}=\mathbf{b}\) 求解时,对增广矩阵 \(\mathbf{B}=(\mathbf{A} \mid \mathbf{b})\) 施行初等变换().
			A. 只能施行初等行变换
			B. 允许进行随意的一种初等列变换
			C. 允许行及列的初等变换
			D. 允许进行初等列变换}
		{对于增广矩阵,允许进行初等行变换.故选 A.}
	\end{question}
	
	\begin{question}
		{选择题}
		{线性方程组}
		{中等}
		{21. 若方程组 \(\left\{\begin{array}{l}x_1+x_2-x_3=1 \\ 2 x_1+3 x_2+a x_3=3 \\ x_1+a x_2+3 x_3=2\end{array}\right.\) 有无穷多解,则 \(a\) 的取值为().
			A. 2
			B. 1
			C. 3
			D. 4}
		{\(\left(\begin{array}{cccc}1 & 1 & -1 & 1 \\ 2 & 3 & a & 3 \\ 1 & a & 3 & 2\end{array}\right) \rightarrow\left(\begin{array}{cccc}1 & 1 & -1 & 1 \\ 0 & 1 & a+2 & 1 \\ 0 & a-1 & 4 & 1\end{array}\right) \rightarrow\left(\begin{array}{cccc}1 & 1 & -1 & 1 \\ 0 & 1 & a+2 & 1 \\ 0 & 0 & -(a+3)(a-2) & 2-a\end{array}\right)\),当 \(a=2\) 时,\(R(A)=R(\bar{A})=2<3\),方程组有无穷多解.}
	\end{question}
	
	\begin{question}
		{选择题}
		{线性方程组}
		{容易}
		{22. 线性非齐次方程组 \(\mathbf{A}_{m \times n} \mathbf{x}=\mathbf{b}\) 有解的充要条件为().
			A. \(R(\mathbf{A})=R(\mathbf{A} \mid \mathbf{b})\)
			B. \(R(\mathbf{A})>R(\mathbf{A} \mid \mathbf{b})\)
			C. \(R(\mathbf{A})<R(\mathbf{A} \mid \mathbf{b})\)
			D. \(R(\mathbf{A})=R(\mathbf{b})\)}
		{由线性非齐次方程组解的判定法则可得.}
	\end{question}
	
	\begin{question}
		{选择题}
		{线性方程组}
		{中等}
		{23. 若 \(\left\{\begin{array}{l}x_1+2 x_2-x_3+3 x_4=4 \\ x_1+x_2-3 x_3+5 x_4=5 \\ x_2+2 x_3-2 x_4=2 \lambda\end{array}\right.\),则方程组有解的条件是 \(\lambda\) 等于().
			A. \(-\frac{1}{2}\)
			B. \(\frac{1}{2}\)
			C. -1
			D. 1}
		{\(\left(\begin{array}{cccc|c}1 & 2 & -1 & 3 & 4 \\ 1 & 1 & -3 & 5 & 5 \\ 0 & 1 & 2 & -2 & 2 \lambda\end{array}\right) \rightarrow\left(\begin{array}{cccc|c}1 & 2 & -1 & 3 & 4 \\ 0 & 1 & 2 & -2 & -1 \\ 0 & 0 & 0 & 0 & 2 \lambda+1\end{array}\right)\) 若方程组有解,则 \(\lambda=-\frac{1}{2}\).}
	\end{question}
	
	\begin{question}
		{选择题}
		{线性方程组}
		{中等}
		{24. 若 \(\left\{\begin{array}{l}x_1+x_2+x_3=0 \\ 2 x_1-x_2-a x_3=0 \\ x_1-2 x_2+3 x_3=0\end{array}\right.\) 有非零解的充要条件是 \(a\) 的取值为().
			A. -4
			B. -3
			C. -2
			D. -1}
		{\(\left(\begin{array}{ccc}1 & 1 & 1 \\ 2 & -1 & -a \\ 1 & -2 & 3\end{array}\right) \rightarrow\left(\begin{array}{ccc}1 & 1 & 1 \\ 0 & -3 & -a-2 \\ 0 & -3 & 2\end{array}\right) \rightarrow\left(\begin{array}{ccc}1 & 1 & 1 \\ 0 & -3 & -a-2 \\ 0 & 0 & a+4\end{array}\right)\) 则当 \(a=-4\)时,方程组有非零解.}
	\end{question}
	
	\begin{question}
		{选择题}
		{线性方程组}
		{中等}
		{25. 若方程组 \(\left\{\begin{array}{l}x_1+x_2-x_3=1 \\ 2 x_1+3 x_2+a x_3=3 \\ x_1+a x_2+3 x_3=2\end{array}\right.\) 无解,则 \(a\) 的取值为().
			A. -3
			B. -2
			C. -1
			D. -4}
		{\(\left(\begin{array}{cccc}1 & 1 & -1 & 1 \\ 2 & 3 & a & 3 \\ 1 & a & 3 & 2\end{array}\right) \rightarrow\left(\begin{array}{cccc}1 & 1 & -1 & 1 \\ 0 & 1 & a+2 & 1 \\ 0 & a-1 & 4 & 1\end{array}\right) \rightarrow\left(\begin{array}{cccc}1 & 1 & -1 & 1 \\ 0 & 1 & a+2 & 1 \\ 0 & 0 & -(a+3)(a-2) & 2-a\end{array}\right)\),当 \(a=-3\) 时,\(R(\mathbf{A})=2<R(\overline{\mathbf{A}})=3\),方程组无解.}
	\end{question}
	
	\begin{question}
		{选择题}
		{线性方程组}
		{中等}
		{26. 若齐次线性方程组 \(\left\{\begin{array}{l}2 x_1-x_2+3 x_3=0 \\ 3 x_1-4 x_2+7 x_3=0 \\ -x_1+2 x_2+k x_3=0\end{array}\right.\) 有非零解,则 \(k\) 的取值为().
			A. -3
			B. -2
			C. -4
			D. -1}
		{对于方程组的系数矩阵作初等变换 \(A=\left(\begin{array}{ccc}2 & -1 & 3 \\ 3 & -4 & 7 \\ -1 & 2 & k\end{array}\right) \rightarrow\left(\begin{array}{ccc}1 & -2 & -k \\ 0 & 2 & 3 k+7 \\ 0 & 3 & 2 k+3\end{array}\right) \rightarrow\left(\begin{array}{ccc}1 & -2 & -k \\ 0 & 1 & \frac{3 k+7}{2} \\ 0 & 0 & -\frac{5 k+15}{2}\end{array}\right)\) 当 \(k=-3\) 时,\(\operatorname{det} \mathbf{A}=0\),方程组有非零解.}
	\end{question}
	
	\begin{question}
		{选择题}
		{线性方程组}
		{困难}
		{27. 若齐次线性方程组 \(\left\{\begin{array}{l}3 x+2 y-z=0 \\ k x+7 y-2 z=0 \\ 2 x-y+3 z=0\end{array}\right.\) 仅有零解,则 \(k\) 应满足().
			A. \(k \neq \frac{63}{5}\)
			B. \(k=\frac{63}{5}\)
			C. \(k \neq-\frac{63}{5}\)
			D. \(k=-\frac{63}{5}\)}
		{若齐次线性方程组仅有零解,则只需方程组的系数行列式 \(|\mathbf{A}| \neq 0\).即 \(\left|\begin{array}{ccc}3 & 2 & -1 \\ k & 7 & -2 \\ 2 & -1 & 3\end{array}\right|=\left|\begin{array}{ccc}1 & 3 & -4 \\ k & 7 & -2 \\ 2 & -1 & 3\end{array}\right|=-7\left|\begin{array}{ccc}1 & 3 & -4 \\ 0 & 7-3 k & -2+4 k \\ 0 & 1 & -\frac{11}{7}\end{array}\right|=7\left|\begin{array}{ccc}1 & 3 & -4 \\ 0 & 1 & -\frac{11}{7} \\ 0 & 0 & -\frac{5}{7} k+9\end{array}\right|\).}
	\end{question}
	
	
	
	\begin{question}
		{判断题}
		{线性方程组}
		{容易}
		{1.\(n\) 阶矩阵 \(\mathbf{A}\) 可逆的充分必要条件是 \(\mathbf{A}\) 可以表示为若干初等矩阵的乘积().}
		{因为初等矩阵是可逆的,故充分性显然.反之,设 \(\mathbf{A}\) 可逆,则 \(\mathbf{A}\) 可以经过有限次初等变换变为单位矩阵 \(\mathbf{E}\) ,即存在初等矩阵 \(\mathbf{P}_1, \mathbf{P}_2, \ldots, \mathbf{P}_{\mathbf{s}} \mathbf{Q}_1, \mathbf{Q}_2, \ldots, \mathbf{Q}_t\) 使得 \(\mathbf{P}_1 \mathbf{P}_2 \ldots \mathbf{P}_s \mathbf{A} \mathbf{Q}_1 \mathbf{Q}_2 \ldots \mathbf{Q}_{\mathbf{t}}=\mathbf{E}\) 所以, \(\mathbf{A}=\mathbf{P}_1^{-1} \mathbf{P}_2^{-1} \ldots \mathbf{P}_s^{-1} \mathbf{E} \mathbf{Q}_1^{-1} \mathbf{Q}_2^{-1} \ldots \mathbf{Q}_{\mathbf{t}}\) 即 \(A\) 可以表示为若干初等矩阵的乘积.}
	\end{question}
	
	\begin{question}
		{判断题}
		{线性方程组}
		{中等}
		{2.设 \(\mathbf{A}, \mathbf{B}\) 均为 \(n\) 阶矩阵,\(\mathbf{A}\) 与 \(\mathbf{B}\) 等价,若 \(|\mathbf{A}|=0\),则 \(|\mathbf{B}|=0\)().}
		{因为 \(\mathbf{A}\) 与 \(\mathbf{B}\) 等价,所以存在可逆矩阵 \(\mathbf{P}, \mathbf{Q}\),使得 \(\mathbf{P A Q} = \mathbf{B}\),从而 \(|\mathbf{P A Q}| = |\mathbf{B}| \Rightarrow |\mathbf{P}| \cdot |\mathbf{A}| \cdot |\mathbf{Q}| = |\mathbf{B}|\),可知 \(\mathbf{A}\) 与 \(\mathbf{B}\) 相差一个非零倍数,故若 \(|\mathbf{A}|=0\),则 \(|\mathbf{B}|=0\).}
	\end{question}
	
	\begin{question}
		{判断题}
		{线性方程组}
		{中等}
		{3.初等矩阵的乘积仍然是初等矩阵(). }
		{举例:若 \(\mathbf{A}=\begin{pmatrix}1 & 1 \\ 0 & 1\end{pmatrix}, \mathbf{B}=\begin{pmatrix}1 & 0 \\ 2 & 1\end{pmatrix}\),\(\mathbf{A}, \mathbf{B}\) 是初等矩阵,但 \(\mathbf{AB} = \begin{pmatrix}3 & 1 \\ 2 & 1\end{pmatrix}\) 不是初等矩阵,故结论错误.}
	\end{question}
	
	\begin{question}
		{判断题}
		{线性方程组}
		{中等}
		{4.初等矩阵的伴随矩阵仍然是初等矩阵(). }
		{举例:若 \(\mathbf{A}=\begin{pmatrix}0 & 0 & 1 \\ 0 & 1 & 0 \\ 1 & 0 & 0\end{pmatrix}\) 是初等矩阵,但其伴随矩阵为 \(\mathbf{A}^*=\begin{pmatrix}0 & 0 & -1 \\ 0 & -1 & 0 \\ -1 & 0 & 0\end{pmatrix}\),不是初等矩阵,故结论错误.}
	\end{question}
	
	\begin{question}
		{判断题}
		{线性方程组}
		{中等}
		{5.可逆矩阵只经过列初等变换就可化为单位矩阵(). }
		{因为可逆矩阵 \(\mathbf{A}\) 与单位矩阵 \(\mathbf{E}\) 等价,所以存在可逆矩阵 \(\mathbf{P}, \mathbf{Q}\),使得 \(\mathbf{PAQ} = \mathbf{E}\),即 \(\mathbf{AQ} = \mathbf{P}^{-1}\),再乘以 \(\mathbf{P}\) 得 \(\mathbf{AQP} = \mathbf{E}\).由于 \(\mathbf{Q}, \mathbf{P}\) 可逆,所以 \(\mathbf{A}\) 只经过列初等变换即可化为单位矩阵 \(\mathbf{E}\).}
	\end{question}
	
	\begin{question}
		{判断题}
		{线性方程组}
		{容易}
		{6.矩阵 \(\mathbf{A}=\begin{pmatrix}1 & 0 & 1 \\ 0 & 1 & 0 \\ 0 & 0 & 0\end{pmatrix}\) 是初等矩阵(). }
		{不能通过一次初等变换将单位矩阵变为该矩阵,故 \(\mathbf{A}\) 不是初等矩阵.}
	\end{question}
	
	\begin{question}
		{判断题}
		{线性方程组}
		{容易}
		{7.矩阵 \(\mathbf{A}=\begin{pmatrix}0 & 0 & 1 \\ 0 & 1 & 0 \\ 1 & 0 & 0\end{pmatrix}\) 是初等矩阵(). }
		{交换单位矩阵的第1行与第3行可以得到该矩阵,故该矩阵是初等矩阵,结论正确.}
	\end{question}
	
	\begin{question}
		{判断题}
		{线性方程组}
		{容易}
		{8.初等矩阵 \(\mathbf{E}(i,j)\) 的行列式等于 \(-1\)(). }
		{初等矩阵 \(\mathbf{E}(i,j)\) 是由互换单位矩阵的第 \(i\) 列与第 \(j\) 列得到,互换列会使行列式变号,故 \(|\mathbf{E}(i,j)| = -1\).}
	\end{question}
	
	\begin{question}
		{判断题}
		{线性方程组}
		{容易}
		{9.单位矩阵是初等矩阵(). }
		{单位矩阵可以看作把单位矩阵的某列的 \(0\) 倍加到另一列上,属于初等变换,故单位矩阵是初等矩阵.}
	\end{question}
	
	\begin{question}
		{判断题}
		{线性方程组}
		{中等}
		{10.初等矩阵 \(\mathbf{A}\) 的逆矩阵是初等矩阵 \(\mathbf{A}\)(). }
		{只有交换单位矩阵第 \(i\) 行与第 \(j\) 行所得到的初等矩阵的逆等于其本身,其余两类初等矩阵的逆不等于本身.例如,\(\begin{pmatrix}1 & 0 \\ 2 & 1\end{pmatrix}\) 是初等矩阵,但其逆为 \(\begin{pmatrix}1 & 0 \\ -2 & 1\end{pmatrix}\),不是其本身.}
	\end{question}
	
	
	\begin{question}
		{判断题}
		{线性方程组}
		{容易}
		{11. 初等矩阵的之和仍然是初等矩阵(). }
		{举例:设 \(\mathbf{A}=\begin{pmatrix}1 & 1 \\ 0 & 1\end{pmatrix},\ \mathbf{B}=\begin{pmatrix}1 & 0 \\ 2 & 1\end{pmatrix}\),\(\mathbf{A},\ \mathbf{B}\) 是初等矩阵,但 \(\mathbf{A}+\mathbf{B}=\begin{pmatrix}2 & 1 \\ 2 & 2\end{pmatrix}\),不是初等矩阵.}
	\end{question}
	
	\begin{question}
		{判断题}
		{线性方程组}
		{中等}
		{12. \(n\) 阶矩阵 \(\mathbf{A}\) 可逆的充分必要条件是 \(\mathbf{A}\) 可以表示为若干初等矩阵之和(). }
		{正确命题应为:\(n\) 阶矩阵 \(\mathbf{A}\) 可逆的充分必要条件是 \(\mathbf{A}\) 可以表示为若干初等矩阵的乘积.}
	\end{question}
	
	\begin{question}
		{判断题}
		{线性方程组}
		{困难}
		{13. 设矩阵 \(\mathbf{A}, \mathbf{B}, \mathbf{X}\) 满足 \(\mathbf{A}\mathbf{X}=\mathbf{B}\),其中 \(\mathbf{A}=\begin{pmatrix}4 & 1 & -2 \\ 2 & 2 & 1 \\ 3 & 1 & -1\end{pmatrix},\ \mathbf{B}=\begin{pmatrix}1 & -3 \\ 2 & 2 \\ 3 & -1\end{pmatrix}\),则矩阵 \(\mathbf{X}=\begin{pmatrix}10 & 2 \\ -15 & -3 \\ 12 & 4\end{pmatrix}\)(). }
		{由于 \(\mathbf{A}\) 可逆,\(\mathbf{X} = \mathbf{A}^{-1}\mathbf{B}\),可通过对增广矩阵 \((\mathbf{A} \mid \mathbf{B})\) 进行初等行变换得到结果,计算可得 \(\mathbf{X} = \begin{pmatrix}10 & 2 \\ -15 & -3 \\ 12 & 4\end{pmatrix}\),故判断正确.}
	\end{question}
	
	\begin{question}
		{判断题}
		{线性方程组}
		{中等}
		{14. 设矩阵 \(\mathbf{A}=\begin{pmatrix}0 & 2 & 0 \\ 0 & 0 & -3 \\ 4 & 0 & 0\end{pmatrix}\),则 \(\mathbf{A}^{-1}=\begin{pmatrix}0 & 0 & \frac{1}{4} \\ \frac{1}{2} & 0 & 0 \\ 0 & -\frac{1}{3} & 0\end{pmatrix}\)(). }
		{将增广矩阵 \((\mathbf{A} \mid \mathbf{E})\) 进行初等行变换后可得逆矩阵 \(\mathbf{A}^{-1}=\begin{pmatrix}0 & 0 & \frac{1}{4} \\ \frac{1}{2} & 0 & 0 \\ 0 & -\frac{1}{3} & 0\end{pmatrix}\),判断正确.}
	\end{question}
	
	\begin{question}
		{判断题}
		{线性方程组}
		{困难}
		{15. 设 \(\mathbf{A}\) 是 \(m \times n\) 矩阵,\(\mathbf{B}\) 是 \(n \times s\) 矩阵,若 \(r(\mathbf{B})=n\),则 \(r(\mathbf{AB})=r(\mathbf{A})\)(). }
		{因为 \(r(\mathbf{B})=n\),说明 \(\mathbf{B}\) 的列向量线性无关,\(\mathbf{B}\) 可化为标准形,存在可逆矩阵 \(\mathbf{Q}\) 使得 \(\mathbf{B}\mathbf{Q} = \begin{pmatrix} \mathbf{E}_n & \mathbf{B}_1 \end{pmatrix}\),于是 \(\mathbf{ABQ} = (\mathbf{A},\ \mathbf{AB}_1)\),故 \(r(\mathbf{AB}) = r(\mathbf{A})\).判断正确.}
	\end{question}
	
	\begin{question}
		{判断题}
		{线性方程组}
		{容易}
		{16. 设 \(\mathbf{A}=\begin{pmatrix}1 & 0 & 1 \\ 0 & 2 & 0 \\ 0 & 0 & 1\end{pmatrix}\),\(\mathbf{B}=\mathbf{A}-\mathbf{E}\),则 \(r(\mathbf{B})=1\)(). }
		{由 \(\mathbf{B}=\mathbf{A}-\mathbf{E}=\begin{pmatrix}0 & 0 & 1 \\ 0 & 1 & 0 \\ 0 & 0 & 0\end{pmatrix}\),行简化得 \(r(\mathbf{B})=2\),判断错误.}
	\end{question}
	
	
	\begin{question}
		{判断题}
		{线性方程组}
		{中等}
		{17.设 \(\mathbf{A}^*\) 是 \(n\) 阶方阵 \(\mathbf{A}\) 的伴随矩阵,若 \(r(\mathbf{A})=n\) ,则 \(r\left(\mathbf{A}^*\right)=n\)().}
		{当 \(r(\mathbf{A})=n\) 时, \(\mathbf{A}\) 可逆,且 \(\mathbf{A}^*=|\mathbf{A}| \mathbf{A}^{-1}\) 也可逆,所以 \(r\left(\mathbf{A}^*\right)=n\) .}
	\end{question}
	
	\begin{question}
		{判断题}
		{线性方程组}
		{中等}
		{18.设 \(\mathbf{A}\) 是 \(n\) 阶方阵,则 \(r(\mathbf{AB})=r(\mathbf{BA})()\) . }
		{举例:设 \(\mathbf{A}=\left(\begin{array}{ll}1 & 0 \\ 0 & 0\end{array}\right), \mathbf{B}=\left(\begin{array}{ll}0 & 0 \\ 1 & 0\end{array}\right)\) ,则 \(r(\mathbf{AB})=0, r(\mathbf{BA})=1\) ,由此知结论错误.}
	\end{question}
	
	\begin{question}
		{判断题}
		{线性方程组}
		{中等}
		{19.若 \(\mathbf{A}\) 的所有 \(r\) 阶子式都为零,则 \(\mathbf{A}\) 的所有 \(r+1\) 阶子式也都为零(). }
		{对 \(\mathbf{A}\) 的任一 \(r+1\) 阶子式按第一行展开,等于若干个 \(r\) 阶子式的线性组合,所以等于零.}
	\end{question}
	
	\begin{question}
		{判断题}
		{线性方程组}
		{容易}
		{20.凡是秩相等的矩阵一定是等价矩阵().}
		{正确命题应该是:凡是秩相等的同阶矩阵一定是等价矩阵.}
	\end{question}
	
	\begin{question}
		{判断题}
		{线性方程组}
		{中等}
		{21.设 \(\mathbf{A}\) 为三阶方阵,若 \(r(\mathbf{A})=1\) ,则 \(r\left(\mathbf{A}^*\right)=0()\) .}
		{因为 \(r(\mathbf{A})=1\) ,所以 \(\mathbf{A}\) 的任意二阶子式都为零,从而 \(\mathbf{A}\) 的任意元素的代数余子式都为零,故 \(r\left(\mathbf{A}^*\right)=0\) .}
	\end{question}
	
	\begin{question}
		{判断题}
		{线性方程组}
		{中等}
		{22.若三阶方阵 \(\mathbf{A}=\left(\begin{array}{ccc}1 & 2 & 3 \\ 2 & 3 & -5 \\ 4 & 7 & 1\end{array}\right)\) ,则 \(r(\mathbf{A})=2()\) . }
		{由 \(\mathbf{A}=\left(\begin{array}{ccc}1 & 2 & 3 \\ 2 & 3 & -5 \\ 4 & 7 & 1\end{array}\right) \xrightarrow[r_2-2 r_1]{r_3-2 r_2}\left(\begin{array}{ccc}1 & 2 & 3 \\ 0 & -1 & -11 \\ 0 & -1 & -11\end{array}\right) \xrightarrow{r_3-r_2}\left(\begin{array}{ccc}1 & 2 & 3 \\ 0 & -1 & -11 \\ 0 & 0 & 0\end{array}\right)\) ,故 \(r(\mathbf{A})=2\).}
	\end{question}
	
	\begin{question}
		{判断题}
		{线性方程组}
		{容易}
		{23.如果 \(\mathbf{Ax}=\mathbf{b}\) 有唯一解,则 \(\mathbf{Ax}=\mathbf{0}\) 只有零解(). }
		{如果 \(\mathbf{Ax}=\mathbf{b}\) 有唯一解,则 \(\mathbf{Ax}=\mathbf{0}\) 只有零解.}
	\end{question}
	
	\begin{question}
		{判断题}
		{线性方程组}
		{容易}
		{24.齐次方程组 \(\mathbf{Ax}=\mathbf{0}\) 可能没有解(). }
		{齐次方程组 \(\mathbf{Ax}=\mathbf{0}\) 恒有解(零解). }
	\end{question}
	
	\begin{question}
		{判断题}
		{线性方程组}
		{中等}
		{25.若 \(\mathbf{Ax}=0\) 只有零解,且 \(\mathbf{A}\) 是 \(n\) 阶矩阵,则 \(R(\mathbf{A})=n-1()\) .}
		{若 \(\mathbf{Ax}=0\) 只有零解,则 \(R(\mathbf{A})=n\) .}
	\end{question}
	
	\begin{question}
		{判断题}
		{线性方程组}
		{中等}
		{26.若 \(\mathbf{Ax}=\mathbf{0}\) 仅有零解,则 \(\mathbf{Ax}=\mathbf{b}\) 无解(). }
		{\(r(\mathbf{A})=n\) ,则 \(\mathbf{Ax}=\mathbf{b}\) 有唯一解.}
	\end{question}
	
	
	\begin{question}
		{判断题}
		{线性方程组}
		{中等}
		{27.若 \(\mathbf{A x}=\mathbf{0}\) 有非零解,则 \(\mathbf{A x}=\mathbf{b}\) 有无穷多个解(). }
		{由题设条件不能判定方程组 \(\mathbf{A x}=\mathbf{b}\) 的系数矩阵与增广矩阵的秩是否相等,故 \(\mathbf{A x}=\mathbf{b}\) 可能无解.}
	\end{question}
	
	\begin{question}
		{判断题}
		{线性方程组}
		{容易}
		{28.非齐次线性方程组 \(\mathbf{A X}=\mathbf{b}\) 求解时,对增广矩阵 \(\mathbf{B}=(\mathbf{A} \mid \mathbf{b})\) 只能施行初等行变换,得到的仍是等价方程(). }
		{对于增广矩阵,允许进行初等行变换.}
	\end{question}
	
	\begin{question}
		{判断题}
		{线性方程组}
		{容易}
		{29.当线性方程组 \(\mathbf{A} \mathbf{x}=\mathbf{b}\) 的系数矩阵的秩小于增广矩阵的秩时,方程组无解(). }
		{根据线性方程组解的判定易得该结论正确.}
	\end{question}
	
	\begin{question}
		{判断题}
		{线性方程组}
		{中等}
		{30.线性方程组 \(\left\{\begin{aligned} 3 x_1+x_2+x_3 & =5 \\ 3 x_1+2 x_2+3 x_3 & =3 \\ x_2+2 x_3 & =0\end{aligned}\right.\) 有解(). }
		{由于 \(\overline{\mathbf{A}}=\left(\begin{array}{lll|l}3 & 1 & 1 & 5 \\ 3 & 2 & 3 & 3 \\ 0 & 1 & 2 & 0\end{array}\right) \rightarrow\left(\begin{array}{lll|l}3 & 1 & 1 & 5 \\ 0 & 1 & 2 & -2 \\ 0 & 1 & 2 & 0\end{array}\right) \rightarrow\left(\begin{array}{lll|l}3 & 1 & 1 & 5 \\ 0 & 1 & 2 & -2 \\ 0 & 0 & 0 & 2\end{array}\right)\) ,\(r(\mathbf{A})=2 \neq r(\overline{\mathbf{A}})=3\) ,所以方程组无解.}
	\end{question}
	
	\begin{question}
		{判断题}
		{线性方程组}
		{容易}
		{31.当线性方程组 \(\mathbf{A} \mathbf{x}=\mathbf{b}\) 的系数矩阵与增广矩阵的秩相等时,方程组有唯一的解(). }
		{由方程组系数与解的关系可知该结论错误,此时方程组有解,但不一定唯一,需进一步判断未知数个数与秩是否相等.}
	\end{question}
	
	
	\begin{question}
		{填空题}
		{线性方程组}
		{容易}
		{1.设矩阵 \(\mathbf{A}=\left(\begin{array}{ccc}2 & 2 & 3 \\ 1 & -1 & 0 \\ -1 & 2 & 1\end{array}\right)\) 则 \(A\) 的逆矩阵 \(\mathbf{A}^{-1}=()\) . }
		{\[
			\begin{aligned}
				& (\mathbf{A} \mid \mathbf{E})=\left(\begin{array}{ccc|ccc}
					2 & 2 & 3 & 1 & 0 & 0 \\
					1 & -1 & 0 & 0 & 1 & 0 \\
					-1 & 2 & 1 & 0 & 0 & 1
				\end{array}\right) \xrightarrow[r_1-2 r_2]{r_3+r_2}\left(\begin{array}{ccc|ccc}
					0 & 4 & 3 & 1 & -2 & 0 \\
					1 & -1 & 0 & 0 & 1 & 0 \\
					0 & 1 & 1 & 0 & 1 & 1
				\end{array}\right) \\
				& \xrightarrow{r_1-3 r_3}\left(\begin{array}{ccc|cc}
					0 & 1 & 0 & 1 & -5 \\
					1 & -1 & 0 & -3 \\
					0 & 1 & 1 & 1 & 0 \\
					0 & 1 & 1
				\end{array}\right) \xrightarrow[r_3-r_1]{r_2+r_1}\left(\begin{array}{lll|ccc}
					0 & 1 & 0 & 1 & -5 & -3 \\
					1 & 0 & 0 & 1 & -4 & -3 \\
					0 & 0 & 1 & -1 & 6 & 4
				\end{array}\right) \text { 知 } \mathbf{A}^{-1}= \\
				& \left(\begin{array}{ccc}
					1 & -4 & -3 \\
					1 & -5 & -3 \\
					-1 & 6 & 4
				\end{array}\right) .
			\end{aligned}
			\]} 
	\end{question}
	
	\begin{question}
		{填空题}
		{线性方程组}
		{中等}
		{2. 若矩阵 \(\mathbf{X}\) 满足 \(\left(\begin{array}{ll}3 & 5 \\ 1 & 2\end{array}\right) \mathbf{X}=\left(\begin{array}{ccc}4 & -1 & 2 \\ 3 & 0 & -1\end{array}\right)\) ,则 \(\mathbf{X}=()\) . }
		{\[
			\begin{aligned}
				& \left(\begin{array}{cc|ccc}3 & 5 & 4 & -1 & 2 \\ 1 & 2 & 3 & 0 & -1\end{array}\right) \xrightarrow{r_1 - 3r_2} \left(\begin{array}{cc|ccc}0 & -1 & -5 & -1 & 5 \\ 1 & 2 & 3 & 0 & -1\end{array}\right) \\
				& \xrightarrow{-r_1} \left(\begin{array}{cc|ccc}0 & 1 & 5 & 1 & -5 \\ 1 & 2 & 3 & 0 & -1\end{array}\right) \xrightarrow{r_2 - 2r_1} \left(\begin{array}{cc|ccc}0 & 1 & 5 & 1 & -5 \\ 1 & 0 & -7 & -2 & 9\end{array}\right) \\
				& \text{故 } \mathbf{X}=\left(\begin{array}{ccc}-7 & -2 & 9 \\ 5 & 1 & -5\end{array}\right) . 
			\end{aligned}
			\]}
	\end{question}
	
	\begin{question}
		{填空题}
		{线性方程组}
		{容易}
		{3. 设 \(\mathbf{A}=\left(\begin{array}{lll}1 & 1 & 1 \\ 2 & 2 & 5 \\ 1 & 1 & t\end{array}\right)\) 的秩 \(r(\mathbf{A})=2\) ,则 \(t=()\) . }
		{\[
			\begin{aligned}
				& \mathbf{A} = \left(\begin{array}{ccc}1 & 1 & 1 \\ 2 & 2 & 5 \\ 1 & 1 & t\end{array}\right) \xrightarrow{r_2 - 2r_1,\, r_3 - r_1} \left(\begin{array}{ccc}1 & 1 & 1 \\ 0 & 0 & 3 \\ 0 & 0 & t - 1\end{array}\right) \\
				& \text{由 } r(\mathbf{A})=2 \text{ 知 } t - 1 = 0 \Rightarrow t = 1. 
			\end{aligned}
			\]}
	\end{question}
	
	\begin{question}
		{填空题}
		{线性方程组}
		{中等}
		{4. 设 \(\mathbf{A}=\left(\begin{array}{ccc}1 & -2 & 3k \\ -1 & 2k & -3 \\ k & -2 & 3\end{array}\right)\) 的秩 \(r(\mathbf{A})=1\) ,则 \(k=()\) . }
		{\[
			\begin{aligned}
				& \mathbf{A} = \left(\begin{array}{ccc}1 & -2 & 3k \\ -1 & 2k & -3 \\ k & -2 & 3\end{array}\right) \rightarrow \left(\begin{array}{ccc}1 & -2 & 3k \\ 0 & 2(k - 1) & 3(k - 1) \\ k - 1 & 0 & 3(1 - k)\end{array}\right) \\
				& \text{由 } r(\mathbf{A}) = 1 \text{ 知 } k - 1 = 0 \Rightarrow k = 1. 
			\end{aligned}
			\]}
	\end{question}
	
	\begin{question}
		{填空题}
		{线性方程组}
		{容易}
		{5. 从矩阵 \(\mathbf{A}\) 中划去一行得到矩阵 \(\mathbf{B}\) ,则 \(\mathbf{A}, \mathbf{B}\) 的秩 \(r(\mathbf{A}), r(\mathbf{B})\) 的大小关系为(). }
		{\[
			\begin{aligned}
				& \text{设 } r(\mathbf{B}) = r,\text{ 则 } \mathbf{B} \text{ 存在某个 } r \text{ 阶非零子式,} \\
				& \text{由于 } \mathbf{B} \text{ 是由 } \mathbf{A} \text{ 删去一行所得,故 } \mathbf{A} \text{ 同样存在某个 } r \text{ 阶非零子式} \\
				& \Rightarrow r(\mathbf{A}) \geq r(\mathbf{B}). 
			\end{aligned}
			\]}
	\end{question}
	
	\begin{question}
		{填空题}
		{线性方程组}
		{困难}
		{6. 已知方程组 \(\left[\begin{array}{ccc}1 & -1 & 3 \\ 0 & 1 & a-2 \\ 3 & a & 23\end{array}\right] \left[\begin{array}{l}x_1 \\ x_2 \\ x_3\end{array}\right] = \left[\begin{array}{c}2 \\ a \\ 16\end{array}\right]\) 有穷多解,则 \(a=()\) . }
		{\[
			\begin{aligned}
				& \text{增广矩阵为 } \left[\begin{array}{ccc|c}1 & -1 & 3 & 2 \\ 0 & 1 & a-2 & a \\ 3 & a & 23 & 16\end{array}\right] \\
				& \xrightarrow{r_3 - 3r_1} \left[\begin{array}{ccc|c}1 & -1 & 3 & 2 \\ 0 & 1 & a-2 & a \\ 0 & a+3 & 14 & 10\end{array}\right] \\
				& \xrightarrow{r_3 - (a+3)r_2} \left[\begin{array}{ccc|c}1 & -1 & 3 & 2 \\ 0 & 1 & a-2 & a \\ 0 & 0 & 20 - a - a^2 & 10 - 3a - a^2\end{array}\right] \\
				& \text{要求有穷多解,需 } r(\mathbf{A}) = r(\overline{\mathbf{A}}) < 3 \Rightarrow 20 - a - a^2 = 0 \Rightarrow a = -5. 
			\end{aligned}
			\]}
	\end{question}
	
	\begin{question}
		{计算题}
		{线性方程组}
		{容易}
		{1.若矩阵 \(X\) 满足 \(\mathbf{A X}+\mathbf{E}=\mathbf{A}^2+\mathbf{X}\) 其中矩阵 \(\mathbf{A}=\left(\begin{array}{ccc}1 & 0 & 1 \\ 0 & 2 & 0 \\ 1 & 0 & 1\end{array}\right)\) 求矩阵 \(X\)(\qquad).}
		{因为 \(\mathbf{A X}+\mathbf{E}=\mathbf{A}^2+\mathbf{X} \Rightarrow(\mathbf{A}-\mathbf{E}) \mathbf{X}=\mathbf{A}^2-\mathbf{E}, \mathbf{A}-\mathbf{E}=\left(\begin{array}{lll}0 & 0 & 1 \\ 0 & 1 & 0 \\ 1 & 0 & 0\end{array}\right)\) ,所以 \(\mathbf{A}-\mathbf{E}\) 可逆,故 \(\mathbf{X}=(\mathbf{A}-\mathbf{E})^{-1}\left(\mathbf{A}^2-\mathbf{E}\right)=(\mathbf{A}-\mathbf{E})^{-1}(\mathbf{A}-\mathbf{E})(\mathbf{A}+\mathbf{E})=\mathbf{A}+\mathbf{E}=\) \(\left(\begin{array}{lll}2 & 0 & 1 \\ 0 & 3 & 0 \\ 1 & 0 & 2\end{array}\right)\).}
	\end{question}
	
	\begin{question}
		{计算题}
		{线性方程组}
		{中等}
		{2.设 \(\mathbf{A}=\left(\begin{array}{ccc}3 & 0 & 1 \\ 1 & 1 & 0 \\ 0 & 1 & 4\end{array}\right)\) 且有 \(\mathbf{A X}=2 \mathbf{X}+\mathbf{A}\) ,求矩阵 \(\mathbf{X}\)(\qquad).}
		{由 \(\mathbf{A X}=2 \mathbf{X}+\mathbf{A} \Rightarrow(\mathbf{A}-2 \mathbf{E}) \mathbf{X}=\mathbf{A} \Rightarrow \mathbf{X}=(\mathbf{A}-2 \mathbf{E})^{-1} \mathbf{A}\) ,又 \(\mathbf{A}-2 \mathbf{E}=\left(\begin{array}{ccc}1 & 0 & 1 \\ 1 & -1 & 0 \\ 0 & 1 & 2\end{array}\right)\) ,可知 \((\mathbf{A}-2 \mathbf{E} \mid \mathbf{A})=\left(\begin{array}{ccc|ccc}1 & 0 & 1 & 3 & 0 & 1 \\ 1 & -1 & 0 & 1 & 1 & 0 \\ 0 & 1 & 2 & 0 & 1 & 4\end{array}\right) \xrightarrow{r_2 - r_1,\, r_3 - r_1 + r_2} \left(\begin{array}{ccc|ccc}1 & 0 & 1 & 3 & 0 & 1 \\ 0 & -1 & -1 & -2 & 1 & -1 \\ 0 & 0 & 1 & -2 & 2 & 3\end{array}\right) \xrightarrow{r_2 + r_3,\, r_1 - r_3} \left(\begin{array}{ccc|ccc}1 & 0 & 0 & 5 & -2 & -2 \\ 0 & -1 & 0 & -4 & 3 & 2 \\ 0 & 0 & 1 & -2 & 2 & 3\end{array}\right)\) 故 \(\mathbf{X}=\left(\begin{array}{ccc}5 & -2 & -2 \\ 4 & -3 & -2 \\ -2 & 2 & 3\end{array}\right)\).}
	\end{question}
	
	\begin{question}
		{计算题}
		{线性方程组}
		{中等}
		{3.若矩阵 \(\mathbf{X}\) 满足 \(\mathbf{A X}+\mathbf{B}=\mathbf{X}\) ,其中 \(\mathbf{A}=\left(\begin{array}{ccc}0 & 1 & 0 \\ -1 & 1 & 1 \\ -1 & 0 & -1\end{array}\right),\quad \mathbf{B}=\left(\begin{array}{cc}1 & -1 \\ 2 & 0 \\ 5 & -3\end{array}\right)\) ,求矩阵 \(\mathbf{X}\)(\qquad).}
		{由 \(\mathbf{A X}+\mathbf{B}=\mathbf{X} \Rightarrow (\mathbf{E}-\mathbf{A}) \mathbf{X}=\mathbf{B}\) ,因为 \(|\mathbf{E}-\mathbf{A}| \neq 0\) ,所以 \(\mathbf{E}-\mathbf{A}\) 可逆,从而 \(\mathbf{X}=(\mathbf{E}-\mathbf{A})^{-1} \mathbf{B}\).由 \((\mathbf{E}-\mathbf{A} \mid \mathbf{B})=\left(\begin{array}{ccc|cc}-1 & -1 & 0 & 1 & -1 \\ 1 & 0 & -1 & 2 & 0 \\ 1 & 0 & -2 & 5 & -3\end{array}\right) \xrightarrow{r_2 - r_1,\, r_3 - r_2} \left(\begin{array}{ccc|cc}1 & -1 & 0 & 1 & -1 \\ 0 & 1 & -1 & 1 & 1 \\ 0 & 0 & 1 & 1 & -1\end{array}\right) \xrightarrow{r_2 + r_3,\, r_1 + r_2 + r_3} \left(\begin{array}{ccc|cc}1 & 0 & 0 & 3 & -1 \\ 0 & 1 & 0 & 2 & 0 \\ 0 & 0 & 1 & 1 & -1\end{array}\right)\) ,得 \(\mathbf{X}=\left(\begin{array}{cc}3 & -1 \\ 2 & 0 \\ 1 & -1\end{array}\right)\).}
	\end{question}
	
	\begin{question}
		{计算题}
		{线性方程组}
		{中等}
		{4.确定 \(\lambda\) 的值,使得矩阵 \(\mathbf{A}=\left(\begin{array}{cccc}1 & \lambda & -1 & 2 \\ 2 & -1 & \lambda & 5 \\ 1 & 10 & -6 & 1\end{array}\right)\) 的秩最小(\qquad).}
		{对 \(\mathbf{A}^T\) 做行变换,得 \(\mathbf{A}^T=\left(\begin{array}{ccc}1 & 2 & 1 \\ \lambda & -1 & 10 \\ -1 & \lambda & -6 \\ 2 & 5 & 1\end{array}\right) \xrightarrow{r_2 - 10 r_1,\, r_3 + 6 r_1,\, r_4 - r_1} \left(\begin{array}{ccc}1 & 2 & 1 \\ \lambda - 10 & -21 & 10 \\ 5 & \lambda + 12 & 0 \\ 1 & 3 & 0\end{array}\right) \xrightarrow{r_2 + 7 r_4,\, r_3 - 5 r_4} \left(\begin{array}{ccc}1 & 2 & 1 \\ \lambda - 3 & 0 & 0 \\ 0 & \lambda - 3 & 0 \\ 1 & 3 & 0\end{array}\right)\) ,故当 \(\lambda = 3\) 时,矩阵秩最小,且 \(r(\mathbf{A})=2\).}
	\end{question}
	
	\begin{question}
		{计算题}
		{线性方程组}
		{困难}
		{5.若矩阵 \(\mathbf{A}=\left(\begin{array}{cccc}1 & -1 & 1 & 2 \\ 3 & \lambda & -1 & 2 \\ 5 & 3 & \mu & 6\end{array}\right)\) ,且 \(r(\mathbf{A})=2\) ,求 \(\lambda\) 与 \(\mu\) 的值(\qquad).}
		{对 \(\mathbf{A}^T\) 做行变换:\(\mathbf{A}^T=\left(\begin{array}{ccc}1 & 3 & 5 \\ -1 & \lambda & 3 \\ 1 & -1 & \mu \\ 2 & 2 & 6\end{array}\right) \xrightarrow{r_2 + r_1,\, r_3 - r_1,\, r_4 - 2r_1} \left(\begin{array}{ccc}1 & 3 & 5 \\ 0 & \lambda + 3 & 8 \\ 0 & -4 & \mu - 5 \\ 0 & -4 & -4\end{array}\right) \xrightarrow{r_4 - r_3} \left(\begin{array}{ccc}1 & 3 & 5 \\ 0 & \lambda + 3 & 8 \\ 0 & -4 & \mu - 5 \\ 0 & 0 & 1 - \mu\end{array}\right) \xrightarrow{r_3 + r_4} \left(\begin{array}{ccc}1 & 3 & 5 \\ 0 & \lambda + 3 & 8 \\ 0 & -4 & -4 \\ 0 & 0 & 1 - \mu\end{array}\right) \xrightarrow{r_2 + 2r_3} \left(\begin{array}{ccc}1 & 3 & 5 \\ 0 & \lambda - 5 & 0 \\ 0 & -4 & -4 \\ 0 & 0 & 1 - \mu\end{array}\right)\) ,为了秩为2,需有 \(\lambda = 5, \mu = 1\).}
	\end{question}
	
	
	\begin{question}
		{计算题}
		{线性方程组}
		{中等}
		{6. 求矩阵 \(\mathbf{A}=\left(\begin{array}{cccc}1 & \lambda & -1 & 2 \\ 2 & -1 & \lambda & 5 \\ 1 & 10 & -6 & 1\end{array}\right)\) 的秩,其中 \(\lambda\) 为参数(\qquad).}
		{ \(\mathbf{A}^T=\left(\begin{array}{ccc}1 & 2 & 1 \\ \lambda & -1 & 10 \\ -1 & \lambda & -6 \\ 2 & 5 & 1\end{array}\right) \xrightarrow[r_4-r_1]{\substack{r_2-10 r_1 \\ r_3+6 r_1}}\left(\begin{array}{ccc}1 & 2 & 1 \\ \lambda-10 & -21 & 0 \\ 5 & \lambda+12 & 0 \\ 1 & 3 & 0\end{array}\right) \xrightarrow[r_4-r_1]{\substack{r_2+7 r_4 \\ r_3-5 r_4}} \left(\begin{array}{ccc}1 & 2 & 1 \\ \lambda-3 & 0 & 0 \\ 0 & \lambda-3 & 0 \\ 0 & 1 & -1\end{array}\right)\) 所以当 \(\lambda=3\) 时,\(r(\mathbf{A})=2\) ;当 \(\lambda \neq 3\) 时,\(r(\mathbf{A})=3\) .}
	\end{question}
	
	\begin{question}
		{计算题}
		{线性方程组}
		{中等}
		{7. 讨论矩阵 \(\mathbf{A}=\left(\begin{array}{lll}x & 1 & 1 \\ 1 & x & 1 \\ 1 & 1 & x\end{array}\right)\) 的秩(\qquad).}
		{ \(\mathbf{A}^T=\left(\begin{array}{lll}x & 1 & 1 \\ 1 & x & 1 \\ 1 & 1 & x\end{array}\right) \xrightarrow[r_3-r_2]{\substack{r_1+r_2 \\ r_1+r_3}}\left(\begin{array}{ccc}x+2 & x+2 & x+2 \\ 1 & x & 1 \\ 0 & 1-x & x-1\end{array}\right)\) ,所以当 \(x=1\) 时, \(r(\mathbf{A})=1\) :当 \(x=-2\) 时,\(r(\mathbf{A})=2\) ;当 \(x \neq 1\) 且 \(x \neq-2\) 时;\(r(\mathbf{A})=3\) .}
	\end{question}
	
	
	\begin{question}
		{计算题}
		{线性方程组}
		{中等}
		{8. 设线性方程组为 \(\left\{\begin{array}{l}x_1-3 x_2-x_3=0 \\ x_1-4 x_2+a x_3=b \\ 2 x_1-x_2+3 x_3=5\end{array}\right.\) ,问:\(a 、 b\) 取何值时,方程组无解、有惟一解、有无穷多解 ().}
		{ \(\overline{\mathbf{A}}=\left(\begin{array}{cccc}1 & -3 & -1 & 0 \\ 1 & -4 & a & b \\ 2 & -1 & 3 & 5\end{array}\right) \rightarrow\left(\begin{array}{cccc}1 & -3 & -1 & 0 \\ 0 & 1 & 1 & 1 \\ 0 & 0 & a+2 & b+1\end{array}\right)\) 当 \(a \neq-2\) 时,方程组有惟一解;当 \(a=-2, b \neq-1\) 时,方程组无解;当 \(a=-2, b=-1\) 时,\(r(A)=r(\bar{A})=2<3\) ,方程组有无穷多组解.}
	\end{question}
	
	\begin{question}
		{计算题}
		{线性方程组}
		{中等}
		{9. 设线性方程组 \(\left\{\begin{array}{r}x_1+x_3=2 \\ x_1+2 x_2-x_3=0 \\ 2 x_1+x_2-a x_3=b\end{array}\right.\) 讨论当 \(a, b\) 为何值时,方程组无解,有惟一解,有无穷多解.}
		{ \(\overline{\mathbf{A}}=\left(\begin{array}{cccc}1 & 0 & 1 & 2 \\ 1 & 2 & -1 & 0 \\ 2 & 1 & -a & b\end{array}\right) \rightarrow\left(\begin{array}{cccc}1 & 0 & 1 & 2 \\ 0 & 2 & -2 & -2 \\ 0 & 1 & -a-2 & b-4\end{array}\right) \rightarrow \left(\begin{array}{cccc}1 & 0 & 1 & 2 \\ 0 & 1 & -1 & -1 \\ 0 & 0 & -a-1 & b-3\end{array}\right)\) ,所以当 \(a=-1\) 且 \(b \neq 3\) 时,方程组无解;当 \(a \neq-1\) 时,方程组有唯一解:当 \(a=-1\) 且 \(b=3\) 时,方程组有无穷多解.}
	\end{question}
	
	
	\begin{question}
		{计算题}
		{线性方程组}
		{中等}
		{10. 设线性方程组 \(\left\{\begin{array}{c}x_1+2 x_3=-1 \\ -x_1+x_2-3 x_3=2 \\ 2 x_1-x_2+5 x_3=0\end{array}\right.\) ,求其系数矩阵和增广矩阵的秩,并判断其解的情况().}
		{ 因为 \(\overline{\mathbf{A}}=\left(\begin{array}{cccc}1 & 0 & 2 & -1 \\ -1 & 1 & -3 & 2 \\ 2 & -1 & 5 & 0\end{array}\right) \rightarrow\left(\begin{array}{cccc}1 & 0 & 2 & -1 \\ 0 & 1 & -1 & 1 \\ 0 & -1 & 1 & 2\end{array}\right) \rightarrow \left(\begin{array}{cccc}1 & 0 & 2 & -1 \\ 0 & 1 & -1 & 1 \\ 0 & 0 & 0 & 3\end{array}\right)\) ,所以 \(r(\mathbf{A})=2, r(\overline{\mathbf{A}})=3\) .又因为 \(r(\mathbf{A})<r(\overline{\mathbf{A}})\) ,所以方程组无解.}
	\end{question}
	
	
	\begin{question}
		{计算题}
		{线性方程组}
		{中等}
		{11.设齐次线性方程组 \(\left\{\begin{array}{c}x_1-3 x_2+2 x_3=0 \\ 2 x_1-5 x_2+3 x_3=0 \\ 3 x_1-8 x_2+\lambda x_3=0\end{array}\right.\) ,问 \(\lambda\) 取何值时方程组有非零解}
		{因为系数矩阵 \[\mathbf{A}=\left(\begin{array}{ccc}1 & -3 & 2 \\ 2 & -5 & 3 \\ 3 & -8 & \lambda\end{array}\right) \rightarrow\left(\begin{array}{ccc}1 & -3 & 2 \\ 0 & 1 & -1 \\ 0 & 1 & \lambda-6\end{array}\right) \rightarrow\left(\begin{array}{ccc}1 & 0 & -1 \\ 0 & 1 & -1 \\ 0 & 0 & \lambda-5\end{array}\right),\] 所以当 \(\lambda=5\) 时, 方程组有非零解.}
	\end{question}
	
	
	\begin{question}
		{计算题}
		{线性方程组}
		{中等}
		{12.当 \(\lambda\) 取何值时,线性方程组 \(\left\{\begin{array}{r}x_1+2 x_2+x_3=0 \\ 2 x_1+3 x_2+x_3=0 \\ 3 x_1+x_2+\lambda x_3=0\end{array}\right.\) 有非零解(\qquad). }
		{因为系数矩阵 \(\overline{\mathbf{A}}=\left(\begin{array}{ccc}1 & 2 & 1 \\ 2 & 3 & 1 \\ 3 & 1 & \lambda\end{array}\right) \rightarrow\left(\begin{array}{ccc}1 & 2 & 1 \\ 0 & -1 & -1 \\ 0 & -5 & \lambda-3\end{array}\right) \rightarrow\left(\begin{array}{ccc}1 & 0 & -1 \\ 0 & 1 & 1 \\ 0 & 0 & \lambda+2\end{array}\right)\) 所以当 \(\lambda=-2\) 时,线性方程组有非零解.}
	\end{question}
	
	
		\begin{question}
		{选择题}
		{向量}
		{容易}
		{1.已知向量 \(\alpha=(3,5,7,9), \beta=(-1,5,2,8)\) ,如果 \(\alpha-\xi=\beta\) , \(\xi=()\). 
			A. \((4,0,5,1)\)
			B. \((4,5,0,-1)\)
			C. \((0,-4,-5,9)\)
			D. \((4,0,5,-9)\)}
		{因 \(\alpha-\xi=\beta\), 所以 \(\xi=\alpha-\beta=(3,5,7,9)-(-1,5,2,8)\) \(=(4,0,5,1)\). }
	\end{question}
	
	
	\begin{question}
		{选择题}
		{向量}
		{容易}
		{2. 已知向量 \(\boldsymbol{\alpha}_1=(1,2,3), \boldsymbol{\alpha}_2=(3,2,1), \boldsymbol{\alpha}_3=(-2,0,2), \boldsymbol{\alpha}_4=(1,2,4)\),则 \(5\boldsymbol{\alpha}_1+2\boldsymbol{\alpha}_2-\boldsymbol{\alpha}_3-\boldsymbol{\alpha}_4=()\). 
			A. \((12,12,11)\)
			B. \((11,11,12)\)
			C. \((12,11,12)\)
			D. \((11,12,12)\)}
		{\(5\boldsymbol{\alpha}_1+2\boldsymbol{\alpha}_2-\boldsymbol{\alpha}_3-\boldsymbol{\alpha}_4=5(1,2,3)+2(3,2,1)-(-2,0,2)-(1,2,4)=(12,12,11)\)}
	\end{question}
	
	\begin{question}
		{选择题}
		{向量}
		{中等}
		{3. 设 \(\boldsymbol{\alpha}=(2,1,1,2), \boldsymbol{\beta}=(-1,2,3,-2)\),则 \(3\boldsymbol{\alpha}+4\boldsymbol{\beta}=()\). 
			A. \((2,11,15,-2)\)
			B. \((-2,11,15,10)\)
			C. \((2,15,11,10)\)
			D. \((2,11,-15,10)\)}
		{\(3\boldsymbol{\alpha}+4\boldsymbol{\beta}=3(2,1,1,2)+4(-1,2,3,-2)=(2,11,15,-2)\)}
	\end{question}
	
	\begin{question}
		{选择题}
		{向量}
		{容易}
		{4. 已知向量 \(\boldsymbol{\alpha}_1=(1,2,3), \boldsymbol{\alpha}_2=(3,2,1), \boldsymbol{\alpha}_3=(-2,0,2), \boldsymbol{\alpha}_4=(1,2,4)\),则 \(\boldsymbol{\alpha}_1+\boldsymbol{\alpha}_2-\boldsymbol{\alpha}_3-\boldsymbol{\alpha}_4=()\). 
			A. \((5,2,-2)\)
			B. \((11,11,12)\)
			C. \((12,11,12)\)
			D. \((11,12,12)\)}
		{\(\boldsymbol{\alpha}_1+\boldsymbol{\alpha}_2-\boldsymbol{\alpha}_3-\boldsymbol{\alpha}_4=(1,2,3)+(3,2,1)-(-2,0,2)-(1,2,4)=(5,2,-2)\)}
	\end{question}
	
	\begin{question}
		{选择题}
		{向量}
		{容易}
		{5. 已知向量 \(\boldsymbol{\alpha}=(3,5,7,4), \boldsymbol{\beta}=(-1,5,2,0)\),如果 \(\boldsymbol{\alpha}+\boldsymbol{\xi}=\boldsymbol{\beta}\),则 \(\xi=()\). 
			A. \((-4,0,-5,-4)\)
			B. \((-4,-5,0,-9)\)
			C. \((0,-4,-5,-9)\)
			D. \((-4,0,-5,9)\)}
		{因 \(\boldsymbol{\alpha}+\boldsymbol{\xi}=\boldsymbol{\beta}\),所以 \(\boldsymbol{\xi}=\boldsymbol{\beta}-\boldsymbol{\alpha}=(-1,5,2,0)-(3,5,7,4)=(-4,0,-5,-4)\)}
	\end{question}
	
	\begin{question}
		{选择题}
		{向量}
		{中等}
		{6. 设 \(\boldsymbol{\alpha}=(2,1,-2), \boldsymbol{\beta}=(-4,2,3), \boldsymbol{\gamma}=(-8,8,5)\),若存在数 \(k\) 使得 \(2\boldsymbol{\alpha}+k\boldsymbol{\beta}=\gamma\),则 \(k=()\). 
			A. 3
			B. 2
			C. -1
			D. 0}
		{将 \(\boldsymbol{\alpha}, \boldsymbol{\beta}, \gamma\) 代入 \(2\boldsymbol{\alpha}+k\boldsymbol{\beta}=\gamma\),得 \(2(2,1,-2)+k(-4,2,3)=(-8,8,5)\),由此得 \(4-4k=-8,2+2k=8,-4+3k=5\),它有唯一解 \(k=3\)}
	\end{question}
	
	\begin{question}
		{选择题}
		{向量}
		{中等}
		{7. 设向量 \(\boldsymbol{\alpha}=(-1,0,1,-2), \boldsymbol{\beta}=(2,0,-2,4)\),若存在向量 \(\boldsymbol{\gamma}\) 使得 \(3\beta+\gamma=4\alpha\),则 \(\gamma=()\). 
			A. \((-10,0,10,-20)\)
			B. \((10,0,-10,20)\)
			C. \((-10,0,-10,20)\)
			D. \((10,0,-10,-20)\)}
		{由于 \(3\boldsymbol{\beta}+\boldsymbol{\gamma}=4\boldsymbol{\alpha}\),则 \(\boldsymbol{\gamma}=4\boldsymbol{\alpha}-3\boldsymbol{\beta}=4(-1,0,1,-2)-3(2,0,-2,4)=(-10,0,10,-20)\)}
	\end{question}
	
	\begin{question}
		{选择题}
		{向量}
		{困难}
		{8. 设 \(3\boldsymbol{\alpha}+4\boldsymbol{\beta}=(2,1,1,2), 2\boldsymbol{\alpha}+3\boldsymbol{\beta}=(-1,2,3,1)\),则 \(\boldsymbol{\alpha}, \boldsymbol{\beta}\) 分别为(). 
			A. \((10,-5,-9,2),(-7,4,7,-1)\)
			B. \((10,5,-9,2),(7,-4,7,-1)\)
			C. \((-10,5,9,-2),(-7,4,7,1)\)
			D. \((10,-5,9,-2),(7,-4,7,1)\)}
		{设 \(\boldsymbol{\alpha}=(a_1,a_2,a_3,a_4), \boldsymbol{\beta}=(b_1,b_2,b_3,b_4)\),由方程组解得 \(a_1=10,a_2=-5,a_3=-9,a_4=2,b_1=-7,b_2=4,b_3=7,b_4=-1\),即 \(\boldsymbol{\alpha}=(10,-5,-9,2), \boldsymbol{\beta}=(-7,4,7,-1)\)}
	\end{question}
	
	\begin{question}
		{选择题}
		{向量}
		{中等}
		{9. 将向量 \(\boldsymbol{\alpha}=(-2,3,2)\) 表成向量 \(\varepsilon_1=(-1,0,0), \varepsilon_2=(0,2,0), \varepsilon_3=(0,0,3)\) 的线性组合,则 \(\boldsymbol{\alpha}=()\). 
			A. \(2\varepsilon_1+\frac{3}{2}\varepsilon_2+\frac{2}{3}\varepsilon_3\)
			B. \(-2\varepsilon_1+\frac{3}{2}\varepsilon_2-\frac{2}{3}\varepsilon_3\)
			C. \(2\varepsilon_1-\frac{3}{2}\varepsilon_2+\frac{2}{3}\varepsilon_3\)
			D. \(2\varepsilon_2-\frac{2}{3}\varepsilon_3\)}
		{设 \(\boldsymbol{\alpha}=k_1\varepsilon_1+k_2\varepsilon_2+k_3\varepsilon_3\),比较各分量得 \(k_1=2,k_2=\frac{3}{2},k_3=\frac{2}{3}\),所以 \(\alpha=2\varepsilon_1+\frac{3}{2}\varepsilon_2+\frac{2}{3}\varepsilon_3\)}
	\end{question}
	
	\begin{question}
		{选择题}
		{向量}
		{中等}
		{10. 将向量 \(\boldsymbol{\beta}=(-5,3,-1)\) 表成向量组 \(\boldsymbol{\alpha}_1=(1,0,1), \boldsymbol{\alpha}_2=(0,1,0), \boldsymbol{\alpha}_3=(0,0,1)\) 的线性组合为(). 
			A. \(-5\boldsymbol{\alpha}_1+3\boldsymbol{\alpha}_2+4\boldsymbol{\alpha}_3\)
			B. \(5\boldsymbol{\alpha}_1-3\boldsymbol{\alpha}_2+4\boldsymbol{\alpha}_3\)
			C. \(-5\boldsymbol{\alpha}_1-3\boldsymbol{\alpha}_2-4\boldsymbol{\alpha}_3\)
			D. \(5\boldsymbol{\alpha}_1+3\boldsymbol{\alpha}_2-4\boldsymbol{\alpha}_3\)}
		{设 \(\boldsymbol{\beta}=k_1\boldsymbol{\alpha}_1+k_2\boldsymbol{\alpha}_2+k_3\boldsymbol{\alpha}_3\),比较各分量得 \(k_1=-5,k_2=3,k_3=4\),所以 \(\beta=-5\alpha_1+3\alpha_2+4\alpha_3\)}
	\end{question}
	
	\begin{question}
		{选择题}
		{向量}
		{困难}
		{11. 设向量组 \(\boldsymbol{\alpha}_1, \boldsymbol{\alpha}_2, \boldsymbol{\alpha}_3\) 线性无关,则下面的向量组中,线性无关的是(). 
			A. \(\boldsymbol{\alpha}_1+2\boldsymbol{\alpha}_2, 3\boldsymbol{\alpha}_2+3\boldsymbol{\alpha}_3, \boldsymbol{\alpha}_1+2\boldsymbol{\alpha}_3\)
			B. \(\boldsymbol{\alpha}_1-\boldsymbol{\alpha}_2, \boldsymbol{\alpha}_2-\boldsymbol{\alpha}_3, \boldsymbol{\alpha}_3-\boldsymbol{\alpha}_1\)
			C. \(\boldsymbol{\alpha}_1+\boldsymbol{\alpha}_2, \boldsymbol{\alpha}_2-\boldsymbol{\alpha}_3, \boldsymbol{\alpha}_1+\boldsymbol{\alpha}_3\)
			D. \(\boldsymbol{\alpha}_1+\boldsymbol{\alpha}_2, \boldsymbol{\alpha}_2+\boldsymbol{\alpha}_3, \boldsymbol{\alpha}_3-\boldsymbol{\alpha}_1\)}
		{对于选项A,行列式\(\left|\begin{array}{lll}1 & 0 & 1 \\ 2 & 3 & 0 \\ 0 & 3 & 2\end{array}\right|=12 \neq 0\),故向量组线性无关;其他选项行列式为0,线性相关}
	\end{question}
	
	\begin{question}
		{选择题}
		{向量}
		{困难}
		{12. 设向量组 \(\boldsymbol{\alpha}_1, \boldsymbol{\alpha}_2, \boldsymbol{\alpha}_3\) 线性无关,则下面的向量组中,线性无关的是(). 
			A. \(\boldsymbol{\alpha}_1+\boldsymbol{\alpha}_2+\boldsymbol{\alpha}_3, 2\boldsymbol{\alpha}_1-3\boldsymbol{\alpha}_2+10\boldsymbol{\alpha}_3, 3\boldsymbol{\alpha}_1+5\boldsymbol{\alpha}_2-5\boldsymbol{\alpha}_3\)
			B. \(\boldsymbol{\alpha}_1+\boldsymbol{\alpha}_3, \boldsymbol{\alpha}_1+\boldsymbol{\alpha}_2, \boldsymbol{\alpha}_3-\boldsymbol{\alpha}_2\)
			C. \(2\boldsymbol{\alpha}_1+\boldsymbol{\alpha}_3, 2\boldsymbol{\alpha}_1+\boldsymbol{\alpha}_2, \boldsymbol{\alpha}_3-\boldsymbol{\alpha}_2\)
			D. \(\boldsymbol{\alpha}_1-\boldsymbol{\alpha}_2, \boldsymbol{\alpha}_2-\boldsymbol{\alpha}_3, \boldsymbol{\alpha}_3-\boldsymbol{\alpha}_1\)}
		{对于选项A,行列式\(\left|\begin{array}{ccc}1 & 2 & 3 \\ 1 & -3 & 5 \\ 1 & 10 & -5\end{array}\right| \neq 0\),故向量组线性无关;其他选项行列式为0,线性相关}
	\end{question}
	
	\begin{question}
		{选择题}
		{向量}
		{中等}
		{13. 下面的向量组中线性相关的是(). 
			A. \(\boldsymbol{\alpha}_1=(1,-1,1), \boldsymbol{\alpha}_2=(1,2,3), \boldsymbol{\alpha}_3=(3,3,7)\)
			B. \(\boldsymbol{\beta}_1=(1,2,3), \boldsymbol{\beta}_2=(2,3,1), \boldsymbol{\beta}_3=(3,1,2)\)
			C. \(\varepsilon_1=(1,2,0), \varepsilon_2=(0,3,3), \varepsilon_3=(1,0,2)\)
			D. \(\boldsymbol{\alpha}=(1,1,1), \boldsymbol{\beta}=(2,-3,10), \gamma=(3,5,-5)\)}
		{对于选项A,行列式\(\left|\begin{array}{ccc}1 & 1 & 3 \\ -1 & 2 & 3 \\ 1 & 3 & 7\end{array}\right|=0\),故向量组线性相关;其他选项行列式不为0,线性无关}
	\end{question}
	
	\begin{question}
		{选择题}
		{向量}
		{中等}
		{14. 下面的向量组中线性无关的是(). 
			A. \(\boldsymbol{\alpha}=(1,2,0), \boldsymbol{\beta}=(0,2,3), \gamma=(1,0,3)\)
			B. \(\boldsymbol{\alpha}=(1,1,0), \boldsymbol{\beta}=(0,1,1), \gamma=(1,2,1)\)
			C. \(\boldsymbol{\alpha}=(1,1,0), \boldsymbol{\beta}=(0,1,1), \gamma=(-1,0,1)\)
			D. \(\boldsymbol{\alpha}=(1,1,1), \boldsymbol{\beta}=(2,-3,22), \gamma=(3,5,-5)\)}
		{对于选项A,行列式\(\left|\begin{array}{lll}1 & 0 & 1 \\ 2 & 2 & 0 \\ 0 & 3 & 3\end{array}\right|=12 \neq 0\),故向量组线性无关;其他选项行列式为0,线性相关}
	\end{question}
	
	\begin{question}
		{选择题}
		{向量}
		{中等}
		{15. 把向量 \(\boldsymbol{\beta}=(3,2,1)\) 表成向量组 \(\boldsymbol{\alpha}_1=(1,1,1), \boldsymbol{\alpha}_2=(-1,1,0), \boldsymbol{\alpha}_3=(0,-1,0)\) 的线性组合,则有 \(\boldsymbol{\beta}=()\). 
			A. \(\alpha_1-2\alpha_2-3\alpha_3\)
			B. \(\boldsymbol{\alpha}_1+2\boldsymbol{\alpha}_2+3\boldsymbol{\alpha}_3\)
			C. \(\mathbf{a}_1-2\boldsymbol{\alpha}_2+3\boldsymbol{\alpha}\)
			D. \(\mathbf{a}_1+2\boldsymbol{\alpha}_2-3\boldsymbol{\alpha}_3\)}
		{设 \(\boldsymbol{\beta}=k_1\boldsymbol{\alpha}_1+k_2\boldsymbol{\alpha}_2+k_3\boldsymbol{\alpha}_3\),解得 \(k_1=1,k_2=-2,k_3=-3\),所以 \(\boldsymbol{\beta}=\boldsymbol{\alpha}_1-2\boldsymbol{\alpha}_2-3\boldsymbol{\alpha}_3\)}
	\end{question}
	
	\begin{question}
		{选择题}
		{向量}
		{中等}
		{16. 将向量 \(\boldsymbol{\beta}=(8,3,-2)\) 表成向量 \(\boldsymbol{\alpha}_1=(1,2,0), \boldsymbol{\alpha}_2=(2,0,-1), \boldsymbol{\alpha}_3=(0,1,-1)\) 的线性组合,则 \(\boldsymbol{\beta}=()\). 
			A. \(2\boldsymbol{\alpha}_1+3\boldsymbol{\alpha}_2-\boldsymbol{\alpha}_3\)
			B. \(\boldsymbol{\alpha}_1-3\boldsymbol{\alpha}_2+\boldsymbol{\alpha}_3\)
			C. \(-2\boldsymbol{\alpha}_1+3\boldsymbol{a}_2-\boldsymbol{\alpha}_3\)
			D. \(\boldsymbol{\alpha}_1+2\boldsymbol{\alpha}_2+3\boldsymbol{\alpha}_3\)}
		{设 \(\boldsymbol{\beta}=k_1\boldsymbol{\alpha}_1+k_2\boldsymbol{\alpha}_2+k_3\boldsymbol{\alpha}_3\),解得 \(k_1=2,k_2=3,k_3=-1\),所以 \(\boldsymbol{\beta}=2\boldsymbol{\alpha}_1+3\boldsymbol{\alpha}_2-\boldsymbol{\alpha}_3\)}
	\end{question}
	
	\begin{question}
		{选择题}
		{向量}
		{中等}
		{17. 将向量 \(\boldsymbol{\alpha}=(4,3,5)\) 表成向量组 \(\varepsilon_1=(1,1,0), \boldsymbol{\varepsilon}_2=(0,1,1), \boldsymbol{\varepsilon}_3=(1,0,1)\) 的线性组合,则有 \(\boldsymbol{\alpha}=()\). 
			A. \(\varepsilon_1+2\varepsilon_2+3\varepsilon_3\)
			B. \(-\varepsilon_1+2\varepsilon_2+3\varepsilon_3\)
			C. \(\varepsilon_1-2\varepsilon_2-3\varepsilon_3\)
			D. \(-\varepsilon_1+2\varepsilon_2-3\varepsilon_3\)}
		{设 \(\boldsymbol{\alpha}=k_1\varepsilon_1+k_2\varepsilon_2+k_3\varepsilon_3\),解得 \(k_1=1,k_2=2,k_3=3\),所以 \(\boldsymbol{\alpha}=\varepsilon_1+2\varepsilon_2+3\varepsilon_3\)}
	\end{question}
	
	\begin{question}
		{选择题}
		{向量}
		{困难}
		{18. 将向量 \(\boldsymbol{\beta}=(1,1,-1,-1)\) 表成向量组 \(\boldsymbol{\alpha}_1=(1,1,1,1), \boldsymbol{\alpha}_2=(1,-1,1,-1), \boldsymbol{\alpha}_3=(1,-1,-1,1), \boldsymbol{\alpha}_4=(1,1,3,-1)\) 的线性组合,则有 \(\boldsymbol{\beta}=()\). 
			A. \(\beta\) 不能由 \(\boldsymbol{\alpha}_1, \boldsymbol{\alpha}_2, \boldsymbol{\alpha}_3, \boldsymbol{\alpha}_4\) 线性表出
			B. \(3\boldsymbol{\alpha}_1+\boldsymbol{a}_2-5\boldsymbol{\alpha}_3-9\boldsymbol{\alpha}_4\)
			C. \(-\boldsymbol{\alpha}_1+3\boldsymbol{\alpha}_2-4\boldsymbol{\alpha}_3+9\boldsymbol{\alpha}_4\)
			D. \(\boldsymbol{\alpha}_1+2\boldsymbol{a}_2-3\boldsymbol{\alpha}_3-2\boldsymbol{\alpha}_4\)}
		{设 \(\boldsymbol{\beta}=k_1\boldsymbol{\alpha}_1+k_2\boldsymbol{\alpha}_2+k_3\boldsymbol{\alpha}_3+k_4\boldsymbol{\alpha}_4\),该线性方程组无解,所以 \(\beta\) 不能由 \(\boldsymbol{\alpha}_1, \boldsymbol{\alpha}_2, \boldsymbol{\alpha}_3, \boldsymbol{\alpha}_4\) 线性表出}
	\end{question}
	
	\begin{question}
		{选择题}
		{向量}
		{中等}
		{19. 向量组 \(\boldsymbol{\alpha}_1=(1,2,0), \boldsymbol{\alpha}_2=(2,4,0), \boldsymbol{\alpha}_3=(3,6,1)\) 的秩为 (). 
			A. 2
			B. 1
			C. 3
			D. 4}
		{通过初等变换可知秩为2}
	\end{question}
	
	\begin{question}
		{选择题}
		{向量}
		{中等}
		{20. 向量组 \(\boldsymbol{\alpha}_1=(1,1,0), \boldsymbol{\alpha}_2=(2,3,1), \boldsymbol{\alpha}_3=(3,5,2)\) 的秩为(). 
			A. 2
			B. 1
			C. 3
			D. 4}
		{通过初等变换可知秩为2}
	\end{question}
	
	\begin{question}
		{选择题}
		{向量}
		{中等}
		{21. 向量组 \(\boldsymbol{\alpha}_1=(1,1,1,1), \boldsymbol{\alpha}_2=(1,1,-1,-1), \boldsymbol{\alpha}_3=(1,-1,-1,1), \boldsymbol{\alpha}_4=(-1,-1,-1,1)\) 的秩为(). 
			A. 4
			B. 1
			C. 3
			D. 2}
		{通过初等变换可知秩为4}
	\end{question}
	
	\begin{question}
		{选择题}
		{向量}
		{中等}
		{22. 向量组 \(\boldsymbol{\alpha}_1=(1,1,-1,-1), \boldsymbol{\alpha}_2=(1,2,0,3), \boldsymbol{\alpha}_3=(1,0,0,0), \boldsymbol{\alpha}_4=(0,1,0,0)\) 的秩为(). 
			A. 4
			B. 1
			C. 2
			D. 3}
		{通过初等变换可知秩为4}
	\end{question}
	
	\begin{question}
		{选择题}
		{向量}
		{中等}
		{23. 向量组 \(\boldsymbol{\beta}_1=(1,-2,3,-1), \boldsymbol{\beta}_2=(2,-1,1,0), \boldsymbol{\beta}_3=(1,-5,8,-3), \boldsymbol{\beta}_4=(3,-6,9,-3)\) 的秩为(). 
			A. 2
			B. 1
			C. 3
			D. 4}
		{通过初等变换可知秩为2}
	\end{question}
	
	\begin{question}
		{选择题}
		{向量}
		{中等}
		{24. 向量组 \(\boldsymbol{\beta}_1=(1,-1,1,-1), \boldsymbol{\beta}_2=(1,2,3,1), \boldsymbol{\beta}_3=(3,3,7,1), \boldsymbol{\beta}_4=(4,5,10,2)\) 的秩为(). 
			A. 2
			B. 1
			C. 3
			D. 7}
		{通过初等变换可知秩为2}
	\end{question}
	
	\begin{question}
		{选择题}
		{向量}
		{中等}
		{25. 若 \(\boldsymbol{\alpha}_1、\boldsymbol{\alpha}_2\) 是齐次线性方程组 \(\mathbf{A}\mathbf{x}=\mathbf{0}\) 的解,则下列不一定是 \(\mathbf{A}\mathbf{x}=\mathbf{0}\) 的解的是(). 
			A. \(\boldsymbol{\alpha}_1\boldsymbol{\alpha}_2^T\)
			B. \(\boldsymbol{\alpha}_1+\boldsymbol{\alpha}_2\)
			C. \(\boldsymbol{\alpha}_1-\boldsymbol{\alpha}_2\)
			D. \(k_1\boldsymbol{\alpha}_1+k_2\boldsymbol{\alpha}_2\)}
		{\(\boldsymbol{\alpha}_1\boldsymbol{\alpha}_2^T\) 不一定是 \(\mathbf{A}\mathbf{x}=\mathbf{0}\) 的解}
	\end{question}
	
	\begin{question}
		{选择题}
		{向量}
		{中等}
		{26. 设 \(\boldsymbol{\beta}_1, \boldsymbol{\beta}_2\) 是非齐次线性方程组 \(\mathbf{A}\mathbf{X}=\mathbf{b}\) 的两个解向量,则下列向量中仍为该方程组解的是(). 
			A. \(\frac{1}{5}(3\boldsymbol{\beta}_1+2\boldsymbol{\beta}_2)\)
			B. \(\boldsymbol{\beta}_1+\boldsymbol{\beta}_2\)
			C. \(\frac{1}{2}\boldsymbol{\beta}_1+\boldsymbol{\beta}_2\)
			D. \(\boldsymbol{\beta}_1-\boldsymbol{\beta}_2\)}
		{\(\mathbf{A}(\frac{1}{5}(3\boldsymbol{\beta}_1+2\boldsymbol{\beta}_2))=\mathbf{b}\),其他选项不满足}
	\end{question}
	
	\begin{question}
		{选择题}
		{向量}
		{中等}
		{27. 已知 \(\boldsymbol{\alpha}_1=(1,1,-1)^T, \boldsymbol{\alpha}_2=(1,2,0)^T\) 是齐次方程组 \(\mathbf{A}\mathbf{x}=\mathbf{0}\) 的基础解系,那么下列向量中 \(\mathbf{A}\mathbf{x}=\mathbf{0}\) 的解向量是(). 
			A. \((2,1,-3)^T\)
			B. \((1,-1,3)^T\)
			C. \((2,2,-5)^T\)
			D. \((2,-2,6)^T\)}
		{只有 \((2,1,-3)\) 可以表示为 \(\boldsymbol{\alpha}_1, \boldsymbol{\alpha}_2\) 的线性组合,是 \(\mathbf{A}\mathbf{x}=\mathbf{0}\) 的解}
	\end{question}
	
	\begin{question}
		{选择题}
		{向量}
		{困难}
		{28. 设 \(\eta_1, \eta_2, \eta_3, \eta_4\) 是齐次线性方程组 \(\mathbf{A}\mathbf{x}=\mathbf{0}\) 的基础解系,则 \(\mathbf{A}\mathbf{x}=\mathbf{0}\) 的基础解系还可以是(). 
			A. \(\eta_1+\eta_2, \eta_2-\eta_3, \eta_3+\eta_4, \eta_4+\eta_1\)
			B. \(\eta_1+\eta_2, \eta_2+\eta_3+\eta_4, \eta_1-\eta_2+\eta_3\)
			C. \(\eta_1+\eta_2, \eta_2+\eta_3, \eta_3+\eta_4, \eta_4+\eta_1\)
			D. \(\eta_1-\eta_2, \eta_2+\eta_3, \eta_3-\eta_4, \eta_4+\eta_1\)}
		{选项A的行列式\(\left|\begin{array}{cccc}1 & 1 & 0 & 0 \\ 0 & 1 & -1 & 0 \\ 0 & 0 & 1 & 1 \\ 1 & 0 & 0 & 1\end{array}\right|=2 \neq 0\),线性无关;其他选项不满足条件}
	\end{question}
	
	
	
	
	\begin{question}
		{判断题}
		{向量}
		{容易}
		{1.设 \(\boldsymbol{\alpha}, \boldsymbol{\beta}\) 分别为两个 \(n\) 维向量,若 \(\boldsymbol{\alpha}\) 和 \(\boldsymbol{\beta}\) 的各分量对应相等,则称 \(\boldsymbol{\alpha}\) 与 \(\boldsymbol{\beta}\) 相等(\qquad). }
		{由向量相等概念可知该结论正确. }
	\end{question}
	
	
	\begin{question}
		{判断题}
		{向量}
		{容易}
		{2. 5 维向量中只有一个分量为 0 的向量称为零向量(). }
		{所有分量都为 0 的向量才称为零向量,所以结论错误.}
	\end{question}
	
	\begin{question}
		{判断题}
		{向量}
		{容易}
		{3. 两个向量一定可以做加法运算(). }
		{必须两个维数相等的向量才能做加法运算,所以结论错误.}
	\end{question}
	
	\begin{question}
		{判断题}
		{向量}
		{中等}
		{4. 若向量 \(\boldsymbol{\alpha}=(3,2,1), \boldsymbol{\beta}=(-3,-2,0)\) ,则 \(2 \boldsymbol{\alpha}+\boldsymbol{\beta}=(3,2,1)\)(\qquad). }
		{由向量的运算法则可知 \(2\boldsymbol{\alpha}+\boldsymbol{\beta}=2(3,2,1)+(-3,-2,0)=(3,2,2)\),所以结论错误.}
	\end{question}
	
	\begin{question}
		{判断题}
		{向量}
		{容易}
		{5. 所有分量为 0 的向量称为零向量(). }
		{所有分量都为 0 的向量称为零向量,结论正确.}
	\end{question}
	
	\begin{question}
		{判断题}
		{向量}
		{中等}
		{6. 设向量 \(\boldsymbol{\alpha}=(2,-2,1,0), \boldsymbol{\beta}=(1,-1,-2,3)\) ,则 \(\boldsymbol{\alpha}-2 \boldsymbol{\beta}=(0,0,5,-6)\)(). }
		{根据向量的运算法则可知该结论正确.}
	\end{question}
	
	\begin{question}
		{判断题}
		{向量}
		{中等}
		{7. 已知向量 \(\varepsilon_1=(1,0,0,0), \varepsilon_2=(0,1,0,0), \varepsilon_3=(0,0,1,0), \varepsilon_4=(0,0,0,1)\) ,则 \(2\varepsilon_1-\varepsilon_2+5\varepsilon_3+\varepsilon_4=(2,0,5,1)\)(). }
		{线性运算得 \(=2(1,0,0,0)-(0,1,0,0)+5(0,0,1,0)+(0,0,0,1)=(2,-1,5,1)\),结论错误.}
	\end{question}
	
	\begin{question}
		{判断题}
		{向量}
		{困难}
		{8. 已知 \(\boldsymbol{\alpha}_1=(2,5,1,3), \boldsymbol{\alpha}_2=(10,1,5,10), \boldsymbol{\alpha}_3=(4,1,-1,1)\) ,若 \(3(\boldsymbol{\alpha}_1-\xi)+2(\boldsymbol{\alpha}_2+\xi)=5(\boldsymbol{\alpha}_3+\xi)\),则 \(\xi=(1,2,3,4)\)(). }
		{根据向量运算和方程变换得出 \(\xi=(1,2,3,4)\),结论正确.}
	\end{question}
	
	\begin{question}
		{判断题}
		{向量}
		{中等}
		{9. 设 \(\boldsymbol{\alpha}=(1,2,3,4), \boldsymbol{\beta}=(2,-1,5,1)\) ,若 \(\gamma-2\boldsymbol{\alpha}=\boldsymbol{\beta}\),则 \(\gamma=(1,5,3,1)\)(). }
		{解得 \(\gamma=2\boldsymbol{\alpha}+\boldsymbol{\beta}=2(1,2,3,4)+(2,-1,5,1)=(4,3,11,9)\),结论错误.}
	\end{question}
	
	\begin{question}
		{判断题}
		{向量}
		{容易}
		{10. 只含有零向量的向量组线性无关(). }
		{含有零向量的向量组必线性相关,结论错误.}
	\end{question}
	
	\begin{question}
		{判断题}
		{向量}
		{容易}
		{11. 向量组只含一个向量,则该组线性相关当且仅当该向量为零向量(). }
		{结论正确,由线性相关定义得知.}
	\end{question}
	
	\begin{question}
		{判断题}
		{向量}
		{容易}
		{12. 向量组只含一个向量,则该组线性无关当且仅当该向量不为零(). }
		{结论正确,由线性相关定义得知.}
	\end{question}
	
	\begin{question}
		{判断题}
		{向量}
		{中等}
		{13. 若 \(\boldsymbol{\alpha}_1,\dots,\boldsymbol{\alpha}_m\) 线性无关,\(\boldsymbol{\beta}, \boldsymbol{\alpha}_1,\dots,\boldsymbol{\alpha}_m\) 线性相关,则 \(\boldsymbol{\beta}\) 可由 \(\boldsymbol{\alpha}_1,\dots,\boldsymbol{\alpha}_m\) 唯一线性表示(). }
		{结论错误,线性相关意味着可表示,但表示法不唯一.}
	\end{question}
	
	\begin{question}
		{判断题}
		{向量}
		{容易}
		{14. 含有零向量的向量组 \(0, \alpha_1, \alpha_2, \cdots, \alpha_m\) 一定线性无关(). }
		{含零向量的向量组必线性相关,结论错误.}
	\end{question}
	
	\begin{question}
		{判断题}
		{向量}
		{容易}
		{15. 含有零向量的向量组 \(0, \alpha_1, \alpha_2, \cdots, \alpha_m\) 一定线性相关(). }
		{含零向量的向量组必线性相关,结论正确.}
	\end{question}
	
	\begin{question}
		{判断题}
		{向量}
		{容易}
		{16. 设 \(\boldsymbol{\beta}=(2,3), \boldsymbol{\alpha}_1=(1,0), \boldsymbol{\alpha}_2=(0,1)\) ,则 \(\boldsymbol{\beta}\) 可由 \(\boldsymbol{\alpha}_1, \boldsymbol{\alpha}_2\) 线性表示(). }
		{由计算 \(\boldsymbol{\beta}=2\boldsymbol{\alpha}_1+3\boldsymbol{\alpha}_2\),结论正确.}
	\end{question}
	
	\begin{question}
		{判断题}
		{向量}
		{中等}
		{17. 设 \(\boldsymbol{\beta}=(2,-3), \boldsymbol{\alpha}_1=(1,0), \boldsymbol{\alpha}_2=(0,1)\) ,则 \(\boldsymbol{\alpha}_1, \boldsymbol{\alpha}_2, \boldsymbol{\beta}\) 线性无关(). }
		{结论错误,\(\boldsymbol{\beta}\) 可由其余两个向量线性表示,故三者线性相关.}
	\end{question}
	
	\begin{question}
		{判断题}
		{向量}
		{中等}
		{18. 向量组中只要有非零向量,就一定存在极大线性无关组(). }
		{结论正确,非零向量可构成线性无关组.}
	\end{question}
	
	\begin{question}
		{判断题}
		{向量}
		{中等}
		{19. 一个向量组的极大线性无关组是唯一的(). }
		{结论错误,极大线性无关组不一定唯一.}
	\end{question}
	
	\begin{question}
		{判断题}
		{向量}
		{中等}
		{20. 一个向量组的极大线性无关组不一定是唯一的(). }
		{结论正确,极大线性无关组的个数可能不唯一.}
	\end{question}
	
	\begin{question}
		{判断题}
		{向量}
		{中等}
		{21. 向量组的任意一个极大线性无关组都与向量组本身等价(). }
		{结论正确,由等价定义知任一极大线性无关组等价于原组.}
	\end{question}
	
	\begin{question}
		{判断题}
		{向量}
		{中等}
		{22. 向量组的任意一个极大线性无关组不一定与向量组本身等价(). }
		{结论错误,极大线性无关组都与原组等价.}
	\end{question}
	
	\begin{question}
		{判断题}
		{向量}
		{中等}
		{23. 向量组的任意两个极大线性无关组不一定等价(). }
		{结论错误,它们都等价于原组,故互相等价.}
	\end{question}
	
	\begin{question}
		{判断题}
		{向量}
		{中等}
		{24. 向量组的任意两个极大线性无关组等价(). }
		{结论正确,它们都与原组等价,故彼此等价.}
	\end{question}
	
	\begin{question}
		{判断题}
		{向量}
		{中等}
		{25. 向量组的任意两个极大线性无关组都包含相同个数的向量(). }
		{结论正确,等价向量组包含的向量个数相同.}
	\end{question}
	
	\begin{question}
		{判断题}
		{向量}
		{中等}
		{26. 向量组的任意两个极大线性无关组包含的向量个数不一定相同(). }
		{结论错误,它们包含相同个数的向量.}
	\end{question}
	
	
	\begin{question}
		{判断题}
		{向量}
		{容易}
		{27.等价的向量组可以有不同的秩(\qquad).}
		{等价的向量组有相同的秩,因此该说法错误.}
	\end{question}
	
	\begin{question}
		{判断题}
		{向量}
		{中等}
		{28.设 \(\mathbf{A}\) 是 \(n\) 维向量组,若 \(\boldsymbol{\alpha}_1, \boldsymbol{\alpha}_2, \cdots, \boldsymbol{\alpha}_r\) 与 \(\boldsymbol{\beta}_1, \boldsymbol{\beta}_2, \cdots, \boldsymbol{\beta}_s\) 都是 \(\mathbf{A}\) 的极大线性无关组,则 \(r = s\)(\qquad).}
		{向量组的任意两个极大线性无关组都包含相同个数的向量,因此该说法正确.}
	\end{question}
	
	\begin{question}
		{判断题}
		{向量}
		{中等}
		{29.若 \(\alpha_1, \alpha_2, \cdots, \alpha_r\) 为 \(n\) 维向量组 \(\mathbf{A}\) 的一个极大线性无关组,则 \(\mathbf{A}\) 中任意向量 \(\alpha\) 都可由 \(\alpha_1, \alpha_2, \cdots, \alpha_r\) 线性表示,但表示法不唯一(\qquad).}
		{由极大线性无关组定义知,该表示法是唯一的,因此该说法错误.}
	\end{question}
	
	\begin{question}
		{判断题}
		{向量}
		{困难}
		{30.设齐次线性方程组 \(\mathbf{Ax} = \mathbf{0}\) 是非齐次线性方程组 \(\mathbf{Ax} = \mathbf{b}\) 的导出组,\(\eta_1, \eta_2\) 是 \(\mathbf{Ax} = \mathbf{b}\) 的解,则 \(\eta_1 + \eta_2\) 是 \(\mathbf{Ax} = \mathbf{0}\) 的解(\qquad).}
		{由线性方程组解的结构可知,该说法错误.}
	\end{question}
	
	\begin{question}
		{判断题}
		{向量}
		{容易}
		{31.若 \(\eta_1, \eta_2\) 是 \(\mathbf{Ax} = \mathbf{0}\) 的解,则 \(10\eta_1 + 100\eta_2\) 也是它的解(\qquad).}
		{由于齐次线性方程组的解构成向量空间,任意线性组合仍为解,因此该说法正确.}
	\end{question}
	
	\begin{question}
		{判断题}
		{向量}
		{中等}
		{32.若 \(\mathbf{x}_1\) 是 \(\mathbf{Ax} = \mathbf{b}\) 的解,\(\mathbf{x}_2\) 是 \(\mathbf{Ax} = \mathbf{0}\) 的解,则 \(k\mathbf{x}_1 + \mathbf{x}_2\)(\(k\) 为任意常数)是 \(\mathbf{Ax} = \mathbf{b}\) 的解(\qquad).}
		{代入可得 \(\mathbf{A}(k\mathbf{x}_1 + \mathbf{x}_2) = k\mathbf{b} + \mathbf{0} = k\mathbf{b}\),因此只有当 \(k=1\) 时该说法才成立,所以该说法错误.}
	\end{question}
	
	\begin{question}
		{判断题}
		{向量}
		{容易}
		{33.如果 \(\gamma\) 是齐次线性方程组 \(\mathbf{Ax} = \mathbf{0}\) 的一个解,那么 \(k\gamma\)(\(k\) 为实数)也是方程组的解(\qquad).}
		{齐次线性方程组的解构成向量空间,实数倍仍为解,因此该说法正确.}
	\end{question}
	
	\begin{question}
		{判断题}
		{向量}
		{中等}
		{34.如果 \(\gamma_1, \gamma_2\) 是非齐次线性方程组 \(\mathbf{Ax} = \mathbf{b}\) 的两个解,则 \(\gamma_1 - \gamma_2\) 是其导出组 \(\mathbf{Ax} = \mathbf{0}\) 的解(\qquad).}
		{非齐次线性方程组的任意两个解之差为对应齐次方程组的解,因此该说法正确.}
	\end{question}
	
	\begin{question}
		{判断题}
		{向量}
		{容易}
		{35.如果 \(\gamma_1, \gamma_2\) 是齐次线性方程组 \(\mathbf{Ax} = \mathbf{0}\) 的两个解,那么 \(\gamma_1 + \gamma_2\) 也是方程组的解(\qquad).}
		{齐次线性方程组的解构成向量空间,和仍为解,因此该说法正确.}
	\end{question}
	
	
	
	
	\begin{question}
		{填空题}
		{向量}
		{容易}
		{1.设有两个向量 \(\boldsymbol{\alpha}=(1,0,0), \boldsymbol{\beta}=(0,1,0)\) ,则 \(2 \boldsymbol{\alpha}+\boldsymbol{\beta}=()\). }
		{由向量的运算法则可知 \(2 \boldsymbol{\alpha}+\boldsymbol{\beta}=(2,1,0)\). }
	\end{question}
	
	
	\begin{question}
		{填空题}
		{向量}
		{容易}
		{2.设有两个向量 \(\boldsymbol{\alpha}=(1,0,1), \boldsymbol{\beta}=(0,1,0)\) ,则 \(\boldsymbol{\alpha}-2 \boldsymbol{\beta}=()\) .}
		{由向量的运算法则可知 \(\boldsymbol{\alpha}-2 \boldsymbol{\beta}=(1,-2,1)\) .}
	\end{question}
	
	\begin{question}
		{填空题}
		{向量}
		{中等}
		{3.设向量 \(\boldsymbol{\beta}=(-1,4), \boldsymbol{\alpha}_1=(1,1), \boldsymbol{\alpha}_2=(0,1)\) ,则 \(\beta\) 可由 \(\boldsymbol{\alpha}_1, \boldsymbol{\alpha}_2\) 线性表示为(\qquad). }
		{设 \(\boldsymbol{\beta}=m \boldsymbol{\alpha}_{\mathbf{1}}+n \boldsymbol{\alpha}_{\mathbf{2}}\) ,则有 \(\boldsymbol{\beta}=(m, m)+(0, n)=(m, m+n)\) ,对照参数可知 \(m=-1, n=5\) ,故 \(\boldsymbol{\beta}=-\boldsymbol{\alpha}_1+5 \boldsymbol{\alpha}_2\) .}
	\end{question}
	
	\begin{question}
		{填空题}
		{向量}
		{中等}
		{4.设向量 \(\boldsymbol{\beta}=(-3,2,1), \boldsymbol{\alpha}_1=(1,0,0), \boldsymbol{\alpha}_2=(0,1,0), \boldsymbol{\alpha}_3=(0,0,1)\) ,则 \(\beta\) 可由 \(\boldsymbol{\alpha}_1, \boldsymbol{\alpha}_2, \boldsymbol{\alpha}_3\) 线性表示为().}
		{设 \(\boldsymbol{\beta}=m \boldsymbol{\alpha}_1+n \boldsymbol{\alpha}_2+l \boldsymbol{\alpha}_3\) ,则有 \(\boldsymbol{\beta}=m(1,0,0)+n(0,1,0)+l(0,0,1)=(m, n, l)\) ,对照参数可知 \(m=-3, n=2, l=1\) ,故 \(\beta=-3 \alpha_1+2 \alpha_2+\alpha_3\) .}
	\end{question}
	
	\begin{question}
		{填空题}
		{向量}
		{中等}
		{5.设向量 \(\boldsymbol{\beta}=(3,2,1), \boldsymbol{\alpha}_1=(1,0,0), \boldsymbol{\alpha}_2=(0,0,1), \boldsymbol{\alpha}_3=(0,1,0)\) ,则 \(\boldsymbol{\beta}\) 可由 \(\boldsymbol{\alpha}_1, \boldsymbol{\alpha}_2, \boldsymbol{\alpha}_3\) 线性表示为(). }
		{设 \(\boldsymbol{\beta}=m \boldsymbol{\alpha}_{1}+n \boldsymbol{\alpha}_{2}+l \boldsymbol{\alpha}_3\) ,则有 \(\boldsymbol{\beta}=m(1,0,0)+n(0,0,1)+l(0,1,0)=(m, l, n)\) ,对照参数可知 \(m=3, n=1, l=2\) ,故 \(\beta=3 \alpha_1+\alpha_2+2 \alpha_3\) .}
	\end{question}
	
	\begin{question}
		{填空题}
		{向量}
		{容易}
		{6.向量组 \(\boldsymbol{\alpha}_1=(1,0,0), \boldsymbol{\alpha}_2=(0,1,0), \boldsymbol{\alpha}_3=(0,0,1)\) 的秩为(\qquad). }
		{向量组 \(\boldsymbol{\alpha}_1=(1,0,0), \boldsymbol{\alpha}_2=(0,1,0), \boldsymbol{\alpha}_3=(0,0,1)\) 的秩为 3.}
	\end{question}
	
	\begin{question}
		{填空题}
		{向量}
		{中等}
		{7.若 \(\boldsymbol{\alpha}_1, \boldsymbol{\alpha}_2, \cdots, \boldsymbol{\alpha}_{r}\) 与 \(\boldsymbol{\beta}_1, \boldsymbol{\beta}_2, \cdots, \boldsymbol{\beta}_s\) 都是向量组 \(\mathbf{A}\) 的极大线性无关组,则 \(r\) 与 \(s\) 的关系为(\qquad). }
		{向量组 \(\mathbf{A}\) 的任意两个极大线性无关组包含的向量的个数相同.}
	\end{question}
	
	\begin{question}
		{填空题}
		{向量}
		{中等}
		{8.向量组 \(\mathbf{A}\) 的秩为 \(r\) ,则向量组中任意 \(r+1\) 个向量构成的向量组都(\qquad). }
		{向量组 \(\mathbf{A}\) 的秩为 \(r\) ,则向量组中任意 \(r+1\) 个向量构成的向量组都是线性相关的.}
	\end{question}
	
	\begin{question}
		{填空题}
		{向量}
		{中等}
		{9.若向量组 \(\mathbf{A}\) 中有 \(r\) 个向量使得 \(\mathbf{A}\) 中每个向量都可由这 \(r\) 个向量唯一线性表示,则向量组 \(\mathbf{A}\) 的秩为(). }
		{由题意知这 \(r\) 个向量是向量组 \(\mathbf{A}\) 的一个极大线性无关组,故二者的秩相同. }
	\end{question}
	
	\begin{question}
		{填空题}
		{向量}
		{困难}
		{10.向量组 \(\boldsymbol{\alpha}=(1,0,-1), \boldsymbol{\beta}=(-2,2,0), \gamma=(3,-5,2)\) 的秩为(\qquad).}
		{以 \(\boldsymbol{\alpha}, \boldsymbol{\beta}, \gamma\) 为行向量作矩阵,对该矩阵进行初等行变换,即 \(\left(\begin{array}{ccc}1 & 0 & -1 \\ -2 & 2 & 0 \\ 3 & -5 & 2\end{array}\right) \rightarrow\left(\begin{array}{ccc}1 & 0 & -1 \\ 0 & 2 & -2 \\ 0 & -5 & 5\end{array}\right) \rightarrow\left(\begin{array}{ccc}1 & 0 & -1 \\ 0 & 1 & -1 \\ 0 & 0 & 0\end{array}\right)\), 可知该矩阵的秩为 2 ,故该向量组的秩也为 2 .}
	\end{question}
	
	\begin{question}
		{填空题}
		{向量}
		{困难}
		{11.向量组 \(\boldsymbol{\alpha}=(1,1,3,1), \boldsymbol{\beta}=(3,-1,2,4), \boldsymbol{\gamma}=(2,2,7,-1)\) 的秩为(\qquad). }
		{以 \(\boldsymbol{\alpha}, \boldsymbol{\beta}, \gamma\) 为行向量作矩阵,对该矩阵进行初等行变换,即 \(\left(\begin{array}{cccc}1 & 1 & 3 & 1 \\ 3 & -1 & 2 & 4 \\ 2 & 2 & 7 & -1\end{array}\right) \rightarrow\left(\begin{array}{cccc}1 & 1 & 3 & 1 \\ 0 & -4 & -7 & 1 \\ 0 & 0 & 1 & -3\end{array}\right)\) ,可知该矩阵的秩为 3 ,故该向量组的秩也为 3 .}
	\end{question}
	
	\begin{question}
		{填空题}
		{向量}
		{中等}
		{12.线性方程组 \(x_1+x_2+x_3=0\) 的一个基础解系是(). }
		{由基础解系的概念可得 \(x_1=-x_2-x_3\) ,所以可得 \(\zeta_1=\left(\begin{array}{c}-1 \\ 1 \\ 0\end{array}\right), \zeta_2=\left(\begin{array}{c}-1 \\ 0 \\ 1\end{array}\right)\) 为一个基础解系.}
	\end{question}
	
	\begin{question}
		{填空题}
		{向量}
		{中等}
		{13.方程组 \(\left\{\begin{array}{l}x_1+x_2=0 \\ x_3-x_4=0\end{array}\right.\) 的基础解系是(). }
		{因为 \(\left\{\begin{array}{l}x_1=-x_2 \\ x_3=x_4\end{array}\right.\) 可得基础解系 \(\left(\begin{array}{c}-1 \\ 1 \\ 0 \\ 0\end{array}\right)\) 及 \(\left(\begin{array}{l}0 \\ 0 \\ 1 \\ 1\end{array}\right)\) .}
	\end{question}
	
	\begin{question}
		{填空题}
		{向量}
		{困难}
		{14.已知方程组 \(\left(\begin{array}{ccc}1 & 2 & 1 \\ 2 & 3 & a+2 \\ 1 & 2 & a^2-2 a-2\end{array}\right)\left(\begin{array}{l}x_1 \\ x_2 \\ x_3\end{array}\right)=\left(\begin{array}{c}1 \\ 3 \\ a-2\end{array}\right)\) 无解,则 \(a=()\) . }
		{化增广矩阵为阶梯型 \(\left(\begin{array}{cccc}1 & 2 & 1 & 1 \\ 2 & 3 & a+2 & 3 \\ 1 & 2 & a^2-2 a-2 & a-2\end{array}\right) \rightarrow\left(\begin{array}{cccc}1 & 2 & 1 & 1 \\ 0 & -1 & a & 1 \\ 0 & 0 & (a-3)(a+1) & a-3\end{array}\right)\) ,可见 \(a=-1\) 时方程组无解,\(a=3\) 时方程组有无穷多解.}
	\end{question}
	
	
	
	\begin{question}
		{计算题}
		{向量}
		{容易}
		{1.设向量 \(\boldsymbol{\alpha}=(2,1,0), \boldsymbol{\beta}=(0,3,1), \gamma=(0,1,0)\) ,则 \(2 \boldsymbol{\alpha}+\) \(3 \beta-\gamma=()\). }
		{由向量的运算法则可知 \(2 \boldsymbol{\alpha}+3 \boldsymbol{\beta}-\gamma=(4,10,3)\). }
	\end{question}
	
	
	
	\begin{question}
		{计算题}
		{向量}
		{中等}
		{2.将向量 \(\boldsymbol{\beta}=(3,5,-6)\) 表示为向量组 \(\boldsymbol{\alpha}_1=(1,0,1), \boldsymbol{\alpha}_2=(1,1,1), \boldsymbol{\alpha}_3=(0,-1,-1)\) 的线性组合为(\qquad). }
		{设 \(\boldsymbol{\beta}=m \boldsymbol{\alpha}_{\mathbf{1}}+\boldsymbol{n} \boldsymbol{\alpha}_{\mathbf{2}}+\boldsymbol{l} \boldsymbol{\alpha}_{\mathbf{3}}\) ,代入 \(\boldsymbol{\alpha}_1=(1,0,1), \boldsymbol{\alpha}_2=\) \((1,1,1) \boldsymbol{\alpha}_3=(0,-1,-1)\) ,并比较参数得 \(m=-11, n=14, l=9\) ,故 \(\beta=-11 \alpha_1+14 \alpha_2+9 \alpha_3\) .}
	\end{question}
	
	\begin{question}
		{计算题}
		{向量}
		{中等}
		{3.求向量组 \(\boldsymbol{\alpha}_1=(2,4,2), \boldsymbol{\alpha}_2=(1,1,0), \boldsymbol{\alpha}_3=(2,3,1), \boldsymbol{\alpha}_4=(3,5,2)\) 的秩(). }
		{以向量 \(\boldsymbol{\alpha}_1, \boldsymbol{\alpha}_2, \boldsymbol{\alpha}_3, \boldsymbol{\alpha}_4\) 为行向量作矩阵 \(\mathbf{A}\) ,则 \(\mathbf{A}\) 得秩就是 \(\boldsymbol{\alpha}_1, \boldsymbol{\alpha}_2, \boldsymbol{\alpha}_3, \boldsymbol{\alpha}_4\) 的秩,现在对 \(\mathbf{A}\) 的行或列进行初等变换化成: \(\mathbf{A}=\) \(\left(\begin{array}{lll}2 & 4 & 2 \\ 1 & 1 & 0 \\ 2 & 3 & 1 \\ 3 & 5 & 2\end{array}\right) \xrightarrow{r_1 \times \frac{1}{2}}\left(\begin{array}{lll}1 & 2 & 1 \\ 1 & 1 & 0 \\ 2 & 3 & 1 \\ 3 & 5 & 2\end{array}\right) \xrightarrow[\substack{r_3-2 r_1 \\ r_4-3 r_1}]{r_2-r_1}\left(\begin{array}{ccc}1 & 2 & 1 \\ 0 & 1 & 1 \\ 0 & -1 & -1 \\ 0 & -1 & -1\end{array}\right)\) \(\rightarrow\left(\begin{array}{lll}1 & 2 & 1 \\ 0 & 1 & 1 \\ 0 & 0 & 0 \\ 0 & 0 & 0\end{array}\right)\) ,进而可知矩阵的秩为 2 ,所以向量组的秩也为 2 .}
	\end{question}
	
	\begin{question}
		{计算题}
		{向量}
		{中等}
		{4.求向量组 \(\boldsymbol{\alpha}_1=(1,-2,3,-1), \boldsymbol{\alpha}_2=(2,-1,1,0), \boldsymbol{\alpha}_3=(1,-5,8,-3)\) 的一个极大线性无关组(). }
		{以 \(\boldsymbol{\alpha}_1, \boldsymbol{\alpha}_2, \boldsymbol{\alpha}_3\) 为列向量作矩阵 \(\mathbf{A}\) ,作初等行变换,将其化为
			\(A=\left(\begin{array}{lll}1 & 2 & 1 \\ -2 & -1 & -5 \\ 3 & 1 & 8 \\ -1 & 0 & -3\end{array}\right) \rightarrow\left(\begin{array}{lll}1 & 2 & 1 \\ 0 & 3 & -3 \\ 0 & -5 & 5 \\ 0 & 2 & -2\end{array}\right) \rightarrow\) \(\left(\begin{array}{lll}1 & 2 & 1 \\ 0 & 1 & -1 \\ 0 & 0 & 0 \\ 0 & 0 & 0\end{array}\right)\) .所以 \(\boldsymbol{\alpha}_1, \boldsymbol{\alpha}_2, \boldsymbol{\alpha}_3\) 的秩为 \(2, \boldsymbol{\alpha}_{\mathbf{1}}, \boldsymbol{\alpha}_{\mathbf{2}}\) 是 \(\boldsymbol{\alpha}_1, \boldsymbol{\alpha}_2, \boldsymbol{\alpha}_3\) 的一个极大线性无关组.}
	\end{question}
	
	\begin{question}
		{计算题}
		{向量}
		{困难}
		{5.求向量组 \(\boldsymbol{\alpha}_1=(3,0,1,2), \boldsymbol{\alpha}_2=(1,4,7,2), \boldsymbol{\alpha}_3=(1,10,17,4), \boldsymbol{\alpha}_4=(4,1,3,3)\) 的一个极大线性无关组(). }
		{对矩阵 \(\mathbf{A}=\left[\boldsymbol{\alpha}_1, \boldsymbol{a}_2, \boldsymbol{\alpha}_3, \boldsymbol{\alpha}_4\right]\) 作初等行变换,将其化为行最简形矩阵,有 \(\mathbf{A}=\left[\boldsymbol{\alpha}_1, \boldsymbol{\alpha}_2, \boldsymbol{\alpha}_3, \boldsymbol{\alpha}_4\right]=\left[\begin{array}{cccc}3 & 1 & 1 & 4 \\ 0 & 4 & 10 & 1 \\ 1 & 7 & 17 & 3 \\ 2 & 2 & 4 & 3\end{array}\right] \xrightarrow{r_3 \leftrightarrow r_1}\) \(\left[\begin{array}{cccc}1 & 7 & 17 & 3 \\ 0 & 4 & 10 & 1 \\ 3 & 1 & 1 & 4 \\ 2 & 2 & 4 & 3\end{array}\right] \xrightarrow{r_3-3 r_1, r_4-2 r_1}\)
			\(\left[\begin{array}{cccc}1 & 7 & 17 & 3 \\ 0 & 4 & 10 & 1 \\ 0 & -20 & -50 & -5 \\ 0 & -12 & -30 & -3\end{array}\right] \xrightarrow{\frac{1}{4} r_2,-\frac{1}{5} r_3,-\frac{1}{3} r_4}\left[\begin{array}{cccc}1 & 7 & 17 & 3 \\ 0 & 1 & \frac{5}{2} & \frac{1}{4} \\ 0 & 4 & 10 & 1 \\ 0 & 4 & 10 & 1\end{array}\right]\)
			\(\xrightarrow{r_1-7 r_2, r_3-4 r_2, r_4-4 r_2}\left[\begin{array}{cccc}1 & 0 & -\frac{1}{2} & \frac{5}{4} \\ 0 & 1 & \frac{5}{2} & \frac{1}{4} \\ 0 & 0 & 0 & 0 \\ 0 & 0 & 0 & 0\end{array}\right]=\mathbf{B}\) .设 \(\mathbf{B}=\) \(\left[\boldsymbol{\beta}_1, \boldsymbol{\beta}_2, \boldsymbol{\beta}_3, \boldsymbol{\beta}_4\right]\) ,则 \(\mathbf{B}\) 的秩为 \(2, \boldsymbol{\beta}_1, \boldsymbol{\beta}_2\) 线性无关,故 \(\boldsymbol{\beta}_1, \boldsymbol{\beta}_2\) 是 \(\boldsymbol{\beta}_1, \boldsymbol{\beta}_2, \boldsymbol{\beta}_3, \boldsymbol{\beta}_4\) 的极大无关组,所以 \(\boldsymbol{\alpha}_1, \boldsymbol{\alpha}_2\) 是 \(\boldsymbol{\alpha}_1, \boldsymbol{\alpha}_2, \boldsymbol{\alpha}_3, \boldsymbol{\alpha}_4\) 的一个极大线性无关组.}
	\end{question}
	
	\begin{question}
		{计算题}
		{向量}
		{容易}
		{6.求向量组 \(\boldsymbol{\alpha}_1=(2,4,2), \boldsymbol{\alpha}_2=(1,1,0), \boldsymbol{\alpha}_3=(2,3,1), \boldsymbol{\alpha}_4=(3,5,2)\) 的一个极大线性无关组(). }
		{以所给向量 \(\boldsymbol{\alpha}_1, \boldsymbol{\alpha}_2, \boldsymbol{\alpha}_3, \boldsymbol{\alpha}_4\) 为列向量作矩阵 \(\mathbf{A}\) ,并对该矩阵施行
			初等行变换,有 \(\mathbf{A}=\left[\begin{array}{llll}2 & 1 & 2 & 3 \\ 4 & 1 & 3 & 5 \\ 2 & 0 & 1 & 2\end{array}\right] \xrightarrow[r_3-r_1]{r_2-2 r_1}\)
			\[
			\begin{aligned}
				& {\left[\begin{array}{cccc}
						2 & 1 & 2 & 3 \\
						0 & -1 & -1 & -1 \\
						0 & -1 & -1 & -1
					\end{array}\right] \rightarrow\left[\begin{array}{llll}
						2 & 1 & 2 & 3 \\
						0 & 1 & 1 & 1 \\
						0 & 0 & 0 & 0
					\end{array}\right] \rightarrow} \\
				& {\left[\begin{array}{llll}
						1 & 0 & \frac{1}{2} & 1 \\
						0 & 1 & 1 & 1 \\
						0 & 0 & 0 & 0
					\end{array}\right]=\mathbf{B} \text {, 由于矩阵 } \mathbf{A} \text { 与 } \mathbf{B} \text { 的列向量组不仅有相同的秩, }}
			\end{aligned}
			\]
			而且有完全相同的线性关系,故 \(\boldsymbol{\alpha}_1, \boldsymbol{\alpha}_2\) 是 \(\boldsymbol{\alpha}_1, \boldsymbol{\alpha}_2, \boldsymbol{\alpha}_3, \boldsymbol{\alpha}_4\) 的一个极大线性无关组. }
	\end{question}
	
	
	\begin{question}
		{计算题}
		{向量}
		{困难}
		{7.求向量组 \(\boldsymbol{\alpha}_1=(1,-1,0,4), \boldsymbol{\alpha}_2=(0,3,1,2), \boldsymbol{\alpha}_3=(3,0,7,14), \boldsymbol{\alpha}_4=(1,-1,2,0), \boldsymbol{\alpha}_5=(2,1,5,6)\) 的秩().}
		{由向量 \(\boldsymbol{\alpha}_1, \boldsymbol{\alpha}_2, \boldsymbol{\alpha}_3, \boldsymbol{\alpha}_4, \boldsymbol{\alpha}_5\) 为行组成矩阵 \(A\) ,其秩即为该向量组的秩,对其进行行初等变换:
			\[
			\begin{aligned}
				&\left(\begin{array}{cccc}
					1 & -1 & 0 & 4 \\
					0 & 3 & 1 & 2 \\
					3 & 0 & 7 & 14 \\
					1 & -1 & 2 & 0 \\
					2 & 1 & 5 & 6
				\end{array}\right) \xrightarrow{\substack{\alpha_3 - 3\alpha_1\\ \alpha_4 - \alpha_1\\ \alpha_5 - 2\alpha_1}} \left(\begin{array}{cccc}
					1 & -1 & 0 & 4 \\
					0 & 3 & 1 & 2 \\
					0 & 3 & 7 & 2 \\
					0 & 0 & 2 & -4 \\
					0 & 3 & 5 & -2
				\end{array}\right) \rightarrow \left(\begin{array}{cccc}
					1 & -1 & 0 & 4 \\
					0 & 3 & 1 & 2 \\
					0 & 0 & 6 & 0 \\
					0 & 0 & 2 & -4 \\
					0 & 0 & 4 & -4
				\end{array}\right) \rightarrow \left(\begin{array}{cccc}
					1 & -1 & 0 & 4 \\
					0 & 3 & 1 & 2 \\
					0 & 0 & 1 & 0 \\
					0 & 0 & 0 & -4 \\
					0 & 0 & 0 & 0
				\end{array}\right),
			\end{aligned}
			\]
			所以秩为4. }
	\end{question}
	
	\begin{question}
		{计算题}
		{向量}
		{中等}
		{8.求齐次线性方程组的通解:\(\left\{\begin{array}{l}x-y+2z=0 \\ 3x-5y-z=0 \\ 3x-7y-8z=0\end{array}\right.\)().}
		{对系数矩阵行初等变换,得:
			\[
			\left(\begin{array}{rrr}
				1 & -1 & 2 \\
				3 & -5 & -1 \\
				3 & -7 & -8
			\end{array}\right) \rightarrow \left(\begin{array}{ccc}
				1 & -1 & 2 \\
				0 & -2 & -7 \\
				0 & -4 & -14
			\end{array}\right) \rightarrow \left(\begin{array}{ccc}
				1 & -1 & 2 \\
				0 & 2 & 7 \\
				0 & 0 & 0
			\end{array}\right) \rightarrow \left(\begin{array}{ccc}
				1 & -1 & 2 \\
				0 & 1 & \frac{7}{2} \\
				0 & 0 & 0
			\end{array}\right) \rightarrow \left(\begin{array}{ccc}
				1 & 0 & \frac{11}{2} \\
				0 & 1 & \frac{7}{2} \\
				0 & 0 & 0
			\end{array}\right),
			\]
			通解为 \(\left\{\begin{array}{l}x=-\frac{11}{2}z \\ y=-\frac{7}{2}z \\ z=z\end{array}\right.\) ,令 \(z=1\) 得基础解系为 \(\left(-\frac{11}{2},-\frac{7}{2},1\right)^T\) ,通解为 \(k\left(-\frac{11}{2},-\frac{7}{2},1\right)^T\).}
	\end{question}
	
	\begin{question}
		{计算题}
		{向量}
		{困难}
		{9.当 \(\lambda\) 取何值时,方程组 \(\left\{\begin{array}{l}4x+3y+z=\lambda x \\ 3x-4y+7z=\lambda y \\ x+7y-6z=\lambda z\end{array}\right.\) 有非零解?().}
		{原方程组等价于齐次线性方程组:
			\[
			\left\{\begin{array}{l}(4-\lambda)x + 3y + z = 0 \\
				3x + (-4-\lambda)y + 7z = 0 \\
				x + 7y + (-6-\lambda)z = 0
			\end{array}\right.
			\]
			求系数行列式:
			\[
			\left|\begin{array}{ccc}
				4-\lambda & 3 & 1 \\
				3 & -4-\lambda & 7 \\
				1 & 7 & -6-\lambda
			\end{array}\right| = \lambda(75 - 6\lambda - \lambda^2)
			\]
			因此,非零解存在当且仅当 \(\lambda=0\) 或 \(\lambda = -3 \pm 2\sqrt{21}\) 时成立. }
	\end{question}
	
	\begin{question}
		{计算题}
		{向量}
		{中等}
		{10.设四元非齐次线性方程组的系数矩阵的秩为3,已知 \(\xi_1=\left(\begin{array}{l}2\\3\\4\\5\end{array}\right)\) 是其解,且 \(\xi_2+\xi_3=\left(\begin{array}{l}1\\2\\3\\4\end{array}\right)\) ,求该方程组的通解(). }
		{因为 \(n=4, r=3\),原方程组的导出组(齐次部分)有1个基础解向量. \(\xi_1-\xi_2\) 与 \(\xi_1-\xi_3\) 为导出组解,二者之和:
			\[
			\eta = 2\xi_1 - (\xi_2+\xi_3) = 2\left(\begin{array}{l}2\\3\\4\\5\end{array}\right) - \left(\begin{array}{l}1\\2\\3\\4\end{array}\right) = \left(\begin{array}{l}3\\4\\5\\6\end{array}\right)
			\]
			为导出组的解. 通解为:
			\[
			\xi = \xi_1 + k\eta = \left(\begin{array}{l}2\\3\\4\\5\end{array}\right) + k\left(\begin{array}{l}3\\4\\5\\6\end{array}\right), \quad k \in \mathbb{R}.
			\]
		}
	\end{question}
	
	\begin{question}
		{计算题}
		{向量}
		{容易}
		{11.问 \(\lambda\) 为何值时,线性方程组 \(\left\{\begin{array}{l}x_1+x_3=\lambda \\ 4x_1+x_2+2x_3=\lambda+2 \\ 6x_1+x_2+4x_3=2\lambda+3\end{array}\right.\) 有解,并求通解(). }
		{构造增广矩阵:
			\[
			\left(\begin{array}{ccc|c}
				1 & 0 & 1 & \lambda \\
				4 & 1 & 2 & \lambda+2 \\
				6 & 1 & 4 & 2\lambda+3
			\end{array}\right) \rightarrow \left(\begin{array}{ccc|c}
				1 & 0 & 1 & \lambda \\
				0 & 1 & -2 & -3\lambda+2 \\
				0 & 0 & 0 & -\lambda+1
			\end{array}\right)
			\]
			方程有解需增广矩阵秩等于系数矩阵秩,即 \(-\lambda+1=0\),得 \(\lambda=1\). 代入得:
			\[
			\left(\begin{array}{ccc|c}
				1 & 0 & 1 & 1 \\
				0 & 1 & -2 & -1 \\
				0 & 0 & 0 & 0
			\end{array}\right)
			\]
			通解为:
			\[
			\left\{
			\begin{array}{l}
				x_1 = 1 - x_3 \\
				x_2 = -1 + 2x_3 \\
				x_3 = x_3
			\end{array}
			\right., \text{即 } \left(\begin{array}{c}1\\-1\\0\end{array}\right) + t\left(\begin{array}{c}-1\\2\\1\end{array}\right), \quad t \in \mathbb{R}.
			\]
		}
	\end{question}
	
	
		
	
	\begin{question}
		{选择题}
		{二次型}
		{容易}
		{1.矩阵 \(\mathbf{A}=\left(\begin{array}{ll}1 & 3 \\ 0 & 2\end{array}\right)\) 的特征值是(\qquad). 
			A. 1 和 2
			B. -1 和 2
			C. 3 和 -2
			D. 3 和 2}
		{因由 \(|\lambda \mathbf{E}-\mathbf{A}|=\left|\begin{array}{cc}\lambda-1 & -3 \\ 0 & \lambda-2\end{array}\right|=(\lambda-1)(\lambda-2)=\) 0 可解得 \(\lambda=1,2\). }
	\end{question}
	
	
	\begin{question}
		{选择题}
		{二次型}
		{容易}
		{2. 矩阵 \(\mathbf{A}=\left(\begin{array}{cc}-1 & 0 \\ 4 & 3\end{array}\right)\) 的特征值是(\qquad). 
			A. \(-1,3\)
			B. \(1,3\)
			C. \(-1,-3\)
			D. \(1,-3\)}
		{因由 \(|\lambda \mathbf{E}-\mathbf{A}|=\left|\begin{array}{cc}\lambda+1 & 0 \\ -4 & \lambda-3\end{array}\right|=(\lambda+1)(\lambda-3)=0\) 可解得 \(\lambda=-1,3\). }
	\end{question}
	
	\begin{question}
		{选择题}
		{二次型}
		{中等}
		{3. 若 \(\lambda=\frac{1}{2}\) 是 \(n\) 阶可逆矩阵 \(\mathbf{A}\) 的一个特征值,则矩阵 \(\mathbf{A}^2-2 \mathbf{A}+\mathbf{E}\) 有一个特征值等于(\qquad). 
			A. \(\frac{1}{4}\)
			B. \(\frac{1}{2}\)
			C. \(0\)
			D. \(2\)}
		{因为 \(\lambda=\frac{1}{2}\) 是矩阵 \(\mathbf{A}\) 的一个特征值,故由矩阵特征值的性质知 \(f(\mathbf{A})=\mathbf{A}^2-2 \mathbf{A}+\mathbf{E}\),则 \(f(\lambda)=\lambda^2-2 \lambda+1=\frac{1}{4}-1+1=\frac{1}{4}\). }
	\end{question}
	
	\begin{question}
		{选择题}
		{二次型}
		{容易}
		{4. 矩阵 \(\mathbf{A}=\left(\begin{array}{cc}4 & 2 \\ -1 & 1\end{array}\right)\) 的特征值是(\qquad). 
			A. \(2,3\)
			B. \(2,-3\)
			C. \(-2,3\)
			D. \(-2,-3\)}
		{因由 \(|\lambda \mathbf{E}-\mathbf{A}|=\left|\begin{array}{cc}\lambda-4 & -2 \\ 1 & \lambda-1\end{array}\right|=(\lambda-2)(\lambda-3)=0\) 可解得 \(\lambda=2,3\). }
	\end{question}
	
	\begin{question}
		{选择题}
		{二次型}
		{中等}
		{5. 矩阵 \(\mathbf{A}=\left(\begin{array}{ccc}1 & 3 & 0 \\ 1 & -1 & 0 \\ 2 & 4 & 3\end{array}\right)\) 的特征值是(\qquad). 
			A. \(-2,2\) 和 \(3\)
			B. \(-3,-2\) 和 \(2\)
			C. \(-2,-2\) 和 \(3\)
			D. \(2,2\) 和 \(3\)}
		{因由 \(|\lambda \mathbf{E}-\mathbf{A}|=\left(\lambda^2-4\right)(\lambda-3)=0\) 可解得 \(\lambda=-2,2\) 和 \(3\). }
	\end{question}
	
	\begin{question}
		{选择题}
		{二次型}
		{中等}
		{6. 若 \(\lambda=2\) 是 \(n\) 阶可逆矩阵 \(\mathbf{A}\) 的一个特征值,则矩阵 \((2 \mathbf{A})^{-1}\) 有一个特征值等于(\qquad). 
			A. \(\frac{1}{4}\)
			B. \(\frac{1}{2}\)
			C. \(4\)
			D. \(2\)}
		{因 \(\lambda=2\) 是 \(\mathbf{A}\) 的一个特征值,故由特征值的性质知 \((2 \mathbf{A})^{-1}=\frac{1}{2} \mathbf{A}^{-1}=\frac{1}{2} \cdot \frac{1}{2}=\frac{1}{4}\). }
	\end{question}
	
	\begin{question}
		{选择题}
		{二次型}
		{容易}
		{7. 若 \(\lambda=1\) 是 \(n\) 阶可逆矩阵 \(\mathbf{A}\) 的一个特征值,则矩阵 \(\mathbf{A}^2+3 \mathbf{A}+2 \mathbf{E}\) 有一个特征值等于(\qquad). 
			A. \(6\)
			B. \(2\)
			C. \(4\)
			D. \(3\)}
		{因 \(\lambda=1\) 是 \(\mathbf{A}\) 的一个特征值,故由特征值的性质知 \(\mathbf{A}^2+3 \mathbf{A}+2 \mathbf{E}\) 的一个特征值为 \(1^2+3×1+2=6\). }
	\end{question}
	
	\begin{question}
		{选择题}
		{二次型}
		{困难}
		{8. 若 \(\lambda=-1\) 是 \(n\) 阶可逆矩阵 \(\mathbf{A}\) 的一个特征值,则矩阵 \(2 \mathbf{A}^2-\mathbf{A}+3 \mathbf{E}\) 有一个特征值等于(\qquad). 
			A. \(6\)
			B. \(-6\)
			C. \(5\)
			D. \(3\)}
		{因 \(\lambda=-1\) 是 \(\mathbf{A}\) 的一个特征值,故由特征值的性质知 \(f(\mathbf{A})=2 \mathbf{A}^2-\mathbf{A}+3 \mathbf{E}\),则 \(f(\lambda)=2(-1)^2-(-1)+3=6\). }
	\end{question}
	
	\begin{question}
		{选择题}
		{二次型}
		{中等}
		{9. 若 \(1,1,2\) 为 3 阶矩阵 \(\mathbf{A}\) 的特征值,\(\mathbf{E}\) 为 3 阶单位矩阵,则 \(|\mathbf{A}-\mathbf{E}|=\)(\qquad). 
			A. \(0\)
			B. \(1\)
			C. \(2\)
			D. \(3\)}
		{因由特征值的性质知,\(\mathbf{A}-\mathbf{E}\) 的特征值为 \(0,0,1\),故 \(|\mathbf{A}-\mathbf{E}|=0×0×1=0\). }
	\end{question}
	
	\begin{question}
		{选择题}
		{二次型}
		{困难}
		{10. 若 \(1,2,3\) 为 3 阶矩阵 \(\mathbf{A}\) 的特征值,\(\mathbf{E}\) 为 3 阶单位矩阵,则 \(|2 \mathbf{A}+\mathbf{E}|=\)(\qquad). 
			A. \(105\)
			B. \(-105\)
			C. \(15\)
			D. \(-15\)}
		{因由特征值的性质知,\(2 \mathbf{A}+\mathbf{E}\) 的特征值为 \(3,5,7\),故 \(|2 \mathbf{A}+\mathbf{E}|=3×5×7=105\). }
	\end{question}
	
	
	
	
	
	
	\begin{question}
		{选择题}
		{二次型}
		{困难}
		{11.设 \(A=\left(\begin{array}{ccc}-1 & 1 & 0 \\ -4 & x & 0 \\ 1 & 0 & 2\end{array}\right)\) ,已知 \(A\) 的特征值为 \(2,1,1\) ,则 \(x=()\)成立. 
			A. 3
			B. -2
			C. 4
			D. -1}
		{由性质知,\(\lambda_1+\lambda_2+\cdots+\lambda_n=\operatorname{tr} A\) .因 \(A\) 的特征值为 \(2,1,1\) ,所以 \(2+1+1=-1+x+2\) ,由此解得,\(x=3\) .}
	\end{question}
	
	
	
	\begin{question}
		{选择题}
		{二次型}
		{困难}
		{12. \(n\) 阶矩阵 \(\mathbf{A}\) 可逆是 \(\mathbf{A}\) 的特征值都不为零的(). 
			A. 充要条件
			B. 充分条件
			C. 必要条件
			D. 无关条件}
		{设 \(\lambda\) 为 \(\mathbf{A}\) 的特征值,则(1)当 \(\mathbf{A}\) 可逆(即 \(|\mathbf{A}| \neq 0\) )时必有 \(\lambda \neq 0\) (否则由 \(|0 \mathbf{E}-\mathbf{A}|=(-1)^n|\mathbf{A}|=0\) 有 \(|\mathbf{A}|=0\) ,此与 \(|\mathbf{A}| \neq 0\) 矛盾):(2)当 \(\lambda \neq 0\) 时 \(\mathbf{A}\) 必可逆,即 \(|\mathbf{A}| \neq 0\)(否则由 \(|0 \mathbf{E}-\mathbf{A}|=\) \((-1)^n|\mathbf{A}|=0\) 知数 0 也为 \(\mathbf{A}\) 的特征值,此与 \(\lambda \neq 0\) 矛盾).综述(1)、(2)知,应选"充要条件"的结论.}
	\end{question}
	
	
	
	
	\begin{question}
		{填空题}
		{二次型}
		{容易}
		{1.已知三阶矩阵 \(A\) 的特征值为 \(1,2,-3\) ,则 \(\left|A^{-1}\right|=\)().}
		{因为三阶矩阵 \(A\) 的特征值为 \(1,2,-3\) ,所以 \(A \mid=-6\) ,| \(A^{-1} \left\lvert\,=-\frac{1}{6}\right.\)}
	\end{question}
	
	\begin{question}
		{填空题}
		{二次型}
		{容易}
		{2.已知 3 阶矩阵 \(A\) 的特征值为 \(1,-2,3\) ,则 \(2A\) 的特征值为(). }
		{由于 \(A\) 的特征值为 \(1,-2,3\) ,根据性质 \(kA\) 的特征值为 \(k\) 倍特征值,可得 \(2A\) 的特征值为 \(2, -4, 6\).}
	\end{question}
	
	\begin{question}
		{填空题}
		{二次型}
		{中等}
		{3.若 \(\lambda_0\) 为 \(n\) 阶矩阵 \(\mathbf{A}\) 的特征值,\(k\) 为任意实数,则 \(k\mathbf{A}\) 的特征值为(). }
		{设 \(\mathbf{A}\mathbf{x} = \lambda_0 \mathbf{x}\),则 \(k\mathbf{A}\mathbf{x} = k\lambda_0 \mathbf{x}\),说明 \(k\lambda_0\) 是 \(k\mathbf{A}\) 的特征值.}
	\end{question}
	
	\begin{question}
		{填空题}
		{二次型}
		{中等}
		{4.若 \(\lambda=0\) 是矩阵 \(\mathbf{A}=\left(\begin{array}{ccc}1 & 0 & 1 \\ 0 & 2 & 0 \\ 1 & 0 & a\end{array}\right)\) 的特征值,则 \(a=\)(). }
		{由于 \(0\) 是特征值,故 \(|\mathbf{A}| = 0\),计算 \(\left|\begin{array}{ccc}-1 & 0 & -1 \\ 0 & -2 & 0 \\ -1 & 0 & -a\end{array}\right| = -2(a - 1) = 0\),解得 \(a = 1\).}
	\end{question}
	
	\begin{question}
		{填空题}
		{二次型}
		{容易}
		{5.2 阶矩阵 \(\mathbf{A}=\left(\begin{array}{cc}3 & 1 \\ 5 & -1\end{array}\right)\) 的特征值是(). }
		{由特征方程 \(|\lambda \mathbf{E} - \mathbf{A}| = \left|\begin{array}{cc}\lambda - 3 & -1 \\ -5 & \lambda + 1\end{array}\right| = (\lambda - 4)(\lambda + 2)\) 可得特征值为 \(4\) 和 \(-2\).}
	\end{question}
	
	\begin{question}
		{填空题}
		{二次型}
		{中等}
		{6.3 阶矩阵 \(\mathbf{A}=\left(\begin{array}{ccc}-1 & 1 & 0 \\ -4 & 3 & 0 \\ 1 & 0 & 2\end{array}\right)\) 的特征值是(). }
		{由特征多项式 \(\left|\begin{array}{ccc}\lambda + 1 & -1 & 0 \\ 4 & \lambda - 3 & 0 \\ -1 & 0 & \lambda - 2\end{array}\right| = (\lambda - 2)(\lambda - 1)^2\) 可得 \(\lambda_1 = 2\), \(\lambda_2 = \lambda_3 = 1\).}
	\end{question}
	
	\begin{question}
		{填空题}
		{二次型}
		{中等}
		{7.若 \(\lambda=3\) 是矩阵 \(\mathbf{A}\) 的一个特征值,则矩阵 \(\mathbf{A}^2 - 3\mathbf{A} + 6\mathbf{E}\) 有一个特征值等于(). }
		{由特征值的性质可得,代入 \(\lambda = 3\) 有 \(3^2 - 3 \cdot 3 + 6 = 6\),故该矩阵有一个特征值为 6.}
	\end{question}
	
	\begin{question}
		{填空题}
		{二次型}
		{容易}
		{8.设 \(\alpha = (3,-2,-2)\),则 \(\|\alpha\| = \)(). }
		{由模长公式 \(\|\alpha\| = \sqrt{3^2 + (-2)^2 + (-2)^2} = \sqrt{17}\).}
	\end{question}
	
	\begin{question}
		{填空题}
		{二次型}
		{困难}
		{9.写出一个与向量 \(\alpha=\left(\begin{array}{l}1 \\ 1 \\ 1\end{array}\right), \beta=\left(\begin{array}{l}1 \\ -2 \\ 1\end{array}\right)\) 都正交的非零向量(). }
		{设所求向量为 \(\gamma=\left(\begin{array}{l}c_1 \\ c_2 \\ c_3\end{array}\right)\),由 \(<\alpha, \gamma>=0\) 与 \(<\beta, \gamma>=0\) 得:\\
			\(\begin{cases}
				c_1 + c_2 + c_3 = 0 \\
				c_1 - 2c_2 + c_3 = 0
			\end{cases}\) 解得 \(c_1 = -c_3, c_2 = 0\),取 \(c_3 = 1\),得 \(\gamma = \left(\begin{array}{c}-1 \\ 0 \\ 1\end{array}\right)\).}
	\end{question}
	
	
	
	\begin{question}
		{计算题}
		{二次型}
		{容易}
		{1.求3阶矩阵 \(\mathbf{A}=\left(\begin{array}{lll}1 & 0 & 1 \\ 0 & 2 & 0 \\ 1 & 0 & 1\end{array}\right)\) 的特征值(\qquad). }
		{由 \(|\lambda \mathbf{E}-\mathbf{A}|=\left|\begin{array}{ccc}\lambda-1 & 0 & -1 \\ 0 & \lambda-2 & 0 \\ -1 & 0 & \lambda-1\end{array}\right|=(\lambda-\)
			2)\(\left|\begin{array}{cc}\lambda-1 & -1 \\ -1 & \lambda-1\end{array}\right|=(\lambda-2)\left|\begin{array}{cc}\lambda-2 & \lambda-2 \\ -1 & \lambda-1\end{array}\right|=(\lambda-\)
			\(2)^2\left|\begin{array}{cc}1 & 1 \\ -1 & \lambda-1\end{array}\right|=\lambda(\lambda-2)^2=0\) 可解得特征值 \(\lambda_1=0\) 和 \(\lambda_2=\) \(\lambda_3=2\) .}
	\end{question}
	
	\begin{question}
		{计算题}
		{二次型}
		{容易}
		{2.已知 \(1 、 2 、 3\) 为 3 阶矩阵 \(\mathbf{A}\) 的特征值,\(E\) 为 3 阶单位矩阵,求 \(\left|\mathbf{A}^*\right|\)(\qquad). }
		{因 \(1、2、3\) 为 \(\mathbf{A}\) 的特征值,故由特征值的性质有 \(|\mathbf{A}| = 1 \times 2 \times 3 = 6\) ,从而结合等式 \(\mathbf{A}^* \mathbf{A} = |\mathbf{A}| \mathbf{E}\) 有 \(\left|\mathbf{A}^*\right| \cdot \left| \mathbf{A} \right| = \left| \mathbf{A}^* \mathbf{A} \right| = \left| |\mathbf{A}| \mathbf{E} \right| = |\mathbf{A}|^3 = 6^3 = 216\) ,
		即 \(6 \cdot \left|\mathbf{A}^*\right| = 216\) ,所以 \(\left|\mathbf{A}^*\right| = 36\) .}
	\end{question}
	
	\begin{question}
		{计算题}
		{二次型}
		{中等}
		{3.已知 \(3\) 阶矩阵 \(\mathbf{A}=\left(\begin{array}{ccc}2 & 1 & 0 \\ -1 & 0 & 0 \\ -2 & -1 & 2\end{array}\right)\) ,求 \(\mathbf{A}\) 的伴随方阵 \(\mathbf{A}^*\) 的特征值(\qquad). }
		{由 \(|\lambda \mathbf{E}-\mathbf{A}| = \left|\begin{array}{ccc} \lambda - 2 & -1 & 0 \\ 1 & \lambda & 0 \\ 2 & 1 & \lambda - 2 \end{array}\right| = (\lambda - 2) \left|\begin{array}{cc} \lambda - 2 & -1 \\ 1 & \lambda \end{array}\right| = (\lambda - 2)(\lambda^2 - 2\lambda + 1) = (\lambda - 2)(\lambda - 1)^2\) 可得 \(\mathbf{A}\) 的特征值为 \(\lambda_1 = \lambda_2 = 1\),\(\lambda_3 = 2\).
		根据特征值与伴随矩阵特征值的关系:\(\lambda_i^* = \dfrac{|\mathbf{A}|}{\lambda_i}\) ,其中 \(|\mathbf{A}| = 1 \cdot 1 \cdot 2 = 2\),故 \(\mathbf{A}^*\) 的特征值为 \(\lambda_1^* = \lambda_2^* = \dfrac{2}{1} = 2\),\(\lambda_3^* = \dfrac{2}{2} = 1\).}
	\end{question}

	\begin{question}
		{计算题}
		{二次型}
		{中等}
		{4.求 \(2\) 阶矩阵 \(\mathbf{A}=\left(\begin{array}{cc}3 & 4 \\ 5 & 2\end{array}\right)\) 的特征值与特征向量(\qquad). }
		{由 \(|\lambda \mathbf{E} - \mathbf{A}| = \left|\begin{array}{cc} \lambda - 3 & -4 \\ -5 & \lambda - 2 \end{array}\right| = (\lambda - 3)(\lambda - 2) - (-4)(-5) = \lambda^2 - 5\lambda - 20 = 0\),解得特征值为 \(\lambda_1 = 7\),\(\lambda_2 = -2\).当 \(\lambda_1 = 7\) 时,由齐次线性方程组 \((7\mathbf{E} - \mathbf{A})\mathbf{x} = \mathbf{0}\) ,即 \(\left\{\begin{array}{l}4x_1 - 4x_2 = 0 \\ -5x_1 + 5x_2 = 0\end{array}\right.\) 可得 \(x_1 = x_2\),对应特征向量为 \(k_1 \begin{pmatrix}1 \\ 1\end{pmatrix}\) ;当 \(\lambda_2 = -2\) 时,由 \((-2\mathbf{E} - \mathbf{A})\mathbf{x} = \mathbf{0}\) ,即 \(\left\{\begin{array}{l}-5x_1 - 4x_2 = 0 \\ -5x_1 - 4x_2 = 0\end{array}\right.\) 可得 \(x_1 = -\dfrac{4}{5}x_2\),对应特征向量为 \(k_2 \begin{pmatrix}4 \\ -5\end{pmatrix}\).}
	\end{question}

	\begin{question}
		{计算题}
		{二次型}
		{困难}
		{5.求 \(3\) 阶矩阵 \(\mathbf{A}=\left(\begin{array}{ccc}1 & 2 & 3 \\ 2 & 1 & 3 \\ 3 & 3 & 6\end{array}\right)\) 的特征值与特征向量(\qquad). }
		{由 \(|\lambda \mathbf{E} - \mathbf{A}| = \left|\begin{array}{ccc} \lambda - 1 & -2 & -3 \\ -2 & \lambda - 1 & -3 \\ -3 & -3 & \lambda - 6 \end{array}\right| = \lambda(\lambda + 1)(\lambda - 9) = 0\),得特征值 \(\lambda_1 = 0\),\(\lambda_2 = -1\),\(\lambda_3 = 9\).当 \(\lambda_1 = 0\) 时,解 \(\mathbf{A}\mathbf{x} = 0\) 得 \(\left\{\begin{array}{l}x_1 = -x_3 \\ x_2 = -x_3\end{array}\right.\),特征向量为 \(k_1 \begin{pmatrix}1 \\ 1 \\ -1\end{pmatrix}\);当 \(\lambda_2 = -1\) 时,解 \((-\mathbf{E} - \mathbf{A})\mathbf{x} = 0\),得 \(\left\{\begin{array}{l}x_1 = -x_2 \\ x_3 = 0\end{array}\right.\),特征向量为 \(k_2 \begin{pmatrix}1 \\ -1 \\ 0\end{pmatrix}\);当 \(\lambda_3 = 9\) 时,解 \((9\mathbf{E} - \mathbf{A})\mathbf{x} = 0\),得 \(\left\{\begin{array}{l}x_1 = \dfrac{1}{2}x_3 \\ x_2 = \dfrac{1}{2}x_3\end{array}\right.\),特征向量为 \(k_3 \begin{pmatrix}1 \\ 1 \\ 2\end{pmatrix}\).}
	\end{question}


\end{document}

